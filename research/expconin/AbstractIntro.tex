\begin{abstract}
Estimating divergences in a consistent way is of great importance in many
machine learning tasks. Although this is a fundamental problem in nonparametric
statistics, to the best of our knowledge there has been no finite sample
exponential inequality convergence bound derived for any divergence estimators.
The main contribution of our work is to provide such a bound for an estimator
of R\'enyi-$\alpha$ divergence for a smooth H\"older class of densities on the
$d$-dimensional unit cube $[0,1]^d$.
%We also demonstrate the applicability of the estimator in mutual information estimation and
We also illustrate our theoretical results with a numerical experiment.
\end{abstract}

\section{Introduction}
There are several important problems in machine learning and statistics that
require the estimation of the distance or divergence between distributions. In
the past few decades many different kinds of divergences have been defined to
measure the discrepancy between distributions. They include the
Kullback--Leibler (\acro{KL}) \citep{kullback51KL},
R\'enyi-$\alpha$ \citep{renyi61measures,renyi70probability},
Tsallis-$\alpha$ \citep{villmann10mathematical},
Bregman \citep{bregman67divergence},
%Jensen--Shannon \citep{lin91divergence},
$L_p$,
maximum mean discrepancy \citep{Borgwardt06MMD},
Csisz\'ar's-$f$ divergence \citep{csiszar67information}
and many others.

Most machine learning algorithms operate on finite dimensional feature vectors.
Using divergence estimators one can develop machine learning algorithms (such
as regression, classification, clustering, and  others) that can operate on
sets and distributions \citep{poczos12kernelimages,oliva13ICML}. Under certain
conditions, divergences can estimate entropy and mutual information. Entropy
estimators are important in
goodness-of-fit testing \citep{goria05new},
parameter estimation in semi-parametric models \citep{Wolsztynski85minimum},
studying fractal random walks \citep{Alemany94fractal},
and texture classification \citep{hero02alpha,hero2002aes}.
Mutual information estimators have been used
in feature selection \citep{peng05feature},
clustering \citep{aghagolzadeh07hierarchical},
%causality detection \citep{Hlavackova07causality},
optimal experimental design \citep{lewi07realtime}, %poczos09identification
f\acro{MRI} data processing \citep{chai09exploring},
prediction of protein structures \citep{adami04information},
boosting and facial expression recognition \citep{Shan05conditionalmutual}.
Both entropy estimators and mutual information estimators have been used for
independent component and subspace analysis
\citep{radical03,szabo07undercomplete_TCC}, %poczos05geodesic %Hulle08constrained
as well as for image registration
\citep{kybic06incremental,hero02alpha,hero2002aes}.
For further applications, see
\citet{Leonenko-Pronzato-Savani2008}.

In this paper we will focus on the estimation of R\'enyi-$\alpha$ divergences.
This important class contains the Kullback--Leibler divergence as the
$\alpha\to 1$ limit case and can also be related to the Tsallis-$\alpha$,
Jensen-Shannon, and Hellinger divergences.\footnote{Some of the divergences
mentioned in the paper are distances as well. To simplify the treatment we will
call all of them divergences.}
It can be shown that many information theoretic quantities
(including entropy, conditional entropy, and mutual information)
can be computed as special cases of R\'enyi-$\alpha$ divergence.

In our framework, we assume that the underlying distributions are not given
explicitly. Only two finite, independent and identically distributed (i.i.d.)
samples are given from some unknown, continuous, nonparametric distributions.

Although many of the above mentioned divergences were
defined a couple of decades ago, interestingly there are still many open
questions left to be answered about the properties of their estimators. In
particular, even simple questions, such as rates are unknown for many
estimators, and to the best of our knowledge no finite sample exponential
concentration bounds have ever been derived for divergence estimators.

\textbf{Our main contribution} is to derive an exponential concentration bound
for a particular consistent, nonparametric, R\'enyi-$\alpha$ divergence
estimator. We illustrate the behaviour of the estimator with a numerical
experiment.

\subsection*{Organization}
In the next section we discuss related work (Section~\ref{sec:related}).
In Section~\ref{sec:Problem} we formally
define the R\'enyi-$\alpha$ divergence estimation problem, and
introduce the notation and the assumptions used in the paper.
Section~\ref{sec:estimator} presents the divergence estimator that we study in
the paper.
Section~\ref{sec:main-result} contains our main theoretical contibutions
concerning the exponential concentration bound of the divergence estimator.
Section~\ref{sec:Proofs} contains the proofs of our theorems.
To illustrate the behaviour of the estimator, we provide a simple numerical
experiment in Section~\ref{sec:numerical}.
We draw conclusions in Section~\ref{sec:Conclusion}.

\section{Related Work} \label{sec:related}

Probably the closest work to our contribution is \citet{liu12exponential}, who
derived exponential-concentration bound for an estimator of the two-dimensional
Shannon entropy. We generalize these results in several aspects:
\begin{enumerate}
\item The estimator of \citet{liu12exponential} operates in the unit square $[0,1]^2$. Our
estimator operates on the $d$-dimensional unit hypercube $[0,1]^d$.
\item In \citet{liu12exponential} the exponential concentration inequality was proven for
densities in the H\"{o}lder class $\Sigma_\kappa(2,L,2)$, whereas our inequality
applies for densities in the H\"{o}lder class $\Sigma_\kappa(\beta,L,r)$ for
any fixed $\beta \geq 0, r \geq 1$ (see Section~\ref{sec:Assumptions} for definitions of these
H\"older classes).
% (the estimator converges as $O\left( n^{-\frac{\beta}{d + \beta}} \right)$).
\item While \citet{liu12exponential} estimated the Shannon entropy using one i.i.d. sample set, in this paper
we estimate the R\'enyi-$\alpha$ divergence using two i.i.d. sample sets.
\end{enumerate}

To the best of our knowledge, only very few consistent nonparametric
estimators exist for R\'enyi-$\alpha$ divergences: \citet{poczos11AISTATS}
proposed a $k$-nearest neighbour based estimator and proved the weak
consistency of the estimator, but they did not study the convergence rate of
the estimator.

\citet{Wang-Kulkarni-Verdu2009} provided an estimator for
the $\alpha \to 1$ limit case only, i.e., for the
\acro{KL}-divergence. They did not study the convergence rate either, and
there is also an apparent error in this work; they applied the reverse Fatou
lemma under conditions when it does not hold. This error originates in the work
 \citet{kozachenko87statistical} and can also be found in other works.
Recently, \citet{perez08estimation} has proposed another consistency proof for
this estimator, but it also contains some errors: he applies the strong law of
large numbers under conditions when it does not hold and also assumes that
convergence in probability implies almost sure convergence.

\citet{hero02alpha,hero2002aes} also investigated the
R\'enyi divergence estimation problem but assumed that one of the two
density functions is known. \citet{gupta12parametric} developed
algorithms for estimating the Shannon entropy and the \acro{KL}
divergence for certain parametric
families.

Recently, \citet{nguyen10estimating}
developed methods for estimating $f$-divergences using their
variational characterization properties. They estimate the likelihood
ratio of the two underlying densities and plug that into the
divergence formulas. This approach involves solving a convex
minimization problem over an infinite-dimensional function space. For
certain function classes defined by reproducing kernel Hilbert spaces
(\acro{RKHS}), however, they were able to reduce the computational
load from solving infinite-dimensional problems to solving
$n$-dimensional problems, where $n$ denotes the sample size. When $n$
is large, solving these convex problems can still be very demanding.
They studied the convergence rate of the estimator, but did not derive
exponential concentration bounds for the estimator.

\citet{2010arXiv1012.4188S,Laurent96Efficient,birge95estimation} studied the
estimation of non-linear functionals of density. They, however, did not study
the R\'enyi divergence estimation and did not derive exponential concentration
bounds either. Using ensemble estimators, \citet{sricharan12ensemble} derived
fast rates for entropy estimation, but they did not investigate the divergence
estimation problem.

\citet{Leonenko-Pronzato-Savani2008} and \citet{goria05new}
considered Shannon and R\'enyi-$\alpha$ entropy estimation from a
single sample.\footnote{The original presentations of these works
contained some errors; \citet{leonenko10correction} provide
corrections for some of these theorems.} In this work, we study
divergence estimators using two independent samples.   Recently,
\citet{pal10estimation} 
proposed a method for consistent R\'enyi information estimation, but
this estimator also uses one sample only and cannot be used for estimating
R\'enyi divergences.

Further information and useful reviews of several
different divergences can be found, e.g.,
in \citet{villmann10mathematical}. %\citet{cichocki09nonnegative}
