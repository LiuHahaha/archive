%% This document created by Scientific Word (R) Version 3.5

\documentclass{article}%
\usepackage{makeidx}
\usepackage{amsmath}
\usepackage{graphicx}
\usepackage{amsfonts}
\usepackage{amssymb}%
\usepackage{enumerate}%
\usepackage{fullpage}%
\setcounter{MaxMatrixCols}{30}
%TCIDATA{OutputFilter=latex2.dll}
%TCIDATA{Version=5.50.0.2960}
%TCIDATA{CSTFile=LaTeX article (bright).cst}
%TCIDATA{Created=Monday, August 27, 2007 13:53:03}
%TCIDATA{LastRevised=Wednesday, September 07, 2011 07:21:54}
%TCIDATA{<META NAME="GraphicsSave" CONTENT="32">}
%TCIDATA{<META NAME="SaveForMode" CONTENT="1">}
%TCIDATA{BibliographyScheme=Manual}
%TCIDATA{<META NAME="DocumentShell" CONTENT="General\SW\Blank - Standard LaTeX Article">}
%TCIDATA{Language=American English}
%BeginMSIPreambleData
\providecommand{\U}[1]{\protect\rule{.1in}{.1in}}
%EndMSIPreambleData
\providecommand{\U}[1]{\protect\rule{.1in}{.1in}}
\providecommand{\U}[1]{\protect\rule{.1in}{.1in}}
\providecommand{\U}[1]{\protect\rule{.1in}{.1in}}
\providecommand{\U}[1]{\protect\rule{.1in}{.1in}}
\providecommand{\U}[1]{\protect\rule{.1in}{.1in}}
\providecommand{\U}[1]{\protect\rule{.1in}{.1in}}
\providecommand{\U}[1]{\protect\rule{.1in}{.1in}}
\newtheorem{theorem}{Theorem}
\newtheorem{acknowledgement}[theorem]{Acknowledgement}
\newtheorem{algorithm}[theorem]{Algorithm}
\newtheorem{axiom}[theorem]{Axiom}
\newtheorem{case}[theorem]{Case}
\newtheorem{claim}[theorem]{Claim}
\newtheorem{conclusion}[theorem]{Conclusion}
\newtheorem{condition}[theorem]{Condition}
\newtheorem{conjecture}[theorem]{Conjecture}
\newtheorem{corollary}[theorem]{Corollary}
\newtheorem{criterion}[theorem]{Criterion}
\newtheorem{definition}[theorem]{Definition}
\newtheorem{example}[theorem]{Example}
\newtheorem{exercise}[theorem]{Exercise}
\newtheorem{lemma}[theorem]{Lemma}
\newtheorem{notation}[theorem]{Notation}
\newtheorem{problem}[theorem]{Problem}
\newtheorem{proposition}[theorem]{Proposition}
\newtheorem{remark}[theorem]{Remark}
\newtheorem{solution}[theorem]{Solution}
\newtheorem{summary}[theorem]{Summary}
\newenvironment{proof}[1][Proof]{\textbf{#1.} }{\ \rule{0.5em}{0.5em}}
\begin{document}

\begin{center}
\textbf{Shashank Singh}

\textbf{sss1@andrew.cmu.edu}

\textbf{21-373 \quad Algebraic Structures, Fall 2011}

\textbf{Assignment 1}

\textbf{Due: Friday, September 16}\\
\end{center}
%DONE
\textbf{Exercise 1:} 

Suppose $G$ is a group such that, $\forall g \in G$,
$g^2 = e$. Then, by definition of the inverse, $\forall g \in G$, $g = g^{-1}$.
As shown in class (Remark 3.4), $\forall a, b \in G$,
$(ab)^{-1} = b^{-1}a^{-1}$, so that $ab = (ab)^{-1} = b^{-1}a^{-1} = ba$. Thus,
$G$ is Abelian. \\\\
%DONE
\textbf{Exercise 2:}
\begin{enumerate}[i.]
	\item  Suppose $G$ is a group with $|G| = 2n$ for some $n \in
\mathbb{N}$. Clearly, the only element in $G$ of order less than $2$ is $e$.
$\forall g \in G$ of order greater than $2$, two sets $A$ and $B$ can be
constructed such that $A \cup B = G$, $A \cap B = \emptyset$, $g \in A$ if and
only if $g^{-1} \in B$; that is, the elements of order greater than $2$ can be
"split" into two disjoint sets, such that the elements of each are the
inverses of the elements of the other. Since inversion ($^{-1}$) gives a
bijection between these two sets, $|A| = |B| = k$ for some $k \in \mathbb{N}$,
so that the number of elements of order greater than $2$ is $2k$. Thus, for
some $k \in \mathbb{N}$, the number of elements of order $2$ in $G$
is given by $2n - (2k + 1) = 2(n - k - 1) + 1,$ which is odd, since
$n - k - 1 \in \mathbb{N}$.

%DONE
	\item  Let $n \in \mathbb{N}$ be odd, and let $G$ be an
Abelian group of order $2n = 2(2k + 1) = 4k + 2$, for some $k \in \mathbb{N}$.
By the result of part i., $G$ contains at least one element $g$ with $g^2 = e$.
Suppose, for sake of contradiction, that $\exists$ distinct $g_1, g_2 \in G$
with $g_1^2 = g_2^2 = e$. Then, since $G$ is Abelian, it is clear that $\{e,
g_1,g_2,g_1g_2\}$ is a subgroup of $G$ (since all of its elements are of order
$2$), and that it has order $4$. By Lagrange's Theorem, then, $4$ divides the
order of $G$. However, this is impossible, since $4$ cannot divide $4k + 2$.\\
Note that this is not necessarily the case if $P$ is non-Abelian. Consider, for
instance, the group of permutations on $3$ elements, denoted $P_3$. $P_3$ is
of order $3! = 6 = 2n$, where $n = 3$ is odd. However, the $3$ transpositions
in $P_3$ (denoted here by their cycle decomposition), $(2\;1)(3)$, $(3\;1)(2)$,
and $(3\;2)(1)$, are all of order $2$.

\end{enumerate}
%DONE
\textbf{Exercise 3:} 
\begin{enumerate}[i.]
	\item Let $G$ be a group, and suppose, for sake of
contradiction, that $\exists$ proper subgroups $A,B \subset G$ with
$G = A \cup B$. Then, $\exists a \in A$ with $a \not \in B$, and
$\exists b \in B$ with $b \not \in A$, since, if either were not the case,
then $G = A \cup B = A$ or $G = A \cup B = B$, violating the supposition that
$A$ and $B$ are \emph{proper} subgroups. Since $G$ is a group, $ab \in G$,
so $ab \in A$ or $ab \in B$. If the former, then, since $a^{-1} \in A$,
$b = eb = \left( a^{-1}a \right) b = a^{-1}(ab) \in A$, and, if the latter,
then, since $b^{-1} \in B$, $a = ae  = a \left( bb^{-1} \right) = (ab)b^{-1}
\in B$. In either case, the existence of $a$ and $b$ as chosen above is
contradicted, and so no such $A$ and $B$ can exist.
	\item Consider the group $G = \{e, a, b, c\}$, under the
operation determined by the following table (with $e$ as the identity
element): \\ \\
\begin{tabular}{ c | c c c}
$\star$ & $a$ & $b$ & $c$ \\
\hline
    $a$ & $e$ & $c$ & $b$ \\
    $b$ & $c$ & $e$ & $a$ \\
    $c$ & $b$ & $a$ & $e$ \\
\end{tabular}\\ \\

Then, $G$ is the union of the three groups $\{e,a\}$, $\{e,b\}$, $\{e,c\}$,
under the same operation.

\end{enumerate}
%DONE
\textbf{Exercise 4:} 

Let $S$ denote the set of infinite groups with
a finite number of subgroups. Suppose, for sake of contradiction, that
$S \neq \emptyset$. Let $\#: S \rightarrow \mathbb{N}$ such that, $\forall G \in
S$, $\#(G)$ gives the number of subgroups of $G$. The elements of $S$ can be
well-ordered by the number of subgroups each has; that is, $\exists G \in S$
such that, $\forall A \in S$,  $\#(G) \leq \#(A)$. Then, if $A \subset G$ is a
proper subgroup of $G$, $A$ is finite, since, otherwise, $A \in S$
and $\#(A) < \#(G)$, contradicting the choice of $G$. Therefore, since $G$ has
a finite number of proper subgroups, all of which are finite, the union of the
proper subgroups of $G$ is finite, and, since $G$ is infinite, $\exists g \in
G$ such that $g$ is not contained in any proper subgroup of $G$. However,
since $G$ is a group, $G_g = \{g^n | n \in \mathbb{Z}\}$ (the cyclic group of
$g$) is a subgroup of $G$ such that $g \in G_g$, contradicting the choice of
$g$. Thus, $S = \emptyset$, so all infinite groups have an infinite number of
subgroups; that is, in contrapositive, all groups with a finite number of
subgroups are finite. \\\\
%DONE
\textbf{Exercise 5:} 
\begin{enumerate}[ii.]
	\item This is not necessarily the case if $P$ is
non-Abelian. Consider, for instance, the group of permutations on $3$ elements,
denoted $P_3$. Denoting permutations by their cycle decomposition, $(2\;1)(3)$
is of order $2$, and $(1\;2\;3)$ is of order $3$, but no element of $P_3$ is of
order greater than $3$, let alone $6 = \mbox{lcm}(2,3)$. \\

\end{enumerate}
%DONE
\textbf{Exercise 6:} 
\begin{enumerate}[i.]
	\item Let $G$ be an Abelian group, and let $H = \{g \in G |
g^n = e$, for some $n \in \mathbb{N}\backslash\{0\} \subseteq G$.  
Letting $e$ denote the identity on $G$, $e \in H$, since $e^1 = e$.
Suppose $a, b \in H$, with $a^m = b^n = e$. Then, since $G$ is Abelian,
$(ab)^{mn} = a^{mn}b^{mn} = \left(a^m\right)^n \left(b^n\right)^m = e^ne^m
= e$. Thus, $H$ is closed under the operation on $G$.
Suppose $g \in H$, with $g^k = e$. Then $\left(g^{-1}\right)^k = \left(g^k
\right)^{-1} = e^{-1} = e$, so $g^{-1} \in H$.
Thus, $H \leq G$.
%DONE
	\item  Calculating $A^2$, $A^3$, and $A^4$ shows that $A$ is
of order $4$, and calculating $B^2$ and $B^3$ shows that $B$ is of order $3$.
However, a simple proof by induction shows that, $\forall n \in \mathbb{N}$,
$\left((AB)_{1,2}\right)^n = n$ ($(AB)_{1,2}$ denotes the second element of the
first row of $AB$). Thus, since the corresponding entry of the $2 \times 2$
identity matrix is $0$, $AB$ is of infinite order.
%DONE
	\item Let $a = (1,1), b = (0,-1)$. Then, $\forall n \in
\mathbb{N}$, $na = (0,n) \neq (0,0)$ or $na = (1,n) \neq (0,0)$, and $nb =
(0,-n) \neq (0,0)$, so $a$ and $b$ are both of infinite order. However, $a + b
= (1,0) \neq (0,0)$, so that $2(a + b) = (0,0)$, the identity element of
$\mathbb{Z}_2 \times \mathbb{Z}$.
\end{enumerate}
\end{document}
