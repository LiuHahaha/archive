\documentclass{article}%
\usepackage{makeidx}
\usepackage{amsmath}
\usepackage{graphicx}
\usepackage{amsfonts}
\usepackage{amssymb}%
\usepackage{enumerate}%
\usepackage{fullpage}%
\setcounter{MaxMatrixCols}{30}
\providecommand{\U}[1]{\protect\rule{.1in}{.1in}}
\providecommand{\U}[1]{\protect\rule{.1in}{.1in}}
\providecommand{\U}[1]{\protect\rule{.1in}{.1in}}
\providecommand{\U}[1]{\protect\rule{.1in}{.1in}}
\providecommand{\U}[1]{\protect\rule{.1in}{.1in}}
\providecommand{\U}[1]{\protect\rule{.1in}{.1in}}
\providecommand{\U}[1]{\protect\rule{.1in}{.1in}}
\providecommand{\U}[1]{\protect\rule{.1in}{.1in}}
\newenvironment{proof}[1][Proof]{\textbf{#1.} }{\ \rule{0.5em}{0.5em}}
\allowdisplaybreaks
\begin{document}

\begin{center}
\textbf{Shashank Singh}

\textbf{sss1@andrew.cmu.edu}

\textbf{21-373 \quad Honors Algebraic Structures, Fall 2011}

\textbf{Assignment 7}

\textbf{Due: Friday, November 11}\\

\textbf{Extension granted until Saturday, November 12}
\end{center}

%DONE
\textbf{Exercise 43: i.} Suppose, for sake of contradiction, that
$2\mathbb{Z}$ and $3\mathbb{Z}$ were isomorphic, so that there exists an
isomorphism $f: 2\mathbb{Z} \rightarrow 3\mathbb{Z}$. Note that, $\forall x
\in \mathbb{Z}$, $2x = x^2$ if and only if $x = 0$ or $x = 2$, and that $f(2)
 \neq 0$, since $f$ is bijective, and $f(0) = 0$ (because $f(0) = f(0) 
= f(0) + f(0)$). Thus, since $f$ is an isomorphism, $f(2) + f(2) = f(2 + 2) =
f(4) = f(2*2) = f(2)*f(2)$, so that $\exists x = f(2) \in 3\mathbb{Z}$.
However, since $x \neq 0$, $x = 2$, contradicting the choice of $x \in
3\mathbb{Z}$. Thus, $2\mathbb{Z}$ and $3\mathbb{Z}$ are not isomorphic.
\qquad $\blacksquare$ \\

%DONE
\textbf{ii.} Suppose, for sake of contradiction, that $\mathbb{Z}[x]$ and
$\mathbb{Q}[x]$ were isomorphic, so that there exists an isomorphism $f:
\mathbb{Q} \rightarrow \mathbb{Z}$. Note that no element of $\mathbb{Z}$
(except $1$) is a unit and that no non-zero element of $\mathbb{Z}$ is
nilpotent, and thus, by the result of Exercise 37 (from Assignment 6),
$\forall P \in \mathbb{Z}[]$, $P$ is a unit if and only if $P = 1$. Then,
since $f$ is an isomorphism, $1 = f(1) = f(2*2^{-1}) = f(2)f(2^{-1})$.
However, this implies that $f(2)$ is a unit in $\mathbb{Z}[x]$, which is
a contradiction, since $f(2) \neq 1$ (as $f$ is a bijection, and $f(1) = 1$).
\qquad $\blacksquare$ \\

%DONE
\textbf{Exercise 44: i.} Let $J \subseteq \mathbb{Z}$ be the set of
polynomials $P \in \mathbb{Z}[x]$ such that $P$ has a constant term which is a
multiple of $3$. Then, for $P \in \mathbb{Z}[x]$, $P \in J$ if and only if $P$
can be written in the form $3n + xA$, for some $A \in \mathbb{Z}[x]$,
$n \in \mathbb{Z}$. Thus, suppose $P, Q \in J$, with $P = 3m + xA$,
$Q = 3n + xB$, for some
$A, B \in \mathbb{Z}[x]$, $m,n \in \mathbb{Z}$. $J$ inherits associativity and
commutativity of addition and multiplication and distributivity of
multiplication over addition from $\mathbb{Z}$. Since $0 = 3(0) + x(0)$,
$0 \in J$. Since $-P = 3(-m) + x(-A)$, $(-P) \in J$. $P + Q = 3(m + n) +
x(A + B)$, and $PQ = 3(3mn) + x(3mB + 3nA + AB)$, $(P + Q), PQ \in J$. Suppose
$C \in \mathbb{Z}[x]$, so that $C = k + xD$, for some $D \in \mathbb{Z}[x]$,
$k \in \mathbb{Z}$. Then, $CP = PC = 3(mk) + x(3mD + kA + xAD) \in J$, so $J$
is an ideal of $\mathbb{Z}[x]$. \qquad $\blacksquare$ \\

%DONE
\textbf{ii.} For $P, Q \in \mathbb{Z}[x]$ such that $P = 1$ and $Q = x^2$, $P$
has a coefficient of $x^2$ which is a multiple of $3$, but $PQ = x^2$ does
not. Thus, the given set is not an ideal of $\mathbb{Z}[x]$.
\qquad $\blacksquare$ \\

%DONE
\textbf{iii.} Let $J \subseteq \mathbb{Z}[x]$ be the set of polynomials $P \in
\mathbb{Z}[x]$ such that the coefficients of the constant, linear, and
quadratic terms of $P$ are zero. Then, for $P \in \mathbb{Z}[x]$, $P \in J$ if
and only if $P = x^3A$, for some $A \in \mathbb{Z}[x]$. Thus, suppose $P, Q
\in J$, with $P = x^3A, Q = x^3B$, for some $A, B \in \mathbb{Z}[x]$. $J$
inherits associativity and commutativity of addition and multiplication and
distributivity of multiplication over addition from $\mathbb{Z}$.
$0 = x^3(0)$, so $0 \in J$. $-P = x^3(-A)$, so $(-P) \in J$.
$P + Q = x^3(A + B)$, and $PQ = x^3(x^3AB)$, so $(P + Q), PQ \in J$. Suppose
$C \in \mathbb{Z}[x]$. Then, $CP = PC = x^3(AC) \in J$, so $J$ is an ideal of
$\mathbb{Z}[x]$. \qquad $\blacksquare$ \\

%DONE
\textbf{iv.} For $P, Q \in \mathbb{Z}[x]$ such that $P = 1$ and $Q = x$, only
even powers of $x$ appear in $P$, but an odd power of $x$ appears in $PQ = x$.
Thus, the given set is not an ideal of $\mathbb{Z}[x]$.
\qquad $\blacksquare$ \\

%DONE
\textbf{v.} Let $J \subseteq \mathbb{Z}[x]$ be the set of polynomials $P \in
\mathbb{Z}[x]$ such that the sum of all coefficients of $P$ is zero.
Then, for $P \in \mathbb{Z}[x]$, $P \in J$ if and only if $P(1) = 0$, so that
$P = (x - 1)A$, for some $A \in \mathbb{Z}[x]$. Thus, suppose $P, Q
\in J$, with $P = (x - 1)A, Q = (x - 1)B$, for some $A, B \in \mathbb{Z}[x]$.
$J$ inherits associativity and commutativity of addition and multiplication
and distributivity of multiplication over addition from $\mathbb{Z}$.
$0 = (x - 1)(0)$, so $0 \in J$. $-P = (x - 1)(-A)$, so $(-P) \in J$.
$P + Q = (x - 1)(A + B)$, and $PQ = (x - 1)(x - 1)AB)$, so
$(P + Q), PQ \in J$. Suppose $C \in \mathbb{Z}[x]$. Then,
$CP = PC = (x -1)(AC) \in J$, so $J$ is an ideal of $\mathbb{Z}[x]$.
\qquad $\blacksquare$ \\

%DONE
\textbf{vii.} For $P, Q \in \mathbb{Z}[x]$ such that $P = 1$ and $Q = x$,
$P^{\prime}(0) = 0$, but $(PQ)^{\prime}(0) = 1 \neq 0$. Thus, the given set is
not an ideal of $\mathbb{Z}[x]$. \qquad $\blacksquare$ \\

%DONE
\textbf{Exercise 45:} Let $R$ be a commutative, unital ring, and, for some
$n \in \mathbb{N}$, let $P_1, P_2, \ldots, P_n$ be prime ideals of $R$.

%DONE
\textbf{i.} Let $A$ be an ideal with the specified conditions. Since $a_2,
a_3, \ldots, a_n \in A$ and $A$ is an ideal and thus closed under
multiplication, $a_2a_3 \cdots a_n \in A$, and, since $a_1 \in A$ and $A$ is
closed under addition, $b = a_1 + a_2a_3 \cdots a_n \in A$. Suppose, for sake
of contradiction, that, for some $i \in \mathbb{N}$ with $2 \leq i \leq n$,
$b \in P_i$. Since $P_i$ is an ideal, $a_2a_3 \cdots a_n \in P_i$. Thus,
$(-a_2a_3 \cdots a_n) \in P_i$, so $a_1 = (b + -a_2a_3 \cdots a_n \in P_i$.
his contradicts the choice of $a_1$ with $a_1 \not \in A_i$.
\qquad $\blacksquare$ \\

%DONE
\textbf{ii.} For $n = 1$, it follows trivially from $B \subset
\bigcup_{i = 1}^n P_i = P_1$ that $B \subset P_1$. Suppose, as an inductive
hypothesis, that, for some $n \in \mathbb{N}$,
$B \subset \bigcup_{i = 1}^n P_i$ implies $B \subset P_i$, for some
$i \in \mathbb{N}^*$. Suppose $B \subset \bigcup_{i = 1}^{n + 1} P_i$.
If, for
each $i \in \mathbb{N}$ with $1 \leq i \leq n + 1$, $\exists a_i \in B
\cap P_i$, then, as shown in part i., $\exists b \in B$ with
$b \not \in \bigcup_{i = 1}^{n + 1} P_i$, contradicting the choice of $B$.
Otherwise, for
some $i \in \mathbb{N}$ with $1 \leq i \leq n$, $B \cap P_i = \emptyset$,
$B \subset \bigcup_{i = 1}^n P_i$ (up to some re-indexing of
$P_1, \ldots, P_{n +1}$), so that, by the inductive hypothesis, for some
$i \in \mathbb{N}$ with $1 \leq i \leq n + 1$, $B \subset P_i$.
Thus, by the Principle of Mathematical Induction, $\forall n \in \mathbb{N}$,
if $B \subset \bigcup_{i = 1}^n P_i$ for prime ideals $P_1, P_2, \ldots, P_n$
of $R$, for some $i \in \mathbb{N}$ with $1 \leq i \leq n$, $B \subset P_i$.
\qquad $\blacksquare$ \\

%DONE
\textbf{Exercise 46:} Let $R$ be a ring with at least one non-zero element,
and such that, for each non-zero $a \in R$, $\exists ! b \in R$, written
$b = \psi(a)$, such that $aba = a$.

%DONE
\textbf{i.} Let $r, x, y \in R$, and let $b = \psi(r)$. If $rx = ry$, then
$r(x - y) = 0$. Suppose, for sake of contradiction, that $x - y \neq 0$.
Then, $b + (x - y) \neq b$. However, $r(b + (x - y))r = rbr + r(x - y)r
 = rbr + 0r = rbr = r$, which contradicts the given that $\psi(r)$ is unique.
Thus, $x - y = 0$, so $x = y$. \qquad $\blacksquare$ \\

%DONE
\textbf{ii.} Suppose that, for some $a,b \in R$, $aba = a$. Then, $bab =
babab$. By the result of part i., then, $b = bab$ (note that, as $\psi(a)$ is
defined only for non-zero $a$, $a, b \neq 0$, and consequently, since
$aba = a$, $ba \neq 0$, so that the result of part i. applies).
\qquad $\blacksquare$ \\

%DONE
\textbf{iii.} Let non-zero $a \in R$. Let $c_1 = a\psi(a)$, $c_2 = \psi(a)a$,
so that $ac_1 = a = c_1a$, and let $b \in R$. Then, $ba = bc_1a$, so that, by
the result of part i., $b = bc_1$, and, similarly, $ab = ac_1b$, so that
$b = c_1b$. Thus, $c_1$ is a mutiplicative identity in $R$. A similar proof
shows that $c_2$ is a multiplicative identity of $R$, so that $c_1 = c_2$.
Furthermore, $\forall r \in R$, $\psi(r)r = 1 = r\psi(r)$, so that every
non-zero element of $R$ has an inverse. Therefore, $R$ is a division ring.
\qquad $\blacksquare$ \\

%DONE
\textbf{Exercise 47:} Let $p$ be an odd prime, let $R \subset \mathbb{Q}$ be
the set of rationals whose denominators in reduced form are not divisible by
$p$, and let $J \subset R$ be the set of such rational whose numerator in
reduced form is a multiple of $p$.

%DONE
\textbf{i.} Let $q, r \in R$, with $q = \frac{a}{b}$ and $r = \frac{c}{d}$,
each in reduced form. Associativity and commutativity of addition and
multiplication and distributivity of multiplication of over addition in $R$
are inherited from $\mathbb{Q}$. Since, $0 = \frac{0}{1}$ in reduced form, and
$p$ does not divide $1$, $0 \in R$. $p$ does not divide $b$, so $p$ does not
divide the denominator of $-q = \frac{-a}{b}$. If $p$ divided the
denominator of either $p + q = \frac{ad + bc}{bd}$ or $pq = \frac{ac}{bd}$,
then, by definition of prime, $p$ would have to divide either $b$ or $d$
(noting that reducing a fraction can only eliminate factors of its
denominator), which it does not, by choice of $p, q \in R$. Thus, since
$R \subset \mathbb{Q}$, $R$ is a subring of $\mathbb{Q}$.

Let $q, r \in J$, with $q = \frac{pa}{b}$ and $r = \frac{pc}{d}$.
Associativity and commutativity of addition and multiplication and
distributivity of multiplication of over addition in $R$ are inherited from
$\mathbb{Q}$. Clearly, since $0 = \frac{0}{1}$ in reduced form, $0 \in J$.
Since $-p = \frac{-pa}{b}$, $(-p) \in J$. $p + q = \frac{pad + pcb}{bd}$, and
$pq = \frac{p^2ac}{bd}$, so $(p + q), pq \in J$ (as $p$ does not divide $bd$,
as explained above). Suppose $s \in R$, with $s = \frac{e}{f}$ in reduced form.
Then $qs = \frac{pae}{bf}$. Since $p$ does not divide $bf$, and $p$ divides
$pae$, $qs \in J$. Thus, since $J \subset R$, $J$ is an ideal of $R$.
\qquad $\blacksquare$ \\

%DONE
\textbf{ii.} Note that, $\forall i \in \mathbb{N}$ with $0 < n < p$,
$\frac{1}{i} + \frac{1}{p - i} = \frac{p}{i(p - i)}$. Furthermore, since $p$
is prime and $i, p - i < p$, $\frac{p}{i(p - i)}$ is in reduced form. The
sum $\sum_{i = 1}^{p - 1} \frac{1}{i}$ can be re-written as
$\sum_{i = 1}^{\frac{p - 1}{2}}\left( \frac{1}{i} + \frac{1}{p - i}\right)
 = p\sum_{i = 1}^{\frac{p - 1}{2}}\frac{1}{i(p - i)}$. Since the denominator
of each term of the sum is not divisible by $p$, the denominator of the sum,
which is the product of the denominators of the terms of the sum, in not
divisible by $p$ (as $p$ is prime). Thus, $p$ divides the numerator of the
sum expressed in reduced form, so
$\sum_{i = 1}^{p - 1} \frac{1}{i} \equiv 0 \pmod p$. \qquad $\blacksquare$ \\


%DONE
\textbf{Exercise 48:} Let $p$ be a prime greater than $3$, and
$k = \lfloor \frac{2p}{3} \rfloor$. Let $b = \sum_{i = 1}^k {p \choose i}$.
$p^2$ divides $b$ if and only if $\frac{b}{p} \equiv 0 \pmod p$. Note that, by
definition of the binomial coefficient,
\[\frac{b}{p} = \sum_{i = 1}^k
    \frac{(p - 1)(p - 2) \cdots (p - (i - 1))}{(1)(2) \cdots (i)}
 \equiv \sum_{i = 1}^k \frac{(-1)(-2) \cdots (-(i - 1))}{(1)(2) \cdots (i)}
    \pmod p
 = \sum_{i = 1}^k \frac{(-1)^{i + 1}}{i}\]

Since $p$ is a prime greater than $3$, either $p \equiv 1 \pmod 6$ or
$p \equiv 5 \pmod$ (since, otherwise, it would be divisible by either $2$ or
$3$). In the first case, for some $n \in \mathbb{N}$, $p = 6n + 1$ and
$k = 4n$, so that, letting $m = \frac{k}{2} + 1$, $m = 2k + 1$. In the
second case, let for some $n \in \mathbb{N}$, $p = 6n + 5$ and
$k = 4n + 3$, so that, letting $m = \frac{k + 1}{2}$, $m = 2n + 2$.
In either case, $m = p - (k + 1)$, so that $p - m = k + 1$. Furthermore, by
this choice of $m$, adding and subtracting twice the sum of the even terms of
the sequence gives:

\[\frac{b}{p} = \sum_{i = 1}^k \frac{1}{i} - 2\sum_{i = 1}^m \frac{1}{2i}
 = \sum_{i = 1}^k \frac{1}{i} - \sum_{i = 1}^m \frac{1}{i}
 = \sum_{i = 1}^k \frac{1}{i} + \sum_{i = 1}^m \frac{1}{(-i)}
 \equiv \sum_{i = 1}^k \frac{1}{i} + \sum_{i = 1}^m \frac{1}{(p-i)} \pmod p
 = \sum_{i = 1}^{p - 1} \frac{1}{i}
\]

Then, by the result of part ii. of Exercise 47, $\frac{b}{p} \equiv 0 \pmod p$.
\qquad $\blacksquare$ \\
\end{document}
