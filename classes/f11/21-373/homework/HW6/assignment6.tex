\documentclass{article}%
\usepackage{makeidx}
\usepackage{amsmath}
\usepackage{graphicx}
\usepackage{amsfonts}
\usepackage{amssymb}%
\usepackage{enumerate}%
\usepackage{fullpage}%
\setcounter{MaxMatrixCols}{30}
\providecommand{\U}[1]{\protect\rule{.1in}{.1in}}
\providecommand{\U}[1]{\protect\rule{.1in}{.1in}}
\providecommand{\U}[1]{\protect\rule{.1in}{.1in}}
\providecommand{\U}[1]{\protect\rule{.1in}{.1in}}
\providecommand{\U}[1]{\protect\rule{.1in}{.1in}}
\providecommand{\U}[1]{\protect\rule{.1in}{.1in}}
\providecommand{\U}[1]{\protect\rule{.1in}{.1in}}
\providecommand{\U}[1]{\protect\rule{.1in}{.1in}}
\newenvironment{proof}[1][Proof]{\textbf{#1.} }{\ \rule{0.5em}{0.5em}}
\allowdisplaybreaks
\begin{document}

\begin{center}
\textbf{Shashank Singh}

\textbf{sss1@andrew.cmu.edu}

\textbf{21-373 \quad Honors Algebraic Structures, Fall 2011}

\textbf{Assignment 6}

\textbf{Due: Monday, October 24}\\
\end{center}

%DONE
\textbf{Exercise 36:} Let $R$ be an integral domain equipped with such
functions $V$ and $W$. Let non-zero $\xi, \eta \in R$. Since, $\forall y \in
R$, $\eta y \in R$, $\{V(\xi \eta y) \, | \, y \in R, y \neq 0\} \subseteq
\{V(\xi \eta y) \, | \, y \in R, y \neq 0\}$, so $W(\xi \eta) \geq W(\xi)$.

Suppose $a, b \in R$, with $b \neq 0$. Let $y \in R$ such that $y$ minimizes
$V(by)$, so that $W(b) = V(by)$. By construction of $V$, $\exists q_*, r_1 \in
R$ such that $ay = byq_* + r_1$, and either $r_1 = 0$ or $r_1 \neq 0$ and
$V(r_1) < V(by)$. Since $r_* = (a - qb)y$, $\exists r_* \in R$ such that
$r_* = r_*y$. Thus, since $R$ is an integral domain, so that the multiplicative
cancellation property holds, $a = q_*b + r_*$. Furthermore, either $r_* = 0$ or
$r_* \neq 0$ and $W(r_*) \leq W(r_*y) = W(r_1) \leq V(r_1) < V(by) \leq W(b)$.
\qquad $\blacksquare$ \\


%DONE
\textbf{Exercise 37:} Let $R$ be a commutative unital ring.

%DONE
\textbf{i.} For some $n \in \mathbb{N}$, let $a_1, a_2, \ldots, a_n \in R$ be
nilpotent. Then, for each $i \in \mathbb{N}$ with $1 \leq i \leq n$,
letting $p_i = 0 + 0x + 0x^2 + \ldots + a_ix^i + \ldots + 0x^n = a_ix^n$,
$p_i$ is also nilpotent (since $x$ commutes with the elements of the ring, so
that, if, for some $m \in \mathbb{N}$ with $n \geq 1$, $a_i^m = 0$, then
$p_i^m = \left(a_ix^i\right)^m = a_i^m x^{im} = 0 x^{im} = 0$). Furthermore,
since it was shown in the previous assignment that the sum of two nilpotent
elements is also nilpotent, by a simple induction on $n$, it is shown that
$a_1x + a_2x^2 + \ldots + a_nx^n = \sum_{i = 1}^n p_i$ is also nilpotent.
Furthermore, since $1$ is a unit, it follows from the result of part ii. that
$1 + a_1x + \ldots a_nx^n$ is a unit. \qquad $\blacksquare$

%DONE
\textbf{ii.} Suppose $P = a_0 + a_1x + \ldots + a_nx^n$ is a unit in $R[x]$.
Then, $\exists P^{-1} \in R[x]$ such that $PP^{-1} = P^{-1}P = 1$. Thus,
$PP^{-1} = a_0P^{-1} + x(a_1 + a_2x + \ldots + a_nx^{n - 1})P^{-1}$. Since the
multiplicative identity is $1$, which has no terms containing $x$, $PP^{-1}
 = a_0P^{-1} = a_0b_0 + xa_0(b_1 + b_2x + \ldots + b_nx^{n - 1})$, where
$b_0 + b_1x + \ldots + b_nx^n = P^{-1}$, so, similarly, $PP^{-1} = a_0b_0 =
b_0a_0 = 1$. Thus, $b_0 = a_0^{-1}$, so $a_0$ is a unit.

For $n = 0$, clearly $a_1, a_2, \ldots, a_n$ are vacuously nilpotent. Suppose,
as an inductive hypothesis, that, for some $n \in \mathbb{N}$, if $P =
a_0 + a_1x + \ldots + a_nx^n$ is a unit in $R[x]$, then, $a_1,\ldots,a_n$ are
nilpotent in $R$. Suppose, moreover, that $P_2 = P + a_{n + 1}x^{n + 1}$ is a
unit in $R$. Then, $\exists P_2^{-1} \in R[x]$ such that $P_2^{-1}P_2 =
P_2P_2^{-1} = 1$. Then $P_2P_2^{-1} = PP_2^{-1} + a_{n + 1}x^{n + 1}P_2^{-1}$.
Since no term of $PP_2^{-1}$ can be of degree $n + 1 + deg(P_2^{-1})$, and
the coefficient of $x^{n + 1}$ in $P_2P_2^{-1}$ must be $0$, $a_{n + 1}$ must
be nilpotent. Furthermore, $a_{n + 1}P_2^{-1} = 0$, so $PP_2^{-1} = 1$.
Therefore, $P$ is a unit in $R[x]$, and thus, by the inductive hypothesis,
$a_1,\dots,a_n$ are nilpotent.

\textbf{Lemma:} Suppose $u$ is a unit in $R$ and $x$ is nilpotent in $R$.
Then, $a = u + x$ is a unit.
\textbf{Proof:} Let $u$ be a unit in $R$ and let $x$ be nilpotent in $R$.
Then, $\exists u^{-1} \in R$ such that $uu^{-1} = u^{-1}u = 1$, and
$\exists n \in \mathbb{N}$ such that $x^n = 0$. Let $a = u + x$, and let
$a^{-1} = c^n \left( \sum_{i = 0}^{n - 1}
\left(-x\right)^iu^{n - (i + 1)}\right)$. Then, as is shown by induciton on
$n$, $a^{-1}a = aa^{-1} = (u + x) c^n \left( \sum_{i = 0}^{n - 1}
\left(-x\right)^iu^{n - (i + 1)}\right) = 1$. Therefore, $a$ is a unit.

Suppose $a_0$ is a unit in $R$ and $a_1, \ldots, a_n$ are nilpotent in $R$.
Then, by the result of part i., $a_1x + a_2x^2 + \ldots + a_nx^n$ is
nilpotent in $R[x]$, so, by the above lemma, $a_0 + a_1x + \ldots + a_nx^n$
is a unit.

Thus, $a_0 + a_1x + \ldots + a_nx^n$ is a unit in $R[x]$ if and only if $a_0$
is a unit in $R$ and $a_1, \ldots, a_n$ are nilpotent in $R$.
\qquad $\blacksquare$ \\

%DONE
\textbf{Exercise 39: i.} Let $x, y \in \mathbb{Z}$, such that $x^3 = y^2 + 2$.
Clearly, $x$ and $y$ have the same parity (in the sense of even and odd), as,
otherwise, $x^3$ and $(y^2 + 2) \equiv y^2$ would be different $\pmod 2$.
Suppose, for sake of contradiction, that $x$ and $y$ are both even. Then,
for some $n, m \in \mathbb{Z}$, $x^3 = 8n^3$ and $y^2 = 4m^2$, so that
$x^3 \equiv 0 \pmod 4$, and yet $y^2 + 2 \equiv 2 \pmod 4$. This is
impossible if indeed $x^3 = y^2 + 2$, so $x$ and $y$ are both odd. \qquad
$\blacksquare$ \\

%DONE
\textbf{Exercise 40:} Let $R$ be a unital ring.

\textbf{i.} Let $P$ be a prime ideal, and let $A$ be an ideal.
Suppose, for
$n = 1$, that $A^n \subseteq P$. Then, clearly, $A = A^n \subseteq P$. Suppose,
as an inductive hypothesis, that, for some $n \in \mathbb{N}$, in $A^n \subseteq
P$, then $A \subseteq P$. Suppose, furthermore, that $A^{n + 1} \subseteq P$.
Since $P$ is a prime ideal, $A$ is an ideal, and $A^{n + 1} = A^nA$, either
$A^n \subseteq P$, or
$A \subseteq P$. By the inductive hypothesis, in the former case, $A \subseteq
P$. In the latter case, trivially, $A \subseteq P$. Thus, by induction on $n$,
$\forall n \in \mathbb{N}$ with $n \geq 1$, if $A^n \subseteq P$, then
$A \subseteq P$. \qquad $\blacksquare$ \\

%DONE
\textbf{Exercise 41: i.} Let $J$ be a prime ideal of a commutative ring $R$.

Suppose $j \in Rad(J)$. Then, $\exists n \in \mathbb{N}$ such that $j^n \in
J$. Since $R$ is commutative, $(j) = jR$. Thus, also since $R$ is commutative,
$(j)^n \subseteq j^nR$. Therefore, $(j)^n \subseteq J$. Since $(j)$ is an
ideal, by the result of Exercise 40 i., then, $(j) \subseteq J$. Therefore,
since $j \in (j)$, $j \in J$.

Suppose $j \in J$. Then, for $n = 1$, $n \in \mathbb{N}$ and $j^n = j \in J$,
so $j \in Rad(J)$.

Thus, $Rad(J) = J$. \qquad $\blacksquare$ \\

%DONE
\textbf{ii.} Suppose $R = \mathbb{Z}$, let $J$ be an ideal of $R$, and let
$I = \cap_{A \in S} A$, where $S$ is the set of prime ideals of $R$ which
contain $J$. Note that it shown in class that the ideals of $\mathbb{Z}$ are
precisely those subsets $I \subseteq \mathbb{Z}$ of the form $m\mathbb{Z}$,
where $m \in \mathbb{N}$ (and furthermore, $I$ is a prime ideal if and only if
$m$ is prime).

Suppose $j \in Rad(J)$. Then, for some $n \in \mathbb{N}$, $j^n \in J$, so
that, for any $A \in S$, $j^n \in A$. Thus, $j \in Rad(A)$, so, since $A$ is a
prime ideal, and thus, by the result of part i., $A = Rad(A)$, $j \in A$.

Suppose, on the other hand, that $j \in I$. Let $m \in \mathbb{N}$ such that
$J = m\mathbb{Z}$, and let $p_1,p_2,\ldots,p_k$, for some $k \in \mathbb{N}$,
be the prime factorization of $m$. Clearly, the prime ideals containing $J$
are $p_1\mathbb{Z}, p_2\mathbb{Z}, \ldots, p_n\mathbb{Z}$. Let $\alpha_1,
\alpha_2, \ldots, \alpha_k$ be the multiplicities of $p_1, p_2, \ldots, p_k$,
respectively, and let $n = \max\{\alpha_1, \alpha_2, \ldots, \alpha_k\}$.
Then, $m$ divides $j^n$, so $j^n \in m\mathbb{Z} = J$. Therefore, $j \in
Rad(J)$.

Thus, $\cap_{A \in S} A = Rad(J)$. \qquad $\blacksquare$ \\

%DONE
\textbf{Exercise 42:} If an element $P = q_0 + q_1x + \ldots \in
\mathbb{Q}[[x]]$ is of the form $\frac{A}{B}$ for some $A, b \in
\mathbb{Z}[[x]]$, then it is not the case that, $\forall i \in \mathbb{N}$,
$q_i = \frac{1}{i!}$. For, suppose, for sake of contradiction, that it were
the case that, $\forall i \in \mathbb{N}$, $q_i = \frac{1}{i!}$, so that
$P \in \mathbb{Q}$, and yet $P = \frac{A}{B}$ for $A,B \in \mathbb{Z}$. Then,
letting $A = a_0 + a_1x + \ldots, B = b_0 + b_1x + \ldots$, $BP = A$;
i.e., $\forall k \in \mathbb{N}, a_k = \sum_{i = 0}^k b_iq_{k - i}
 = \sum_{i = 0|}^k \frac{b_k}{(k - i)!}$. However, multiplying by $(k - 1)!$\
and subtracting all but one term of the summation gives $\frac{b_0}{k}
= (k - 1)!a_k - \sum_{i = 1}^k b_i\frac{(k - 1)!}{(k - i)!}$. Since, for
$i \geq 1$, $(k - i)!$ divides $(k - 1)!$ and $b_i \in \mathbb{Z}$, every term
of the summation is an integer, so that the right hand side for the equation
is an integer, $\forall k \in \mathbb{N}$. However, clearly, since, for
$k = b_0 + 1$, $k$ does not divide $b_0$, the left hand side is not always an
integer. This is a contradiction, so $P$ is not of the form $\frac{A}{B}$ for
$A, B \in \mathbb{Z}[[X]]$.

Since there exists an element $P \in \mathbb{Q}[[x]]$ (so that, consequently,
$P$ is in the field of fractions of $\mathbb{Q}[[x]]$), such that $P$ is not
in the field of fractions of $\mathbb{Z}[[x]]$, the field of fractions of
$\mathbb{Z}[[x]]$ is strictly smaller than the field of fractions of
$\mathbb{Q}[[x]]$. \qquad $\blacksquare$

\end{document}
