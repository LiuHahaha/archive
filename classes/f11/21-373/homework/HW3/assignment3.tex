\documentclass{article}%
\usepackage{makeidx}
\usepackage{amsmath}
\usepackage{graphicx}
\usepackage{amsfonts}
\usepackage{amssymb}%
\usepackage{enumerate}%
\usepackage{fullpage}%
\setcounter{MaxMatrixCols}{30}
\providecommand{\U}[1]{\protect\rule{.1in}{.1in}}
\providecommand{\U}[1]{\protect\rule{.1in}{.1in}}
\providecommand{\U}[1]{\protect\rule{.1in}{.1in}}
\providecommand{\U}[1]{\protect\rule{.1in}{.1in}}
\providecommand{\U}[1]{\protect\rule{.1in}{.1in}}
\providecommand{\U}[1]{\protect\rule{.1in}{.1in}}
\providecommand{\U}[1]{\protect\rule{.1in}{.1in}}
\providecommand{\U}[1]{\protect\rule{.1in}{.1in}}
\newenvironment{proof}[1][Proof]{\textbf{#1.} }{\ \rule{0.5em}{0.5em}}
\begin{document}

\begin{center}
\textbf{Shashank Singh}

\textbf{sss1@andrew.cmu.edu}

\textbf{21-373 \quad Algebraic Structures, Fall 2011}

\textbf{Assignment 3}

\textbf{Due: Friday, September 30}\\
\end{center}

%DONE
\textbf{Exercise 15:} As can be seen by enumerating the elements of each, $S_4$
has no elements of order $12$, whereas $D_{12}$ does have an element of order
$12$ (e.g, the element corresponding to rotation of the dodecahedron by
$\frac{pi}{6}$). As discussed in Remark 5.4, if two groups are isomorphic,
$\forall n \in \mathbb{N}$, each group must have the same number of elements of
order $n$. Thus, $S_4 \not \simeq D_12$. \qquad $\blacksquare$

%DONE
\textbf{Exercise 16:} The elements of order $4$ in $S_4$, denoted by their
cycle decompositions, are \[(1 \, 2\, 3 \, 4), (1 \, 2 \, 4 \, 3),
(1 \, 3 \, 2 \, 4), (1 \, 3 \, 4 \, 2), (1 \, 4 \, 2 \, 3),
(1 \, 4 \, 3 \, 2).\]

The elements of order $2$ in $S_4$, denoted by their
cycle decompositions, are \[(1 \, 2), (1 \, 3), (1 \, 4), (2 \, 3), (2 \, 4),
(3 \, 4), (1 \, 2) (3 \, 4), (1 \, 3) (2 \, 4), (1 \, 4) (3 \, 4).\]


%DONE
\textbf{Exercise 17:} Suppose some element $a \in S_n$, for some $n \in
\mathbb{N}$, can be written as a $k$-cycle. Then, $a^i$ can be written as a
$k$-cycle if and only if $(i,k) = 1$ (i.e., $i$ and $k$ are relatively prime).
This follows directly from the fact that the order of any permutation is the
least common multiple of the lengths of the cycles in its cycle decomposition.

As a consequence, for $\sigma = (1 \, 2 \, 3 \, 4 \, 5 \, 6 \, 7 \, 8)$,
$\sigma^i$ can be written as an $8$-cycle if and only if $(i,8) = 1$, for
$\tau = (1 \, 2 \, 3 \, 4 \, 5 \, 6 \, 7 \, 8 \, 9 \, 10 \, 11 \, 12)$,
$\tau^i$ can be written as a $12$-cycle if and only if $(i,12) = 1$, and, for
$\omega = (1 \, 2 \, 3 \, 4 \, 5 \, 6 \, 7 \, 8 \, 9 \, 10 \, 11 \, 12 \, 13
\, 14)$, $\omega^i$ can be written as a $14$-cycle if and only if $(i,14) =
1$. \qquad $\blacksquare$

%DONE
\textbf{Exercise 19:} Let $m \geq 1$, and let $q_1, q_2, \ldots, q_m \in
\mathbb{Q}$. By definition of $H$, $H$ is a group. Suppose $b_1, b_2, \ldots,
b_m$, are the denominators of some $q_1, q_2, \ldots, q_m$. Then, clearly,
$q_1, q_2, \ldots, q_m$ can be written as multiples of $\frac{1}{D}$.
Furthermore, the sum and difference of any two $q_i, q_j$ can be written as
a multiple of $\frac{1}{D}$, so $H \subseteq K$. Thus, $H \leq K$.
\qquad $\blacksquare$

Since $H \leq K$, and $K$ can be generated by a single element, $H$ can be
generated by a single element (some multiple of $\frac{1}{D}$), so $H$ is
cyclic. \qquad $\blacksquare$

%DONE
\textbf{Exercise 20:} Let $a \in \mathbb{Q}$, so that $a = \frac{p}{q}$, for
some $p, q \in \mathbb{N}$. Suppose $k \in \mathbb{N}$. Then, since
$\mathbb{N}$ is closed under multiplication, $kq \in \mathbb{N}$, and so
$\frac{p}{kq} \in \mathbb{Q}$. Therefore, since $k\frac{p}{kq} = \frac{p}{q}
 = a$, $\mathbb{Q}$ is divisible. \qquad $\blacksquare$

Suppose $G$ is a finite Abelian group. Then, if $|G| > 1$, $G$ is not
divisible. Let $k = |G|$. Since $|G| > 1$, $\exists a \in G$ with $a \neq e$.
Suppose $b \in G$. Thenm $kb = e \neq a$, so no finite, Abelian group is
divisible. \qquad $\blacksquare$

Let $G_1, G_2$ be Abelian groups, and let $G = G_1 \times G_2$.

Suppose $G$ is
divisible, and let $a_1 \in G_1, a_2 \in G_2, k \in \mathbb{N}$. Since $G$ is
divisible, $\exists (b_1,b_2) \in \mathbb{G}$ such that $k(b_1,b_2) =
(g_1,g_2)$. Thus, $kb_1 = a_1$, and $kb_2 = a_2$, so $G_1, G_2$ are divisible.

Suppose $G_1, G_2$ are divisible, and let $a \in G$, so that $a = (a_1,a_2)$,
for some $a_1 \in G_1, a_2 \in G_2$. Let $k \in \mathbb{N}$. Then,
$\exists b_1 \in G_1, b_2 \in G_2$ such that $kb_1 = a_1, kb_2 = a_2$. Thus,
$k(b_1,b_2) = (a_1,a_2) = a$, so $G$ is divisible. \qquad $\blacksquare$

%DONE
\textbf{Exercise 21:} Any symmetric rigid motion can be determined uniquely by
mapping a fixed edge $E_0$ of the platonic solid to another (non necessarily
distinct edge $E_1$ of the platonic solid (which can be done in $E$ ways), and
then choosing which vertex of $E_1$ edge to move to which vertex of $E_2$
(which can be done in $2$ ways). Thus, there are $2E$ rigid motion symmetries
of any platonic solid. \qquad $\blacksquare$

Label the vertices of the tetrahedron, any four non-adjacent vertices of the
cube, and any four non-adjacent faces of the octahedron with the numbers $1,
2, 3$, and $4$. Then, any permutation of $\{1, 2, 3, 4\}$ corresponds to a
rigid motion symmetry of the tetrahedron, cube, or octahedron, respectively,
(in particular, the permutation $\pi$ corresponds to symmetry of the the rigid
motion moving the vertex or face labelled $i$ to the vertex or face,
respectively, labelled $\pi(i)$). This can be shown formally by noting the
correspondence for transpositions, and then observing that $S_4$ is
generated by its subset of transpositions, and the group of rigid motion
symmetries is generated by the motion symmetries corresponding to switching
two vertices or faces in a solid. \qquad $\blacksquare$

\end{document}
