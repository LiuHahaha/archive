\documentclass{article}%
\usepackage{makeidx}
\usepackage{amsmath}
\usepackage{graphicx}
\usepackage{amsfonts}
\usepackage{amssymb}%
\usepackage{enumerate}%
\usepackage{fullpage}%
\setcounter{MaxMatrixCols}{30}
\providecommand{\U}[1]{\protect\rule{.1in}{.1in}}
\providecommand{\U}[1]{\protect\rule{.1in}{.1in}}
\providecommand{\U}[1]{\protect\rule{.1in}{.1in}}
\providecommand{\U}[1]{\protect\rule{.1in}{.1in}}
\providecommand{\U}[1]{\protect\rule{.1in}{.1in}}
\providecommand{\U}[1]{\protect\rule{.1in}{.1in}}
\providecommand{\U}[1]{\protect\rule{.1in}{.1in}}
\providecommand{\U}[1]{\protect\rule{.1in}{.1in}}
\newenvironment{proof}[1][Proof]{\textbf{#1.} }{\ \rule{0.5em}{0.5em}}
\allowdisplaybreaks
\begin{document}

\begin{center}
\textbf{Shashank Singh}

\textbf{sss1@andrew.cmu.edu}

\textbf{21-373 \quad Honors Algebraic Structures, Fall 2011}

\textbf{Assignment 9}

\textbf{Due: Wednesday, November 30}\\
\end{center}

%DONE
\textbf{Exercise 57:} Let $K$ be a finite field, and let $\forall a,b \in K$, if
$a^2 = b^2$, then $(a + b)(a - b)$, so that $a = -b$. Therefore, if $x$ is a
square in $K$, then there are at most two distinct elements of which $x$ is a
square. Thus, if $n = |K|$, for $S = \{x^2 : x \in K\}$, $|S| \geq
 \lceil \frac{n + 1}{2} \rceil$. Let $T = \{k - x : x \in S\}$. Then,
$f(x) = k - x$ is an injection from $S$ to $T$, so that
$|T| \geq \lceil \frac{n + 1}{2} \rceil$.
Therefore, since $S, T \subseteq K$, $S \cap T \neq \emptyset$ (for, if $S$
and $T$ were disjoint, then $|K| \geq n + 1$, contradicting the choice of
$n$). Then, for $x \in S \cap T$, $x = a^2$ and $k - x = b^2$, for some
$a,b \in K$, so that $k = a^2 + b^2$.

%DONE
\textbf{Exercise 58:} Let $E$ be a field, and let $F = E(x)$. Let $P,Q \in
E[x]$ such that $P$ and $Q$ are relatively prime.

%DONE
\textbf{i.} Let $n$ be the degree of $P$, and let $m$ be the degree of $Q$.
For $R = \frac{P}{Q} Q - P$, $R \in E[x]$, and $R(x) = 0$. Thus, $x$ is
algebraic over $E(\frac{P}{Q})$. Furthermore, since $[F : E(\frac{P}{Q})]$ is
the degree of which $x$ is algebraic over $E(\frac{P}{Q})$ (as shown on the
previous assignment), $[F : E(\frac{P}{Q})] \leq degree(R) = \max\{degree(P),
degree(Q)\}$. It remains to show that $\frac{P}{Q} Q - P$ is irreducible, so
that, since there is a unique monic, irreducible polynomial such of which $x$
is a root (and $R$ can be made monic be dividing by the coefficient of the
leading term), $[F : E(\frac{P}{Q})] \geq degree(R)$.

%DONE
\textbf{ii} Note that, since $P,Q$ are relatively prime, at least one is
non-constant.
Suppose $\max\{degree(P),degree(Q)\} = 1$. Then $\frac{P}{Q} =
\frac{ax + b}{cx + d}$, for some $a,b,c,d \in E$. Thus, for
$\tau(x) = \frac{b - dx}{cx - a}$, $\tau = \sigma^{-1}$. Thus, $\sigma$ is
bijective, so that, since $\sigma$ is an endomorphism, $\sigma$ is an
automorphism.

Suppose, on the other hand, that $\sigma$ is an automorphism.
Since $\sigma$ maps $F$ to $E(\frac{P}{Q})$ and is bijective, $|F| \leq
|E(\frac{P}{Q})$, so that, since $E(\frac{P}{Q}) \subseteq |F|$,
$E(\frac{P}{Q}) = F$. Therefore, $[F : E(\frac{P}{Q})] = 1$, so that, by the
result of part i., $\max\{degree(P),degree(Q)\} = 1$.

%DONE
\textbf{Exercise 59:} Note that, in the scope of this exercise, $p$ denotes
a prime integer. Clearly, $\not \exists x \in \mathbb{Z}$ such that
$x^2 = -1$, $x^2 = 2$, $x^2 = -2$, or $x^4 = -1$. Thus, since, as shown below,
all factorizations of $x^4 + 1$ require the existence of some such element,
$x^4 + 1$ is irreducible in $\mathbb{Z}$.

%DONE
\textbf{i.} If $p = 2$, then, since $1 = -1$, $1$ is a root of $x^4 + 1$.

%DONE
\textbf{ii.} $(x^2 + b)(x^2 - b) = x^4 - b^2$. Thus, $x^4 + 1$ factors as
$(x^2 + b)(x^2 - b)$ in $\mathbb{Z}_p$ if and only if $(-1)$ is a quadratic
residue for $p$. Indeed, as
given, since $p = 8n + 5 = 4(2n + 1) + 1$, $(-1)$ is a quadratic residue for
$p$, so, assuming $p$ is not of the form $8n + 1$, $x^4 + 1$ factors as
$(x^2 + b)(x^2 - b)$ in $\mathbb{Z}_p$ if and only if $p = 8n + 1$ for some
$n \in \mathbb{N}$.

%DONE
\textbf{iii.} $(x^2 + ax + 1)(x^2 - ax + 1) = x^4 + (2 - a^2)x^2 + 1$. Thus,
$x^4 + 1$ factors as $(x^2 + ax + 1)(x^2 - ax + 1)$ in $\mathbb{Z}_p$ if and
only if $2$ is a quadratic residue for $p$. Indeed, as given, since
$p = 8n + 7 = 8(n + 1) - 1$, $2$ is a quadratic residue for $p$, so, assuming
$p$ is not of the form $8n + 1$, $x^4 + 1$ factors as
$(x^2 + ax + 1)(x^2 - ax + 1)$ if and only if $p = 8n + 7$.

%DONE
\textbf{iv.} $(x^2 + ax - 1)(x^2 - ax - 1) = x^4 - (2 + a^2)x^2 + 1$. Thus,
$x^4 + 1$ factors as $(x^2 + ax - 1)(x^2 - ax - 1)$ if and only if $(-2)$ is a
quadratic residue for $p$. The Legendre symbol of $(-2)$ shows that, since
$\left(\frac{-2}{p}\right) =
    \left(\frac{-1}{p}\right)\left(\frac{2}{p}\right)$,
$-2$ is a quadratic residue of $p$ if and only if either both or neither of $2$ and
$(-1)$ are quadratic residues for $p$. Since $p$ is prime, for some $n \in
\mathbb{N}$, $p = 8n + 1$, $p = 8n + 3$, $p = 8n + 5$, or $p = 8n + 7$. In the
latter two cases, as shown above, one, but not both, of $2$ and $(-1)$ is a quadratic
residue for $p$. Thus, assuming $p$ is not of the form $8n + 1$, $x^2 + 1$
factors as $(x^2 + ax - 1)(x^2 - ax - 1)$ if and only if $p = 8n + 3$.

%DONE
\textbf{Exercise 60: i.} Let $E$ be a field of characteristic $0$, and suppose
$P \in E[x]$ is such that $P(x^2 + 1) = (P(x))^2 + 1$ and $P(0) = 0$. Let
$S = |\{x \in \mathbb{N} : P(x) = x\}|$. Suppose, for sake of contradiction,
that, for some $n \in \mathbb{N}$, $|S| = n$. Let $k = \max S$. Then,
$P(k^2 + 1) = P(k)^2 + 1 = k^2 + 1 \not \in S$, contradicting the choice of $k$.
Thus, there are infinitely many solutions $x$ to $P(x) = x$, so that
$(P - x)$ has infinitely many roots. Since the number of roots of a non-zero
polynomial is bounded by the degree of the polynomial, then, $(P - x) = 0$, so
that $P - x$.

%DONE
\textbf{ii.} Let $E$ be a field of characteristic $p$. Clearly, for $P = x$
(the ``trivial solution''), $P \in E[x]$ satisfies $P(x^2 + 1) = (P(x))^2 + 1$
and $P(0) = 0$. Suppose that, for some $P \in E[x]$, $P$ satisfies
$P(x^2 + 1) = (P(x))^2 + 1$ and $P(0) = 0$. As shown in class, since $E$ is a
finite field (as it is of positive characteristic), the Frobenius homomorphism
on $E$ is an automorphism on $E$, so that it is bijective. Thus, if $f$ is the
Frobenius homomorphism on $E$, $P(f(x)) = P(x^p)$ satisfies $P(x^2 + 1) =
(P(x))^2 + 1$ and $P(0) = 0$. Furthermore, $P(f(f(x))), P(f(f(f(x)))), \ldots,
P(f^i(x)), \ldots$ all satisfy the constraint, so that there are an infinite
number of solutions $P \in E[x]$ to the constraint.

%DONE
\textbf{Exercise 61:} Let $n \in \mathbb{N}$ with $n \geq 2$. It is easily
shown by induction on $n$ that $(x + x^{n + 2}) (1 - x^{n + 1} + x^{2(n + 1)} -
  \ldots + (-1)^n (x^{n + 1})^{n - 2}) + (-1)^{n + 1} (x^{n^2})$. Thus, for
$P(x,y,z) = z(1 - y + y^2 - \ldots + (-1)^n y^{n - 2}) + (-1)^{n + 1} x^2$,
$x \equiv P(x^n,x^{n + 1},x + x^{n + 2})$.

%DONE
\textbf{Exercise 62:} Let $n,m \in \mathbb{N}$ with $n = 2m$ and $m > 1$ is
odd, and let $\theta = e^{2 \pi i / n}$. Then, $\theta^m + 1 = e^{\pi i} + 1
 = 0$. It is easily shown by induction that, $\forall x \in \mathbb{C}$,
$\forall k \in \mathbb{N}$, if $k$ is odd, then $x^k = (x + 1)\left(
\sum_{i = 1}^k (-1)^{i + 1} x^{k - i} \right)$. Thus, $\theta^m = (\theta + 1)
\left(\sum_{i = 1}^m (-1)^{i + 1} \theta^{m - i} \right)$. Since
$\theta \neq -1$ (as $m > 1$), $0 = \sum_{i = 1}^m (-1)^{i + 1}
\theta^{m - i}$ and thus $1 = \sum_{i = 1}^{m - 1} (-1)^{i} \theta^{m - i}$.
$1 = \sum_{i = 1}^{m - 1} (-1)^{i} \theta^{m - i}$.

Separating even and odd terms gives $1 = (1 - \theta)
(\sum_{i = 0}^{\frac{m - 3}{2}}\theta^{2i + 1})$. Thus, $(1 - \theta)^{-1}
 = (\sum_{i = 0}^{\frac{m - 3}{2}}\theta^{2i + 1})$.
\end{document}
