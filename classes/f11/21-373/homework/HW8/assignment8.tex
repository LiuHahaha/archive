\documentclass{article}%
\usepackage{makeidx}
\usepackage{amsmath}
\usepackage{graphicx}
\usepackage{amsfonts}
\usepackage{amssymb}%
\usepackage{enumerate}%
\usepackage{fullpage}%
\setcounter{MaxMatrixCols}{30}
\providecommand{\U}[1]{\protect\rule{.1in}{.1in}}
\providecommand{\U}[1]{\protect\rule{.1in}{.1in}}
\providecommand{\U}[1]{\protect\rule{.1in}{.1in}}
\providecommand{\U}[1]{\protect\rule{.1in}{.1in}}
\providecommand{\U}[1]{\protect\rule{.1in}{.1in}}
\providecommand{\U}[1]{\protect\rule{.1in}{.1in}}
\providecommand{\U}[1]{\protect\rule{.1in}{.1in}}
\providecommand{\U}[1]{\protect\rule{.1in}{.1in}}
\newenvironment{proof}[1][Proof]{\textbf{#1.} }{\ \rule{0.5em}{0.5em}}
\allowdisplaybreaks
\begin{document}

\begin{center}
\textbf{Shashank Singh}

\textbf{sss1@andrew.cmu.edu}

\textbf{21-373 \quad Honors Algebraic Structures, Fall 2011}

\textbf{Assignment 8}

\textbf{Due: Monday, November 21}\\
\end{center}

The following lemma is used in the below proofs.

\textbf{Lemma 1:} Let $E$ be a field, and let $F$ be a field extension of $E$.
Then, if $a \in F$ is agebraic of degree $m \in \mathbb{N}$ over $E$, then
$[E(a) : E] = m$.

%DONE
\textbf{Proof:} Since $a$ is algebraic of degree $m$, for
$S = \{1, a, \ldots, a^{m - 1}\}$, $S$ is linearly independent, as otherwise,
since $\exists$ a non-trivial linear combination
$q_0 + q_1a + \ldots + q_{m - 1}a^{m - 1}$ of elements in $S$,
$\exists Q \in E[x]$ of degree $k < m$ such that $Q(a) = 0$. Therefore,
$[E(a) : E] \geq m$.

Suppose, for sake of contradiction, that $[E(a) : E] > m$, so that, for
$n = [E(a) : E] - m$, $n > 0$. Then, as shown in lecture,
for $B = \{1, a, \ldots, a^{m + n}\}$,
$B$ is a basis of $E(a)$. Since $a$ is algebraic over $E$ of degree $m$,
$\exists P \in E[x]$ of degree $m$ such that $P(a) = 0$. Let
$p_0, p_1, \ldots, p_m$ be the coefficients of $P$, so that
$P(a) = p_0 + p_1a + \ldots + p_ma^m = 0$. Then,
$1 = -(p_0)^{-1}(p_1a + p_2a^2 + \ldots + p_ma^m$, contradicting the linear
independence of $B$. Thus, $[E(a) : E] \leq m$.

Therefore, $[E(a) : E] = m$. \qquad $\blacksquare$ \\

%DONE
\textbf{Exercise 50:} Suppose $u \in E(x_1,x_2,\ldots,x_n)$ is algebraic in
$E$. Then, for some $P, Q \in E[x]$ such that $\frac{P}{Q}$ in reduced form,
$u = \frac{P}{Q}$. Furthermore, for some $k \in \mathbb{N}$,
$0 = r_0 + r_1u + \ldots + r_ku^k = r_0 + r_1\frac{P}{Q} + \ldots + r_k\left(
\frac{P}{Q}\right)^k$. Thus, $\left(\frac{P}{Q}\right)^k = -r_k^{-1}
\left(r_0 + r_1\frac{P}{Q} + \ldots +
r_{k - 1}\left(\frac{P}{Q}\right)^{k - 1}\right)$, so that
$P^k = -r_k^{-1} \left(r_0Q^k + r_1PQ^{k - 1} + \ldots + r_{k - 1}P^{k - 1}Q
\right)$.
Thus, $P$ divides $r_k^{-1}Q^k$. Since $\frac{P}{Q}$ is in reduced form, $P$
and $Q$ have no common factors, so that $P$ divides $r_k^{-1}$. Thus, $P$ is
a constant. Furthermore, $Q^k = -r_0^{-1} \left(r_1PQ^{k - 1} + \ldots +
r_{k - 1}P^{k - 1}Q + r_kP^k\right)$, so that $Q$ divides $r_0^{-1}P^{k}$.
Since $P$ and $Q$ have no common factors, $Q$ divides $r_0^{-1}$ so that $Q$
is also constant. Thus, $\frac{P}{Q}$ is constant, so that $u = \frac{P}{Q}
 \in E$. Thus, if $u$ is algebraic in $E$, then $u \in E$, so that the
contrapositive, that every element of $E(x_1, \ldots, x_n) \backslash E$ is
transcendental, holds. \qquad $\blacksquare$ \\

%DONE
\textbf{Exercise 51:} Let $E$ be a field, let $F$ be a field extension, let
$a,b \in F$ be algebraic over $E$ of degrees $m$ and $n$ respectively, with
$(m,n) = 1$.

By Lemma 1 above, $[E(a) : E] = m$, and $[E(b) : E] = n$. Note that $E(a,b)$
is a field extension both of $E(a)$ and of $E(b)$, which are in turn field
extensions of $E$. Thus, by Lemma 29.5, $m$ and $n$ both divide
$[E(a,b) : E]$. Since $(m,n) = 1$, so that $m$ and $n$ share no
prime factors thus the product of the prime factorizations of $m$ and $n$
divides $[E(a,b) : E]$, so that $[E(a,b) : E] \geq mn$.

Suppose, for sake of contradiction, that $[E(a,b) : E] > mn$, so that, for
$k = [E(a,b) : E] - m$, $l = [E(a,b) : E] - n$, either $k > 0$ or $l > 0$.
Then, as shown in lecture, for \[B = \{1, a, \ldots, a^{m + k}, 1b, ab,
\ldots, a^{m + k}b, \ldots, 1b^{n + l}, ab^{n + l}, \ldots,
a^{m + k}b^{n + l}\},\] $B$ is a basis of $E(a)$. Since $a$ is algebraic of
degree $m$ over $E$, $\exists P \in E[x]$
of degree $m$ such that $P(a) = 0$. Since $b$ is algebraic of degree $n$ over
$E$, $\exists Q \in E[x]$ of degree $n$ such that $Q(b) = 0$. Let
$p_0, p_1, \ldots, p_m$ be the coefficients of $P$ and let
$q_0, q_1, \ldots, q_n$ be the coefficients of $Q$, so that
$P(a) = p_0 + p_1a + \ldots + p_ma^m = 0$ and
$Q(b) = q_0 + q_1b + \ldots + q_nb^n = 0$. Then,
$1 = -(p_0)^{-1}(p_1a + p_2a^2 + \ldots + p_ma^m$ and
$1 = -(q_0)^{-1}(q_1b + q_2b^2 + \ldots + q_nb^n$. In either case, since
either $k > 0$ or $l > 0$, this contradicts the linear independence of $B$.
Thus, $[E(a) : E] \leq m$.

Therefore, $[E(a,b) : E] = mn$. \qquad $\blacksquare$ \\

%DONE
\textbf{Exercise 52:} Let $E$ be a field, and let $F$ be a field extension of
$E$.

%DONE
\textbf{i:} Suppose $u \in F$ is algebraic over $E$. Then, $\exists P \in
E[x]$ such that $P(u) = 0$. Let $n$ be the degree of $P$, and let $p_0, p_1,
\ldots, p_n$ be the coefficients of $P$, so that  $p_0 + p_1u + \ldots +
p_nu^n = 0$. Subtracting odd terms gives $-(p_1u + p_3u^3 + \ldots + p_ku^k)
 = p_0 + p_2u^2 + \ldots + p_ju^j$, where one of $k, u$ is $n$, and the other
is $n - 1$, depending on whether $n$ is even or odd. Then, squaring both sides
of the equation gives $q_2u^2 + q_6u^6 + \ldots + q_{2k}u^{2k}
= q_0 + p_4u^2 + \ldots + p_{2j}u^{2j}$ for some $q_0, q_2, \ldots, q_{2n}
\in E$, so that $q_0 + q_2u^2 + \ldots + q_{2n}u^{2n} = 0$. Thus, $u^2$ is a
root of $q_0 + q_2u + \ldots q_{2n}u^n$, so that $u^2$ is algebraic.
\qquad $\blacksquare$ \\

%DONE
\textbf{ii:} Suppose $v \in F$ of algebraic of odd degree over $E$ (in
particular, let $v$ be algebraic of degree $m \in \mathbb{N}$ over $E$).
Clearly, $E(v^2) \subseteq E(v)$, since any field containing $v$ contains
$v^2$. Suppose $p \in E(v)$, so that $p = e_0 + e_1v + \ldots +
e_{m - 1}v^{m - 1}$ for some $e_0, e_1, \ldots, e_{m - 1} \in E$ (as $\{1,
v, v^2, \ldots, v^{m - 1}\}$ is a basis of $E(v)$. Let
$a = e_0 + e_2v^2 + \ldots + e_{m - 2}v^{m - 2}$,
$b = e_1 + e_3v^2 + \ldots + e_{m - 1}v^{m - 2}$, so that $a, b \in E(v^2)$.
Then, subtracting odd terms gives $a = vb$. Since $m$ is odd, $b \neq 0$
(as, otherwise, there would be a polynomial $P$ of degree $(m - 1)$ with
$P(v) = 0$), so, since $ab^{-1} = v$, $v \in E(v^2)$. Therefore, $p \in
E(v^2)$, so $E(v) \subseteq E(v^2)$ and thus $E(v) = E(v^2)$.

By Lemma 1 above, if $v$ is algebraic of degree $m$ over $E$,
$[E(v) : E] = m$. Since $E(v^2) = E(v)$, $[E(v^2) : E] = m$, so that $v^2$ is
algebraic of degree $m$ over $E$ (in particular, $v^2$ is algebraic of odd
degree over $E$). \qquad $\blacksquare$ \\

%DONE
\textbf{iii:} It is possible to have $w$ algebraic of even degree over $E$ and
$E(v) = E(v^2)$. For instance, let $w = \omega = \frac{-1 + i\sqrt{3}}{2}$,
and let $E = \mathbb{R}$. Clearly, $w$ is not of degree $1$ over $\mathbb{R}$,
since $w \not \in \mathbb{R}$. However, $w^2 + w + 1 = 0$, so $w$ is algebraic
of degree $2$ over $E$. Since $w = -1 - w^2$, $\mathbb{R}(w) =
\mathbb{R}(w^2)$. \qquad $\blacksquare$ \\

%DONE
\textbf{Exercise 53:} Let $E$ be a field, and let $F$ be a field extension of
$E$. Suppose $u, v \in F$ with $v$ algebraic (in particular, of degree $m \in
\mathbb{N}$, over $E(u)$, and $v$ transcendental over $E$. Then, for some
$e_0, e_1, \ldots, e_m \in E(u)$, $e_0 + e_1v + \ldots + e_mv^m = 0$.
Since $e_0, e_1, \ldots, e_m \in E(u)$, $e_0 = p_{0,0} + p_{0,1}u + \ldots +
p_{0,n}u^n, e_1 = p_{1,0} + p_{1,1}u + \ldots + p_{1,n}u^n, \ldots,
e_m = p_{m,0} + p_{m,1}u + \ldots + p_{m,n}u^n$. Let $f_0 = p_{0,0} + p_{1,0}v
 + \ldots + p_{m, 0}v^m, f_1 = p_{0,1} + p_{1,1}v + \ldots + p_{m, 1}v^m,
\ldots, f_m = p_{0,m} + p_{1,m}v  + \ldots + p_{m, m}v^m$. Then, $f_0, f_1,
\ldots, f_m \in E(v)$, and $f_0 + f_1u + \ldots + f_nu^n = 0$. Furthermore,
since $v$ is transcendental over $E$, $f_0 \neq 0$. Thus, $u$ is algebraic
over $E(v)$. \qquad $\blacksquare$ \\

%NOT DONE
\textbf{Exercise 54:} Let $E$ be a field, and let $F = E(x)$. Let $u =
\frac{x^3}{x + 1}$, and let $K = E(u)$. Let $v = x$. Then, since any field
containing $x$ contains $\frac{x^3}{x + 1}$, $K(v) = E(x) = F$. Since $x$ is a
root of $u + uy - y^3 \in E(u) = K$, by Lemma 1 above, $[F : K] \leq 3$.
It remains to show that there does not exist a polynomial $P$ of degree $1$ or
$2$ such that $P(u) = 0$, so that $[F : K] \geq 3$, and thus $[F : K] = 3$. \\

The following lemma is used in the solution of Exercise 56:

\textbf{Lemma 2:} Let $E$ be a field, let $P \in E[x]$ be of degree $n \geq
1$, and let $F$ be a splitting field extension for $P$ over $E$. If the roots
of $P$ are distinct (that is, they are all of multiplicity $1$), and no root
is in $E$, $[F : E] = n!$.

%DONE
\textbf{Proof}: If $n = 1$, then $P$ is already linear, so that $E$ is itself
a splitting field for $P$, and $[E : E] = 1$ which divides $n$. Suppose, as an
inductive hypothesis, that, for some $k \in \mathbb{N}$, the above lemma holds
$\forall n \leq k$. Let $P$ be a polynomial of degree $(k + 1)$, with distinct
factors $f_1,f_2, \ldots, f_{k + 1} \not \in E$. Let $F_{k}$ be a splitting
field of $P/(f_{k + 1})$, and let $F_{k + 1}$. Then,
$[F_{k + 1} :F_k] = k + 1$. Since $[F_k : E] = k!$, $[F_{k + 1} : E]
 = [F_k : E] [F_{k + 1} :F_k] = k!(k + 1) = (k + 1)!$. Thus, by the Principle
of Mathematical Induction, the above lemma holds $\forall n \in \mathbb{N}$.

%DONE
\textbf{Exercise 56:} Order the roots of $P$ $f_1, f_2, \ldots, f_n$ such
that, for some $k \in \mathbb{N}$, $\forall n \in \mathbb{N}$, $i \leq k$ if
and and only if $f_i \not \in E$ and, $\forall j \in \mathbb{N}$ with
$i \neq j \leq k$, $f_i \neq f_j$. That is, pick an ordering such that each of
the roots not in $E$ appears exactly once within the first $k$ terms. By Lemma
2 above, a splitting field extension $F^{\prime}$ of
$(x - f_1)(x - f_2) \ldots (x - f_k)$ is such that $[F^{\prime} : E] = k!$.
Furthermore, it is a splitting field of $P$, since all other factors of $P$
are either already in $E$ or are among $f_0, f_1, \ldots, f_k$, so that they
can be factored into linear terms. Thus, since $k!$ divides $n!$ (as $k \leq
n$), $[F : E]$ divides $n!$.

\end{document}
