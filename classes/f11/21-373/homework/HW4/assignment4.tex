\documentclass{article}%
\usepackage{makeidx}
\usepackage{amsmath}
\usepackage{graphicx}
\usepackage{amsfonts}
\usepackage{amssymb}%
\usepackage{enumerate}%
\usepackage{fullpage}%
\setcounter{MaxMatrixCols}{30}
\providecommand{\U}[1]{\protect\rule{.1in}{.1in}}
\providecommand{\U}[1]{\protect\rule{.1in}{.1in}}
\providecommand{\U}[1]{\protect\rule{.1in}{.1in}}
\providecommand{\U}[1]{\protect\rule{.1in}{.1in}}
\providecommand{\U}[1]{\protect\rule{.1in}{.1in}}
\providecommand{\U}[1]{\protect\rule{.1in}{.1in}}
\providecommand{\U}[1]{\protect\rule{.1in}{.1in}}
\providecommand{\U}[1]{\protect\rule{.1in}{.1in}}
\newenvironment{proof}[1][Proof]{\textbf{#1.} }{\ \rule{0.5em}{0.5em}}
\begin{document}

\begin{center}
\textbf{Shashank Singh}

\textbf{sss1@andrew.cmu.edu}

\textbf{21-373 \quad Algebraic Structures, Fall 2011}

\textbf{Assignment 4}

\textbf{Due: Friday, October 7}\\
\end{center}

%DONE
\textbf{Exercise 22: Lemma:} If $G$ is a finite group and $p = |G|$, the
order of $G$, is prime, then $G$ is cyclic.

\textbf{Proof:} Let $g \in G$, with $g \neq e$ (such a $G$ exists, since $p$
is prime and thus $p > 1$). Since the order of $g$ divides $p$ and is greater
than $1$, it must be $p$, and so, $\forall i,j \in \mathbb{N}$ with $
i < j < p$, $g^i \neq g^j$. Therefore, the $p$ distinct elements $e, g, g^2,
g^3, \ldots, g^{p - 1}$ are precisely the elements of $G$ and $G$ is cyclic.
\qquad $\blacksquare$ \\

As a consequence of the above lemma, a subgroup of $E_{p^n}$ of order $p$ is
determined by determining a single generator element.
If, for some $k_2,k_3,\ldots,k_n,l_2,l_3,\ldots,l_n \in \mathbb{Z}_n$, letting
$b = 1$, $g_1 = (b,k_2,k_2,\ldots,k_n)$, $g_2 = (b,l_2,l_3,\ldots,l_n)$, then
$<g_1> = <g_2>$ if and only if $k_1 = l_1,k_2 = l_2, \ldots, k_n = l_n$. This
follows from the fact that $k_i^n = l_i^n$ $\forall n \in \mathbb{N}$ if
and only if $k_i = l_i$.
Thus, choosing values for $k_2,k_3,\ldots,k_n$ from the $p$ values in
$\mathbb{Z}_p$, there are $p^{n - 1}$ subgroups of the form of $g_1$.
The same reasoning holds if we let $b = 0$, with the exception that $e$ does
not generate a subgroup of order $p$. Thus, there are $2*p^{n - 1} - 1$
subgroups of $E_{p^n}$ of order $p$. \\

%DONE
\textbf{Exercise 23:} Let $G$ be a finite group, and let $n$ be the order of
$G$. Then, since, $\forall g \in G$, the order of $g$ must divide the order of
$G$, $n$ is a natural number such that $\forall g \in G$, $g^n = e$, so
$\exists$ a minimal $k \in \mathbb{N}$ such that, $\forall g \in \mathbb{G}$,
$g^k = e$ (i.e., $k$ is the exponent of $G$).

Note that $k$ is not necessarily the order of any element in $G$; consider,
for instance, the symmetric group $S_3$. The exponent of $S_3$ is $6$, but no
element of $S_3$ has order greater than $3$.

Let $G = \mathbb{Z}_2 \times \mathbb{Z}_2 \times \mathbb{Z}_2 \times \ldots$
Then, $|G| = |\mathcal{P}(\mathbb{N})|$ (the mapping $f: \mathbb{G}
\rightarrow \mathcal{P}(\mathbb{N})$ such that, $\forall g = (g_1,g_2,g_3,
\ldots) \in G$, $f(g) = \{n \in \mathbb{N}: g_n = 1\}$ gives a simple
bijection), but the order of any element $g \in G$ is $1$ or $2$, so that the
exponent of $G$ is $2$. Thus, there exist infinite groups of finite exponent.
\qquad $\blacksquare$ \\

%DONE
\textbf{Exercise 24:} Let $G$ be an Abelian group, let $p \in \mathbb{N}$ be
prime, let $G^p = \{g^p : g \in G\}$, and let $G_p = \{g \in G : g^p \in G\}$.
Clearly, by their definition, and the closure of $G$ under its group operation,
$G^p \subseteq G$ and $G_p \subseteq G$. Furthermore, clearly, $e \in G^p$ and
$e \in G_p$, since $e^p = e$, so $G^p \neq \emptyset$ and $G_p \neq \emptyset$.
Suppose $a, b \in G^p$. Then, $\exists g_1, g_2 \in G$, such that $g_1^p = a,
g_2^p = b$. Thus, $g_1g_2^{-1} \in G$, so, since $G$ is Abelian and thus
$\left(g_1g_2^{-1}\right)^p = g_1^p \left(g_2^p\right)^{-1} = ab^{-1}$,
$ab^{-1} \in G^p$. Suppose, on the other hand, that  $g_1, g_2 \in G_p$, so
that $g_1^p = g_2^p = e$. Then, since $G$ is Abelian, $\left(g_1 g_2^{-1}
\right)^p = g_1^p \left(g_2^p\right)^{-1} = e e^{-1} = e$, so $g_1 g_2^{-1}
\in G_p$. Thus, $G^p \leq G$ and $G_p \leq G$. \qquad $\blacksquare$\\

Let $G, H, K$ be groups, with $G = H \times K$, and let $p \in \mathbb{N}$ be
prime.

Suppose $g \in H^p \times K^p$, so that $\exists h_1 \in H^p, k_1 \in K^p$,
such that $g = (h_1,k_1)$. Then, for some $h_0 \in H, k_0 \in K$, $h_1 =
h_0^p$ and $k_1 = k_0^p$, so that $g = (h_1,h_1) = (h_0,k_0)^p$. Thus,
since $(h_0, k_0) \in G$ and $G$ is a group and consequently closed under its
operation, $g \in G$. Suppose, on the other hand, that $g_1 \in G^p$, so that
$g_1 = g_0^p$, for some $g_0 \in G$. Since $G = H \times K$, $g_0 = (h_0,
k_0)$, for some $h_0 \in H,k_0 \in K$, so that $g_1 = (h_0,k_0)^p = (h_0^p,
k_0^p)$. Thus, since $h_0^p \in H^p$ and $k_0^p \in K^p$, $g_1 \in H^p \times
K^p$. Therefore, $G^p = H^p \times K^p$.

Suppose $g \in H_p \times K_p$, so that, for some $h \in H, k \in K$ with $h^p
 = e_H$, $k^p = e_K$ (where $e_H, e_K$ denote the identities of their
respective groups), $g = (h,k)$. Since $G = H \times K$, $g \in G$.
Furthermore, $(e_H,e_K)$ is the identity of $H \times K$ and thus of $G$ (i.e.,
$(e_H,e_K) = e$), so, since $g^p = (h,k)^p = (h^p,k^p) = (e_H,e_K) = e$, $g \in
G_p$. Suppose, on the other hand, that $g \in G^p$. Then, since $G = H \times
K$, $g \in \left(H \times K\right)^p$, so $\exists h \in H, k \in K$ such that
$g = (h,k)$ and $(h^p,k^p) = (h,k)^p = e = (e_H,e_K)$. Therefore, $h \in H_p,
k$, so $g = (h,k) \in H_p \times K_p$. Therefore, $G_p = H_p \times K_p$.
\qquad $\blacksquare$ \\

%DONE
\textbf{Exercise 27:} Let $G$ be a group of order $pqr$, where $p,q,r \in
\mathbb{N}$ are primes with $p < q < r$. Let $n_p, n_q, n_r$ denote the number
of Sylow-$p$,-$q$, and -$r$ subgroups of $G$, respectively. Suppose, for sake
of contradiction, that $G$ is simple. Then, $n_p, n_q,n_r > 1$, since,
otherwise, as explained in Remark 11.6 a Sylow-$p$,-$q$, or -$r$ subgroup
would be a non-trivial normal subgroup of $G$. By another of the Sylow
theorems, $n_p \cong 1 \pmod p$, $n_q \cong 1 \pmod q$, and $n_r \cong 1 \pmod
r$, so that $n_p > p$, $n_ q > q$, and $n_r > r > q > p$. Then, since $p,q$
are prime and, by another of the Sylow theorems $n_r | pq$, $n_r = pq$.
Similarly, since $n_q | pr$ and $n_p | qr$, $n_q \geq r$ and $n_p \geq q$.
Note that, since the Sylow-$p$ subgroups of $G$ are all cyclic (as shown in
the proof for Exercise 22), they are disjoint except for the identity (i.e., if
$G_1,G_2$ are two distinct Sylow-$p$ subgroups of $G$, then $G_1 \cap G_2 =
\{e\}$). Furthermore, the same holds for any pair of a Sylow-$q$ subgroups or
Sylow-$r$ subgroups, as well as any pair of Sylow-$p$, Sylow-$q$, and
Sylow-$r$ subgroups. Thus, the number of elements in $G$ is at least
\begin{eqnarray*}
|\{e\}| + n_r(r - 1) + n_q(q - 1) + n_p(p - 1) 
& \geq & 1 + pq(r - 1) + r(q - 1) + q(r - 1) \\
& >    & pq(r - 1) + r(q - 1) \\
& \geq & pq(r - 1) + pq = pqr.
\end{eqnarray*}
However, this contradicts the given that $G$ is of order $pqr$. Thus, $G$ cannot
be simple. \\

Let $G$ be a group of order $p^2q$, where $p, q \in \mathbb{N}$ are primes with
$p < q$. Suppose, for sake of contradiction, that $G$ is simple. As explained
previously, since $G$ is simple, letting $n_p$ denote the number of Sylow-$p$
subgroups of $G$ and $n_q$ the number of Sylow-$q$ subgroups of $G$, $n_p, n_q
 > 1$, $n_p | q$, $n_q | p^2$, $n_p \cong 1 \pmod p$, and $n_q \cong 1 \pmod
q$, so that $n_q = p^2$ (since $n_q = p$ or $n_q = p^2$, but $n_q > q > p$),
$n_p = q$, and thus $q | (p^2 - 1)$. Since $q$ is prime, this implies that
$q | (p + 1)$ or $q | (p - 1)$. The latter is impossible, since $p < q$; the
former implies $q = p + 1$, for the same reason. Since either $p$ or $p + 1$
is even, and both $p$ and $q$ are prime, this is only possible if $p = 2$ and 
$q = 3$. However, then, there are $(q - 1)*n_q = 8$ elements of order $q$,
$1$ element of order $1$ (the identity), so there are not enough elements
to construct the requisite number of Sylow-$p$ subgroups ($n_p = 3$). \\

%DONE
\textbf{Exercise 28:} Suppose, for sake of contradiction, that there exists a
a simple group $G$ of order $90$.
As explained in the proof of Exercise 27, by the result of 11.6, since $G$ is
simple, if $p$ is prime, the number of Sylow-$p$ subgroups of $G$, denoted
$n_p$, must be greater than $1$ (i.e., $n_p > 1$). Furthermore, by the Sylow
theorems, $n_p \cong 1 \pmod p$ and $n_p | (90/p)$. Thus, $n_5 \cong 1 \pmod
5$, $n_5 | 18$, and $n_p > 1$, so $n_p = 6$, and, similarly, it can be seen
that $n_3 = 10$. However, since the intersection of different Sylow-$p$
subgroups can contain only the identity, this implies there are $(9 - 1)*10
 = 80$ elements of order $9$, and $(5 - 1)*6 = 24$ elements of order $5$,
which is impossible, since $24 + 80 > 90 = |G|$. Thus, no simple group of order
$90$ can exist.
\end{document}
