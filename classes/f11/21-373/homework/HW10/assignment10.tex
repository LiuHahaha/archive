\documentclass{article}
\usepackage{makeidx}
\usepackage{amsmath}
\usepackage{graphicx}
\usepackage{amsfonts}
\usepackage{amssymb}
\usepackage{enumerate}
\usepackage{fullpage}
\setcounter{MaxMatrixCols}{30}
\providecommand{\U}[1]{\protect\rule{.1in}{.1in}}
\providecommand{\U}[1]{\protect\rule{.1in}{.1in}}
\providecommand{\U}[1]{\protect\rule{.1in}{.1in}}
\providecommand{\U}[1]{\protect\rule{.1in}{.1in}}
\providecommand{\U}[1]{\protect\rule{.1in}{.1in}}
\providecommand{\U}[1]{\protect\rule{.1in}{.1in}}
\providecommand{\U}[1]{\protect\rule{.1in}{.1in}}
\providecommand{\U}[1]{\protect\rule{.1in}{.1in}}
\newenvironment{proof}[1][Proof]{\textbf{#1.} }{\ \rule{0.5em}{0.5em}}
\allowdisplaybreaks
\begin{document}

\begin{center}
\textbf{Shashank Singh}

\textbf{sss1@andrew.cmu.edu}

\textbf{21-373 \quad Honors Algebraic Structures, Fall 2011}

\textbf{Assignment 10}

\textbf{Due: Tuesday, December 6}\\
\end{center}

%DONE
\textbf{Exercise 64:} Let $n \in \mathbb{N}$ with $n \geq 1$, and let
$P = -1 + (x - 1)(x - 2)\ldots(x-n)$. Suppose, for sake of contradiction,
that $P$ is reducible in $\mathbb{Z}[x]$; in particular, suppose that
$P = AB$, for some non-constant $A,B \in \mathbb{Z}[x]$. Let $k_1$ be the
degree of $A$, and let $k_2$ be the degree of $B$. Then, $\forall i
\in \mathbb{N}$ with $1 \leq i \leq n$, $AB(i) = P(i) = -1$. The only integers
whose product is $(-1)$ are $1$ and $(-1)$; thus, $\forall i \in \mathbb{N}$,
either $A(i) = 1$ and $B(i) = -1$, or $A(i) = -1$ and $B(i) = 1$. In either
case, $A(i) + B(i) = 0$, so that $(A + B)$ has at least $n$ roots. Thus,
either $(A + B)$ is of degree $m \geq n$, or $(A + B) = 0$. Since $k_1,
k_2 \geq 1$ and $n = k_1 + k_2$, $k_1, k_2 < n$, so that, since
$m = \max \{k_1, k_2\} < n$, the first case is impossible. Thus,
$(A + B) = 0$, so that $A = -B$. Then, noting that $k_1 = k_2$, if $a$ is the
coefficient of $x^{k_1}$ in $A$ and $b$ is the coefficient of $x^{k_1}$ in
$B$, $a = -b$, so that $ab < 0$ (as $a,b \neq 0$, by definition of degree).
However, this is a contradiction, since $ab$ is the coefficient of $x^n$
in $P$, which is $1$. Thus, $P$ is irreducible in $\mathbb{Z}[n]$.
\qquad $\blacksquare$ \\

%DONE
\textbf{Exercise 65:} Let $n \geq 2$, and let $P = 1 + x + \ldots + x^{n - 1}$.
Then, $P = \frac{x^n - 1}{x - 1}$, so that the roots of $P$ are precisely those
$n^{\mbox{th}}$ roots of unity which are not $1$. Suppose $p$ is composite.
Then, by Lemma 34.12, since $P = \prod_{d | n}$, $P$ is not irreducible.

Suppose, on the other hand, that $n$ is prime. Then, as noted in Remark 34.13,
$P = \Phi_n$, where $\Phi_n$ denotes the $n^{\mbox{th}}$. By Lemma 35.1, then,
$P$ is irreducible. \qquad $\blacksquare$ \\

%DONE
\textbf{Exercise 66: i.} Let $P_1 = x^4 - 2 =
                                                (x - \sqrt[4]{2})
                                                (x + \sqrt[4]{2})
                                                (x - i\sqrt[4]{2})
                                                (x + i\sqrt[4]{2})$.

Since $2$ divides $2$ but not $1$, and $2^2$ does not divide $2$, by
Eisenstein's Criterion, $P$ is irreducible. Let $E = \mathbb{Q}(\sqrt[4]{2})$,
and let $F = E(i\sqrt[4]{2})$. Since any field containing
$\sqrt[4]{2}$ contains $-\sqrt[4]{2}$, and any field containing
$i\sqrt[4]{2}$ contains $-i\sqrt[4]{2}$, $F$ is a splitting field extension
for $P_1$ over $\mathbb{Q}[x]$. Since $P$ is irreducible and monic, $P$ is the
minimal polynomial for $\sqrt[4]{2}$ over $\mathbb{Q}$, so that
$[E : \mathbb{Q}] = 4$. Since $i\sqrt[4]{2} \not \in \mathbb{R}$ and
$E \subseteq \mathbb{R}$, $i\sqrt[4]{2} \not \in E$, so that $[F : E] \geq 2$.
Furthermore, $x^2 + (\sqrt[4]{2})^2 \in E[x]$ is a degree $2$ polynomial, so
that $[F : E] \leq 2$, and thus $[F : E] = 2$. Therefore,
$[F : \mathbb{Q}] = [F : E][E : \mathbb{Q}] = 8$. \qquad $\blacksquare$ \\

%DONE
\textbf{ii.} Let $P_2 = x^4 + 2 = 
                                                (x - \sqrt[4]{-2})
                                                (x + \sqrt[4]{-2})
                                                (x - i\sqrt[4]{-2})
                                                (x + i\sqrt[4]{-2})$.

Let $F = \mathbb{Q}(\sqrt[4]{-2},i\sqrt[4]{-2})$, and let
$E = \mathbb{Q}(\sqrt[4]{2},i\sqrt[4]{2})$, so that $F$ is a splitting field
extension for $P_2$ over $\mathbb{Q}[x]$. Since any field containing $\mathbb{Q}$
contains $\sqrt[4]{-2}$ and $i\sqrt[4]{-2}$ if and only if it contains $\sqrt[4]{2}$
and $i\sqrt[4]{2}$, $F = E$, so that $[F : \mathbb{Q}] = [E : \mathbb{Q}]$. As
shown in part i., $[E : \mathbb{Q}] = 8$, so that $[F :\mathbb{Q}] = 8$.
\qquad $\blacksquare$ \\

%DONE
\textbf{iii.} Let $\xi$ Let $P_3 = x^4 + x^2 + 1 = (x^2 + x + 1)(x^2 - x + 1) =
                                                (x - \sqrt[3]{-1})
                                                (x + \sqrt[3]{-2})
                                                (x - (-1)^{2/3})
                                                (x + (-1)^{2/3}$.

Let $F = \mathbb{Q}(\sqrt[3]{-1})$. Then, clearly, $-\sqrt[3]{-2} \in F$, and
$(\sqrt[3]{-1})^2 = (-1)^{2/3} \in F$, so that $-(-1)^{2/3} \in F$. Thus, $F$
is a splitting field extension of $\mathbb{Q}$. Furthermore, $\sqrt[3]{-1}$ is
a root of the degree $2$ polynomial $x^2 - x + 1 \in \mathbb{Q}[x]$, so that
$[F : \mathbb{Q}] \leq 2$. Since $(-1)^{2/3} \in F$ and
$(-1)^{2/3} \not \in \mathbb{Q}$, $[F : \mathbb{Q}] \geq 2$, so that
$[F : \mathbb{Q}] = 2$. \qquad $\blacksquare$ \\


%NOT DONE
\textbf{iv.} Let $P_4 = (x^3 + 2)(x^3 - 2) =
                                                (x - \sqrt[3]{2})
                                                (x + \sqrt[3]{2})
                                                (x - \sqrt[3]{-2})
                                                (x + \sqrt[3]{-2})
                                                (x - (-1)^{2/3}\sqrt[3]{2})
                                                (x + (-1)^{2/3}\sqrt[3]{-2}$.
Let $E = \mathbb{Q}(\sqrt[3]{2})$, and let $F = E((-1)^{2/3}\sqrt[3]{2})$, so
that $F$ is a splitting field extension for $P_4$ over $\mathbb{Q}$.
Since $\sqrt[3]{2}$ is a root of the degree $3$ polynomial
$x^3 - 2 \in mathbb{Q}[x]$, $[E : \mathbb{Q}] = 3$. However, since
$E \subseteq \mathbb{R}$, and $(-1)^{2/3}\sqrt[3]{2} \not \in \mathbb{R}$,
$(-1)^{2/3}\sqrt[3]{2} \not \in E$, so that $[F : E] \geq 2$. However,
$(-1)^{2/3}\sqrt[3]{2}$ is a root of the degree $2$ polynomial
$(x^2+2^{1/3} x+2^{2/3}) \in E[x]$, so that $[F : E] \leq 2$, and thus
$[F : E] = 2$. Therefore, $[F : \mathbb{Q}]  = [F : E][E : \mathbb{Q}] = 6$.
\qquad $\blacksquare$ \\

%DONE
\textbf{Exercise 67:} Let $E$ be a field. Let $S = E \backslash \{0,1,-1\}$.
Suppose $x \in E$, with $x^2 = 1$. Then, $x^2 + x = x + 1$, so that
$x(x + 1) = x + 1$. Therefore, $x \in \{1,-1\}$. Thus, $\forall y \in S$,
$y \neq n^{-1}$. As a consequence, $\prod_{y \in S} y = 1$, since each term in
the product can be paired with its inverse in the product. Thus, since
$\prod_{x \in E} x = (1)(-1)\prod_{x \in S} x = -1$. \qquad $\blacksquare$ \\

%DONE
\textbf{Exercise 68:}
Polynomials of degrees $2$ and $3$ in $\mathbb{Z}_3[x]$ are reducible
if and only if they have roots in $\mathbb{Z}_3$. Thus, the monic reducible
polynomials of degree $2$ in $\mathbb{Z}_3[x]$ can be counted by counting the
number of ways of choosing the roots. In the case that the roots are distinct,
there are ${3 \choose 2} = 3$ polynomials; otherwise, there are
${3 \choose 1} = 3$ polynomials, so that, of the $3^2 = 9$ monic degree $2$
polynomials in $\mathbb{Z}_3[x]$, $6$ are reducible, and $3$ are irreducible.

Reducible polynomials of degree $3$ in $\mathbb{Z}_3$ either have $3$ roots in
$\mathbb{Z}_3$, or are the product of a linear and an irreducible quadratic
polynomial in $\mathbb{Z}_3$. In the first case, the polynomials can be
counted by counting their roots. There is $1$ polynomial with
$3$ distinct roots, there are $3*2 = 6$ polynomials with $2$
distinct roots, and there are ${3 \choose 1} = 3$ polynomials with $1$
distinct root, so that there are $10$ such monic reducible degree $3$
polynomials. In the second case, since there are $3$ irreducible degree $2$
polynomials and $3$ linear polynomials, there are $3*3 = 9$ such monic
reducible degree $3$ polynomials. Thus, of the $3^3 = 27$ monic degree $3$
polynomials in $\mathbb{Z}_3[x]$, $19$ are reducible and $8$ are irreducible.

A degree $4$ polynomial in $\mathbb{Z}_3[x]$ is reducible if and only if it is a product of two
irreducible degree $2$ polynomials, or it has exactly $1$, $2$, or $4$ roots
in $\mathbb{Z}_3$. There are $3^2 = 9$ degree $4$ polynomials which are
products of two irreducible degree $2$ polynomials. If a degree $4$ polynomial
in $\mathbb{Z}_3[x]$ has exactly one root in $\mathbb{Z}_3$, it is the product
of one linear and one irreducible degree $3$ polynomial. Thus, there are $3*8 = 24$
degree $4$ polynomials with exactly one root in $\mathbb{Z}_3$. If a degree
$4$ polynomial in $\mathbb{Z}_3[x]$ has exactly two roots in $\mathbb{Z}_3$,
it is the product of two linear and one irreducible degree $2$ polynomial.
Thus, there are $3*{3 \choose 2} = 9$
degree $4$ polynomials with exactly $1$ root in $\mathbb{Z}_3$. Lastly, there
are $21$ degree $4$ polynomials in $\mathbb{Z}_3[x]$ with
$4$ roots in $\mathbb{Z}_3$ ($0$ with $4$ distinct roots, $3*3 = 9$ with $3$
distinct roots, $3$ with $2$ distinct roots of the same multiplicity, $6$ with
$2$ distinct roots of different multiplicity, and $3$ with one distinct root).

Thus, there are $81 - (9 + 24 + 9 + 21) = 18$ irreducible monic polynomials of degree $4$
in $\mathbb{Z}_3$. \qquad $\blacksquare$ \\
\end{document}
