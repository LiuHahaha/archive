%% This document created by Scientific Word (R) Version 3.5

\documentclass{article}%
\usepackage{makeidx}
\usepackage{amsmath}
\usepackage{graphicx}
\usepackage{amsfonts}
\usepackage{amssymb}%
\usepackage{enumerate}%
\usepackage{fullpage}%
\setcounter{MaxMatrixCols}{30}
\providecommand{\U}[1]{\protect\rule{.1in}{.1in}}
\providecommand{\U}[1]{\protect\rule{.1in}{.1in}}
\providecommand{\U}[1]{\protect\rule{.1in}{.1in}}
\providecommand{\U}[1]{\protect\rule{.1in}{.1in}}
\providecommand{\U}[1]{\protect\rule{.1in}{.1in}}
\providecommand{\U}[1]{\protect\rule{.1in}{.1in}}
\providecommand{\U}[1]{\protect\rule{.1in}{.1in}}
\providecommand{\U}[1]{\protect\rule{.1in}{.1in}}
\newenvironment{proof}[1][Proof]{\textbf{#1.} }{\ \rule{0.5em}{0.5em}}
\begin{document}

\begin{center}
\textbf{Shashank Singh}

\textbf{sss1@andrew.cmu.edu}

\textbf{21-373 \quad Algebraic Structures, Fall 2011}

\textbf{Assignment 1}

\textbf{Due: Wednesday, September 21}\\
\end{center}

I collaborated with the ``Morewood B Tower'' group in completing this
assignment.\\

%DONE
\textbf{Exercise 8:} Let $G$ be a group, with some subgroup $H$ of index
$[G:H] = 2$. By definision of the index, $H$ has $2$ left cosets, one of which
must be $eH = H$, and the other of which is $gH$, for some $g \in G
\backslash H$. If for some $g_2 \in G$ $g_2 \not \in H$, then $gg_2 \in H$,
since otherwise, $gg_2H$ would be another left coset of $H$ (noting that
$e \in H$, so $g_2 \neq e$. Suppose $g \in G$. If $g \in H$,
then clearly $gH = Hg$. If $g \not \in H$, then, for $h$ such that $gh \in gH$,
$gh \not \in H$, so, since $g^{-1} \not \in H$, $ghg^{-1} \in H$. Thus,
since $Hg^{-1} = Hg$, $gh \in H$. Similarly, if $hg \in Hg$, then
$g^{-1}hg \in H$, so $hg \in gH$. Thus, since $g$ is an arbitrary element of
$G$, since $gH = Hg$, $H$ is a normal subgroup of $G$.

This is not necessarily true if $H$ has index $3$. For instance, let $G$ be the
set of permutations on $3$ elements, and let $H$ be the subgroup
$\{(1\;2)(3), (1)(2)(3)\}$ (where elements are denoted by their cycle
decomposition). Then, since $|G|$ is finite, $H$ has index $|G|/|H| = 3$.
However, for $g = (1\;2\;3)$, $(1\;3)(2) \in gH$, but $(1\;3)(2) \not \in Hg$.
Thus, $gH \neq Hg$, so $G$ has a subgroup $H$ of index $3$ that is not
normal.\\

%DONE
\textbf{Exercise 9:} Let $G$ be a subgroup, and let $H$, $K$ be subgroups of
$G$. \\

Suppose $HK$ is a subgroup of $G$.

Suppose $g \in HK$. Then, since
$g^{-1} \in HK$, $\exists h \in H, k \in K$, such that $g^{-1} = hk$. Since $H$,
$K$ are subgroups, $h^{-1} \in H$, and $k^{-1} \in K$, so, since $g =
(g^{-1})^{-1} = (hk)^{-1} = k^{-1}h^{-1}$, and $k^{-1}h^{-1} \in KH$,
$g \in KH$.\

Suppose, on the other hand, that $g \in KH$, so that $\exists
k \in K, h \in H$ with $g = kh$. Then, $g^{-1} = (kh)^{-1} = h^{-1}k^{-1}$,
so $g^{-1} \in HK$, since $h^{-1} \in H$, and $k^{-1} \in K$ (as $H$,$K$ are
subgroups), so, since $HK$ is a subgroup, $g \in HK$. Thus, $HK = KH$.\\

Suppose, on the other hand, that $HK = KH$. Since $e \in H$ and $e \in K$,
$e = ee \in HK$, so $HK \neq \emptyset$. Suppose $g_1, g_2 \in HK$. Then,
for some $h_1, h_2 \in H$, $k_1,k_2 \in K$, $g_1 = h_1k_1$, $g_2 = h_2k_2$.
(Note, as a necessary aside, that this implies that $HK \subseteq G$, since
$H,K \subseteq G$ and $G$ is a group and consequently close under its
operation.)

Show $h_1k_1k_2^{-1}h_2^{-1} = g_1g_2^{-1} \in HK$. \\

%DONE
\textbf{Exercise 10:} Let $G$ be a finite group, and an let $A \subseteq G$,
such that $\frac{|G|}{2} < |A|$. Suppose $g \in G$. let $gA^{-1} =
\{ga^{-1} | a \in A\}$. Since, $G$ is a group, for $f: A \rightarrow gA^{-1}$,
$G: gA^{-1} \rightarrow A$, such that $\forall a \in A$, $f(a) = ga$, and,
$\forall b \in gA^{-1}$, $g(b) = g^{-1}b$, $f$ and $g$ are inverses and thus
bijective, so $|gA^{-1}| = |A| > \frac{|G|}{2}$. If $A$ and $gA^{-1}$ were
disjoint, then $|A \cup gA^{-1}| = |A| + |gA^{-1}| > \frac{|G|}{2} +
\frac{|G|}{2}$; however, $A \cup gA^{-1} \subseteq G$, so this is not possible,
and $\exists a \in A \cap gA^{-1}$. Furthermore, there must exist $b \in A$
with $a = gb^{-1}$, so that $ab = gb^{-1}b = g$. Thus, $g$ is written and the
product of two elements, $a$ and $b$, in $A$. \\

%DONE
\textbf{Exercise 11:} Suppose $d | n$. Then, since $3 \not | n$,
$3 \not | d$. Thus, $3 | (d - 1)$ or $3 | (d + 1)$, so
$d | (d^2 - 1)$, since $d^2 - 1 = (d + 1)(d - 1)$. Furthermore, since
$2 \not | n$, $2 \not | d$. Thus, $d \equiv 1 \pmod 4$ or
$d \equiv 3 \pmod 4$. In either case, $4 | (d - 1)$ or $(d + 1)$, and
$2$ divides the other, so $8 | (d^2 - 1)$, and thus $24 |
d^2 - 1$. Furthermore, since
\[d + \frac{n}{d} = \frac{d^2 - 1 + n + 1}{d},\]
$24 | (d^2 - 1)v$, $24 | (n + 1)$, and $(24,d) = 1$,
\[24 | d + \frac{n}{d}.\]
Since, if $n = k^2$ for some $k \in \mathbb{N}$, $n = 4k^2$ or
$n = 4(k^2 +4k) + 1$
\[\sum_{\{d: d | n\}} d =
\sum_{\{d : d | n \mbox{ and } d < \sqrt{n}\}} d + \frac{n}{d},\]
which is a sum of multiples of $24$ and consequently a multiples of $24$.\\

%DONE
\textbf{Exercise 12:} Suppose, for sake of contradiction, that, for some $n \in
\mathbb{N}$ with
$n \geq 2$, $n$ divides $2^n - 1$; in particular, let $n$ be the smallest such
natural number. Clearly, $n$ is odd, since, otherwise, it could not divide
$2^n - 1$, which is necessarily odd. Since $|\mathbb{Z}_n| = \phi(n) < n$
and the order of $2$ in $\mathbb{N}$ is at most
$|\mathbb{Z}_n|$, the order of $2$ in $\mathbb{N}$ is some $k \in \mathbb{N}$
with $k \neq n$ and $k$ divides $n$. Thus, $2^k \equiv 1 \pmod n$. However,
since $k$ divides $n$, $2^k \equiv 1 \pmod k$. Since $k \neq n$, this
contradicts the choice of $n$ as the smallest natural number with this
divisibility property. \\

%DONE
\textbf{Exercise 14: (i)} The elements of $G$ which can be expressed in the
form $c^2$, for some $c \in G$, are the $8$ values in the set
\[\{e,a^2,b,b^2,b^3,b^4,b^5,b^6\}.\]
\end{document}
