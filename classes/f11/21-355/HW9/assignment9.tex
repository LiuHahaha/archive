\documentclass{article}
\usepackage{makeidx}
\usepackage{amsmath}
\usepackage{graphicx}
\usepackage{amsfonts}
\usepackage{amssymb}
\usepackage{enumerate}
\usepackage{fullpage}
\setcounter{MaxMatrixCols}{30}
\allowdisplaybreaks
\begin{document}

\begin{center}
\textbf{Shashank Singh}

\textbf{sss1@andrew.cmu.edu}

\textbf{21-355C \quad Real Analysis, Fall 2011}

\textbf{Assignment 9}

\textbf{Due: Wednesday, Novemeber 16}
\end{center}

%DONE
\textbf{Question 0.1.} Let $g$ be a differentiable real function of
$\mathbb{R}$ such that, $\forall x \in \mathbb{R}$, $|g^{\prime}(x)| \leq M$,
for some $M \in \mathbb{R}$. Let $\epsilon = \frac{1}{M + 1} > 0$. Let
$f: \mathbb{R} \rightarrow \mathbb{R}$ such that, $\forall x \in \mathbb{R}$,
$f(x) = x + \epsilon g(x)$. Then, by Theorem 5.3, since the identity and $g$
are differentiable on $\mathbb{R}$, $f$ is differentiable on $\mathbb{R}$ and,
$\forall x \in \mathbb{R}$, $f^{\prime}(x) = 1 + \epsilon g^{\prime}(x)$.
$\forall x \in \mathbb{R}$, since $-(M + 1) < g^{\prime}(X)$, $-1 < \epsilon
g^{\prime}(x)$, so $f^{\prime}(x) > 0$. Suppose, for sake of contradiction,
that $f$ were not injective; in particular, suppose $\exists x_1, x_2 \in
\mathbb{R}$ such that $x_1 \neq x_2$ and $f(x_1) = f(x_2)$. Then, by the
Mean Value Theorem (in particular, by Theorem 5.10), $\exists x \in \mathbb{R}$
such that $f(x_2) - f(x_1) = (x_2 - x_1)f^{\prime}(x)$. Since $x_1 \neq x_2$
and $f(x_1) = f(x_2)$, this implies that $f^{\prime}(x) = 0$, contradicting
the above proof that $f^{\prime}(x) > 0$. Thus, $f$ is injective.
\qquad $\blacksquare$ \\

%DONE
\textbf{Question 0.2.} Let $f: (0, \infty) \rightarrow \mathbb{R}$ be
differentiable its domain, such that $\lim_{x \rightarrow \infty}
f^{\prime}(x) = 0$. Let $g: \mathbb{R} \rightarrow \mathbb{R}$ such that,
$\forall x \in \mathbb{R}$, $g(x) = f(x + 1) - f(x)$. Let $\epsilon \in
\mathbb{R}$ with $\epsilon > 0$. Since $\lim_{x \rightarrow \infty}
f^{\prime}(x) = 0$, $\exists x_0 \in (0,\infty)$ such that, $\forall x \in
\mathbb{R}$ with $x > x_0$, $|f^{\prime}(x)| < \epsilon$. Let $x \in
\mathbb{R}$ with $x > x_0$. Suppose, for sake of contradiction that $g(x) >
\epsilon$. Then, by the Mean Value Theorem (in particular, by Theorem 5.10),
$\exists c \in (x, x + 1)$ such that $|g(x)| = |f(x + 1) - f(x)| =
|((x + 1) - x) f^{\prime}(c)| = |f^{\prime}(c)|$. However, since $c > x_0$,
this contradicts the result that $|f^{\prime}(c)| < \epsilon$. Therefore,
$\forall x \in \mathbb{R}$ with $x > x_0$, $|g(x)| < \epsilon$, so that
$\lim_{x \rightarrow \infty} g(x) = \infty$. \qquad $\blacksquare$ \\

%NOT DONE
\textbf{Question 0.3.}

%NOT DONE
\textbf{Question 0.4.}

%NOT DONE
\textbf{Lemma 0.5.}

%NOT DONE
\textbf{Question 0.6.}

%NOT DONE
\textbf{Question 0.7.}

%NOT DONE
\textbf{Question 0.8.}

%NOT DONE
\textbf{Question 0.9.}

%NOT DONE
\textbf{Question 0.10.}
\end{document}
