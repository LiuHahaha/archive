\documentclass[11pt]{article}
\usepackage{enumerate}
\usepackage[margin=1in]{geometry}
\setlength\topmargin{0.1in}
\usepackage{graphicx}
\usepackage{fancyhdr}
\usepackage{amsmath, amsfonts, amsthm, amssymb}
\usepackage{setspace}

\onehalfspacing
\pagestyle{fancyplain}
\renewcommand{\headrulewidth}{0pt}

%%%%%%%%%%%%%%%%%%%%%%%%HEADER%%%%%%%%%%%%%%%%%%%%%%%%%%%%%
\newcommand{\myname}{Shashank Singh}
\newcommand{\myemail}{sss1@andrew.cmu.edu}
\newcommand{\myclassnum}{15-221}
\newcommand{\mysem}{Summer 2013}
\newcommand{\mysection}{Section M}
\newcommand{\theassignment}{Homework 2}
\newcommand{\thetitle}{Writing Assignment 1 Reflection}
\newcommand{\thedate}{05/30/2013}
%%%%%%%%%%%%%%%%%%%%%%%%%%%%%%%%%%%%%%%%%%%%%%%%%%%%%%%%%%%

%%%%%%%%%%%%%%%%%%%%CONTENT MACROS%%%%%%%%%%%%%%%%%%%%%%%%%
\renewcommand{\qed}{\quad $\blacksquare$}
\newcommand{\mqed}{\quad \blacksquare}
\newcommand{\inv}{^{-1}}
\newcommand{\bv}{\mathbf{v}}
\newcommand{\bx}{\mathbf{x}}
\newcommand{\by}{\mathbf{y}}
\newcommand{\bff}{\mathbf{f}}
\newcommand{\bzero}{\mathbf{0}}
\newcommand{\area}{\operatorname{area}}
\newcommand{\N}{\mathbb{N}} % natural numbers
\newcommand{\Z}{\mathbb{Z}} % integers
\newcommand{\Q}{\mathbb{Q}} % rational numbers
\newcommand{\R}{\mathbb{R}} % real numbers
\newcommand{\C}{\mathcal{C}} % compact functions
%%%%%%%%%%%%%%%%%%%%%%%%%%%%%%%%%%%%%%%%%%%%%%%%%%%%%%%%%%%

\begin{document}

\lhead{\myname\\\myemail\\\thedate}
\rhead{\myclassnum\\\mysem\\\mysection}

\begin{center}
\theassignment

\thetitle
\end{center}

\begin{enumerate}
\item 
I found the non-technical explanation easier to write, because I felt less
compelled to cover technical details of the definition. I felt it necessary to
give the technical audience some explanation of how mathematicians consider
manifolds and how manifolds depend on not only their own structure but also the
structure of the space into which they are embedded (extrinsic dimension).
A mathematician's definition of a manifold might be:\\\\
{\bf Definition:} If $n,k \in \N$ with $k \leq n$, $M \subseteq \R^n$ is a
{\bf $k$-manifold} iff, $\forall x \in M$, $\exists$ open sets
$U_x \subseteq \R^n$ (with $x \in U_x$) and $V_x \subseteq \R^k$ and a
continuous bijection $f_x : V_x \to M \cap U_x$ with $f\inv_x$ continuous.\\\\
A technical reader has some exposure to mathematical notation, and might have
seen some rigorous mathematical definitions. Thus, I struggled for quite a
while with how much of the true definition to include.

For the non-technical audience, it seemed sufficent to explain manifolds
visually, without referencing their mathematical context, because a more
thorough explanation would have to be unreasonably long.

\item
The results claim that both my definitions `should be easily
understood by' 15 to 16 year olds, suggesting that both definitions were
written at an appropriate level in terms of word complexity and sentence
length. This isn't too surprising, since most of my definitions are visual in
nature, and describe simple images. I think the primary difficulty in
understanding my definitions is in generalizing the examples and in
understanding (especially for the technical definition) how and why manifolds
in general serve as interesting mathematical objects. I think, therefore, that
I may have oversimplified the technical definition, at the cost of making the
definition less interesting and useful. It does surprise me slightly that the
readability statistics are so similar for the two definitions, since I had
thought that I had been using slightly more complex vocabulary for the
technical definition (I suppose this is reflected in the 5\% difference in the
percentages of complex words).
\end{enumerate}

Readability statistics are tabulated on the next page.

\begin{tabular}{|l|c|l|c|}
\hline
Non-technical Audience          &           &
Technical Audience              &           \\
\hline
Flesch Kincaid Reading Ease     &   69.3    &
Flesch Kincaid Reading Ease     &   65.1    \\
\hline
Flesch Kincaid Grade Level      &   8.1     &
Flesch Kincaid Grade Level      &   9       \\
\hline
Gunning Fog Score               &   11.9    &
Gunning Fog Score               &   14.3    \\
\hline
SMOG Index                      &   8.6     &
SMOG Index                      &   10.3    \\
\hline
Coleman Liau Index              &   9       &
Coleman Liau Index              &   9.4     \\
\hline
Automated Readability Index     &   7.5     &
Automated Readability Index     &   8.4     \\
\hline
No. of sentences                &   12      &
No. of sentences                &   23      \\
\hline
No. of words                    &   219     &
No. of words                    &   443     \\
\hline
No. of complex words            &   26      &
No. of complex words            &   73      \\
\hline
Percent of complex words        &   11.87\% &
Percent of complex words        &   16.48\% \\
\hline
Average words per sentence      &   18.25   &
Average words per sentence      &   19.26   \\
\hline
Average syllables per word      &   1.41    &
Average syllables per word      &   1.44    \\
\hline
\end{tabular}
\end{document}
