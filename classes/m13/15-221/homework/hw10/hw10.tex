\documentclass[11pt]{article}
\usepackage{enumerate}
\usepackage[margin=1in]{geometry}
\setlength\topmargin{0.1in}
\usepackage{graphicx}
\usepackage{fancyhdr}
\usepackage{amsmath, amsfonts, amsthm, amssymb}
\usepackage{setspace}

\onehalfspacing
\pagestyle{fancyplain}
\renewcommand{\headrulewidth}{0pt}

%%%%%%%%%%%%%%%%%%%%%%%%HEADER%%%%%%%%%%%%%%%%%%%%%%%%%%%%%
\newcommand{\myname}{Shashank Singh}
\newcommand{\myemail}{sss1@andrew.cmu.edu}
\newcommand{\myclassnum}{15-221}
\newcommand{\mysem}{Summer 2013}
\newcommand{\mysection}{Section M}
\newcommand{\theassignment}{Homework 10}
\newcommand{\thetitle}{15-221 Course Assignments}
\newcommand{\thedate}{06/24/2013}
%%%%%%%%%%%%%%%%%%%%%%%%%%%%%%%%%%%%%%%%%%%%%%%%%%%%%%%%%%%

\begin{document}

\lhead{\myname\\\myemail\\\thedate}
\rhead{\myclassnum\\\mysem\\\mysection}

\begin{center}
\theassignment

\thetitle
\end{center}

Homework 6 concerning Informative Abstracts was helpful, because I had never
heard of ``informative'' or ``descriptive'' abstracts. Prior to this assignment
and the relevant discussion, I had been familiar only with what I now know to
be descriptive abstracts, having read many in the course of research.

Since
being exposed to the idea, I have discovered that one of my research advisors
habitually writes an informative abstract for nearly every academic paper he
reads, allowing him to quickly find important information about articles. It
also allows him to quickly summarize an article to anyone with whom he needs to
talk about the article. I have also taken up this habit, and have found it a
very effective way to improve my understanding and retention of the key points
of an article.

Homework 6 gave good practice evaluating what makes an informative abstract
helpful, which I think will help me in writing these abstracts.
\newpage

\end{document}
