\documentclass[11pt]{article}
\usepackage{enumerate}
\usepackage[margin=1in]{geometry}
\setlength\topmargin{0.1in}
\usepackage{graphicx}
\usepackage{fancyhdr}
\usepackage{amsmath, amsfonts, amsthm, amssymb}
\usepackage{setspace}

\onehalfspacing
\pagestyle{fancyplain}
\renewcommand{\headrulewidth}{0pt}

%%%%%%%%%%%%%%%%%%%%%%%%HEADER%%%%%%%%%%%%%%%%%%%%%%%%%%%%%
\newcommand{\myname}{Shashank Singh}
\newcommand{\myemail}{sss1@andrew.cmu.edu}
\newcommand{\myclassnum}{15-221}
\newcommand{\mysem}{Summer 2013}
\newcommand{\mysection}{Section M}
\newcommand{\theassignment}{Homework 8}
\newcommand{\thetitle}{Process Explanations}
\newcommand{\thedate}{06/24/2013}
%%%%%%%%%%%%%%%%%%%%%%%%%%%%%%%%%%%%%%%%%%%%%%%%%%%%%%%%%%%

%%%%%%%%%%%%%%%%%%%%CONTENT MACROS%%%%%%%%%%%%%%%%%%%%%%%%%
%%%%%%%%%%%%%%%%%%%%%%%%%%%%%%%%%%%%%%%%%%%%%%%%%%%%%%%%%%%

\begin{document}

\lhead{\myname\\\myemail\\\thedate}
\rhead{\myclassnum\\\mysem\\\mysection}

\begin{center}
\theassignment

\thetitle
\end{center}

I consider ``The Patent Process'' to be the best process explanation. The
biggest reason is its clear layout, both aesthetically, with good use of
visuals, whitespace, and clearly defined sections, and substantially, with a
clear outline at the beginning, and conceptually, with a clear organization
that allows the reader to immediately know where he or she is in the process.

It elides more technical information then the other two papers, making it
insufficient for some readers, but it outlines the process clearly and is
simply written, significantly improving the reader's retention. Furthermore, it
provides references for further information, somewhat alleviating this
information sparsity.

Both ``Data Compression Using Huffman Coding'' and ``How Internet Routing
Works'' are difficult to grasp in detail because of the lack of visuals and
whitespace. The section titles in ``Data Compression'' do not stand out
\emph{at all} from the text, making it easy for the reader to get lost. Also,
the figures really need descriptive captions, since their contents are by no
means apparent. In fact, despite being more familiar with Huffman Coding than
either the Patent Process or Internet Routing, I found ``Data Compression'' the
most difficult to follow. Both papers also have issues with page numbering
(``Data Compression'' has none, and ``Internet Routing'' begins with 2).

It terms of the ``Planning a Process Explanation'' outline, the biggest asset
of ``The Patent Process'' is the clear enumeration of the steps on the first
page. The flowchart is extraneous (and perhaps even confusing, since it omits
Step 4) but still better than not enumerating the steps. Both the conclusion
and the introduction also provide a clear purpose for engineers to read the
document (although it doesn't explicitly identify the audience), and the
conclusion provides some interesting secondary applications of patents. The
simple presentation of the steps means that the summary at the end of each of
the other two documents is uncessary.

Perhaps a final major reason for the readability of ``The Patent Process'' is
that the process is broken into more steps, and, consequently, the steps are
smaller and simpler than in the other two.

\end{document}
