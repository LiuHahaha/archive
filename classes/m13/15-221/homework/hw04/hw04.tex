\documentclass[11pt]{article}
\usepackage{enumerate}
\usepackage[margin=1in]{geometry}
\setlength\topmargin{0.1in}
\usepackage{graphicx}
\usepackage{fancyhdr}
\usepackage{amsmath, amsfonts, amsthm, amssymb}
\usepackage{setspace}

\onehalfspacing
\pagestyle{fancyplain}
\renewcommand{\headrulewidth}{0pt}

%%%%%%%%%%%%%%%%%%%%%%%%HEADER%%%%%%%%%%%%%%%%%%%%%%%%%%%%%
\newcommand{\myname}{Shashank Singh}
\newcommand{\myemail}{sss1@andrew.cmu.edu}
\newcommand{\myclassnum}{15-221}
\newcommand{\mysem}{Summer 2013}
\newcommand{\mysection}{Section M}
\newcommand{\theassignment}{Homework 4}
\newcommand{\thetitle}{Room 703 Exercise Reflection}
\newcommand{\thedate}{06/12/2013}
%%%%%%%%%%%%%%%%%%%%%%%%%%%%%%%%%%%%%%%%%%%%%%%%%%%%%%%%%%%

%%%%%%%%%%%%%%%%%%%%CONTENT MACROS%%%%%%%%%%%%%%%%%%%%%%%%%
\renewcommand{\qed}{\quad $\blacksquare$}
\newcommand{\mqed}{\quad \blacksquare}
\newcommand{\inv}{^{-1}}
\newcommand{\bv}{\mathbf{v}}
\newcommand{\bx}{\mathbf{x}}
\newcommand{\by}{\mathbf{y}}
\newcommand{\bff}{\mathbf{f}}
\newcommand{\bzero}{\mathbf{0}}
\newcommand{\area}{\operatorname{area}}
\newcommand{\N}{\mathbb{N}} % natural numbers
\newcommand{\Z}{\mathbb{Z}} % integers
\newcommand{\Q}{\mathbb{Q}} % rational numbers
\newcommand{\R}{\mathbb{R}} % real numbers
\newcommand{\C}{\mathcal{C}} % compact functions
%%%%%%%%%%%%%%%%%%%%%%%%%%%%%%%%%%%%%%%%%%%%%%%%%%%%%%%%%%%

\begin{document}

\lhead{\myname\\\myemail\\\thedate}
\rhead{\myclassnum\\\mysem\\\mysection}

\begin{center}
\theassignment

\thetitle
\end{center}

My group consisted of Benjamin Chung, Sung Uk Ryu, Jane Threefoot, Felix Wen,
and myself. We were unable to solve the problem; I'll discuss some reasons
after narrating our process.

First, we each read our clues aloud, as Sung wrote down the salient points of
each clue. Then, we decided that a table would be the best way to organize the
information, and so I began tabulating the information on a piece of paper. Due
to my position, only about half the group was able to see the paper, and I
suspect this may have contributed significantly to our failing to solve the
problem. One complication, for example, was a combination of
communication errors that somehow led us to think that Carr was a teacher and
that Jacob was a Teacher's Aid. This might have been resolved if the person
with the corresponding clues had been able to see the table and notice the
mistake.

The exercise made clear two differences between written and oral
communication:
\begin{enumerate}
\item Oral communication relies much more heavily on short term memory.
\item Important technical details are more easily lost in oral communication.
\end{enumerate}
The first point became evident as I tried to tabulate the clues. It was very
difficult to keep track of the clues in my head, and only once I had organized
them in the table was I able to put the clues together to draw conclusions.
This explains why I was the only group member able to implement a process of
elimination to build on the given clues. The second point was clear in
retrospect from the confusion between Carr and Mr. Jacobs; had I read the title
``Mr.'' I would have immediately realized Jacobs was a teacher's name. Also, I
would have realized that each of the teachers was referenced with their title,
and so ``Carr'' would not have refered to a teacher.

Both of these points would be important for computer scientists to keep in mind
when presenting technically dense material orally. The first point suggests
the importance of organizing information carefully, by presenting material
concisely and in a well-chosen order, alongside with helpful visuals. The
second point suggests that a speaker must consider carefully what and how much
technical information to include and emphasize in a presentation, as the
clarity of important details should not be compromised by adding superfluous
details.

Another thing I noticed is that the group began with little or no planning and
thus kept changing approach, resulting in a lot of wasted time. This really
hints at the importance of an outline when discussing technical material.
\end{document}
