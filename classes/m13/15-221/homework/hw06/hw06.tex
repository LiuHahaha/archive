\documentclass[11pt]{article}
\usepackage{enumerate}
\usepackage[margin=1in]{geometry}
\setlength\topmargin{0.1in}
\usepackage{graphicx}
\usepackage{fancyhdr}
\usepackage{amsmath, amsfonts, amsthm, amssymb}
\usepackage{setspace}

\onehalfspacing
\pagestyle{fancyplain}
\renewcommand{\headrulewidth}{0pt}

%%%%%%%%%%%%%%%%%%%%%%%%HEADER%%%%%%%%%%%%%%%%%%%%%%%%%%%%%
\newcommand{\myname}{Shashank Singh}
\newcommand{\myemail}{sss1@andrew.cmu.edu}
\newcommand{\myclassnum}{15-221}
\newcommand{\mysem}{Summer 2013}
\newcommand{\mysection}{Section M}
\newcommand{\theassignment}{Homework 6}
\newcommand{\thetitle}{Abstracts}
\newcommand{\thedate}{06/16/2013}
%%%%%%%%%%%%%%%%%%%%%%%%%%%%%%%%%%%%%%%%%%%%%%%%%%%%%%%%%%%

%%%%%%%%%%%%%%%%%%%%CONTENT MACROS%%%%%%%%%%%%%%%%%%%%%%%%%
%%%%%%%%%%%%%%%%%%%%%%%%%%%%%%%%%%%%%%%%%%%%%%%%%%%%%%%%%%%

\begin{document}

\lhead{\myname\\\myemail\\\thedate}
\rhead{\myclassnum\\\mysem\\\mysection}

\begin{center}
\theassignment

\thetitle
\end{center}

I consider Abstract 2 to be the best. Abstract 2 focuses on covering the
technical content of the article, starting with the definition of a
tetrachromat, continuing with the biological basis for tetrachromacy, and
finally mentioning the study searching for tetrachromats and some of the
evolutionary context of tetrachromacy. This broad but thorough technical
coverage is crucuial for an informative abstract.

The abstract could have dscussed in detail the Newcastle University study
or tetrachromacy in animals, but, taking into account the word limit, the
abstract did a good job overviewing the information in the article. It also
avoided adding any interpretation or emphasis that was not present in the
article.

Abstract 1 is quite similar to Abstract 2, but fails to even mention the
research study. Also, the last two sentences of Abstract 1 seem strangely
worded to the point of causing some confusion.

Abstract 3, in contrast, discusses the research study in too much depth, at the
cost of neglecting some important facts about tetrachromacy.

Abstract 4 makes the same mistake as Abstract 3, and also places some emphasis
on the wrong facts due to its first sentence.

Abstract 5 fails entirely to present the term ``tetrachromat,'' thus omitting
perhaps the most central fact in the article. Admittedly, the abstract manages
very well to convey the facts without refering to the word.

Abstract 6 is written from an external perspective, which is incorrect for an
informative abstract, although the material coverage is both thorough and
concise.

\end{document}
