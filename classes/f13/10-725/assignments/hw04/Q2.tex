\documentclass[12pt]{article}

\usepackage{hyperref}
\usepackage{amsfonts,amsmath,amssymb,amsthm}
\usepackage{epsfig, graphics, graphicx}
\usepackage{latexsym}
\usepackage{fullpage}
\usepackage[parfill]{parskip}
\usepackage[tight]{subfigure}
\usepackage{hyperref}
\usepackage{enumerate,comment}

\newcommand{\inv}{^{-1}}
\newcommand{\figref}[1]{Fig.~\ref{#1}}
\newcommand{\R}{\mathbb{R}}
\renewcommand{\qed}{\quad \ensuremath{\blacksquare}}
\newcommand{\sgn}{{\operatorname{sign}}}

\begin{document}
Shashank Singh\footnote{sss1@andrew.cmu.edu}
\setcounter{section}{1}
\section{Safe rules for the LASSO [25 points] (Adona)}
%TODO
\begin{enumerate}[(a)]
\item The primal problem can be rewritten as
\[\min_{\beta \in \R^p, z \in \R^n} f(z) + \lambda\|\beta\|_1
    \quad \mbox{such that} \quad z = X\beta,
\]
and hence the dual function is
\begin{align*}
g(u)
 &  = \min_{\beta \in \R^p, z \in \R^n} f(z) + \lambda\|\beta\|_1
        + u^T(z - X\beta)   \\
 &  = \left( \min_{z \in \R^n} f(z) + u^Tz \right)
    + \min_{\beta \in \R^p} \lambda ( \|\beta\|_1 - u^TX\beta/\lambda)  \\
 &  = -\left( \max_{z \in \R^n} u^Tz - f(-z) \right)
    - \lambda \max_{\beta \in \R^p} (u^TX\beta/\lambda - \|\beta\|_1)   \\
 &  = -f^*(-z) - \left( \|\cdot\|_1 \right)^* (X^Tu/\lambda)            \\
 &  = -f^*(-z) - I_{\{v : \|v\|_\infty \leq 1\}}(X^Tu/\lambda)
    = -f^*(-z) - I_{\{v : \|v\|_\infty \leq \lambda\}}(X^Tu)
\end{align*}
Thus, it follows from the definition of the indicator function that dual
problem is
\[\max_{u \in \R^n} g(u)
    = \max_{
        \begin{subarray}{c}
            u \in \R^n\\
            \|X^Tu\|_\infty \leq \lambda
        \end{subarray}} -f^*(-u). \qed
\]

The stationarity KKT condition for $\beta$ gives
\begin{align*}
0
 &  \in \partial (f(z) + \lambda\|\beta\|_1) + u^T \partial (z - X\beta)    \\
 &  = \lambda \partial \|\beta\|_1 - u^T \partial X\beta
 &  = \lambda \partial \|\beta\|_1 - X^Tu,
\end{align*}
so that
\[X^tu \in \lambda \partial \|\beta\|_1
    = \lambda \left\{
        \begin{array}{ll}
            [-1,1]              & : \mbox{ if } \beta_i = 0      \\
            \{\sgn(\beta_i)\}   & : \mbox{ if } \beta_i \neq 0
        \end{array}
    \right..
\]
(this last equality was shown in class). \qed

\item By definition, the dual problem is (after replacing $u$ with $-u$ since
it is unconstrained)
\begin{align*}
\max_{\mu > 0} \min_{u \in \R^n} X_k^Tu + \mu(f^*(u) + \gamma)
    & = \min_{\mu \geq 0} \mu \max_{u \in \R^n}
                                \frac{-X_k^Tu}{\mu} - f^*(u) - \gamma   \\
    & = \min_{\mu \geq 0}
        \mu f\left( -\frac{X_k}{\mu} \right) - \mu\gamma,
\end{align*}
where we used the fact that, since $f$ is convex and continuous, $f^{**} = f$.
\qed

\item Plugging in the LASSO function and noting $\|X_k\| = 1$ gives
\begin{align*}
T_{k,+}
 &  = \min_{\mu > 0} - \mu\gamma
    + \frac{\mu}{2}\left\| Y + \frac{X_k}{\mu}\right\|_2^2  \\
 &  = \frac{1}{2} \min_{\mu > 0} - 2\mu\gamma
    + \mu \|Y\|_2^2 + 2Y^TX_k + \frac{1}{\mu}.
\end{align*}
Then, setting the appropriate derivative with respect to $\mu$ to $0$, we have
\[0 = -2\gamma + \|Y\|_2^2 - \mu^{-2}
    \quad \mbox{ so } \quad
  \mu = \frac{1}{\sqrt{\|Y\|_2^2 - 2\gamma}}.
\]
Plugging this into $T_{k,+}$ gives
\[T_{k,+} = \sqrt{Y^TY - 2\gamma} + Y^TX_k. \qed\]

\item Since all the steps in part (c) would work if we replaced $X_k$ with
$-X_k$, we have
\[T_{k,-} = \sqrt{Y^TY - 2\gamma} - Y^TX_k.\]
Thus,
\[\max(T_{k,+},T_{k,-}) = \sqrt{Y^TY - 2\gamma} + |Y^TX_k|. \qed\]

\item To be feasible, we want $sY^TY \geq 2\gamma$, so choose
\[s = \frac{2\gamma}{\|Y\|_2^2}.\]
Then, $\gamma = \frac{s\|Y\|_2^2}{2}$. \qed

\end{enumerate}
\end{document}
