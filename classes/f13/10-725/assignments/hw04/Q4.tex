\documentclass[11pt]{article}

\usepackage{hyperref}
\usepackage{amsfonts,amsmath,amssymb,amsthm}
\usepackage{epsfig, graphics, graphicx}
\usepackage{latexsym}
\usepackage{fullpage}
\usepackage[parfill]{parskip}
\usepackage[tight]{subfigure}
\usepackage{hyperref}
\usepackage{enumerate,comment}

\newcommand{\inv}{^{-1}}
\newcommand{\figref}[1]{Fig.~\ref{#1}}
\newcommand{\R}{\mathbb{R}}
\renewcommand{\P}{\mathbb{P}}
\newcommand{\Q}{\mathbb{Q}}
\newcommand{\X}{\mathcal{X}}
\renewcommand{\qed}{\quad \ensuremath{\blacksquare}}
\newcommand{\sgn}{{\operatorname{sign}}}
\newcommand{\KL}{{\operatorname{KL}}}
\newcommand{\wx}{{\widetilde{x}}}

\begin{document}
Shashank Singh\footnote{sss1@andrew.cmu.edu}
\setcounter{section}{3}

\section{Statistical estimation and penalty methods [25 points] (Sashank)}

\paragraph{A [KL divergence estimation].}
\begin{enumerate}[(a)]
\item First, note that, $\forall x \in \X$, if
\[0
    = \left( \frac{d}{dg(x)} \log(g(x))p(x) - g(x)q(x) \right) (g^*(x))
    = \frac{p(x)}{g^*(x)} - q(x),
\]
then $g^*(x) = \frac{p(x)}{q(x)}$, so that $g* = \frac{p}{q}$ maximizes
$\log(g^*(x))p(x) - g^*(x)q(x)$. If $\KL(\P\|\Q) = \infty$, then
$\int_\X \log(g^*(x))p(x) - g^*(x)q(x) = \infty$. Thus, assume
$\KL(\P\|\Q) < \infty$. Then,
\begin{align*}
\KL(\P\|\Q)
 &  = \int_\X p(x) \log \frac{p(x)}{q(x)} \, dx    \\
 &  = \int_\X \left( \log \frac{p(x)}{q(x)} \right) p(x) - p(x) + p(x) \, dx \\
 &  = \int_\X \left( \log g^*(x) \right) p(x) - g^*(x)q(x) \, dx + 1    \\
 &  = \int_\X \sup_{g > 0} \left( \log g(x) \right) p(x) - g(x)q(x) \, dx + 1\\
 &  = \sup_{g > 0} \int_\X \left( \log g(x) \right) p(x) - g(x)q(x) \, dx + 1
\end{align*}
(we use the Dominated Convergence Theorem switch the $\sup$ and the integral).
\qed
 
\item Not quite clear on how to do this; don't we need some sort of smoothing
to get an estimate of $\frac{p(x)}{q(x)}$ that is finite almost everywhere?
 
\item If $\frac{p(x)}{q(x)} = w^Tx$, then
\[KL(\P,\Q)
    = \sup_{
        \begin{subarray}{c}
            w \in \R^d\\
            w^Tx = \log \frac{p(x)}{q(x)}
        \end{subarray}} \int_\X w^Tx p(x) \, dx - \int\exp(w^Tx) q(x) \, dx + 1.
\]
This maximization problem is concave and smooth in $w$, so we should be able to
use any simple constrained optimization algorithm. The dual problem is
\[
\inf_{v \geq 0} \sup_{w \in \R^d}
    \int_\X w^Tx p(x) \, dx - \int\exp(w^Tx) q(x) \, dx + 1
        + \int_\X v(x)\left( w^Tx - \frac{p(x)}{q(x)} \right).
\]
\end{enumerate}

%TODO
\paragraph{B [Penalty methods].}
\begin{enumerate}[(i)]
\item Since $f$ and $h$ convex and differentiable, the KKT conditions for $(P)$
are
\begin{align*}
0 & = \nabla f(x^*) + \sum_{i = 1}^m u_i^*\nabla h_i(x^*)   \\
u_i^* h_i(x^*) & = 0, \quad (i \in \{1,\dots,m\})           \\
h(x^*) & \leq 0,                                            \\
\mbox{and} \quad u^* & \geq 0.
\end{align*}
$(P(c))$ is unconstrained and $h^+$ need not be differentiable, so the only
condition for $(P(c))$ is
\begin{align*}
0 & \in \partial \left( f(x_c^*) + cp(x_c^*) \right)
    = \nabla f(x_c^*) + c\sum_{i = 1}^m \partial h_i^+(x_c^*),
\end{align*}
where $x_c^*$ is a primal solution of $(P(c))$.
 
\item We show
$\displaystyle 0  \in \nabla f(\wx) + c\sum_{i = 1}^m \partial h_i^+(\wx)$.
Since $\displaystyle 0 = \nabla f(\wx) + \sum_{i = 1}^m u_i^*\nabla h_i(\wx)$,
it suffices to show
\[u_i^*\nabla h_i(\wx) \in c\partial h_i^+(\wx),\]
for each $i \in \{1,\dots,m\}$. Suppose $h_i(\wx) < 0$. Then, since each
$u_i^*h_i(\wx) = 0$, $u_i^* = 0$, and, since $h_i^+(\wx) = 0$, so that $h_i^+$
achieves a local minimum at $\wx$,
\[u_i^*\nabla h_i(\wx) = 0 \in c\partial h_i^+(\wx).\]
Now suppose $h_i(\wx) = 0$. Then, $h_i^+(\wx) = h_i(\wx)$, so that, since
$h_i^+ \geq h_i$ and $h_i$ is convex,
\[ch_i^+(z) - ch_i^+(\wx) \geq u_i^*h_i(z) - u_i^*h_i(\wx)
    \geq u_i^*\nabla h_i(\wx) \cdot (z - \wx) \quad \quad \forall z \in \R^n\]
(using $c \geq \|u\|_\infty$), and so
\[u_i^*\nabla h_i(\wx) \in c\partial h_i^+(\wx).\]
Since $h(\wx) \leq 0$, this covers all cases. \qed
 
\item Since $u^*$ is a solution to the dual, $u^* \geq 0$. Since the dual is
bounded, there is a solution $x^*$ to the primal. Then, since
$c > \|u^*\|_\infty$ and $u^* \geq 0$, by strong duality,
\begin{align*}
f(\wx) + c\sum_{i = 1}^m h_i^+(\wx)
 &  \geq f(\wx) + \sum_{i = 1}^m u_i^* h_i^+(\wx)
    \geq f(\wx) + \sum_{i = 1}^m u_i^* h_i(\wx)     \\
 &  \geq f(x^*) + \sum_{i = 1}^m u_i^* h_i(x^*)
    = f(x^*)
    = f(x^*) + c\sum_{i = 1}^m h_i^+(x^*)
\end{align*}
since each $u_i^*h_i(x^*) = 0 = h_i^+(x^*)$. Thus, since $\wx$ minimizes the
first expression, $f(\wx) = f(x^*)$ and $\wx$ is feasible, so $\wx$ solves
$(P)$. \qed 
\end{enumerate}
\end{document}
