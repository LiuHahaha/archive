\documentclass[12pt]{article}

\usepackage{hyperref}
\usepackage{amsfonts,amsmath,amssymb,amsthm}
\usepackage{epsfig, graphics, graphicx}
\usepackage{latexsym}
\usepackage{fullpage}
\usepackage[parfill]{parskip}
%\usepackage{mysymbols}
\usepackage[tight]{subfigure}
\usepackage{hyperref}
\usepackage{enumerate,comment}

\newcommand{\figref}[1]{Fig.~\ref{#1}}
\newcommand{\R}{\mathbb{R}}

\newenvironment{qpart}[1]%
{\begin{enumerate}\item[(#1)]}%
{\end{enumerate}}

\newenvironment{qpartp}[2]%
{\begin{enumerate}\item[(#1)] [#2 points] }%
{\end{enumerate}}

\title{Convex Optimization 10-725/36-725 \\Homework 4, due Oct 31}
\date{}

\begin{document}

\maketitle

{\bf Instructions}: 
\begin{itemize}
\item
You must complete Problems 1--3 and {\bf either
Problem 4 or Problem 5} (your choice between the two). 
\item When you
submit the homework, upload a single PDF (e.g., produced by LaTeX,
or scanned handwritten exercises) for the solution of each problem 
separately, to blackboard. You should your name at the top of each 
file, except for the first problem. {\bf Your solution to Problem 1
(mastery set) should appear completely anonymous to a reader.}
\end{itemize}

\bigskip
{\bf N.B.} A few problems may appear quite long, but really, there
is just a lot of text setting up and motivating the problem. This does not
mean more work for you! It's simply trying to give you an idea of where 
the problem is coming from, and why you'd want to solve it in the first 
place.

%%%%%%%%%%%%%%%%%%%%%%%%%%%%%%%%%%%%%%%%%%%%%%%%%%%%%%%%%%%%%%%%%
%%%%%%%%%%%%%%%%%%%%%%%%%%%%%%%%%%%%%%%%%%%%%%%%%%%%%%%%%%%%%%%%%

\documentclass[11pt]{article}

\usepackage{hyperref}
\usepackage{amsfonts,amsmath,amssymb,amsthm}
\usepackage{epsfig, graphics, graphicx}
\usepackage{latexsym}
\usepackage{fullpage}
\usepackage[parfill]{parskip}
%\usepackage{mysymbols}
\usepackage[tight]{subfigure}
\usepackage{hyperref}
\usepackage{enumerate,comment}

\newcommand{\figref}[1]{Fig.~\ref{#1}}
\newcommand{\R}{\mathbb{R}}

\def\prox{\mathrm{prox}}
\newcommand{\argmin}{\operatornamewithlimits{argmin}}   % argmin
\newcommand{\argmax}{\operatornamewithlimits{argmax}}   % argmax
\newcommand{\inv}{{^{-1}}}                              % inverse

\begin{document}
\twocolumn
\section{Mastery set [25 points]}
For any $p \in [1,\infty]$, let
\[B_p := \{x \in \R^n : \|x\|_p \leq 1\}\]
denote the $p$-norm unit ball.
\paragraph{Q1. [2+2+3]}
\begin{enumerate}[(a)]
\item The projection is given in each coordinate by
\[(P_{\R^n_+}(x))_i = \max \{0,x_i\}.\]
 
\item The projection is
\[P_{B_2}(x)
 =  \left\{
        \begin{array}{ll}
            x                   & : \mbox{ if } \|x\|_2 \leq 1  \\
            \frac{x}{\|x\|}     & : \mbox{ else }
        \end{array}
    \right..
\]
 
\item The projection is given in each coordinate by
\[(P_{B_\infty}(x))_i
 =  \left\{
        \begin{array}{ll}
            -1      & : \mbox{ if } x_i < 1  \\
            x_i     & : \mbox{ if } -1 \leq x \leq 1    \\
             1      & : \mbox{ else }
        \end{array}
    \right..
\]
 
\end{enumerate}

\paragraph{Q2. [3+3]}
\begin{enumerate}[(a)]
\item Let
\[x^* := \argmin_{x \in \R^n} \frac{1}{2}\|x - z\|_2^2
    - \lambda \sum_{i = 1}^n \log(|x_i|).\]
Taking an appropriate derivative in each coordinate gives
\[0 = x_i^* - z_i - \frac{\lambda}{x_i^*}
    = (x_i^*)^2 - z_ix_i^* - \lambda.\]
The quadratic formula gives
\[\mbox{\fbox{$\displaystyle x_i^*
    = \frac{z_i \pm \sqrt{z_i^2 + 4\lambda}}{2}$.}}\]
(Since the regularization term decreases away from $0$, we use the negative
value if $z_i < 0$ and the positive value otherwise).

\item Let
\[x^* := \argmin_{x \in \R^n} \frac{1}{2}\|x - z\|_2^2
    + \lambda(x^TAx + b^Tx + c).\]
Taking an appropriate gradient gives
\begin{align*}
0 & = (x^* - z) + \lambda((A + A^T)x^* + b) \\
\mbox{ or } \quad z - \lambda b & = (2\lambda A + I)x^*,
\end{align*}
since $A = A^T$. Since $A$ is positive semidefinite and $\lambda \geq 0$,
$(2\lambda A + I)$ is invertible, so
\[\mbox{\fbox{$\displaystyle x^* = (2\lambda A + I)\inv(z - \lambda b)$.}}\]
\end{enumerate}

\paragraph{Q3. [4+4+4]}
\begin{enumerate}[(a)]
\item The dual program is
\begin{align*}
\min_{u \in \R^4} \quad &   4u_1 + 2u_2         \\
\mbox{ such that } \quad
                        &   u_1 + u_2   \geq 2, \\
                        &   -u_1 - u_2  \geq 1, \\
\mbox{ and } \quad
                        &   u_1, u_2    \geq 0.
\end{align*}
The original problem is unbounded, and the dual is unfeasible.
 
\item The dual program is
\begin{align*}
\min_{u \in \R^4} \quad &   -4u_1 + 2u_2        \\
\mbox{ such that } \quad
                        &   -u_1 + u_2  \geq 2, \\
                        &   -u_1 + u_2  \geq 1, \\
\mbox{ and } \quad
                        &   u_1, u_2    \geq 0.
\end{align*}
The original problem is infeasible, and the dual is unbounded.
 
\item The dual program is
\begin{align*}
\min_{u \in \R^4} \quad &   -4u_1 + 2u_2        \\
\mbox{ such that } \quad
                        &   -u_1 + u_2  \geq 2, \\
                        &   u_1 - u_2   \geq 1, \\
\mbox{ and } \quad
                        &   u_1, u_2    \geq 0.
\end{align*}
The original and dual programs are both infeasible.
 
\end{enumerate}
\end{document}

\documentclass[11pt]{article}

\usepackage{hyperref}
\usepackage{amsfonts,amsmath,amssymb,amsthm}
\usepackage{epsfig, graphics, graphicx}
\usepackage{latexsym}
\usepackage{fullpage}
\usepackage[parfill]{parskip}
\usepackage[tight]{subfigure}
\usepackage{hyperref}
\usepackage{enumerate,comment}

\newcommand{\inv}{^{-1}}
\newcommand{\figref}[1]{Fig.~\ref{#1}}
\newcommand{\R}{\mathbb{R}}
\renewcommand{\P}{\mathbb{P}}
\newcommand{\Q}{\mathbb{Q}}
\newcommand{\X}{\mathcal{X}}
\renewcommand{\qed}{\quad \ensuremath{\blacksquare}}
\newcommand{\sgn}{{\operatorname{sign}}}
\newcommand{\KL}{{\operatorname{KL}}}
\newcommand{\wx}{{\widetilde{x}}}

\begin{document}
Shashank Singh\footnote{sss1@andrew.cmu.edu}
\setcounter{section}{1}
\section{PSD Matrices (Adona) [25 points]}
\textbf{Part A [15 points]: Basic properties of PSD matrices} 
\begin{enumerate}
\item Suppose $v \in \R^{|I|}$. Let $k_1 < \dots < k_{|I|}$ denote the elements
of $I$. Define $v_I \in \R^n$ such that each $(v_I)_{k_i} = v_i$ and
$(v_I)_k = 0$, $\forall k \notin I$. Since $X \succeq 0$,
\[v^T X_I v = v_I^T X v_I \geq 0,\]
and hence $X_I \succeq 0$. \qed

\item If $I = \{i,j\}$, then, since $X_I \succeq 0$,
\[0 \leq \det X_I = X_{ii}X_{jj} - X_{ij}^2,\]
and hence $X_{ij}^2 \leq X_{ii}X_{jj}$.

If $X_{ii} = 0$, then each $0 \leq X_{ji}^2, X_{ij}^2 \leq X_{ii}X_{jj} = 0$,
and so $X_{ji} = X_{ij} = 0$. \qed

\item If $X = Q \Lambda Q\inv$, where $\Lambda$ is a diagonal matrix of
eigenvalues of $X$, then
\[\det X
    = \det (Q \Lambda Q\inv)
    = \det Q \det \Lambda \det(Q\inv)
    = \det Q \det \Lambda \det(Q)\inv
    = \det \Lambda.
\]
Since the determinant of a diagonal matrix is trivially the product of its
diagonal elements, $\det X$ is the product of the eigenvalues of $X$. \qed

\item {\bf Lemma} If $X = Y^TZY$ for some invertible matrix $Y$, then
$X \succeq 0$ if and only if $Z \succeq 0$.

\emph{Proof:} If $X \succeq 0$, then, $\forall v \in \R^n$,
\[0
    \leq (Y\inv v)^TXY\inv v
    = (Y\inv v)^TY^TZYY\inv v
    = v^TZv,
\]
and so $Z \succeq 0$. The converse follows, since $Z = (Y\inv)^TXY$.
\quad $\square$

It is easily checked that $X = Y^TZY$, where
\[Y =
\begin{bmatrix}
    I   &   A\inv B \\
    0   &   I
\end{bmatrix}
\quad \mbox{ and } \quad
Z =
\begin{bmatrix}
    A   &   0   \\
    0   &   C - B^T A\inv B
\end{bmatrix}.
\]
Since $Y$ is triangular without zeros on its diagonal, $Y$ is invertible. Thus,
by the lemma, it suffices to show that $Z \succeq 0$ if and only if
$A \succeq 0$ and $C - B^T A\inv B \succeq 0$. One implication is immediate
from Part 1. On the other hand, it is clear that any eigenvalue of either $A$
or $C - B^T A\inv B$ is an eigenvalue of $Z$, and hence, if
$A, C - B^T A\inv B \succeq 0$, then $Z$ has non-negative eigenvalues and so
$Z \succeq 0$. \qed
\end{enumerate}

\newpage
\textbf{Part B [10 points]: Formulating problems as SDPs}
\begin{enumerate}
\item Let $X \in \R^{n \times n}$ such that each $X_{uv} = x_u^Tx_v$.
Since \[\|x_u - x_v\|^2 = x_u^Tx_u - 2x_u^Tx_v + x_v^Tx_v,\]
the objective function is just
\[\frac{1}{4} X \cdot W\]
where $W_{uv}$ is the weight of $(u,v)$ if $(u,v) \in \mathcal{E}$ and
$W_{uv} = 0$ otherwise.

Each constraint $\|x_u\|^2 = 1$ is simply
\[X \cdot A_u = 1,\]
where $\left( A_u \right)_{uu} = 1$ and all other coordinates of $A_u$ are
zero.

Since each $\|x_u\|^2 = 1$, the constraint $\|x_{s_i} - x_{t_i}\|^2 = 4$ for
$(s_i,t_i) \in T$ is equivalent to $x_{s_i} = - x_{t_i}$, and so it can be
written
\[X \cdot A_{(s_i,t_i)} = 0,\]
where $A_{(s_i,t_i)}$ has $1$ in indices $(s_i,s_i)$ and $(s_i,t_i)$.

Didn't have time to finish this part.

\item Didn't have time to finish this part.

\end{enumerate}
\end{document}

\documentclass[11pt]{article}

\usepackage{hyperref}
\usepackage{amsfonts,amsmath,amssymb,amsthm}
\usepackage{epsfig, graphics, graphicx}
\usepackage{latexsym}
\usepackage{fullpage}
\usepackage[parfill]{parskip}
\usepackage[tight]{subfigure}
\usepackage{hyperref}
\usepackage{enumerate,comment}

\newcommand{\inv}{^{-1}}
\newcommand{\figref}[1]{Fig.~\ref{#1}}
\newcommand{\R}{\mathbb{R}}
\renewcommand{\P}{\mathbb{P}}
\newcommand{\Q}{\mathbb{Q}}
\newcommand{\X}{\mathcal{X}}
\renewcommand{\qed}{\quad \ensuremath{\blacksquare}}
\newcommand{\sgn}{{\operatorname{sign}}}
\newcommand{\KL}{{\operatorname{KL}}}
\def\hbeta{\hat{\beta}}
\def\hu{\hat{u}}

\begin{document}
Shashank Singh\footnote{sss1@andrew.cmu.edu}
\setcounter{section}{2}
\section{Binary sequences of piecewise constant expectation [30 points]}
\begin{enumerate}[(a)]
\item Let $m = n - 1$, and, for each $j \in \{1,\dots,m\}$ let
$d_j := (0,\dots,0,1,-1,0,\dots,0) \in \R^n$, with $d_{j,j} = 1$ and
$d_{j,j + 1} = -1$. Then, for $x_i = e_i$, (4) is clearly equivalent to

\[\min_{\beta \in \R^p} \sum_{i = 1}^n - t_i \cdot x_i^T\beta
    + \log\left(1 + \exp(x_i^T\beta) \right)
    + \lambda \sum_{j = 1}^m |d_j\beta|,
\]
which is of the form considered in Homework 3. $D \in \R^{m \times n}$ is the
matrix with $j^{th}$ row $d_j$. \qed

\item For notational convenience, we note that, if $H : [0,1] \to [0,\infty)$
is the entropy function
\[H(p) = -p\log p - (1 - p)\log(1 - p),\]
then $\displaystyle g(u) = \sum_{t = 1}^n H(y_t(D^Tu)_t)$. Since
$\displaystyle \frac{d}{dp} H(p) = -\log\left( \frac{p}{1 - p} \right)$, by the
Chain Rule,
\[ \nabla g(u)
    = -\sum_{t = 1}^n \log\left( \frac{y_t(D^Tu)_t}{1 - y_t(D^Tu)_t} \right)
        y_t \nabla (D^Tu)_t
    = -\sum_{t = 1}^n \log\left( \frac{y_t(D^Tu)_t}{1 - y_t(D^Tu)_t} \right)
        y_t d_t^T
    = Dc(u),
\]
where, for $t \in \{1,\dots,n\}$,
\[(c(u))_t
    = - \log\left( \frac{y_t(D^Tu)_t}{1 - y_t(D^Tu)_t} \right) y_t.
\]
Since $\displaystyle \frac{d}{dp} \log \left( \frac{p}{1 - p} \right)
    = \frac{1}{p(1 - p)}$,
\[\nabla(c(u))_t
    = -\frac{y_td_t^T}{y_t(D^Tu)_t(1 - y_t(D^Tu)_t)}
    = -\frac{d_t^T}{(D^Tu)_t(1 - y_t(D^Tu)_t)},
\]
so $\nabla c(u) = W(u)D^T$, where $W(u)$ is the diagonal matrix with
\[W_{t,t}(u) = -\frac{1}{(D^Tu)_t(1 - y_t(D^Tu)_t)},\]
and hence $\nabla^2 g(u) = D\nabla c(u) = DW(u)D^T$.

The log barrier function is
\[\phi(u)
    = \sum_{t = 1}^n \log(y_t(D^Tu)_t) + \log(1 - y_t(D^Tu)_t)
    + \sum_{i = 1}^m \log(\lambda - u_i) + \log(u_i + \lambda).
\]
Thus, for each $i \in \{1,\dots,m\}$,
\begin{align*}
(\nabla(\phi(u))_i
 &  = \frac{1}{u_i - \lambda} + \frac{1}{u_i + \lambda}
    + \sum_{t = 1}^n \frac{2y_t(D^Tu)_t - 1}{y_t(D^Tu)_t(y_t(D^Tu)_t - 1)}
    (y_td_t^T)  \\
 &  = \frac{2u_i}{u_i^2  - \lambda^2}
    + \sum_{t = 1}^n \frac{2y_t(D^Tu)_t - 1}{(D^Tu)_t(y_t(D^Tu)_t - 1)}
    d_t^T  \\
\end{align*}
Thus, $\nabla\phi(u) = a(u) + Db(u)$, where for each $i \in \{1,\dots,m\}$,
\[(a(u))_i = \frac{2u_i}{u_i^2  - \lambda^2}
    \quad \mbox{ and } \quad
  (b(u))_i = \sum_{t = 1}^n \frac{2y_t(D^Tu)_t - 1}{(D^Tu)_t(y_t(D^Tu)_t - 1)}.
\]
Since
$\displaystyle (\nabla (a(u))_i)_i = -\frac{2(u_i^2 + 1)}{(u_i^2 - 1)^2}$,
\[\nabla^2 \phi(u)
    = U(u) + DV(u)D^T,
\]
where $U(u)$ and $W(u)$ are diagonal matrices with
\[U_{i,i}(u) = -\frac{2(u_i^2 + 1)}{(u_i^2 - 1)^2}
    \quad \mbox{ and } \quad
  V_{t,t}(u) = \frac{-2(y_t(D^T)_t)^2 + 2y_t(D^T)_t - 1}
                            {(D^Tu)_t^2(y_t(D^Tu)_t - 1)^2}.
\]
The Newton step is
\begin{align*}
      \left[ \nabla^2 (\tau g(u) + \phi(u)) \right]\inv
 &                                      \nabla (\tau g(u) + \phi(u))    \\
 &  = \left[ D(\tau W(u) + V(u))D^T + U(u) \right]\inv
                                        \nabla (\tau D(c(u) + b(u)) + a(u)).
\end{align*}

\item See attached code.

%TODO
\item

\end{enumerate}

\end{document}

\documentclass[11pt]{article}

\usepackage{epsfig, graphics, graphicx}
\usepackage{latexsym}
\usepackage{fullpage}
\usepackage{hyperref}
\usepackage{amsmath,amssymb,enumerate,comment}

\newcommand{\qed}{\quad \ensuremath{\blacksquare}}
\newcommand{\figref}[1]{Fig.~\ref{#1}}
\newcommand{\R}{\mathbb{R}}
\newcommand{\prox}{\operatorname{prox}}
\newcommand{\op}{\operatorname{op}}
\newcommand{\argmin}{\operatornamewithlimits{argmin}}   % argmin
\newcommand{\argmax}{\operatornamewithlimits{argmax}}   % argmax
\newcommand{\e}{\varepsilon}                            % \varepsilon

\begin{document}
Shashank Singh\footnote{sss1@andrew.cmu.edu}
\setcounter{section}{3}
\section{Convergence rate of generalized gradient descent [15 points] (Adona)}
For convenience, we use the notation $\langle \cdot, \cdot \rangle$ to denote
the dot product in several expressions.
\begin{enumerate}[(a)]
\item A second-order Taylor approximation gives, for some $\xi \in \R^n$,
\[g(y) \leq g(x)    + (\nabla g(x))^T(y - x)
                    + \frac12(y - x)^T(\nabla^2f(\xi))(y - x).\]

Since all directional derivatives of $\nabla g$ are bounded in magnitude by $L$
(this is immediate from the definition of directional derivative) and
$\|(\nabla^2f(\xi))(y - x)\|$ is the magnitude of the derivative of $\nabla g$
at $\xi$ in the direction $y - x$,
$\|(\nabla^2f(\xi))(y - x)\| \leq L\|y - x\|$.
Thus, by Cauchy-Schwartz,
\[(y - x)^T(\nabla^2f(\xi))(y - x)
    \leq \|(y - x)\|_2\|(\nabla^2f(\xi))(y - x)\|_2
    \leq L\|(y - x)\|_2^2
.\]
Then, since $f = g + h$,
\begin{equation}
\label{ineq_a}
f(y)
    \leq g(x) + (\nabla g(x)^T) (y - x) + \frac{L}{2} \|y - x\|_2^2
    + h(y). \qed
\end{equation}
 
\item Substituting $y = x^+ = x - tG_t(x)$ into (\ref{ineq_a}) gives
\begin{align}
\label{ineq_b}
\notag
f(x^+)
 &  \leq g(x) + (\nabla g(x)^T) (x - tG_t(x) - x)
    + \frac{L}{2} \|x - tG_t(x) - x\|_2^2 + h(x - tG_t(x)). \\
\notag
 &  = g(x) - t(\nabla g(x)^T) G_t(x)
    + \frac{Lt^2}{2} \|G_t(x)\|_2^2 + h(x - tG_t(x)). \\
 &  \leq g(x) - t\langle \nabla g(x), G_t(x) \rangle
    + \frac{t}{2} \|G_t(x)\|_2^2 + h(x - tG_t(x)),
\end{align}
where the last inequality follows by bounding a factor of $t$ by $1/L$. \qed
 
\item From the definitions of $G_t$ and $\prox_t$, we have
\[x - tG_t(x)
    = \prox_t(x - t\nabla g(x))
    = \argmin_{z \in \R^n} \frac{1}{2t} \| x - t\nabla g(x) - z\|_2^2 + h(z).
\]
The zero subgradient characterization of optimality and definition of
$\argmin$ then imply
\begin{align*}
0
 &  \in \partial \frac{1}{2t} \| x - t\nabla g(x) - (x - tG_t(x))\|_2^2
                                                        + h(x - tG_t(x))    \\
 &  =   \partial \frac{1}{2} \| G_t(x) - \nabla g(x)\|_2^2
                                                + \partial h(x - tG_t(x))   \\
 &  =   \{-(G_t(x) - \nabla g(x)\} + \partial h(x - tG_t(x)),
\end{align*}
and hence $G_t(x) - \nabla g(x) \in \partial h(x - tG_t(x))$. \qed
 
\newpage
\item Since $G_t(x) - \nabla g(x) \in \partial h(x - tG_t(x))$,
\begin{align*}
h(x - tG_x(x))
    & \leq h(z) - \langle G_t(x) - \nabla g(x), z - (x - tG_t(x)) \rangle  \\
    & =    h(z) + \langle G_t(x), x - z             \rangle
                - t\|     G_t(x)                    \|_2^2
                + \langle \nabla g(x), x - z        \rangle
                + t\langle \nabla g(x), G_t(x)      \rangle \\
    & \leq h(z) + \langle G_t(x), x - z             \rangle
                - t\|     G_t(x)                    \|_2^2
                + g(z) - g(x)
                + t\langle \nabla g(x), G_t(x)      \rangle,
\end{align*}
by bilinearity of the inner product and convexity of $g$ (since
$\langle \nabla g, x - z \rangle \leq g(z) - g(x)$). Substituting this bound
for $h(x - tG_x(x))$ in (\ref{ineq_b}) and observing that several terms cancel
gives
\begin{align}
\notag
f(x^+)
 &  \leq h(z) + g(z) + \langle G_t(x), x - z \rangle
    -    \frac{t}{2} \|G_t(x)\|_2^2 \\
\label{ineq_d}
 &  \leq f(z) + \langle G_t(x), x - z \rangle - \frac{t}{2} \|G_t(x)\|_2^2
\end{align}
since $f = g + h$. \qed
 
%TODO
\item Plugging $z = x$ into (\ref{ineq_d}) gives
\begin{align*}
f(x^+)
    \leq f(x) + \langle G_t(x), x - x \rangle - \frac{t}{2} \|G_t(x)\|_2^2
    \leq f(x)
\end{align*}
(and the latter inequality is strict if and only if $G_t(x) \neq 0$), so that
generalized gradient descent does indeed decrease the criterion $f$ in each
iteration. Plugging $z = x^\star$ into (\ref{ineq_d}) gives
\begin{align*}
f(x^+)
    \leq f(x^\star) + \langle G_t(x), x - x^\star \rangle
    -    \frac{t}{2} \|G_t(x)\|_2^2.
\end{align*}
Substituting $G_t(x) = \frac{x - x^+}{t}$ and simplifying gives the desired
result:
\[f(x^+) \leq f(x^\star) + \frac{1}{2t}
    \left( \|x - x^\star\|_2^2 - \|x^+ - x^\star\|_2^2 \right). \qed\]
 
\item Since each $f(x^{(k)}) \leq f(x^{(k - 1)}$, by the result of part (e)
\begin{align*}
k(f(x^{(k)}) - f(x^\star))
    = \sum_{i = 1}^k f(x^{(k)}) - f(x^\star)
 &  \leq \sum_{i = 1}^k f(x^{(i)}) - f(x^\star) \\
 &  \leq \sum_{i = 1}^k \frac{1}{2t} \left(\|x^{(k - 1)} - x^\star\|_2^2
                                        - \|x^{(k)} - x^\star\|_2^2 \right) \\
 &  \leq \frac{\|x^{(0)} - x^\star\|_2^2}{2t},
\end{align*}
since the last sum telescopes. Dividing by $k$ gives the desired result:
\[f(x^{(k)}) - f(x^\star) \leq \frac{\|x^{(0)} - x^\star\|_2^2}{2tk}. \qed\]
 
\end{enumerate}
\end{document}

\section{ Accelerating matrix completion [15 points] (Adona)}

In this problem we will continue exploring the matrix completion algorithms of Q3, in particular how they combine with acceleration and backtracking, and how they behave on a realistic dataset. 

(a) [\textbf{7}] Implement the generalized gradient descent (soft-impute) algorithm of Q3(a) with acceleration. Run both the non-accelerated and accelerated versions for different $\lambda$s in $\Lambda = \text{logspace}(0,3,30)$, and plot the number of iterations the two algorithms took to converge for each value of $\lambda$, starting from $B=0$. As before, run the algorithms until either $\frac{|f_{k+1} - f_{k}|}{f_{k}} < 10^{-4}$ (where $f_k$ is the objective function value at $k^{th}$ iteration), or after a maximum of 500 iterations (whichever occurs first). Further, for fixed $\lambda = 10$, plot and compare the value of the objective at each iteration $\#$ with and without acceleration. What do you observe? Does acceleration help for this problem? 

(b) [\textbf{2}] Do the same as in part (a), but now using warm starts. Is there any difference in the comparison between no acceleration and acceleration?
%(b) [\textbf{2}] Consider the backtracking algorithm of Lecture 8, slide 13. Would backtracking further help in the case of matrix completion? Why or why not?

(c) [\textbf{6}] Next we will explore how matrix completion performs as an algorithm for image reconstruction. Download the image of Mona Lisa from \url{http://www.stat.cmu.edu/~ryantibs/convexopt/homeworks/mona_bw.jpg}. Construct a test image by randomly subsampling $50\%$ of the pixels, and setting the remaining pixels to zero. Run matrix completion with accelerated generalized gradient descent on the subsampled image for a few (10-15) different $\lambda$s in the range $10^{-2} - 10^2$. What do you observe? Show the original image, the subsampled image, and 3-4 reconstructions across the range of $\lambda$s. Which $\lambda$ returns the best results? Repeat this experiment at $20\%$ subsampling level, and show the best reconstructed image. What $\lambda$ did you choose in this case, and was it different from the best $\lambda$ at $50\%$ subsampling level?




\end{document}
