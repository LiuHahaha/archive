\documentclass[11pt]{article}

\usepackage{hyperref}
\usepackage{amsfonts,amsmath,amssymb,amsthm}
\usepackage{epsfig, graphics, graphicx}
\usepackage{latexsym}
\usepackage{fullpage}
\usepackage[parfill]{parskip}
%\usepackage{mysymbols}
\usepackage[tight]{subfigure}
\usepackage{hyperref}
\usepackage{enumerate,comment}

\newcommand{\figref}[1]{Fig.~\ref{#1}}
\newcommand{\R}{\mathbb{R}}

\def\prox{\mathrm{prox}}
\newcommand{\argmin}{\operatornamewithlimits{argmin}}   % argmin
\newcommand{\argmax}{\operatornamewithlimits{argmax}}   % argmax
\newcommand{\inv}{{^{-1}}}                              % inverse
\renewcommand{\qed}{\quad \ensuremath{\blacksquare}}
\newcommand{\sgn}{{\operatorname{sign}}}

\begin{document}
\twocolumn

\section{Mastery set [20 points] (Yifei)}

\textbf{A [5 points]}
Let $\|\cdot\|_1$ be a norm, let $\|\cdot\|_2$ be the
dual norm of $\|\cdot\|_1$, and let $\|\cdot\|_3$ be the dual norm of
$\|\cdot\|_2$. The conjugate of $\|\cdot\|_1$ is the indicator function
$I_{\{z : \|z\|_2 \leq 1\}}$ of the unit ball under $\|\cdot\|_2$. The
conjugate of $I_{\{z : \|z\|_2 \leq 1\}}$ is
\[\left(x \mapsto \max_{\|z\|_2 \leq 1}z^Tx \right) = \|\cdot\|_3.\]
Since every norm is convex and continuous (in finite dimensions),
$\|\cdot\|_3 = \|\cdot\|_1^{**} = \|\cdot\|_1$. \qed

\textbf{B [5 points]}

The KKT conditions give
\begin{align*}
0
 &  \in \partial \left( \frac{1}{2}\|x - y\|_2^2 + \lambda \|x\|_1 \right)  \\
 &  \in x - y + \lambda \left\{
        \begin{array}{ll}
            [-1,1]              & : \mbox{ if } x_i = 0      \\
            \{\sgn(x_i)\}   & : \mbox{ if } x_i \neq 0
        \end{array}
    \right..
\end{align*}

\textbf{C [5 points]} The dual program is
\begin{align*}
\min_{u \in \R^4} \quad &   u_1 + u_2         \\
\mbox{ such that } \quad
                        &   u_1 + u_2/3 - u_3   \geq 1, \\
                        &   u_1/2 + u_2  - u_4  \geq 1, \\
\mbox{ and } \quad
                        &   u_1, u_2 u_3, u_4   \geq 0.
\end{align*}
We have strong duality



Since we have strong duality, we change the optimization problem into an
equivalent feasibility problem by finding point where the dual is equal to the
primal; i.e., we add the constraint
\[u_1 + u_2 \leq x_1 + x_2\]
and find a feasible point over all $6$ primal and dual variables.

\textbf{D [5 points]} The Lagrangian of the dual is
\[L(u,v,x) = -f^*(-A^Tu - C^Tv) - b^Tu - d^Tv + x^Tu.\]
Thus, the dual of the dual is
\begin{align*}
\min_{x \in \R^n, x \leq 0} & \; \max_{u \in \R^m, v \in \R^n}   \\
                            &        -f^*(-A^Tu - C^Tv) - b^Tu - d^Tv + x^Tu \\
\end{align*}
\end{document}
