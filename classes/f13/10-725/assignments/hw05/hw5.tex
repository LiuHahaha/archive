\documentclass[12pt]{article}

\usepackage{hyperref}
\usepackage{amsfonts,amsmath,amssymb,amsthm}
\usepackage{epsfig, graphics, graphicx}
\usepackage{latexsym}
\usepackage{fullpage}
\usepackage[parfill]{parskip}
%\usepackage{mysymbols}
\usepackage[tight]{subfigure}
\usepackage{hyperref}
\usepackage{enumerate,comment}

\newcommand{\figref}[1]{Fig.~\ref{#1}}
\newcommand{\R}{\mathbb{R}}

\newenvironment{qpart}[1]%
{\begin{enumerate}\item[(#1)]}%
{\end{enumerate}}

\newenvironment{qpartp}[2]%
{\begin{enumerate}\item[(#1)] [#2 points] }%
{\end{enumerate}}

\title{Convex Optimization 10-725/36-725 \\Homework 5, due Nov 26}
\date{}

\begin{document}

\maketitle

{\bf Instructions}: 
\begin{itemize}
\item
You must complete Problems 1--3 and {\bf either
Problem 4 or Problem 5} (your choice between the two). 
\item When you
submit the homework, upload a single PDF (e.g., produced by LaTeX,
or scanned handwritten exercises) for the solution of each problem 
separately, to blackboard. You should your name at the top of each 
file, except for the first problem. {\bf Your solution to Problem 1
(mastery set) should appear completely anonymous to a reader.}
\end{itemize}

%% \bigskip
%% {\bf N.B.} A few problems may appear quite long, but really, there
%% is just a lot of text setting up and motivating the problem. This does not
%% mean more work for you! It's simply trying to give you an idea of where 
%% the problem is coming from, and why you'd want to solve it in the first 
%% place.

%%%%%%%%%%%%%%%%%%%%%%%%%%%%%%%%%%%%%%%%%%%%%%%%%%%%%%%%%%%%%%%%%
%%%%%%%%%%%%%%%%%%%%%%%%%%%%%%%%%%%%%%%%%%%%%%%%%%%%%%%%%%%%%%%%%

\documentclass[11pt]{article}

\usepackage{epsfig, graphics, graphicx}
\usepackage{latexsym}
\usepackage{fullpage}
\usepackage{hyperref}
\usepackage{amsmath,amssymb,enumerate,comment}

\newcommand{\qed}{\quad \ensuremath{\blacksquare}}
\newcommand{\figref}[1]{Fig.~\ref{#1}}
\newcommand{\R}{\mathbb{R}}
\newcommand{\D}{\mathcal{D}}
\newcommand{\prox}{\operatorname{prox}}
\newcommand{\conv}{\operatorname{conv}}
\newcommand{\op}{{\operatorname{op}}}

\begin{document}
\twocolumn
\section{Mastery set [25 points]\\(Aaditya)}
\paragraph{Lemma 1:} If $f : D \to \R$ is convex and $m \in \R$, then
$U_f(m) := \{ x \in D : f(x) \leq m \}$ is convex.

\paragraph{Proof:} For $x,y \in U_f(m), \theta \in [0,1]$, if
$z=\theta x+(1-\theta)y$, then
\[f(z) \leq \theta f(x) + (1 - \theta)f(y) \leq m,\] so
$z \in U_f(m)$. \qed

\paragraph{A [2 + 2]} If $S = \emptyset$, then $S$ is trivially convex.
Otherwise, for $x \in S$, $S = U_f(f(x))$ is convex, by Lemma 1. If $f$ is
strictly convex and $\exists x,y \in S$ with $x \neq y$, for
$\theta \in (0,1)$, $z = \theta x + (1 - \theta)y$,
\[f(z) < \theta f(x) + (1 - \theta)f(y) = f(x),\]
contradicting the fact that $x$ minimizes $f$.

\paragraph{B [2 + 2]} If $A$ is orthogonal, $A^TA = I$, whose only singular
value is $1$. For
\[A =
\begin{bmatrix}
    1 & 0 \\
    0 & 1
\end{bmatrix},
B =
\begin{bmatrix}
    0 & 1 \\
    1 & 0
\end{bmatrix},
\]
both $A$ and $B$ are orthogonal, but \fbox{$\tfrac12A + \tfrac12B$} is not
orthogonal.

\paragraph{C [2 + 2 + 2]} By Lemma 1, the unit ball $U_{\|\cdot\|}(1)$ is
convex. Since, by part B, all orthogonal matrices are in the unit ball, it
follows that any convex combination of orthogonal matrices is in the unit ball.
By the triangle inequality, for any norm $\|\cdot\|$,
\begin{align*}
                \|x\| & = \|x - y + y\| \leq \|x - y\| + \|y\|  \\
\mbox{ and }    \|y\| & = \|y - x + x\| \leq \|x - y\| + \|x\|,
\end{align*}
which implies
\[-\|x - y\| \leq \|x\| - \|y\| \leq \|x - y\|,\]
and so $|\|x\| - \|y\|| \leq \|x - y\|$ (this statement is precisely the
Reverse Triangle Inequality. Hence, any norm is Lipschitz with respect to
itself, with Lipschitz constant \fbox{$L = 1$.} \qed

\paragraph{D [5]} We already showed that the convex hull of orthogonal matrices
in contained in the unit ball of $\|\cdot\|_\op$.

If $A$ satisfies
$\|A\|_\op \leq 1$, then $A$ has a singular value decomposition
$A = U\Sigma V^T$, where $U$ and $V$ are orthogonal and $\Sigma$ is diagonal
and has entries in $[-1,1]$. Let $\D$ denote the set of diagonal matrices with
diagonal entries in $\{-1,1\}$. It is geometrically obvious that
$\conv\left(\{-1,1\}^n\right)~=~[-1,1]^n$, and hence $\Sigma \in \conv(\D)$.
Also, any $D \in \D$ is clearly orthogonal. Thus, since
$\Sigma = \sum_{D \in \D} \theta_D D$, where $\sum_{D \in \D} \theta_D = 1$,
\[A
    = U \left( \sum_{D \in \D} \theta_D D \right) V^T
    = \sum_{D \in \D} \theta_D UDV^T
\]
is in the convex hull of orthogonal matrices (as any product of orthogonal
matrices is itself orthogonal). \qed

\paragraph{E [4+2]} Rearranging the first-order definition of strong convexity,
\begin{align*}
f(y) - \langle \nabla f(x), y - x \rangle
    & \geq f(x) + \frac{\lambda}{2}\|y - x\|_2^2    \\
f(x) + \langle \nabla f(y), y - x \rangle
    & \geq f(y) + \frac{\lambda}{2}\|y - x\|_2^2.
\end{align*}
Adding these inequalities and cancelling terms,
\[\langle \nabla f(y) - \nabla f(x), y - x \rangle
    \geq \lambda\|y - x\|_2^2.\]
Then, by Cauchy-Schwarz,
\begin{align*}
\|\nabla f(y) - \nabla f(x)\|
 &  \geq \frac{\langle \nabla f(y) - \nabla f(x), y - x \rangle}{\|y - x\|_2}\\
 &  \geq + \lambda\|y - x\|_2. \qed
\end{align*}
If $\nabla f$ is Lipschitz with Lipschitz constant $L$, then
\[L\|y - x\|_2 \geq \|\nabla f(y) - \nabla f(x)\| \geq + \lambda\|y - x\|_2,\]
so \fbox{$L \geq \lambda$.}
\end{document}

\documentclass[11pt]{article}

\usepackage{hyperref}
\usepackage{amsfonts,amsmath,amssymb,amsthm}
\usepackage{epsfig, graphics, graphicx}
\usepackage{latexsym}
\usepackage{fullpage}
\usepackage[parfill]{parskip}
%\usepackage{mysymbols}
\usepackage[tight]{subfigure}
\usepackage{hyperref}
\usepackage{enumerate,comment}

\newcommand{\figref}[1]{Fig.~\ref{#1}}
\newcommand{\R}{\mathbb{R}}                 % real numbers
\newcommand{\Col}{\mathcal{C}}              % column space of a matrix
\newcommand{\Nul}{\mathcal{N}}              % null space of a matrix
\newcommand{\inv}{{^{-1}}}                  % inverse
\def\prox{\mathrm{prox}}                    % proximal operator
\newcommand{\argmin}{\operatornamewithlimits{argmin}}   % argmin
\newcommand{\argmax}{\operatornamewithlimits{argmax}}   % argmax

\begin{document}
Shashank Singh\footnote{sss1@andrew.cmu.edu}
\setcounter{section}{1}
\section{Projections and proximal operators [25 points] (Adona)}
\paragraph{Q1. [4+4]} 
\begin{enumerate}[a)]
\item For a symmetric matrix $A$ we want to find positive semidefinite matrix
$B$ minimizing the sum of the absolute values of the eigenvalues of $B - A$.
Clearly, if $A = U\Sigma U\inv$ is the singular value decomposition of $A$,
this is achieved by
\[B = U\Sigma^+U\inv.\]
where $\Sigma^+$ is the diagonal matrix with entries
$\Sigma^+_{ii} = \max\{0,\Sigma_{ii}\}$.

\item Let $x^* \in \R^n$ such that $Ax^* + b$ is the projection of $y$ onto
$\{Ax + b : x \in \R^n\}$. Then, $Ax^*$ is the projection of $y - b$ onto the
column space $\Col(A)$, so $(y - b) - Ax^*$ is orthogonal to every vector in
$\Col(A)$ and is hence in the null space $\Nul(A^T)$. Then,
\[A^T((y - b) - Ax^*) = 0, \quad \mbox{and so} \quad A^T(y - b) = A^TAx^*. \]

Since $A$ has full column rank, $A^TA$ is an invertible (square) matrix, so
\[(A^TA)\inv A^T(y - b) = x^*,\]
and hence the projection $p$ of $y$ onto $\{Ax + b : x \in \R^n\}$ is
\[p = Ax^* + b = \mbox{\fbox{$\displaystyle A(A^TA)\inv A^T(y - b) + b$.}}\]
\end{enumerate}

\paragraph{Q2. [5+7+5]}
\begin{enumerate}[a)]
\item Let
\[x^* := \argmin_{x \in \R^n} \frac{1}{2}\|x - z\|_2^2
                                        + \lambda\sum_{i = 1}^n (x_i)_+.\]
Clearly, if $z_i \leq 0$, then $x_i^* = z_i$. If $z_i > 0$, then
$x^*_i \geq 0$. If $x^*_i > 0$, we have
\[0
    = \frac{\partial}{\partial x^*_i}
        \frac12 \sum_{z_i > 0} (x^*_i - z_i)^2 + \lambda x^*_i
    = x^*_i - z_i + \lambda,
\]
so that $x^*_i = z_i - \lambda$ when this is positive. Thus, we have
\[\mbox{\fbox{$\displaystyle x_i^*
 =  \left\{
        \begin{array}{ll}
            z_i             & : \mbox{ if } z_i \leq 0              \\
            0               & : \mbox{ if } 0 < z_i \leq \lambda    \\
            z_i - \lambda   & : \mbox{ if } \lambda < z_i           \\
        \end{array}
    \right.$.}}
\]

\item Let
\[x^* := \argmin_{x \in \R^n} \frac{1}{2}\|x - z\|_2^2 + \lambda\|x\|_2\]
Clearly, $x^* = cz$, for some $c \in [0,1]$. Thus, $c = 0$ or
\[0
    = \frac{d}{dc} \frac{1}{2}\|(c - 1)z\|_2^2 + \lambda\|cz\|_2
    = (c - 1)\|z\|_2^2 + \lambda\|z\|_2,
\]
so $c = 1 - \frac{\lambda}{\|z\|_2}$. Thus,
\[\mbox{\fbox{$\displaystyle x^*
    = \left\{
        \begin{array}{ll}
            \left(1 - \frac{\lambda}{\|z\|_2}\right)z
                                        & \mbox{ if } \lambda \leq \|z\|_2 \\
            0   & \mbox{else}
        \end{array}
    \right.$.}}
\]

\item Let
\[x^* := \argmin_{x \in \R^n} \frac{1}{2}\|x - z\|_2^2 + \lambda\|x\|_\infty\]
and $\rho := \|x\|_\infty$. Clearly, by definitions of the $2$- and
$\infty$-norms, in each coordinate of $x^*$,
\[x_i^*
 =  \left\{
        \begin{array}{ll}
            \rho    & : \mbox{ if } \rho < z_i                  \\
            z_i     & : \mbox{ if } -\rho \leq z_i \leq \rho    \\
            -\rho   & : \mbox{ if } z_i < -\rho
        \end{array}
    \right..
\]
Then, we have
\[0 = \frac{d}{d\rho} \frac{1}{2}\|x^* - z\|_2^2 + \lambda\|x\|_\infty
    = \sum_{|z_i| > \rho} \rho - |z_i| + \lambda
    = -\|x^* - z\|_1 + \lambda,
\]
so $\|x^* - z\|_1 = \lambda$.
\end{enumerate}
\end{document}

\documentclass[11pt]{article}

\usepackage{hyperref}
\usepackage{amsfonts,amsmath,amssymb,amsthm}
\usepackage{epsfig, graphics, graphicx}
\usepackage{latexsym}
\usepackage{fullpage}
\usepackage[parfill]{parskip}
\usepackage[tight]{subfigure}
\usepackage{hyperref}
\usepackage{enumerate,comment}

\newcommand{\inv}{^{-1}}
\newcommand{\figref}[1]{Fig.~\ref{#1}}
\newcommand{\R}{\mathbb{R}}
\renewcommand{\P}{\mathbb{P}}
\newcommand{\Q}{\mathbb{Q}}
\newcommand{\X}{\mathcal{X}}
\renewcommand{\qed}{\quad \ensuremath{\blacksquare}}
\newcommand{\sgn}{{\operatorname{sign}}}
\newcommand{\KL}{{\operatorname{KL}}}
\def\hbeta{\hat{\beta}}
\def\hu{\hat{u}}

\begin{document}
Shashank Singh\footnote{sss1@andrew.cmu.edu}
\setcounter{section}{2}
\section{Binary sequences of piecewise constant expectation [30 points]}
\begin{enumerate}[(a)]
\item Let $m = n - 1$, and, for each $j \in \{1,\dots,m\}$ let
$d_j := (0,\dots,0,1,-1,0,\dots,0) \in \R^n$, with $d_{j,j} = 1$ and
$d_{j,j + 1} = -1$. Then, for $x_i = e_i$, (4) is clearly equivalent to

\[\min_{\beta \in \R^p} \sum_{i = 1}^n - t_i \cdot x_i^T\beta
    + \log\left(1 + \exp(x_i^T\beta) \right)
    + \lambda \sum_{j = 1}^m |d_j\beta|,
\]
which is of the form considered in Homework 3. $D \in \R^{m \times n}$ is the
matrix with $j^{th}$ row $d_j$. \qed

\item For notational convenience, we note that, if $H : [0,1] \to [0,\infty)$
is the entropy function
\[H(p) = -p\log p - (1 - p)\log(1 - p),\]
then $\displaystyle g(u) = \sum_{t = 1}^n H(y_t(D^Tu)_t)$. Since
$\displaystyle \frac{d}{dp} H(p) = -\log\left( \frac{p}{1 - p} \right)$, by the
Chain Rule,
\[ \nabla g(u)
    = -\sum_{t = 1}^n \log\left( \frac{y_t(D^Tu)_t}{1 - y_t(D^Tu)_t} \right)
        y_t \nabla (D^Tu)_t
    = -\sum_{t = 1}^n \log\left( \frac{y_t(D^Tu)_t}{1 - y_t(D^Tu)_t} \right)
        y_t d_t^T
    = Dc(u),
\]
where, for $t \in \{1,\dots,n\}$,
\[(c(u))_t
    = - \log\left( \frac{y_t(D^Tu)_t}{1 - y_t(D^Tu)_t} \right) y_t.
\]
Since $\displaystyle \frac{d}{dp} \log \left( \frac{p}{1 - p} \right)
    = \frac{1}{p(1 - p)}$,
\[\nabla(c(u))_t
    = -\frac{y_td_t^T}{y_t(D^Tu)_t(1 - y_t(D^Tu)_t)}
    = -\frac{d_t^T}{(D^Tu)_t(1 - y_t(D^Tu)_t)},
\]
so $\nabla c(u) = W(u)D^T$, where $W(u)$ is the diagonal matrix with
\[W_{t,t}(u) = -\frac{1}{(D^Tu)_t(1 - y_t(D^Tu)_t)},\]
and hence $\nabla^2 g(u) = D\nabla c(u) = DW(u)D^T$.

The log barrier function is
\[\phi(u)
    = \sum_{t = 1}^n \log(y_t(D^Tu)_t) + \log(1 - y_t(D^Tu)_t)
    + \sum_{i = 1}^m \log(\lambda - u_i) + \log(u_i + \lambda).
\]
Thus, for each $i \in \{1,\dots,m\}$,
\begin{align*}
(\nabla(\phi(u))_i
 &  = \frac{1}{u_i - \lambda} + \frac{1}{u_i + \lambda}
    + \sum_{t = 1}^n \frac{2y_t(D^Tu)_t - 1}{y_t(D^Tu)_t(y_t(D^Tu)_t - 1)}
    (y_td_t^T)  \\
 &  = \frac{2u_i}{u_i^2  - \lambda^2}
    + \sum_{t = 1}^n \frac{2y_t(D^Tu)_t - 1}{(D^Tu)_t(y_t(D^Tu)_t - 1)}
    d_t^T  \\
\end{align*}
Thus, $\nabla\phi(u) = a(u) + Db(u)$, where for each $i \in \{1,\dots,m\}$,
\[(a(u))_i = \frac{2u_i}{u_i^2  - \lambda^2}
    \quad \mbox{ and } \quad
  (b(u))_i = \sum_{t = 1}^n \frac{2y_t(D^Tu)_t - 1}{(D^Tu)_t(y_t(D^Tu)_t - 1)}.
\]
Since
$\displaystyle (\nabla (a(u))_i)_i = -\frac{2(u_i^2 + 1)}{(u_i^2 - 1)^2}$,
\[\nabla^2 \phi(u)
    = U(u) + DV(u)D^T,
\]
where $U(u)$ and $W(u)$ are diagonal matrices with
\[U_{i,i}(u) = -\frac{2(u_i^2 + 1)}{(u_i^2 - 1)^2}
    \quad \mbox{ and } \quad
  V_{t,t}(u) = \frac{-2(y_t(D^T)_t)^2 + 2y_t(D^T)_t - 1}
                            {(D^Tu)_t^2(y_t(D^Tu)_t - 1)^2}.
\]
The Newton step is
\begin{align*}
      \left[ \nabla^2 (\tau g(u) + \phi(u)) \right]\inv
 &                                      \nabla (\tau g(u) + \phi(u))    \\
 &  = \left[ D(\tau W(u) + V(u))D^T + U(u) \right]\inv
                                        \nabla (\tau D(c(u) + b(u)) + a(u)).
\end{align*}

\item See attached code.

%TODO
\item

\end{enumerate}

\end{document}

\documentclass[11pt]{article}

\usepackage{hyperref}
\usepackage{amsfonts,amsmath,amssymb,amsthm}
\usepackage{epsfig, graphics, graphicx}
\usepackage{latexsym}
\usepackage{fullpage}
\usepackage[parfill]{parskip}
\usepackage[tight]{subfigure}
\usepackage{hyperref}
\usepackage{enumerate,comment}

\newcommand{\figref}[1]{Fig.~\ref{#1}}
\newcommand{\R}{\mathbb{R}}
\renewcommand{\qed}{\ensuremath{\quad \blacksquare}}

\begin{document}
Shashank Singh\footnote{sss1@andrew.cmu.edu}
\setcounter{section}{3}
\section{Regularized logistic regression dual [25 points] (Yifei)}
\begin{enumerate}[(a)]
\item To simplify notation, define $b_i := x_i^T\beta$. For each
$i \in \{1,\dots,n\}$, by the choice of $y_i$,
\begin{align*}
\log\left( 1 + \exp(-y_ib_i) \right)
 &  = \log\left(\frac{\exp(t_ib_i) + \exp((1 - t_i)b_i)}{\exp(t_ib_i)}\right)\\
 &  = -\left(t_ib_i - \log\left(\exp(t_ib_i) + \exp((1 - t_i)b_i)\right)\right)
\end{align*}
Now observe that, if $t_i = 0$, then $\exp(t_ib_i) = 1$ and
$\exp((1 - t_i) b_i) = \exp(b_i)$, while, if $t_i = 1$, then
$\exp(t_ib_i) = \exp(b_i)$ and $\exp((1 - t_i)b_i) = 1$. In either case, we
have
\begin{equation}
\label{eq:fit_term}
\log\left( 1 + \exp(-y_ib_i) \right)
    = -\left( t_ib_i - \log\left( 1 + \exp(b_i) \right) \right).
\end{equation}
Also, by definition of the $1$-norm, clearly
\begin{equation}
\label{eq:reg_term}
\|D\beta\|_1 = \sum_{j = 1}^m |(D\beta)_j| = \sum_{j = 1}^m |d_j^T\beta|,
\end{equation}
and the desired equality follows from (\ref{eq:reg_term}) and
(\ref{eq:fit_term}). \qed

\item First a quick derivation:

{\bf Lemma 1:} Define $f : \R \to \R$ by $f(x) = \log(1 + \exp(x))$,
$\forall x \in \R$. Then, the conjugate of $f$ is
\[f^*(y) = y\log(y) + (1 - y)\log(1 - y), \quad \forall y \in [0,1].\]
\emph{Proof:} Note that, if
\[0
    = \frac{d}{dx} yx - \log(1 + \exp(x))
    = y - \frac{\exp(x)}{1 + \exp(x)}
    = y - \frac{1}{\exp(-x) + 1},
\]
and solving for $x$ gives $x = \log \left( \frac{y}{1 - y} \right)$. Thus,
\begin{align*}
f^*(y)
 &  = \min_{x \in \R} yx - \log(1 + \exp(x))    \\
 &  = y\log\left( \frac{y}{1 - y} \right) - \log\left(1 + \frac{y}{1 - y}\right)    \\
 &  = y\left(\log(y) - \log(1 - y)\right) + \log(1 - y)
    = y\log(y) + (1 - y)\log(1 - y). \quad \square
\end{align*}
It follows from the relationships between duals, conjugates, and norms (slides
15 and 16 of Lecture 14) that the dual of the given problem is
\[\max_{\alpha\in\R^m} -\sum_{i = 1}^n
    (y_i(D^T\alpha)_i)\log(y_i(D^T\alpha)_i)
    + (1 - y_i(D^T\alpha)_i)\log(1 - y_i(D^T\alpha)_i),\]
such that $\|\alpha\|_\infty \leq \lambda,0 \leq y_i(D^T\alpha)_i \leq 1$
(since $(D\beta)^T\alpha = \beta^T(D^T\alpha)$). \qed

\item Since the regularization term of the original problem is not generally
smooth, we could only use the subgradient method or generalized gradient
descent. The dual, on the other hand, is smooth, but has inequality
constraints, so we could use gradient descent. In neither case could we use
Newton's or quasi-Newton methods, or conjugate gradient descent.

\item As usual, the solution to the dual problem lower bounds the solution to
the primal problem. Didn't have time to finish this.

\end{enumerate}
\end{document}

{\bf Part B} [10 points] 

Recall that a zero-mean, $P$-dimensional Gaussian distribution is defined by its covariance matrix $\Sigma$. So `learning' such a distribution from some iid samples $X = \{ X^{(1)},\ldots, X^{(N)} \}$ amounts to an estimate $\widehat{\Sigma}$ of $\Sigma$. Since we have an i.i.d. sample from a known family of distributions, a natural approach is to pick the $\widehat{\Sigma}$ which maximizes the likelihood of the sample. However, if $N < P$, the estimate is not well-defined. Further, even if $N \geq P$ but $N$ and $P$ are of comparable size, the maximum likelihood estimate can have high variability, leading to bad predictive performance. Since $P$ is big, we are inclined to restrict attention to sparse distributions, for some appropriate notion of sparsity. A tempting, but unrealistic, kind of sparsity is independence of a large number of the features $i$ and $j$: $\Sigma_{ij} = E_{X \sim N(0,\Sigma)}(X_i X_j) = 0$. A more realistic kind of sparsity is conditional independence of features $i$ and $j$ given the other features. Since the distribution is Gaussian, this happens when $\Sigma^{-1}_{ij} = 0$. Since our sparsity belief/assumption concerns $\Sigma^{-1}$, let's orient our notation around that, starting with the log-likelihood function, which can be shown to be:
$$\ell(K) = \textrm{log det}(K) - \textrm{tr}(SK)$$
where $K$ is a symmetric positive semidefinite $P \times P$ matrix meant to estimate $\Sigma^{-1}$, $S$ is the empirical covariance matrix $S = \frac{1}{N-1} \sum_{i=1}^{N}(X_i - \bar{X})(X_i - \bar{X})^T$ and the sample mean $\bar{X} = \frac{1}{N}\sum_{i=1}^N X_i$.
The maximizer of $\ell(K)$ is generally ill-defined and non-sparse, and so one considers the following $\ell_1$ penalized estimator (and the problem is lovingly called the graphical lasso):
$$
\min_{K \succ 0} -\log \det (K) + Tr(SK) + \lambda \sum_{i \neq j} |K_{ij}|
$$

\begin{enumerate}
\item [3 points] Write down the subgradient method's update equations.

\item [4 points] Write down the proximal gradient update equations.

\item [3 points] Can you accelerate this? If yes, write down the update equations. If no, why not?
\end{enumerate}
\section{ Accelerating matrix completion [15 points] (Adona)}

In this problem we will continue exploring the matrix completion algorithms of Q3, in particular how they combine with acceleration and backtracking, and how they behave on a realistic dataset. 

(a) [\textbf{7}] Implement the generalized gradient descent (soft-impute) algorithm of Q3(a) with acceleration. Run both the non-accelerated and accelerated versions for different $\lambda$s in $\Lambda = \text{logspace}(0,3,30)$, and plot the number of iterations the two algorithms took to converge for each value of $\lambda$, starting from $B=0$. As before, run the algorithms until either $\frac{|f_{k+1} - f_{k}|}{f_{k}} < 10^{-4}$ (where $f_k$ is the objective function value at $k^{th}$ iteration), or after a maximum of 500 iterations (whichever occurs first). Further, for fixed $\lambda = 10$, plot and compare the value of the objective at each iteration $\#$ with and without acceleration. What do you observe? Does acceleration help for this problem? 

(b) [\textbf{2}] Do the same as in part (a), but now using warm starts. Is there any difference in the comparison between no acceleration and acceleration?
%(b) [\textbf{2}] Consider the backtracking algorithm of Lecture 8, slide 13. Would backtracking further help in the case of matrix completion? Why or why not?

(c) [\textbf{6}] Next we will explore how matrix completion performs as an algorithm for image reconstruction. Download the image of Mona Lisa from \url{http://www.stat.cmu.edu/~ryantibs/convexopt/homeworks/mona_bw.jpg}. Construct a test image by randomly subsampling $50\%$ of the pixels, and setting the remaining pixels to zero. Run matrix completion with accelerated generalized gradient descent on the subsampled image for a few (10-15) different $\lambda$s in the range $10^{-2} - 10^2$. What do you observe? Show the original image, the subsampled image, and 3-4 reconstructions across the range of $\lambda$s. Which $\lambda$ returns the best results? Repeat this experiment at $20\%$ subsampling level, and show the best reconstructed image. What $\lambda$ did you choose in this case, and was it different from the best $\lambda$ at $50\%$ subsampling level?




\end{document}
