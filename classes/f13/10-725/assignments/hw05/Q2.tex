\documentclass[11pt]{article}

\usepackage{hyperref}
\usepackage{amsfonts,amsmath,amssymb,amsthm}
\usepackage{epsfig, graphics, graphicx}
\usepackage{latexsym}
\usepackage{fullpage}
\usepackage[parfill]{parskip}
\usepackage[tight]{subfigure}
\usepackage{hyperref}
\usepackage{enumerate,comment}

\newcommand{\inv}{^{-1}}
\newcommand{\figref}[1]{Fig.~\ref{#1}}
\newcommand{\R}{\mathbb{R}}
\renewcommand{\P}{\mathbb{P}}
\newcommand{\Q}{\mathbb{Q}}
\newcommand{\X}{\mathcal{X}}
\renewcommand{\qed}{\quad \ensuremath{\blacksquare}}
\newcommand{\sgn}{{\operatorname{sign}}}
\newcommand{\KL}{{\operatorname{KL}}}
\newcommand{\wx}{{\widetilde{x}}}

\begin{document}
Shashank Singh\footnote{sss1@andrew.cmu.edu}
\setcounter{section}{1}
\section{PSD Matrices (Adona) [25 points]}
\textbf{Part A [15 points]: Basic properties of PSD matrices} 
\begin{enumerate}
\item Suppose $v \in \R^{|I|}$. Let $k_1 < \dots < k_{|I|}$ denote the elements
of $I$. Define $v_I \in \R^n$ such that each $(v_I)_{k_i} = v_i$ and
$(v_I)_k = 0$, $\forall k \notin I$. Since $X \succeq 0$,
\[v^T X_I v = v_I^T X v_I \geq 0,\]
and hence $X_I \succeq 0$. \qed

\item If $I = \{i,j\}$, then, since $X_I \succeq 0$,
\[0 \leq \det X_I = X_{ii}X_{jj} - X_{ij}^2,\]
and hence $X_{ij}^2 \leq X_{ii}X_{jj}$.

If $X_{ii} = 0$, then each $0 \leq X_{ji}^2, X_{ij}^2 \leq X_{ii}X_{jj} = 0$,
and so $X_{ji} = X_{ij} = 0$. \qed

\item If $X = Q \Lambda Q\inv$, where $\Lambda$ is a diagonal matrix of
eigenvalues of $X$, then
\[\det X
    = \det (Q \Lambda Q\inv)
    = \det Q \det \Lambda \det(Q\inv)
    = \det Q \det \Lambda \det(Q)\inv
    = \det \Lambda.
\]
Since the determinant of a diagonal matrix is trivially the product of its
diagonal elements, $\det X$ is the product of the eigenvalues of $X$. \qed

\item {\bf Lemma} If $X = Y^TZY$ for some invertible matrix $Y$, then
$X \succeq 0$ if and only if $Z \succeq 0$.

\emph{Proof:} If $X \succeq 0$, then, $\forall v \in \R^n$,
\[0
    \leq (Y\inv v)^TXY\inv v
    = (Y\inv v)^TY^TZYY\inv v
    = v^TZv,
\]
and so $Z \succeq 0$. The converse follows, since $Z = (Y\inv)^TXY$.
\quad $\square$

It is easily checked that $X = Y^TZY$, where
\[Y =
\begin{bmatrix}
    I   &   A\inv B \\
    0   &   I
\end{bmatrix}
\quad \mbox{ and } \quad
Z =
\begin{bmatrix}
    A   &   0   \\
    0   &   C - B^T A\inv B
\end{bmatrix}.
\]
Since $Y$ is triangular without zeros on its diagonal, $Y$ is invertible. Thus,
by the lemma, it suffices to show that $Z \succeq 0$ if and only if
$A \succeq 0$ and $C - B^T A\inv B \succeq 0$. One implication is immediate
from Part 1. On the other hand, it is clear that any eigenvalue of either $A$
or $C - B^T A\inv B$ is an eigenvalue of $Z$, and hence, if
$A, C - B^T A\inv B \succeq 0$, then $Z$ has non-negative eigenvalues and so
$Z \succeq 0$. \qed
\end{enumerate}

\newpage
\textbf{Part B [10 points]: Formulating problems as SDPs}
\begin{enumerate}
\item Let $X \in \R^{n \times n}$ such that each $X_{uv} = x_u^Tx_v$.
Since \[\|x_u - x_v\|^2 = x_u^Tx_u - 2x_u^Tx_v + x_v^Tx_v,\]
the objective function is just
\[\frac{1}{4} X \cdot W\]
where $W_{uv}$ is the weight of $(u,v)$ if $(u,v) \in \mathcal{E}$ and
$W_{uv} = 0$ otherwise.

Each constraint $\|x_u\|^2 = 1$ is simply
\[X \cdot A_u = 1,\]
where $\left( A_u \right)_{uu} = 1$ and all other coordinates of $A_u$ are
zero.

Since each $\|x_u\|^2 = 1$, the constraint $\|x_{s_i} - x_{t_i}\|^2 = 4$ for
$(s_i,t_i) \in T$ is equivalent to $x_{s_i} = - x_{t_i}$, and so it can be
written
\[X \cdot A_{(s_i,t_i)} = 0,\]
where $A_{(s_i,t_i)}$ has $1$ in indices $(s_i,s_i)$ and $(s_i,t_i)$.

Didn't have time to finish this part.

\item Didn't have time to finish this part.

\end{enumerate}
\end{document}
