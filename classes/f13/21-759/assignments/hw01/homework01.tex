\documentclass[11pt]{article}
\usepackage{enumerate}
\usepackage{fullpage}
\usepackage{fancyhdr}
\usepackage{amsmath, amsfonts, amsthm, amssymb}
\usepackage{color}
\setlength{\parindent}{0pt}
\setlength{\parskip}{5pt plus 1pt}
\pagestyle{empty}

\def\indented#1{\list{}{}\item[]}
\let\indented=\endlist

\newcounter{questionCounter}
\newcounter{partCounter}[questionCounter]
\newenvironment{question}[2][\arabic{questionCounter}]{%
    \setcounter{partCounter}{0}%
    \vspace{.25in} \hrule \vspace{0.5em}%
        \noindent{\bf #2}%
    \vspace{0.8em} \hrule \vspace{.10in}%
    \addtocounter{questionCounter}{1}%
}{}
\renewenvironment{part}[1][\alph{partCounter}]{%
    \addtocounter{partCounter}{1}%
    \vspace{.10in}%
    \begin{indented}%
       {\bf (#1)} %
}{\end{indented}}

%%%%%%%%%%%%%%%%%%%%%%%HEADER%%%%%%%%%%%%%%%%%%%%%%%%%%%%%%
\newcommand{\myname}{Shashank Singh}
\newcommand{\myandrew}{sss1@andrew.cmu.edu}
\newcommand{\myclass}{21-759 Differential Geometry}
\newcommand{\myhwnum}{1}
\newcommand{\duedate}{Friday, September 13, 2013}
%%%%%%%%%%%%%%%%%%%%%%%%%%%%%%%%%%%%%%%%%%%%%%%%%%%%%%%%%%%

%%%%%%%%%%%%%%%%%%%%CONTENT MACROS%%%%%%%%%%%%%%%%%%%%%%%%%
\renewcommand{\qed}{\quad \ensuremath{\blacksquare}}    % QED square
\newcommand{\inv}{^{-1}}                % inverse
\newcommand{\bv}{\mathbf{v}}
\newcommand{\bx}{\mathbf{x}}
\newcommand{\by}{\mathbf{y}}
\newcommand{\bff}{\mathbf{f}}
\newcommand{\bzero}{\mathbf{0}}
\newcommand{\sminus}{\backslash}        % set difference
\renewcommand{\dim}{\operatorname{dim}} % dimension of a vector space
\renewcommand{\det}{\operatorname{det}} % determinant of a matrix
\newcommand{\sgn}{\operatorname{sign}}  % determinant of a matrix
\newcommand{\rnk}{\operatorname{rank}}  % rank of a matrix
\newcommand{\N}{\mathbb{N}}             % natural numbers
\newcommand{\Z}{\mathbb{Z}}             % integers
\newcommand{\Q}{\mathbb{Q}}             % rational numbers
\newcommand{\R}{\mathbb{R}}             % real numbers
\newcommand{\M}{\mathcal{M}}            % manifold M
\newcommand{\D}{\mathcal{D}}            % \smooth real functions on a manifold
\newcommand{\B}{\mathcal{B}}            % basis of a topology
\newcommand{\pow}{\mathcal{P}}          % power set
\newcommand{\e}{\varepsilon}            % \varepsilon
%%%%%%%%%%%%%%%%%%%%%%%%%%%%%%%%%%%%%%%%%%%%%%%%%%%%%%%%%%%

\begin{document}
\thispagestyle{plain}

{\Large Homework \myhwnum} \\
\myclass \\
Name: \myname\footnote{\myandrew} \\
Due: \duedate

I would be willing to present solutions to problems 1,3,4, and 5.

\begin{question}{Problem 1}
\begin{enumerate}[(i)]
\item As a subspace of $\R^n$, $\M_\alpha$ is second countable and Hausdorff.

Let $p \in \M_\alpha$. Since $\alpha$ is regular, without loss of generality,
$(DF)_1 \neq 0$. By the Implicit Function Theorem, there exist open
neighborhoods $U \subseteq \R^{n - 1}$ and $V \subseteq \R$ of
$(p_2,\dots,p_n)$ and $p_1$, respectively, and a smooth function $g : U \to V$
such that $\{(q,g(q)) : q \in U\} = (U \times V) \cap F\inv(\alpha)$. The
function $\phi(x) := (x,g(x))$ on $U$ is clearly a diffeomorphism. Therefore,
$(U, \phi)$ is a coordinate chart for $\M_\alpha$ in the neighborhood
$U \times V$ of $p$, so $\M_\alpha$ is a manifold (of dimension $n-1$). \qed


\item As shown in part (i),  $M_\alpha$ admits a differentiable structure
of the form $\{(U_\gamma,\phi_\gamma)\}$, where $\phi(x) = (x,g_\gamma(x))$ on
$U_\gamma$. Thus, if $(U_\gamma,\phi_\gamma)$ and $(U_\beta,\phi_\beta)$ are
charts with
$W := \phi_\gamma(U_\gamma) \cap \phi_\beta(U_\beta) \neq \emptyset$, then, on
$W$, $\phi_\beta\inv\phi_\gamma \equiv id_n$, the identity on $\R^n$, and hence
\[\det(D(\phi_\beta\inv\phi_\gamma)) = \det(id_n) = 1 > 0,\] so that
$\M_\alpha$ is orientable. \qed
\end{enumerate}
\end{question}

\begin{question}{Problem 2}
If $p_1 = p_2$ and $v_1 \neq v_2$ then, since $T_{p_1}\M$ is homeomorphic to $\R^n$
(which is Hausdorff), there are open sets $V_1,V_2 \subset T_{p_1}\M$
separating $v_1$ and $v_2$, and so the open sets $\M \times V_1$ and $\M \times
V_2$ separate $(p_1,v_1)$ and $(p_2,v_2)$. If $p_1 \neq p_2$, then, since $\M$
is Hausdorff, there are open sets $U_1,U_2 \subset \M$ separating $p_1$ and
$p_2$ and, and so the open sets $U_1 \times T_{p_1}\M$ and
$U_2 \times T_{p_2}\M$ separate $(p_1,v_1)$ and $(p_2,v_2)$. Thus, $T\M$ is
Hausdorff. Since $\M$ and $T_p\M$ have a countable bases $\B$ and $\B_p$,
set $\{U \times V : U \in \B, V \in \B_p, p \in U\}$ is a countable base for
$T\M$.

$\forall p \in \M$, since $T_p\M$ is a vector space for which the vectors
$\frac{\partial}{\partial x_1},\dots,\frac{\partial}{\partial x_n}$ form a
basis, there is a unique linear mapping $D_p : T_p\M \to \R^n$ such that,
$\forall \in T_p\M$, $v = \sum_{i = 1}^n D_i(v) \frac{\partial}{\partial x_i}$.
Then, the function $(p,v) \mapsto (\phi\inv(p),D_p(v))$ is a diffeomorphism
between $\M$ and $\R^{2d}$.

We first compute
\[v\left[\left(\phi\inv\right)_i\right]
 = \sum_{j = 1}^n a^j\frac{\partial}{\partial x_j}
    (\phi\inv \circ \phi)_i (\phi\inv(p))
 = \sum_{j = 1}^n a^j\frac{\partial}{\partial x_j}
    x_i (\phi\inv(p))
 = \sum_{j = 1}^n a^j \delta_i^j
 = a^i.
 \]
We now compute this in an alternate manner:
\[v\left[\left(\phi\inv\right)_i\right]
 = \sum_{j = 1}^n b^j\frac{\partial}{\partial y_j}
    (\phi\inv \circ \psi)_i (\psi\inv(p)).
\]
The result can be summarized in matrix notation:
\[\mbox{\fbox{$a
    = \left[D (\phi\inv \circ \psi)\right] \bigg|_{\psi\inv(p)} b$.}}\]
\end{question}

\begin{question}{Problem 3}
Since $x_i = \left(\phi\inv\right)_i$,
$df = \sum_{i = 1}^n \alpha_i d\left(\phi\inv\right)_i$, and so
\[\frac{\partial}{\partial x_j}[f]
 = \frac{\partial}{\partial x_j}
        \sum_{i = 1}^n \alpha_i \left(\left(\phi\inv\right)_i \circ \phi\right)
            \left( \phi\inv(p) \right)
 = \sum_{i = 1}^n \alpha_i \frac{\partial}{\partial x_j} x_i
            \left( \phi\inv(p) \right)
 = \sum_{i = 1}^n \alpha_i \delta_i^j
 = \alpha_j.
\]
Also, since $y_i = \left(\psi\inv\right)_i$,
\[\frac{\partial}{\partial x_j}[f]
 = \frac{\partial}{\partial x_j}
        \sum_{i = 1}^n \beta_i \left(\left(\psi\inv\right)_i \circ \phi\right)
            \left( \phi\inv(p) \right)
 = \sum_{i = 1}^n \beta_i
        \frac{\partial}{\partial x_j} \left(\psi\inv \circ \phi\right)_i
            \left( \phi\inv(p) \right).
\]
The result can be summarized in matrix notation:
\[\mbox{\fbox{$\alpha
    = \left[D (\psi\inv \circ \phi)\right] \bigg|_{\phi\inv(p)} \beta$.}}\]
\end{question}

\begin{question}{Problem 4}
Suppose that, for some $\alpha \in \R^n$,
$\displaystyle df := \sum_{i = 1}^n \alpha_i dx_i = 0$. Then,
\[0
 = df\left( \sum_{j = 1}^n \alpha_j \frac{\partial}{\partial x_j} \right)
 = \sum_{j = 1}^n \alpha_j \frac{\partial}{\partial x_j}
        \sum_{i = 1}^n \alpha_i x_i
 = \sum_{j = 1}^n \sum_{i = 1}^n \alpha_j \alpha_i \delta_i^j
 = \sum_{i = 1}^n \alpha_i^2,
\]
and so $\alpha = 0$. Thus, $dx_1,\dots,dx_n$ are linearly independent. As the
dual of an $n$-dimensional vector space, $T_p\M^*$ has dimension $n$, and so
the $n$ independent vectors $dx_1,\dots,dx_n$ span $T_p\M^*$. \qed
\end{question}

\begin{question}{Problem 5}
Without loss of generality, we may assume that, for every chart
$(U_\alpha,\phi_\alpha)$ of $\M$, $U_\alpha$ is connected, since, otherwise, we
may replace the chart by identical charts on the connected components of
$U_\alpha$.

Let $p \in \M$, and let $(U_\alpha,\phi_\alpha)$ be a chart at $p$. Since
$df = 0$, for $v_i = \frac{\partial}{\partial x_i}$ and $g$ the identity on
$\R$
\[0
 = df_p(v_i)[g]
 = v_i[g \circ f]
 = v_i[f]
 = \frac{\partial}{\partial x_i} (f \circ \phi_\alpha)(\phi_\alpha\inv(p)),
\]
And so
$[D(f \circ \phi_\alpha)] \bigg|_{\phi_\alpha\inv(p)} \hspace{-0.8cm}= 0$. By a
theorem in multivariable calculus, since $U_\alpha$ is connected,
$f \circ \phi_\alpha \equiv C_\alpha$ on $U_\alpha$, for some constant $C_p \in \R$. Since $\phi$
is a surjection, it follows that $f \equiv C_p$ on $\phi(U)$.

Now let $U := \cup \{\phi_\beta(U_\beta) : C_\beta = C_\alpha\}$,
and let $V := \cup \{\phi_\beta(U_\beta) : C_\beta \neq C_\alpha\}$. As unions
of open sets, both $U$ and $V$ are open, and clearly $\M = U \cup V$ and
$U \cap V = \emptyset$. Thus, since $\M$ is connected and $U \neq \emptyset$,
$V = \emptyset$, and so $f$ is constant. \qed
\end{question}

\begin{question}{Problem 6}
To show that $DF$ is injective (so that $F$ is an immersion), it suffices to
show that
\[DF =
    \begin{bmatrix}
        2x  & y & z & 0 \\
        -2y & x & 0 & z \\
        0   & 0 & x & y
    \end{bmatrix}
\]
satisfies $\rnk(DF) \geq 2$, $\forall (x,y,z)$ with $x^2 + y^2 + z^2 = 1$. If
$x$ is non-zero, then the first and third columns are independent. If $y$ is
non-zero, then the first and fourth columns are independent. If $x = y = 0$,
then $z = \pm1$, and so the last two columns are independent. Thus,
$\rnk(DF) \geq 2$.

We now show $F$ is injective.
If one of $(x_1,y_1,z_1)$ is zero, then, checking several cases and using the
fact that $([0],[0],[0]) \notin P^2$, one can verify that
$(x_1,y_1,z_1) = (\pm x_2,\pm y_2,\pm z_2)$. If all values are nonzero, then
\[\frac{x_1}{x_2} = \frac{y_2}{y_1} = \frac{z_1}{z_2} = \frac{x_2}{x_1},\]
so that $x_1^2 = x_2^2$, and hence $x_1 \sim x_2$ (similarly, $y_1 \sim y_2$
and $z_1 \sim z_2$). It follows that $F$ is injective, and thus $F$ is a
diffeomorphism (it follows from the above work that $F\inv$ is differentiable).
\end{question}

\begin{question}{Problem 7}
\begin{enumerate}[(i)]
\item Suppose the orientation induced on $\M$ by $\{(U_\alpha,\phi_\alpha)\}$
is preserved by all $g \in G$. Let $\{(V_\alpha,\pi_\alpha \circ x_\alpha)\}$
be the quotient differentiable structure on $\M/G$. Then, if
$\pi_\alpha(x_\alpha(V_\alpha)) \cap \pi_\beta(x_\beta(V_\beta))
\neq \emptyset$, then, since $\pi_\beta\inv\pi_\alpha = \Phi_g$, for some
$g \in G$, and $\Phi_g$ preserves orientation on $\M$,
\begin{align*}
\sgn(\det(D((\pi_\beta \circ x_\beta)\inv \circ \pi_\alpha \circ x_\alpha)))
 & = \sgn(\det(D(x_\beta\inv \circ \Phi_g \circ x_\alpha)))   \\
 & = \sgn(\det(D(x_\beta\inv \circ x_\alpha))) = 1,
\end{align*}
and hence $\M/G$ is orientable.

\item Let $G$ be the group $(\{-1,1\},\cdot)$. As discussed in class,
$G \times S^n \to S^n$ is a properly discontinuous action. Since $S^n$ is
orientable, fix an oriented differentiable structure
$\{(U_\alpha,\phi_\alpha)\}$.

Since $\Phi_{-1}$ is the restriction of the antipodal mapping $x \mapsto -x$ on
$\R^{n + 1}$ to $S^n$ and $\phi_\alpha$ and $\phi_\beta$ share orientation,
\[\det(D(\phi_\beta\inv \circ \Phi_{-1} \circ \phi_\alpha))
    = \det(-D (\phi_\beta\inv \circ \phi_\alpha))
    = (-1)^{n + 1} \det(D \phi_\beta\inv \circ \phi_\alpha)),
\]
has sign $(-1)^{n + 1}$. Thus, $\Phi_{-1}$ is orientation preserving if and
only if $n$ is odd, and so, by the result of part (i), $P^n = S^n/G$ is
orientable if and only if $n$ is odd. \qed
\end{enumerate}
\end{question}

\begin{question}{Problem 8}
Suppose $\exists \gamma_0 \in C^\infty([0,t_0),\M)$ with $\gamma_0(0) = p$.
Consider an increasing sequence $t_n \in [0,t_0)$ converging to $t_0$. Since
$\M$ is compact and $\gamma_0 \in C^\infty$, the sequence $\gamma_0(t_n)$ has a
limit $p \in \M$. Thus, the solution $\gamma_0$ can be extended to a solution
$\gamma_1 \in C^\infty([0,t_0],\M)$ (i.e., $\gamma_0$ is not right maximal).

Similarly, any solution on a left-open interval can be left-extended. It
follows by an extension theorem from ODE that there is a solution for all
$t \in \R$.
\end{question}
\end{document}
