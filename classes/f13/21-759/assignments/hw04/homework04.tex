\documentclass[11pt]{article}
\usepackage{enumerate}
\usepackage{fullpage}
\usepackage{fancyhdr}
\usepackage{amsmath, amsfonts, amsthm, amssymb}
\usepackage{color}
\usepackage{textcomp}
\setlength{\parindent}{0pt}
\setlength{\parskip}{5pt plus 1pt}
\pagestyle{empty}

\def\indented#1{\list{}{}\item[]}
\let\indented=\endlist

\newcounter{questionCounter}
\newcounter{partCounter}[questionCounter]
\newenvironment{question}[2][\arabic{questionCounter}]{%
    \setcounter{partCounter}{0}%
    \vspace{.25in} \hrule \vspace{0.5em}%
        \noindent{\bf #2}%
    \vspace{0.8em} \hrule \vspace{.10in}%
    \addtocounter{questionCounter}{1}%
}{}
\renewenvironment{part}[1][\alph{partCounter}]{%
    \addtocounter{partCounter}{1}%
    \vspace{.10in}%
    \begin{indented}%
       {\bf (#1)} %
}{\end{indented}}

%%%%%%%%%%%%%%%%%%%%%%%HEADER%%%%%%%%%%%%%%%%%%%%%%%%%%%%%%
\newcommand{\myname}{Shashank Singh}
\newcommand{\myandrew}{sss1@andrew.cmu.edu}
\newcommand{\myclass}{21-759 Differential Geometry}
\newcommand{\myhwnum}{3}
\newcommand{\duedate}{Monday, October 20, 2013}
%%%%%%%%%%%%%%%%%%%%%%%%%%%%%%%%%%%%%%%%%%%%%%%%%%%%%%%%%%%

%%%%%%%%%%%%%%%%%%%%CONTENT MACROS%%%%%%%%%%%%%%%%%%%%%%%%%
\renewcommand{\qed}{\quad \ensuremath{\blacksquare}}    % QED square
\newcommand{\inv}{^{-1}}                % inverse
\newcommand{\ox}{{\overline{X}}}
\newcommand{\oy}{{\overline{Y}}}
\newcommand{\oz}{{\overline{Z}}}
\newcommand{\bv}{\mathbf{v}}
\newcommand{\bx}{\mathbf{x}}
\newcommand{\by}{\mathbf{y}}
\newcommand{\bff}{\mathbf{f}}
\newcommand{\bzero}{\mathbf{0}}
\newcommand{\sminus}{\backslash}        % set difference
\renewcommand{\dim}{\operatorname{dim}} % dimension of a vector space
\renewcommand{\det}{\operatorname{det}} % determinant of a matrix
\newcommand{\sgn}{\operatorname{sign}}  % determinant of a matrix
\newcommand{\rnk}{\operatorname{rank}}  % rank of a matrix
\newcommand{\tr}{\operatorname{trace}}  % trace of a matrix
\renewcommand{\div}{\operatorname{div}} % divergence of a vector field
\newcommand{\grad}{\operatorname{grad}} % gradient of a scalar function
\newcommand{\N}{\mathbb{N}}             % natural numbers
\newcommand{\Z}{\mathbb{Z}}             % integers
\newcommand{\Q}{\mathbb{Q}}             % rational numbers
\newcommand{\R}{\mathbb{R}}             % real numbers
\newcommand{\M}{\mathcal{M}}            % manifold M
\newcommand{\D}{\mathcal{D}}            % \smooth real functions on a manifold
\newcommand{\B}{\mathcal{B}}            % basis of a topology
\newcommand{\pow}{\mathcal{P}}          % power set
\newcommand{\e}{\varepsilon}            % \varepsilon
\renewcommand{\phi}{\varphi}            % replace \phi with \varphi
%%%%%%%%%%%%%%%%%%%%%%%%%%%%%%%%%%%%%%%%%%%%%%%%%%%%%%%%%%%

\begin{document}
\thispagestyle{plain}

{\Large Homework \myhwnum} \\
\myclass \\
Name: \myname\footnote{\myandrew} \\
Due: \duedate

I would be willing to present solutions to Exercises 9 and 12.

\begin{question}{Problem 1}
Since the extension into $\overline{\M}$, the connection on $\overline{\M}$,
$\Pi_T$, and $(df)\inv$ are all linear (in $X$), $\nabla_X Y$ is linear in $X$.
Also, if $h$ is a smooth function on $\overline\M$, then
\begin{align*}
\nabla_X (hY)|_p
 &  = (df)\inv \left( \Pi_T \nabla_\ox h \oy \right)
    = (df)\inv \left( \Pi_T dh(\ox)\oy + h\nabla_\ox \oy \right)    \\
 &  = (df)\inv \left( dh(\ox)Y + h\nabla_\ox Y \right)
    = dh(X)Y + h\nabla_X Y,
\end{align*}
so we have the Leibnitz Rule.
\begin{align*}
g(\nabla_X Y,Z) + g(Y,\nabla_X Z)
 &  = \overline g(\Pi_T \nabla_\ox \oy, df Z)
    + \overline g(df Y, \Pi_T \nabla_\ox \oz)          \\
 &  = \overline g(\nabla_\ox df Y, df Z)
    + \overline g(df Y, \nabla_\ox df Z)          \\
 &  = \ox\overline g(df Y,df Z)
    = Xg(Y,Z),
\end{align*}
so we have symmetry. Finally, the connection is torsion-free since
\begin{align*}
\nabla_X Y - \nabla_Y X
 &  = (df)\inv \left( \Pi_T \nabla_\ox \oy - \nabla_\oy \ox \right) \\
 &  = (df)\inv \left( \Pi_T [\ox,\oy] \right) \\
 &  = (df)\inv [\Pi_T \ox,\Pi_T \oy]
    = [X,Y]. \qed
\end{align*}
\end{question}

\begin{question}{Exercise 5}
\begin{enumerate}[a)]
\item The flow is the solution to the differential equation
\[\left\{
    \begin{array}{l}
        \frac{d}{dt} \phi(t,p) = A\phi(t,p)   \\
        \phi(0,p) = p
    \end{array}
\right.
\]
Since $A$ is linear and constant, we have from ODEs that
$\phi(t,p) = \exp(tA) p$.

Let $p \in \M, w,v \in T_{\phi(t,p)}\M$. Then, since $\exp(tA)$ is linear,
\[g(d\exp(tA)w,d\exp(tA)v)
    = g(\exp(tA)w,\exp(tA)v)
    = g(w,\exp(tA^T)\exp(tA)v)
.\]
Since $[\exp(tA^t)]\inv = \exp(-tA)$, $\exp(tA)$ is an isometry if and only if
$A^T = -A$. \qed

\item
Since $\frac{d}{dt} \phi(t,p) = X(p) = 0$, $\phi(t,p) = p$ for all $t \in \R$.
Let $d : \M^2 \to \R$ denote the metric induced by $g$. Then, since
$q \mapsto \phi(t,q)$ is an isometry for $t \in (-\e,\e)$,
\[d(\phi(t,q),p) = d(\phi(0,q),p).\]
Thus, $\frac{d}{dt} d(\phi(t,q),p) \big|_{t = 0} = 0$, and hence
$X(\phi(t,q)) = \frac{d}{dt}\phi(t,q)$ is tangent to the geodesic sphere
centered at $p$.

\item Since $f\inv : M \to N$ is an isometry and, $\forall q \in Y$,
$X(f\inv(q)) = df\inv_q(Y(q))$, one direction suffices.

Assume $X$ is a Killing field, and let $\e > 0$ such that the flow
$\phi_X : (-\e,\e) \times U \to \M$ is an isometry. By local uniqueness,
a solution to
\[\left\{
    \begin{array}{l}
        \frac{d}{dt} \phi_Y(t,q) = Y(\phi_Y(t,q))   \\
        \phi_Y(0,q) = q
    \end{array}
\right.
\]
is precisely $\phi_Y(t,f(p)) = f(\phi_X(t,p))$, since
\[\frac{d}{dt} \phi_Y(t,f(p))
    = df_{\phi_X(t,p)} \frac{d}{dt} \phi_X(t,p)
    = df_{\phi_X(t,p)} X(\phi_X(t,p))
    = Y(f(X(\phi_X(t,p))))
    = Y(\phi_Y(t,f(p))).
\]
Also, if $t \in (-\e,\e)$, since $f$ is an isometry,
\begin{align*}
d(\phi_Y(t,f(p)),\phi_Y(t,f(q)))
 &  = d(\phi_X(t,p),\phi_X(t,q))    \\
 &  = d(\phi_X(0,p),\phi_X(0,q))
    = d(\phi_Y(0,f(p)),\phi_Y(0,f(q))),
\end{align*}
so $q \mapsto \phi(t,q)$ is an isometry.

%TODO
\item ($\Rightarrow$) Suppose, first, that $q \in U$ with $X(q) \neq 0$. If
$X(q) \neq 0$, then we can let $S$ be a submanifold of $U$ passing through $q$
and normal to $X(q)$, with $\dim S = \dim M - 1$. We can choose coordinates
$(x_1,\dots,x_{n - 1})$ in a neighborhood $V \subseteq U$ of $q$ such that
$(x_1,\dots,x_{n - 1},t)$ are coordinates in a neighborhood
$V \times (-\e,\e) \subseteq U$ and $X = \frac{\partial}{\partial t}$. Defining
$X_i = \frac{\partial}{\partial x_i}$, we have
\[\langle \nabla_{X_j} X, X_i \rangle + \langle \nabla_{X_i} X, X_j \rangle
    = X\langle X_i, X_j \rangle - \langle [X,X_i], X_j \rangle
        - \langle [X,X_j], X_i \rangle
    = \frac{\partial}{\partial t} \langle X_i, X_j \rangle
    = 0
,\]
where the last equality uses the fact that $X$ is a Killing field.

Now, if $X(q) = 0$, then either $q \in \overline{\{q \in U : X(q) \neq 0\}}$ or
there is a neighborhood $q$ on which $X \equiv 0$. In the first case, we have
the desired equation by continuity, and, in the second case, the equation holds
trivially.

($\Leftarrow$) Under the same setup,
\[\frac{\partial}{\partial t} \langle X_i, X_j \rangle
    = X\langle X_i, X_j \rangle - \langle [X,X_i], X_j \rangle
        - \langle [X,X_j], X_i \rangle
    = \langle \nabla_{X_j} X, X_i \rangle + \langle \nabla_{X_i} X, X_j \rangle
    = 0
,\]
so that $\langle X_i, X_j \rangle$ is constant and hence $X$ is Killing. \qed

\item Using the same coordinates as in part d, since
\[0
    = \frac{\partial}{\partial t} \langle X_i, X_j \rangle
    = \frac{\partial}{\partial t} g_{i,j}
    = \frac{\partial}{\partial x_n} g_{i,j}
,\]
$g_{i,j}$ does not depend on $x_n$.
\end{enumerate}
\end{question}

\begin{question}{Exercise 7}
Let $U := \exp_p((-\e,\e))$, and let $\{e_i\}_{i = 1}^n$ be an orthonormal
basis of $T_p\M$. If $q \in U$, then there is a geodesic $\exp_q(v)$ from $p$
to $q$. Since $\exp$ is a local isometry, $\{d(\exp_q)e_i\}_{i = 1}^n$ is an
orthonormal basis of $T_q\M$, so define $E_i(q) := d(\exp_q)e_i$. Since
$\nabla_{E_i} E_j(p) = 0$ and the Riemann connection is a function of the
metric (and hence is invariant under isometries)
$\nabla_{E_i} E_j \equiv 0$ on $U$.
\end{question}

\begin{question}{Exercise 8}
\begin{enumerate}[a)]
\item Since $\{E_i(p)\}_{i = 1}^n$ is an orthonormal basis of $T_p\M$,
\[\grad f(p)
    = \sum_{i = 1}^n \langle \grad f(p), E_i(p) \rangle E_i(p)
    = \sum_{i = 1}^n d f_p(E_i(p)) E_i(p)
    = \sum_{i = 1}^n (E_i(f)) E_i(p).
\]
It follows from the Levi-Civita formula for the Riemann connection that, in a
geodesic frame, the trace of the mapping $Y(p) \mapsto \nabla_Y X(p)$ is
\[\sum_{i = 1}^n E_i(f_i)(p).\]

\item If $\M = \R^n$, then $(E_i(f)) = \frac{\partial f}{\partial x_i}$ and
$E_i(p) = e_i$, $\forall p \in \M$, so it follows from part a) that
\[\grad f = \sum_{i = 1}^n \frac{\partial f}{\partial x_i} e_i
    \quad \mbox{ and } \quad
\div X = \sum_{i = 1}^n \frac{\partial f_i}{\partial x_i}.\]

\end{enumerate}
\end{question}

\begin{question}{Exercise 9}
\begin{enumerate}[a)]
\item By part a) of Problem 8,
\[\Delta f(p)
    = (\div \grad f)(p)
    = \div \left( \sum_{i = 1}^n (E_i(f))E_i(p) \right)
    = \sum_{i = 1}^n (E_i(E_i(f)) (p)
.\]
By part b) of Problem 8, it follows that, if $M = \R^n$, then
\[\Delta f(p)
    = \sum_{i = 1}^n \frac{\partial f}{\partial x_i}
        \left( \sum_{j = 1}^n \frac{\partial f}{\partial x_j} e_j \right)_i
    = \sum_{i = 1}^n \frac{\partial f}{\partial x_i}
                                            \frac{\partial f}{\partial x_i}
    = \sum_{i = 1}^n \frac{\partial^2 f}{\partial x_i^2}.
\]

\item Using the normal product rule for derivatives,
\begin{align*}
\Delta(fg) (p)
 &  = \sum_{i = 1}^n (E_i(E_i(fg)) (p)
    = \sum_{i = 1}^n (E_i (fE_i(g) + gE_i(f))) (p)  \\
 &  = \sum_{i = 1}^n fE_i(E_i(g))(p) + gE_i(E_i(f))(p) + E_i(f)E_i(g)  \\
 &  = (f\Delta g)(p) + (g\Delta f)(p)
                        + 2\left\langle \grad f(p), \grad g(p) \right\rangle.
    \qed
\end{align*}
\end{enumerate}
\end{question}

\begin{question}{Exercise 11}
Let $p \in \M$ and let $E_i$ be a geodesic frame at $p$ and let $\omega_i$
denote the differential $1$-form defined on the same neighborhood of $p$ as the
geodesic frame by $\omega_i(E_j) = \delta_{i,j}$. Also, define the $n$-form
$\nu := \wedge_{i = 1}^n \omega_i$. Then, for any
$v_1,\dots,v_n \in T_p\M$,
\[(\nu_p(v_1,\dots,v_n))^2
    = \det\left( [\omega_i(v_j)]^2 \right)
    = \det\left( [\langle E_i(p), v_j) \rangle]^2 \right)
    = \det\left( [\langle v_i, v_j \rangle] \right),
\]
so $\nu$ is the volume element on $M$. Also, if
$\theta_i
    := \omega_1 \wedge \dots \wedge \hat\omega_i \wedge \dots \wedge \omega_n$
and $X = \sum_{i = 1}^n f_iE_i$, then
\begin{align*}
(i(X)\nu)_p(v)
 &  = \nu_p(X(p),v)  \\
 &  = \sum_{\sigma \in S_n} \sgn(\sigma)
                                \prod_{i = 1}^n \omega_i(v_{\sigma(i)})
    = \sum_{i = 1}^n (-1)^{i + 1} \omega_i(X(p)) \theta_i(v)
    = \sum_{i = 1}^n (-1)^{i + 1} f_i \theta_i(v)
\end{align*}
(where $v = (v_2,\dots,v_n)$ and $v_1 = X(p)$). It follows from properties of
the exterior derivative that
\[d(i(X)\nu)
    = \sum_{i = 1}^n (-1)^{i + 1} (df_i \wedge \theta_i + f_i \wedge d\theta_i)
    = \left( \sum_{i = 1}^n E_i(f_i) \right)\nu
        + \sum_{i = 1}^n (-1)^{i + 1} f_i \wedge d\theta_i.
\]
Since
\[d\omega_k(E_i,E_j)
    = E_i\omega_k(E_j) - E_j\omega_k(E_i) - \omega_k([E_i,E_j])
    = \omega_k(\nabla_{E_i} E_j - \nabla_{E_j} E_i)
    = \omega_k(0)
    = 0,
\]
$(d\theta_i)_p = 0$, and so
\[d(i(X)\nu)(p) = \left( \sum_{i = 1}^n E_i(f_i) \right)\nu. \qed\]
\end{question}

\begin{question}{Exercise 12}
By the result of Exercise 11 and Stokes' Theorem, if $\nu$ is a volume form,
then
\[\int_M \Delta f\nu
    = \int_M \div \grad f \nu
    = \int_M d(i(\grad f)\nu)
    = 0.
\]
Since $\nabla f \geq 0$, $\nabla f = 0$. Then, applying Stokes Theorem and the
result of part b) of Exercise 9,
\[0
    = \int_M d(i(\grad f^2/2) \nu)
    = \int_M \Delta (f^2/2) \nu
    = \int_M \|\grad f\|^2 \nu,
\]
so $0 = \grad f = df$. By Problem 5 from Assignment 1, then, $f \equiv C$ on
$\M$, for some $C \in \R$. \qed
\end{question}
\end{document}
