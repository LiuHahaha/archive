\documentclass[11pt]{article}
\usepackage{enumerate}
\usepackage{fullpage}
\usepackage{fancyhdr}
\usepackage{amsmath, amsfonts, amsthm, amssymb}
\usepackage{color}
\usepackage{textcomp}
\setlength{\parindent}{0pt}
\setlength{\parskip}{5pt plus 1pt}
\pagestyle{empty}

\def\indented#1{\list{}{}\item[]}
\let\indented=\endlist

\newcounter{questionCounter}
\newcounter{partCounter}[questionCounter]
\newenvironment{question}[2][\arabic{questionCounter}]{%
    \setcounter{partCounter}{0}%
    \vspace{.25in} \hrule \vspace{0.5em}%
        \noindent{\bf #2}%
    \vspace{0.8em} \hrule \vspace{.10in}%
    \addtocounter{questionCounter}{1}%
}{}
\renewenvironment{part}[1][\alph{partCounter}]{%
    \addtocounter{partCounter}{1}%
    \vspace{.10in}%
    \begin{indented}%
       {\bf (#1)} %
}{\end{indented}}

%%%%%%%%%%%%%%%%%%%%%%%HEADER%%%%%%%%%%%%%%%%%%%%%%%%%%%%%%
\newcommand{\myname}{Shashank Singh}
\newcommand{\myandrew}{sss1@andrew.cmu.edu}
\newcommand{\myclass}{21-759 Differential Geometry}
\newcommand{\myhwnum}{2}
\newcommand{\duedate}{Monday, September 26, 2013}
%%%%%%%%%%%%%%%%%%%%%%%%%%%%%%%%%%%%%%%%%%%%%%%%%%%%%%%%%%%

%%%%%%%%%%%%%%%%%%%%CONTENT MACROS%%%%%%%%%%%%%%%%%%%%%%%%%
\renewcommand{\qed}{\quad \ensuremath{\blacksquare}}    % QED square
\newcommand{\inv}{^{-1}}                % inverse
\newcommand{\bv}{\mathbf{v}}
\newcommand{\bx}{\mathbf{x}}
\newcommand{\by}{\mathbf{y}}
\newcommand{\bff}{\mathbf{f}}
\newcommand{\bzero}{\mathbf{0}}
\newcommand{\sminus}{\backslash}        % set difference
\renewcommand{\dim}{\operatorname{dim}} % dimension of a vector space
\renewcommand{\det}{\operatorname{det}} % determinant of a matrix
\newcommand{\sgn}{\operatorname{sign}}  % determinant of a matrix
\newcommand{\rnk}{\operatorname{rank}}  % rank of a matrix
\newcommand{\tr}{\operatorname{trace}}  % trace of a matrix
\newcommand{\N}{\mathbb{N}}             % natural numbers
\newcommand{\Z}{\mathbb{Z}}             % integers
\newcommand{\Q}{\mathbb{Q}}             % rational numbers
\newcommand{\R}{\mathbb{R}}             % real numbers
\newcommand{\M}{\mathcal{M}}            % manifold M
\newcommand{\D}{\mathcal{D}}            % \smooth real functions on a manifold
\newcommand{\B}{\mathcal{B}}            % basis of a topology
\newcommand{\pow}{\mathcal{P}}          % power set
\newcommand{\e}{\varepsilon}            % \varepsilon
%%%%%%%%%%%%%%%%%%%%%%%%%%%%%%%%%%%%%%%%%%%%%%%%%%%%%%%%%%%

\begin{document}
\thispagestyle{plain}

{\Large Homework \myhwnum} \\
\myclass \\
Name: \myname\footnote{\myandrew} \\
Due: \duedate

I would be willing to present a solution to problem 5 or the first part of
problem 1 (showing the Lie algebra associated with $SL(n)$ is $sl(n)$.

\begin{question}{Problem 1}
Let $I$ denote the $n \times n$ identity matrix. We first show that
$T_ISL(n) = sl(n)$. Suppose $\gamma : (-\e,\e) \to SL(n)$ is differentiable,
with $\gamma(0) = I$. We showed in class that,
\[\tr(\gamma'(0))
    = \left( \frac{d}{dt}\det( \gamma(t) \right)\bigg|_{t = 0}
    = \left( \frac{d}{dt} 1 \right)\bigg|_{t = 0}
    = 0,\]
and hence, by definition of the tangent space, $T_ISL(n) \subseteq sl(n)$.

Suppose $A \in sl(n)$. Since $\det(I) = 1$ and the function
$t \mapsto \det(I + tA)$ is continuous, $\exists \e > 0$ such that
$\det(I + tA) \neq 0$, $\forall t \in (-\e,\e)$. Thus, we define
$\gamma : (-\e,\e) \to SL(n)$ for all $t \in (-\e,\e)$.
\[\gamma(t) = \frac{I + tA}{\det(I + tA)}.\]
Clearly the image of $\gamma$ indeed lies in $SL(n)$, and $\gamma(0) = I$.
Noting first that
\[\left( \frac{d}{dt} \det(I + tA) \right)\bigg|_{t = 0}
    = \tr\left( \frac{d}{dt} I + tA \right) \bigg|_{t = 0}
    = \tr(A)
    = 0,\]
we calculate
\[\gamma'(0)
    = \frac{\det(I + tA)A - (I + tA)\frac{d}{dt}\det(I + tA)}
           {(\det(I + tA))^2}\bigg|_{t = 0}
    = \frac{1\cdot A - 0\cdot I}{1^2}
    = A,
\]
and hence $A \in T_ISL(n)$. \qed

We now show $T_ISO(n) = so(n)$. Suppose $\gamma : (-\e,\e) \to SO(n)$ is
differentiable, with $\gamma(0) = I$. Then,
\[0
    = \left( \frac{d}{dt} I \right)\bigg|_{t = 0}
    = \left( \frac{d}{dt} \gamma(t)\gamma^T(t) \right)\bigg|_{t = 0}
    = \left( \gamma'(t)\gamma^T(t) + \gamma(t)(\gamma^T)'(t) \right)\bigg|_{t = 0}
    = \gamma'(0) + (\gamma^T)'(0).
\]
Thus, $(\gamma'(0))^T = -\gamma'(0)$, and so $T_ISO(n) \subseteq so(n)$.

I didn't have time to finish this problem.
\end{question}

\newpage
\begin{question}{Problem 2}
Since that vector spaces over a field are isomorphic if and only if they have
the same dimension, it suffices to show that the three spaces have the same
dimension. First, note
\[\dim((V \otimes W)^*)
    = \dim(V \otimes W)
    = \dim(V)\dim(W)
    = \dim(V^*)\dim(W^*)
    = \dim(V^* \otimes W^*).
\]
Since this problem isn't being graded, I didn't finish writing a thorough
solution, but it is easy to show that a bilinear function
$f \in L_2(V \otimes W)$ is defined by its values on each pair $(v,w)$ for $v
\in \B_V, w \in \B_W$, where $\B_V$ and $\B_W$ are bases of $V$ and $W$,
respectively. It follows that $\dim(L_2(V,W)) = \dim(V)\dim(W)$. \qed
\end{question}

\begin{question}{Problem 3}
I wasn't able to finish this problem.
\end{question}

\begin{question}{Problem 4}
\begin{enumerate}[(i)]
\item
Suppose $S_1$ and $S_2$ are $(0,s_1)$ and $(0,s_2)$ tensors, respectively. If
$p \in \M$, $v_1,\dots,v_{s_1 + s_2} \in T_p\M$,
\begin{align*}
\Phi^*(S_1 \otimes S_2)|_p (v_1,\dots,v_{s_1 + s_2})
 &  = (S_1 \otimes S_2)|_{\Phi(p)} (D\Phi v_1,\dots,D\Phi v_{s_1 + s_2})    \\
 &  = S_1|_{\Phi(p)}(D\Phi v_1,\dots,D\Phi v_{s_1})
        S_2|_{\Phi(p)}(D\Phi v_{s_1 + 1},\dots,D\Phi v_{s_2})               \\
 &  = \Phi^*S_1|_{\Phi(p)} (v_1,\dots,v_{s_2})
        \Phi^*S_2|_{\Phi(p)} (v_{s_1 + 1},\dots,v_{s_1 + s_2})              \\
 &  = (\Phi^*(S_1) \otimes \Phi^*(S_2))_p (v_1,\dots,v_{s_1 + s_2}).
\end{align*}
It follows that $\Phi^*(S_1 \otimes S_2) = \Phi^*(S_1) \otimes \Phi^*(S_2)$.
\qed

\item Suppose $\omega_1$ and $\omega_2$ are $k$- and $l$-forms on $\M$,
respectively. If $p \in \M$, $v_1,\dots,v_{s_1 + s_2} \in T_p\M$,
\begin{align*}
\Phi^*(\omega_1 \wedge \omega_2)|_p (v_1,\dots,v_{s_1 + s_2})
 &  = (S_1 \wedge S_2)|_{\Phi(p)} (D\Phi v_1,\dots,D\Phi v_{s_1 + s_2})    \\
 &  = S_1|_{\Phi(p)}(D\Phi v_1,\dots,D\Phi v_{s_1})
        S_2|_{\Phi(p)}(D\Phi v_{s_1 + 1},\dots,D\Phi v_{s_2})               \\
 &  = \Phi^*S_1|_{p} (v_1,\dots,v_{s_2})
        \Phi^*S_2|_{p} (v_{s_1 + 1},\dots,v_{s_1 + s_2})              \\
 &  = (\Phi^*(S_1) \wedge \Phi^*(S_2))_p (v_1,\dots,v_{s_1 + s_2}).
\end{align*}
It follows that $\Phi^*(S_1 \wedge S_2) = \Phi^*(S_1) \wedge \Phi^*(S_2)$.
\qed

\newpage
\item If $\omega$ is a $0$-form, $p \in \M, v \in T_p\M$, then
\[\Phi^*(d\omega)|_p(v)
    = d\omega|_{\Phi(p)}(D\Phi v)
    = d\omega|_{\Phi(p)}(D\Phi v)
    = D\Phi v [\omega]
    = v [\omega \circ \Phi]
    = (d\Phi^*\omega) (v)
\]
Suppose that, for some $n \in \N$, $\forall k \leq n$, for all $k$-forms
$\omega$, $\Phi^*(d\omega) = d\Phi^*(\omega)$, and let $\omega$ be an
$(n + 1)$-form. Then, $\omega = \omega_1 \wedge \omega_2$, for some $k_1$- and
$k_2$-forms $\omega_1$ and $\omega_2$, respectively, with $k_1,k_2 \leq n$.
Thus, by part (ii) and since the pullback is clearly linear by its definition,
\begin{align*}
\Phi^*(d\omega)
   = \Phi^*(d(\omega_1 \wedge \omega_2))
 & = \Phi^*(d\omega_1 \wedge \omega_2 + (-1)^{k_1}\omega_1 \wedge d\omega_2)\\
 & = \Phi^*(d\omega_1) \wedge \Phi^*(\omega_2)
        + (-1)^{k_1}\Phi^*(\omega_1) \wedge \Phi^*(d\omega_2) \\
 & = d\Phi^*(d\omega_1) \wedge \Phi^*(\omega_2)
        + (-1)^{k_1}\Phi^*(\omega_1) \wedge d\Phi^*(\omega_2) \\
 & = d\left(\Phi^*(\omega_1) \wedge \Phi^*(\omega_2)\right) \\
 & = d\Phi^*(\omega_1 \wedge \omega_2)
   = d\Phi^*\omega
\end{align*}

By induction on $n$, the exterior derivative and pullback commute for all
$n$-forms. \qed
\end{enumerate}
\end{question}

\begin{question}{Problem 5}
Since $\omega$ is a $1$-form, $\forall p \in S^2$, $\omega|_p \in (T_pS^2)^*$.
Thus, we can choose a dual vector $v \in T_pS^2$ with the property that $v = 0$
implies $\omega|_p = 0$. In particular, since $\omega$ is smooth, we can choose
$v_p$ such that the map $p \mapsto (p,v_p)$ is smooth. Since this map defines a
vector field on $S^2$ and one cannot ``comb a coconut'', $\exists p \in S^2$
with $v_p = 0$, and hence $\omega|_p = 0$.

Recall now that $SO(3)$ is the group of rotations in $\R^3$, and hence,
$\forall q \in S^2$, $\exists \phi_q \in SO(3)$ such that $\phi(q) = p$. Then,
for any $v \in T_qS^2$,
\[\omega|_q(v) = \phi^*\omega|_q(v) = \omega|_{\Phi(q)}(D\phi v) = 0,\]
and hence $\omega = 0$. \qed
\end{question}
\end{document}
