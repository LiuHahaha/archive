\documentclass[11pt]{article}
\usepackage{enumerate}
\usepackage{fullpage}
\usepackage{fancyhdr}
\usepackage{amsmath, amsfonts, amsthm, amssymb}
\usepackage{color}
\usepackage{textcomp}
\setlength{\parindent}{0pt}
\setlength{\parskip}{5pt plus 1pt}
\pagestyle{empty}

\def\indented#1{\list{}{}\item[]}
\let\indented=\endlist

\newcounter{questionCounter}
\newcounter{partCounter}[questionCounter]
\newenvironment{question}[2][\arabic{questionCounter}]{%
    \setcounter{partCounter}{0}%
    \vspace{.25in} \hrule \vspace{0.5em}%
        \noindent{\bf #2}%
    \vspace{0.8em} \hrule \vspace{.10in}%
    \addtocounter{questionCounter}{1}%
}{}
\renewenvironment{part}[1][\alph{partCounter}]{%
    \addtocounter{partCounter}{1}%
    \vspace{.10in}%
    \begin{indented}%
       {\bf (#1)} %
}{\end{indented}}

%%%%%%%%%%%%%%%%%%%%%%%HEADER%%%%%%%%%%%%%%%%%%%%%%%%%%%%%%
\newcommand{\myname}{Shashank Singh}
\newcommand{\myandrew}{sss1@andrew.cmu.edu}
\newcommand{\myclass}{21-759 Differential Geometry}
\newcommand{\myhwnum}{3}
\newcommand{\duedate}{Monday, October 20, 2013}
%%%%%%%%%%%%%%%%%%%%%%%%%%%%%%%%%%%%%%%%%%%%%%%%%%%%%%%%%%%

%%%%%%%%%%%%%%%%%%%%CONTENT MACROS%%%%%%%%%%%%%%%%%%%%%%%%%
\renewcommand{\qed}{\quad \ensuremath{\blacksquare}}    % QED square
\newcommand{\inv}{^{-1}}                % inverse
\newcommand{\bv}{\mathbf{v}}
\newcommand{\bx}{\mathbf{x}}
\newcommand{\by}{\mathbf{y}}
\newcommand{\bff}{\mathbf{f}}
\newcommand{\bzero}{\mathbf{0}}
\newcommand{\sminus}{\backslash}        % set difference
\renewcommand{\dim}{\operatorname{dim}} % dimension of a vector space
\renewcommand{\det}{\operatorname{det}} % determinant of a matrix
\newcommand{\sgn}{\operatorname{sign}}  % determinant of a matrix
\newcommand{\rnk}{\operatorname{rank}}  % rank of a matrix
\newcommand{\tr}{\operatorname{trace}}  % trace of a matrix
\newcommand{\N}{\mathbb{N}}             % natural numbers
\newcommand{\Z}{\mathbb{Z}}             % integers
\newcommand{\Q}{\mathbb{Q}}             % rational numbers
\newcommand{\R}{\mathbb{R}}             % real numbers
\newcommand{\M}{\mathcal{M}}            % manifold M
\newcommand{\D}{\mathcal{D}}            % \smooth real functions on a manifold
\newcommand{\B}{\mathcal{B}}            % basis of a topology
\newcommand{\pow}{\mathcal{P}}          % power set
\newcommand{\e}{\varepsilon}            % \varepsilon
%%%%%%%%%%%%%%%%%%%%%%%%%%%%%%%%%%%%%%%%%%%%%%%%%%%%%%%%%%%

\begin{document}
\thispagestyle{plain}

{\Large Homework \myhwnum} \\
\myclass \\
Name: \myname\footnote{\myandrew} \\
Due: \duedate

I would be willing to present solutions to problems 3, 4, and 5.

\begin{question}{Problem 1}
First, let $g : T_eG \times T_eG \to \R$ be an inner product (on the Lie
algebra associated with $G$). We now extend this to a Riemann metric tensor on
$G$ by defining
\[g_p(v,w) := g_e(DL_{g\inv} v, DL_{g\inv} v).\]
$\forall p \in G, v,w \in T_pG$.
Let
\[\omega := \sqrt{|g|} dx_1 \wedge \dots \wedge dx_n,\]
be the usual volume form induced by $g$. Then, for any open set $A$ of $G$,
define
\[\mu(A) := \int_A\]
For any $p \in G$, since $L_p$ is an isometry with respect to $g$,
$L_p^*\omega = \omega$, so
\[\mu(L_p(A))
    = \int_{L_p(A)} \omega
    = \int_A L_p^*\omega
    = \int_A \omega
    = \mu(A),
\]
and we have left-invariance, as desired. \qed

If $\mu_2$ is another left-invariant Borel measure on $G$, define, for any open
set $A$,
\[\phi(A) := \frac{\mu(A)}{\nu(A)}\]
(note, $\mu(A),\nu(A) \neq 0$). Then, $\forall p \in G$,
\[\phi(L_p(A))
    = \frac{\mu(pA)}{\nu(pA)}
    = \frac{\mu(A)}{\nu(A)}
    = \phi(A),\]
and so $\phi$ is constant on $G$. From construction of the Borel
$\sigma$-algebra, it follows that $\mu = c\nu$, for some constant $c \in \R$,
on all Borel sets, so $\mu$ is unique up to constant multiples. \qed
\end{question}

\newpage
\begin{question}{Problem 2}
\begin{enumerate}[(i)]
\item If $X$ is a vector field and $\Phi_t$ is the associated flow,
\begin{align*}
L_X(\alpha \wedge \beta)
 &  = \frac{d}{dt} \bigg|_{t = 0} \Phi_t^*(\alpha \wedge \beta)             \\
 &  = \frac{d}{dt} \bigg|_{t = 0} \Phi_t^*(\alpha) \wedge \Phi_t^*(\beta)   \\
 &  = \frac{d}{dt} \bigg|_{t = 0} \Phi_t^*(\alpha) \wedge \beta
        + \alpha \wedge \frac{d}{dt} \bigg|_{t = 0} \Phi_t^*(\beta)         \\
 &  = (L_X\alpha) \wedge \beta + \alpha \wedge (L_X\beta),
\end{align*}
where the third equality follows from the product rule and equation (4) in the
notes (definition of the wedge product). \qed

\item

\item If $X$ is a vector field, since $d^2 = 0$, by Cartan's formula,
\[L_X \circ d
    = d \circ i_X \circ d + i_X \circ d \circ d
    = d \circ d \circ i_X + d \circ i_X \circ d
    = d \circ L_X. \qed
\]

\end{enumerate}
\end{question}

\newpage
\begin{question}{Problem 3}
Fix any $p \in U$. If $q \in U$ and $\gamma : [0,1] \to U$ is a smooth curve
from $p$ to $q$ (i.e., $\gamma(0) = p, \gamma(1) = q$), define
\[f(q) := \int_\gamma \phi^*\omega,\]
where $\phi^*\omega$ denotes the pullback of $\omega$ by $\phi^*$.

\paragraph{Lemma 1} This $f$ is well defined on $U$.

\emph{Proof:} Let $q \in U$. Since $U$ is simply connected, it is pathwise
connected, and hence there is a smooth curve from $p$ to $q$.

Suppose now that $\gamma_1,\gamma_2 : [0,1] \to U$ are smooth curves from $p$
to $q$. Since $U$ is simply connected, there is a smooth homotopy
$H : [0,1]^2 \to U$ between $\gamma_1$ and $\gamma_2$ (i.e.,
$H(0,t) = \gamma_1(t)$ and $H(1,t) = \gamma_2(t)$, and $H(s,0) = p$ and
$H(s,1) = q$, $\forall s,t \in [0,1]$.) Note that the image
$\M_2 := H([0,1]^2)$ is a manifold whose boundary is the image of the
``concatenation'' $\gamma$ of $\gamma_1$ and $\gamma_2$, where $\gamma_2$ is
oriented backwards. Thus, by Stokes' Theorem, and properties of the pullback,
\[ \int_{\gamma_1} \phi^*\omega - \int_{\gamma_2} \phi^*\omega
    = \int_{\gamma} \phi^*\omega
    = \int_{\M_2} d(\phi^*\omega)
    = \int_{\M_2} \phi^*(d\omega)
    = \int_{\M_2} \phi^*(0)
    = 0. \quad \square
\]
Then, $\forall x \in \M$,
\[d(f \circ \phi\inv)_x\left( \frac{\partial}{\partial x_i} \right)
    = \frac{\partial}{\partial x_i} (f \circ \phi\inv \circ \phi) (\phi\inv(x))
    = \frac{\partial}{\partial x_i} f (\phi\inv(x))
    = \phi^*\omega\bigg|_{\phi\inv(x)}
    \hspace{-5mm}= \omega_x\left(\frac{\partial}{\partial x_i}\right). \qed\]
\end{question}

\begin{question}{Problem 4}
\[\int_\M \omega
    = \int_\M d\alpha
    = \int_{\partial \M} \alpha
    = \int_\emptyset \alpha = 0. \qed
\]
Define $f : \R^2 \sminus \{(0,0)\} \to \R^2$ by
\[f(x,y) := \left( \frac{x}{x^2 + y^2}, \frac{-y}{x^2 + y^2} \right),\]
and let $\omega$ denote the $1$-form on $S^1$ corresponding to $f$. Then,
\[\int_{S^1} \omega = 2\pi \neq 0,\]
and so, by the previous result, $\omega$ cannot be exact, since $S^1$ is a
compact manifold.
\end{question}

\newpage
\begin{question}{Problem 5}
Note that a parametrization of the catenoid is given by
\[(\cosh(z)\cos(t),\cosh(z)\sin(t),z) \quad z \in \R, t \in (0,2\pi)\]
and a parametrization of the helicoid is given by
\[(t\cos(z),t\sin(z),z) \quad z,t \in \R.\]
Define $\Phi : \R^2 \to \R^2$ defined by
\[\Phi(t,z) := (t,\sinh z).\]
The Jacobian of $\Phi$ is
\[J_\Phi
    = \begin{bmatrix}
        1   &   0       \\
        0   &   \cosh v \\
    \end{bmatrix},
\]
so $\det(J_\Phi)$ is non-zero everywhere and hence $\Phi$ is a local
diffeomorphism. It is easily checked that, under this reparametrization of the
helicoid, for both the catenoid and the helicoid, that
\begin{align*}
\left\langle\frac{\partial}{\partial z},
                                    \frac{\partial}{\partial z}\right\rangle
    & = \cosh^2(t), \\
\left\langle\frac{\partial}{\partial z},
                                    \frac{\partial}{\partial t}\right\rangle
    & = 0,          \\
\left\langle\frac{\partial}{\partial t},
                                    \frac{\partial}{\partial t}\right\rangle
    & = \cosh^2(t). \\
\end{align*}

The catenoid and helicoid are not globally isometric, as they are not even
homeomorphic. For example, the helicoid is clearly simply connected, whereas
the catenoid is not. \qed
\end{question}

\begin{question}{Problem 6}
I wasn't able to finish this problem.
\end{question}
\end{document}
