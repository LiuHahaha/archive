\documentclass[11pt]{article}
\usepackage{enumerate}
\usepackage{fullpage}
\usepackage{fancyhdr}
\usepackage{amsmath, amsfonts, amsthm, amssymb}
\usepackage{color}
\setlength{\parindent}{0pt}
\setlength{\parskip}{5pt plus 1pt}
\pagestyle{empty}

\def\indented#1{\list{}{}\item[]}
\let\indented=\endlist

\newcounter{questionCounter}
\newcounter{partCounter}[questionCounter]
\newenvironment{question}[2][\arabic{questionCounter}]{%
    \setcounter{partCounter}{0}%
    \vspace{.25in} \hrule \vspace{0.5em}%
        \noindent{\bf #2}%
    \vspace{0.8em} \hrule \vspace{.10in}%
    \addtocounter{questionCounter}{1}%
}{}
\renewenvironment{part}[1][\alph{partCounter}]{%
    \addtocounter{partCounter}{1}%
    \vspace{.10in}%
    \begin{indented}%
       {\bf (#1)} %
}{\end{indented}}

%%%%%%%%%%%%%%%%%%%%%%%HEADER%%%%%%%%%%%%%%%%%%%%%%%%%%%%%%
\newcommand{\myname}{Shashank Singh\footnote{sss1@andrew.cmu.edu}}
\newcommand{\myclass}{21-740 Introduction to Functional Analysis II}
\newcommand{\myhwnum}{5}
\newcommand{\duedate}{Friday, Decemeber 6, 2013}
%%%%%%%%%%%%%%%%%%%%%%%%%%%%%%%%%%%%%%%%%%%%%%%%%%%%%%%%%%%

%%%%%%%%%%%%%%%%%%%%CONTENT MACROS%%%%%%%%%%%%%%%%%%%%%%%%%
\renewcommand{\qed}{\quad \ensuremath{\blacksquare}}
\newcommand{\inv}{^{-1}}
\newcommand{\bv}{\mathbf{v}}
\newcommand{\bx}{\mathbf{x}}
\newcommand{\by}{\mathbf{y}}
\newcommand{\bff}{\mathbf{f}}
\newcommand{\bzero}{\mathbf{0}}
\newcommand{\bxi}{\boldsymbol{\xi}}
\newcommand{\boldeta}{\boldsymbol{\eta}}
\newcommand{\dist}{\operatorname{dist}} % distance from or between sets
\newcommand{\area}{\operatorname{area}} % area of a polygon
\newcommand{\Gr}{\operatorname{Gr}}     % graph of a function
\renewcommand{\sp}{\operatorname{span}} % span of a set
\newcommand{\bdry}{\operatorname{bdry}} % boundary of a set
\newcommand{\sminus}{\backslash}        % set difference
\newcommand{\N}{\mathbb{N}}             % natural numbers
\newcommand{\Z}{\mathbb{Z}}             % integers
\newcommand{\Q}{\mathbb{Q}}             % rational numbers
\newcommand{\R}{\mathbb{R}}             % real numbers
\newcommand{\C}{\mathbb{C}}             % complex numbers
\newcommand{\D}{\mathcal{D}}            % domain of an operator
\newcommand{\Cmp}{\mathcal{C}}          % space of compact linear operators\s
\newcommand{\K}{\mathbb{K}}             % underlying field of a linear space
\newcommand{\Ran}{\mathcal{R}}          % range of a linear operator
\newcommand{\Nul}{\mathcal{N}}          % null-space of a linear operator
\renewcommand{\L}{\mathcal{L}}          % space of bounded linear functions
\newcommand{\pow}[1]{\mathcal{P}\left(#1\right)}    % power set of #1
\newcommand{\e}{\varepsilon}            % \varepsilon
\newcommand{\wto}{\rightharpoonup}      % weak convergence
\newcommand{\wsto}{\stackrel{*}{\rightharpoonup}}   % weak-* convergence
\renewcommand{\Re}{\operatorname{Re}}   % real part of a complex number
\newcommand{\tT}{\widetilde{T}}         % for P3
\newcommand{\A}{\mathcal{A}}            % for P3
\renewcommand{\S}{\mathcal{S}}          % Schwartz space
%%%%%%%%%%%%%%%%%%%%%%%%%%%%%%%%%%%%%%%%%%%%%%%%%%%%%%%%%%%

\begin{document}
\thispagestyle{plain}

{\Large Homework \myhwnum} \\
\myclass \\
Name: \myname \\
Due: \duedate

\begin{question}{Problem 1}
[I'm not sure I exactly analyzed the right properties of the semigroup here, or
that I was sufficiently rigorous in these results\dots]

Let $X = H^2(\R^n)$, and let $\D(A) = X$. Define $A : \D(A) \to X$ by
\[(Af)(x) = \Delta f(x) - V(x)f(x), \quad \forall f \in X, x \in \R^n\]
(noting $u_t = i A u$). $A$ is self-adjoint on $H^2(\R^n)$, since, using
integration by parts,
\begin{align*}
(Af,g)
	= \int_{\R^n} (\Delta f) g - (V f) g
	= \int_{\R^n} \Delta f (\Delta g) - f (V g)
	= (f,Ag), \quad \forall f,g \in H^2(\R^n).
\end{align*}
Since we are in a Hilbert space, $A$ generates a semigroup
$T : [0,\infty) \to \L(X;X)$ defined by
\[T(t) := \int_{-\infty}^\infty e^{\lambda tA} \, dP(\lambda).\]
Then, we can define the semigroup $U : [0,\infty) \to \L(X;X)$ generated by
$iA$ as
\[U(t) := \int_{-\infty}^\infty e^{i \lambda t} \, dP(\lambda).\]
Note that $\forall t \geq 0$,
\[U^*(t)
	= \int_{-\infty}^\infty e^{-i\lambda t} dE_\lambda
	= U\inv(t),
\]
and so $U$ is unitary. A consequence should be that the Schrodinger equation
is time-reversible. Another immediate consequence is that $U$ is a contraction
semigroup. To gain more information, we should study the spectrum of $A$.
\end{question}

\begin{question}{Problem 2}
[I didn't have time to do this question.]
\end{question}

\newpage
\begin{question}{Problem 3}
Suppose $\exists N \in \N \cup \{0\}, K \in \R$ such that
\[\|L\phi\|_X \leq L\||\phi\||_N, \quad \forall \phi \in \S(\R^n),\]
and suppose $\phi \in \S(\R^n)$ and $\{\phi_k\}_{k = 1}^\infty$ is a sequence
in $\S(\R^n)$ with $\phi_k \to \phi$ as $k \to \infty$.
By Remark 13.13, $\forall \alpha, \beta \in M_n$ with
$|\alpha|,|\beta| \leq N$,
\[\|P_\beta D^\alpha \phi_k - P_\beta D^\alpha \phi\|_\infty \to 0\]
as $k \to \infty$. Then, since
$S := |\{\alpha, \beta \in M_n : |\alpha|,|\beta| \leq N\}|$ is finite,
\[m_k := \sup_{|\alpha|,|\beta| \leq N}
	\|P_\beta D^\alpha \phi_k - P_\beta D^\alpha \phi\|_\infty
		\to 0
\]
as $k \to \infty$. Thus,
\begin{align*}
\|L\phi_k - L\phi\|_X
 &	= \|L(\phi_k - \phi)\|_X
 	\leq K \|| \phi_k - \phi \||_N	\\
 &	= K \sum_{|\alpha|,|\beta| \leq N} \|P_\beta D^\alpha \phi_k - P_\beta D^\alpha \phi\|_\infty
	\leq K S m_k
\to 0.
\end{align*}
as $k \to \infty$, and so $L$ is continuous.
I didn't time to do the converse.

\end{question}

\begin{question}{Problem 4}
Such a tempered distribution does indeed exist.
Define the functional $u : \S(\R^n) \to \C$ by
\[u (\phi) := (L \check \phi)(0), \quad \forall \phi \in \S(\R^n).\]
If $\{\phi_k\}_{k = 1}^\infty$ is a sequence in $\S(\R^n)$ converging to
$\phi \in \S(\R^n)$ in the usual metric $\rho$ on $\S(\R^n)$, then, since
$L$ is continuous, $\rho(L\phi_k,L\phi) \to 0$ as $k \to \infty$. By definition
of $\rho$, this implies $\|L\phi_k - L\phi\|_\infty \to 0$, and hence
$u(\phi_k) \to u(\phi)$, as $k \to \infty$. Thus, $u$ is continuous and hence
$u \in \S'(\R^n)$. Furthermore, $\forall \phi \in \S(\R^n), x \in \R^n$,
\[
(L \phi)(x)
    = (\tau_{-x} (L \phi))(0)
    = (L ({\tau_{-x} \phi}))(0)
    = u(\tau_{x} \check \phi)
    = (u * \phi)(x). \qed
\]
\end{question}

\newpage
\begin{question}{Problem 5}
[I ended up able to prove necessary conditions on $r$ for the embedding, which
I give below. I wasn't able to show sufficient conditions, as the question asked.]

$\forall \lambda > 0$, define the dilation operator
$\delta_\lambda u(x) := u(\lambda x)$. Then,
\[\|\delta_\lambda u\|_r = \frac{1}{\lambda^{n/r}} \|u\|_r.\]
Also, by properties of the Fourier Transform and a Change of Variables, for
$\lambda \geq 1$

\begin{align*}
\|\delta_\lambda u\|_{s,2}^2
 &  = \int_{\R^n} Q_s(\xi) |\widehat{\delta_\lambda u(\xi)}|^2\, d\xi
    = \int_{\R^n} Q_s(\xi) \left|
                        \frac{1}{\lambda^n} \delta_{1/\lambda}
                                        \widehat{u(\xi)} \right|^2\, d\xi   \\
 &  = \frac{\lambda^n}{\lambda^{2n}}
                \int_{\R^n} Q_s(\lambda \xi) \left|
                                            \widehat{u(\xi)} \right|^2\, d\xi
    = \frac{\lambda^{2s}}{\lambda^{n}}
                \int_{\R^n} (\lambda^{-2} + |\xi|^2)^s \left|
                                            \widehat{u(\xi)} \right|^2\, d\xi
    \leq \frac{\lambda^{2s}}{\lambda^{n}} \|u\|_{s,2}^2.
\end{align*}
If we have a constant $C > 0$ such that
\[\|\delta_\lambda u\|_r \leq C \|\delta_\lambda u\|_{s,2}^2, \quad \forall u \in H^s(\R^n),\]
then, $\forall \lambda \geq 1$,
\[\frac{1}{\lambda^{n/r}}\|u\|_r \leq C
    \frac{\lambda^s}{\lambda^{n/2}}\|u\|_{s,2}.\]
Thus, in order to avoid a contradiction when taking $\lambda \to \infty$, we
must have
\[s > n/2 - n/r.\]
Solving this inequality for $r$ gives
\[r \leq \frac{2n}{n - 2s}.\]
Due to tail behavior of functions in $H^s$,
$H^s(\R^n) \not \hookrightarrow L^r(\R^n)$ for $r < 2$,
Thus, if $H^s(\R^n) \hookrightarrow L^r(\R^n)$,
\[r \in \left[ 2, \frac{2n}{n - 2s}\right].\]
\end{question}
\end{document}
