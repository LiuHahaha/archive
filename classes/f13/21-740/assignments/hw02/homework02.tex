\documentclass[11pt]{article}
\usepackage{enumerate}
\usepackage{fullpage}
\usepackage{fancyhdr}
\usepackage{amsmath, amsfonts, amsthm, amssymb}
\usepackage{color}
\setlength{\parindent}{0pt}
\setlength{\parskip}{5pt plus 1pt}
\pagestyle{empty}

\def\indented#1{\list{}{}\item[]}
\let\indented=\endlist

\newcounter{questionCounter}
\newcounter{partCounter}[questionCounter]
\newenvironment{question}[2][\arabic{questionCounter}]{%
    \setcounter{partCounter}{0}%
    \vspace{.25in} \hrule \vspace{0.5em}%
        \noindent{\bf #2}%
    \vspace{0.8em} \hrule \vspace{.10in}%
    \addtocounter{questionCounter}{1}%
}{}
\renewenvironment{part}[1][\alph{partCounter}]{%
    \addtocounter{partCounter}{1}%
    \vspace{.10in}%
    \begin{indented}%
       {\bf (#1)} %
}{\end{indented}}

%%%%%%%%%%%%%%%%%%%%%%%HEADER%%%%%%%%%%%%%%%%%%%%%%%%%%%%%%
\newcommand{\myname}{Shashank Singh\footnote{sss1@andrew.cmu.edu}}
\newcommand{\myclass}{21-740 Introduction to Functional Analysis II}
\newcommand{\myhwnum}{2}
\newcommand{\duedate}{Wednesday, October 9, 2013}
%%%%%%%%%%%%%%%%%%%%%%%%%%%%%%%%%%%%%%%%%%%%%%%%%%%%%%%%%%%

%%%%%%%%%%%%%%%%%%%%CONTENT MACROS%%%%%%%%%%%%%%%%%%%%%%%%%
\renewcommand{\qed}{\quad \ensuremath{\blacksquare}}
\newcommand{\inv}{^{-1}}
\newcommand{\bv}{\mathbf{v}}
\newcommand{\bx}{\mathbf{x}}
\newcommand{\by}{\mathbf{y}}
\newcommand{\bff}{\mathbf{f}}
\newcommand{\bzero}{\mathbf{0}}
\newcommand{\bxi}{\boldsymbol{\xi}}
\newcommand{\boldeta}{\boldsymbol{\eta}}
\newcommand{\dist}{\operatorname{dist}}
\newcommand{\area}{\operatorname{area}}
\newcommand{\vspan}{\operatorname{span}}
\newcommand{\Gr}{\operatorname{Gr}}     % graph of a function
\renewcommand{\sp}{\operatorname{span}} % span of a set
\newcommand{\bdry}{\operatorname{bdry}} % boundary of a set
\newcommand{\sminus}{\backslash}        % set difference
\newcommand{\N}{\mathbb{N}}             % natural numbers
\newcommand{\Z}{\mathbb{Z}}             % integers
\newcommand{\Q}{\mathbb{Q}}             % rational numbers
\newcommand{\R}{\mathbb{R}}             % real numbers
\newcommand{\C}{\mathbb{C}}             % complex numbers
\newcommand{\Cmp}{\mathcal{C}}          % space of compact linear functions
\newcommand{\K}{\mathbb{K}}             % underlying field of a linear space
\newcommand{\Ran}{\mathcal{R}}          % range of a linear operator
\newcommand{\Nul}{\mathcal{N}}          % null-space of a linear operator
\renewcommand{\L}{\mathcal{L}}          % space of bounded linear functions
\newcommand{\pow}[1]{\mathcal{P}\left(#1\right)}    % power set of #1
\newcommand{\e}{\varepsilon}            % \varepsilon
\newcommand{\wto}{\rightharpoonup}      % weak convergence
\newcommand{\wsto}{\stackrel{*}{\rightharpoonup}}   % weak-* convergence
%%%%%%%%%%%%%%%%%%%%%%%%%%%%%%%%%%%%%%%%%%%%%%%%%%%%%%%%%%%

\begin{document}
\thispagestyle{plain}

{\Large Homework \myhwnum} \\
\myclass \\
Name: \myname \\
Due: \duedate

\begin{question}{Problem 1}
For any $m,n \in \N$ with $m > n$, the define the operator
$A_{m,n} := A_m - A_n$. Using the fact that $A_{m,n} \geq 0$, it is easy to
adapt the proof of the Cauchy-Schwarz Inequality to show, $\forall x,y \in X$,
\begin{equation}
\label{ineq:schwarz}
(A_{m,n} x,y)^2 \leq (A_{m,n} x,x)(A_{m,n} y,y).
\end{equation}
Plugging $y = A_{m,n} x$ into this inequality gives
\begin{align}
\notag
\|(A_m - A_n)x\|^4
     = (A_{m,n} x,A_{m,n} y)^2
 &   \leq (A_{m,n}x,x)(A_{m,n}^2x,A_{m,n}x)  \\
\notag
 &   \leq (A_{m,n}x,x)\|A_{m,n}\|^3\|x\|^2        \\
\label{ineq:work}
 &   \leq (A_{m,n}x,x)\|I - A_1\|^3\|x\|^2.
\end{align}
where the last line follows from $A_1 \leq A_n \leq A_m \leq I$. Since the
sequence  $\{(A_nx,x)\}_{n = 1}^\infty$ is increasing and bounded by $1$,
it converges and is hence Cauchy. It follows from (\ref{ineq:work}) that the
sequence $\{A_nx\}_{n = 1}^\infty$ is Cauchy. Since $X$ is complete, we can
define $L : X \to X$ by
\[Lx = \lim_{n \to \infty} A_n x.\]
Clearly, $L$ is linear and self-adjoint, and $A_1, \leq L \leq I$. Thus, for
any $x \in X$, $-\|A\| \leq (A_1x,x) \leq (Lx,x) \leq \|x\|^2$, and so, since
$\|L\| = \sup \{(Lx,x) : x \in X, \|x\| = 1\}$, $L$ is bounded. \qed

I didn't use the hypothesis that $A_mA_n = A_nA_m$, but the proof \emph{seems}
correct to me.
\end{question}

\newpage
\begin{question}{Problem 2}
Note that, if $A = 0$, the operator $B = 0$ clearly has the desired properties.
Thus, we assume $A \neq 0$. We first prove the result in the case $A \leq I$,
and then generalize this.
\vspace{-5mm}
\paragraph{Step 1} Assume $A \leq I$. Define $B_0 = 0$ and
\[B_{n + 1} = B_n + \frac{1}{2}(A - B_n^2) \quad \mbox{ for all } n \in \N.\]
It is easily checked by induction that each $B_n$ is as a linear combination of
powers of $A$, and it follows that, $\forall n,m \in \N$, $B_nB_m = B_mB_n$
(since powers of $A$ commute), $B_n^* = B_n$. If for some $n \in \N$,
$0 \leq B_n \leq A$, $\forall x \in X$, then
\[B_{n + 1} = B_n + \frac{1}{2}(A - B_n^2) \geq B_n\]
and, since $B_n \leq I$, so that $B_n^2 \leq B_n$,
\[2B_{n + 1} = 2B_n - B_n^2 + A \leq B_n + A \leq 2A,\]
and hence $B_{n + 1} \leq A$. By induction, then, $\forall n \in \N$,
$0 \leq B_n \leq B_{n + 1} \leq I$. It follows from Problem 1 that
$\{B_n\}_{n = 1}^\infty$ has a limit $B \in \L(X;X)$ (in the strong operator
topology) with $B^* = B$. It is also clear from the construction of this limit
in Problem 1 that $B \geq 0$ (since each $B_n \geq 0$), and that if
$CA = AC$, so that each $CB_n = B_nC$, then $CB = BC$. Also, $\forall x \in X$,
\[Bx
    = \lim_{n \to \infty} B_{n + 1} x
    = \lim_{n \to \infty} \left( B_n + \frac{1}{2} (A - B_n^2)x \right)
    = Bx + \frac{1}{2} (A - B^2)X,
\]
and it follows that $Ax - B^2x = 0$, so that $B^2 = A$.
\vspace{-5mm}
\paragraph{Step 2:} ($A \not\leq I$)
Define $A_1 := \frac{A}{\|A\|}$. For all
$x \in X$, applying Cauchy-Schwarz
\vspace{-3mm}
\[0
    \leq \frac{1}{\|A\|} (Ax,x)
    = (A_1x,x)
    \leq \|A_1\| \|x\|^2
    = \|x\|^2
    = (Ix,x),
\]
and hence $0 \leq A_1 \leq I$. Also, $A_1^* = A_1$ Thus, as already shown,
$\exists B_1 \in \L(X;X)$ such that $B_1^* = B_1$, $B_1 \geq 0$, $B_1^2 = A_1$.
Moreover, if $AC = CA$, then clearly $A_1C = CA_1$, and so $B_1C = CB_1$.

It follows that the operator $B := \sqrt{\|A\|}B_1$ has the desired properties:
$B^* = B$, $B \geq 0$, $B^2 = \sqrt{\|A\|}^2 B_1^2 = A$, and, if $AC = CA$ then
$BC = AC$.

Finally, we show that the positive operator $B$ with $B^2 = A$ is unique.
Suppose $B^2 = C^2 = A$. Define $D := B - C$ and let $x \in X$, $y := Dx$.
Then,
\vspace{-2mm}
\[(By,y) + (Cy,y)
    = ((B + C)y,y)
    = ((B + C)(B - C)x,y)
    = ( (A - A)x,y)
    = 0.
\]
Since $B$ and $C$ are positive, it follows from (\ref{ineq:schwarz}) that
$By = Cy = 0$. Thus,
\vspace{-2mm}
\[\|(B - C)x\|^2
    = \|Dx\|^2
    = (Dx,Dx)
    = (D^2x,x)
    = (Dy,x) = 0,
\]
and hence, since $x$ was chosen arbitrarily $B = C$. \qed
\end{question}

\begin{question}{Problem 4}
If $\alpha = 0$, then, $\forall x \in X$, by Bessel's Inequality,
\[\sum_{n = 1}^\infty |(x_n,x)|^2 \leq \|x\|^2,\]
and so $(x_n,x) \to 0$ as $n \to \infty$. Therefore, $x_n \wto 0$ (weakly) as
$n \to \infty$.

I believe that, if $|\alpha| \in (0,1)$, then $\{x_n\}_{n = 1}^\infty$ need not
converge weakly, but somehow, I wasn't able to produce a counterexample.

I managed to show that, if $\{x_n\}_{n = 1}^\infty$ converges weakly to
$x \in X$, then the sequence
\[\{y_n\}_{n = 1}^\infty
    := \left\{\frac{1}{n} \sum_{k = 1}^n\right\}_{n = 1}^\infty\]
has a subsequence converging strongly to $x$, which might be helpful. In
particular, as $n \to \infty$ $\|y_n\|^2 \to \alpha$, implying
$\|x\|^2 = \alpha$, which I think might help produce a contradiction.
\end{question}

\begin{question}{Problem 6}
For any $\lambda \in \R$, define $B_\lambda \in \L(X;X)$ by
\[(B_\lambda x)_i = \left| \frac{1}{i} - \lambda \right| x_i,
    \quad \forall i \in \N, x \in l^2.\]
Then,
\[(B_\lambda^2 x)_i = \left( \frac{1}{i} - \lambda \right)^2 x_i
    = \left( (A - \lambda I)^2 x \right)_i,
    \quad \forall i \in \N, x \in l^2,\]
and hence $B_\lambda = \sqrt{L(\lambda)^2} = |L(\lambda)|$. Noting that
$B_\lambda x_i = L(\lambda) x_i$ if $i < 1/\lambda$ and
$(B_\lambda x)_i = - (L(\lambda) x)_i$ otherwise, we then have
\[L(\lambda)^+
    = \frac{1}{2} (B_\lambda + L(\lambda))
    = \left\{
        \begin{array}{ll}
            \left( \frac{1}{i} - \lambda \right) x_i & \mbox{ if } i < \frac{1}{\lambda}  \\
            0 & \mbox{ else}
        \end{array}
    \right., \quad \quad \forall i \in \N, \lambda \in \R, x \in l^2,
\]
It follows immediate that $\forall \lambda \in \R$,
\[\Nul(L(\lambda)^+)
    = \{x \in l^2 : x_1 = x_2 = \cdots = x_{i - 1} = 0\}
    = cl(\sp\{e_j : j \geq i\}),
\]
where $i = \lceil1/\lambda\rceil$ is the least integer with $i \geq 1/\lambda$.
\qed
\end{question}

\newpage
\begin{question}{Problem 8}
Suppose, for sake of contradiction, that, for some $T \in \Cmp(X;X)$,
$B := R + T$ is normal. We first show that $\sigma(R + T)$ is uncountable, and
then show that $\sigma(B)$ must be countable, giving a contradiction.

Let $L = R^*$ denote the left-shift operator. If $\lambda \in \C$ with
$|\lambda| < 1$ and $x = (1,\lambda,\lambda^2,\ldots) \in l^2$, then
$(L - \lambda I)x = 0$, and it follows that
$\{\lambda \in \C : |\lambda| < 1\} \subseteq \sigma_p(L)\}$. Since
$(L - \lambda I) = (R - \bar{\lambda} I)^*$ is bijective if and only if
$(R - \bar{\lambda} I)$ is bijective,
$\{\lambda \in \C : |\lambda| < 1\} \subseteq \sigma(R)$, and hence
$\sigma(R)$ is uncountable. In particular, since $R$ has no eigenvalues,
$\sigma(R)\sminus\sigma_p(R)$ is uncountable, and so, by the result of Problem
7, $\sigma(R + T)$ is uncountable.

\paragraph{Lemma 1} If $T \in \C(X;X)$, then $\sigma(T)$ is countable.

{\emph Proof:} For all $n \in \N$, define
$\Lambda_n := \{\lambda \in \sigma(T) : |\lambda| \geq 1/n\}$. Then,
\[\sigma(T) = \bigcup_{n \in \N} \Lambda_n.\]
Since $0$ is the only accumulation point of $\sigma(T)$, each $\Lambda_n$ is
finite, and hence $\sigma(T)$ is countable. $\square$

Now note that
\[I = LR = R^*R = (B - T)^*(B - T) = B^*B - B^*T - T^*B + T^*T,\]
and hence
\[B^*B = I + B^*T + T^*B - T^*T.\]
Let $C := B^*T + T^*B - T^*T$, and note that, since any finite sum of compact
operators or composition of compact and bounded operators is compact, $C$ is
compact. Also, $I + C - \lambda I = C - (1 + \lambda)I$, and so
$\sigma(I + C) = \{\lambda - 1 : \lambda \in \sigma(C)\}$. By Lemma 1, then,
$\sigma(B^*B) = \sigma(I + C)$ is countable. Since $B$ is normal, it should
follow that $\sigma(B)$ is countable, although I wasn't able to show this
clearly. \qed
\end{question}

\begin{question}{Problem 11}
The inclusion $\bdry(\sigma(A)) \subseteq \sigma_{ap}(A)$ holds.

Suppose $\lambda \in \bdry(\sigma_p(A))$. By definition of the
boundary, there is a sequence $\{\lambda_n\}_{n = 1}^\infty$ with each
$\lambda_n \in \K \sminus \sigma(A) = \rho(A)$. Since $\sigma(A)$ is closed,
$\lambda \in \sigma(A)$. By (the contrapositive of) part (iii) of
Proposition 5.6, $\forall i \in \N$,
\[\|(\lambda_n I - A)\inv\| \geq \frac{1}{|\lambda_n - \lambda|} \to \infty
\quad \mbox{as } n \to \infty.\]
It follows that there is a sequence $\{y_n\}_{n = 1}^\infty$ in $Y$ such that
each $y_n \to 0$ as $n \to \infty$ and each
$x_n := (A - \lambda_n I)\inv y_n$ has $\|x_n\| = 1$. Thus,
\[\lim_{n \to \infty} \|(\lambda I - A) x_n\| \to 0,\]
and hence $\lambda \in \sigma_{ap}(A)$. \qed
\end{question}
\end{document}
