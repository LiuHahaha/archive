\documentclass[11pt]{article}
\usepackage{enumerate}
\usepackage{fullpage}
\usepackage{fancyhdr}
\usepackage{amsmath, amsfonts, amsthm, amssymb}
\usepackage{color}
\setlength{\parindent}{0pt}
\setlength{\parskip}{5pt plus 1pt}
\pagestyle{empty}

\def\indented#1{\list{}{}\item[]}
\let\indented=\endlist

\newcounter{questionCounter}
\newcounter{partCounter}[questionCounter]
\newenvironment{question}[2][\arabic{questionCounter}]{%
    \setcounter{partCounter}{0}%
    \vspace{.25in} \hrule \vspace{0.5em}%
        \noindent{\bf #2}%
    \vspace{0.8em} \hrule \vspace{.10in}%
    \addtocounter{questionCounter}{1}%
}{}
\renewenvironment{part}[1][\alph{partCounter}]{%
    \addtocounter{partCounter}{1}%
    \vspace{.10in}%
    \begin{indented}%
       {\bf (#1)} %
}{\end{indented}}

%%%%%%%%%%%%%%%%%%%%%%%HEADER%%%%%%%%%%%%%%%%%%%%%%%%%%%%%%
\newcommand{\myname}{Shashank Singh\footnote{sss1@andrew.cmu.edu}}
\newcommand{\myclass}{21-740 Introduction to Functional Analysis II}
\newcommand{\myhwnum}{3}
\newcommand{\duedate}{Friday, November 8, 2013}
%%%%%%%%%%%%%%%%%%%%%%%%%%%%%%%%%%%%%%%%%%%%%%%%%%%%%%%%%%%

%%%%%%%%%%%%%%%%%%%%CONTENT MACROS%%%%%%%%%%%%%%%%%%%%%%%%%
\renewcommand{\qed}{\quad \ensuremath{\blacksquare}}
\newcommand{\inv}{^{-1}}
\newcommand{\bv}{\mathbf{v}}
\newcommand{\bx}{\mathbf{x}}
\newcommand{\by}{\mathbf{y}}
\newcommand{\bff}{\mathbf{f}}
\newcommand{\bzero}{\mathbf{0}}
\newcommand{\bxi}{\boldsymbol{\xi}}
\newcommand{\boldeta}{\boldsymbol{\eta}}
\newcommand{\dist}{\operatorname{dist}} % distance from or between sets
\newcommand{\area}{\operatorname{area}} % area of a polygon
\newcommand{\Gr}{\operatorname{Gr}}     % graph of a function
\renewcommand{\sp}{\operatorname{span}} % span of a set
\newcommand{\bdry}{\operatorname{bdry}} % boundary of a set
\newcommand{\sminus}{\backslash}        % set difference
\newcommand{\N}{\mathbb{N}}             % natural numbers
\newcommand{\Z}{\mathbb{Z}}             % integers
\newcommand{\Q}{\mathbb{Q}}             % rational numbers
\newcommand{\R}{\mathbb{R}}             % real numbers
\newcommand{\C}{\mathbb{C}}             % complex numbers
\newcommand{\D}{\mathcal{D}}            % domain of an operator
\newcommand{\Cmp}{\mathcal{C}}          % space of compact linear operators\s
\newcommand{\K}{\mathbb{K}}             % underlying field of a linear space
\newcommand{\Ran}{\mathcal{R}}          % range of a linear operator
\newcommand{\Nul}{\mathcal{N}}          % null-space of a linear operator
\renewcommand{\L}{\mathcal{L}}          % space of bounded linear functions
\newcommand{\pow}[1]{\mathcal{P}\left(#1\right)}    % power set of #1
\newcommand{\e}{\varepsilon}            % \varepsilon
\newcommand{\wto}{\rightharpoonup}      % weak convergence
\newcommand{\wsto}{\stackrel{*}{\rightharpoonup}}   % weak-* convergence
%%%%%%%%%%%%%%%%%%%%%%%%%%%%%%%%%%%%%%%%%%%%%%%%%%%%%%%%%%%

\begin{document}
\thispagestyle{plain}

{\Large Homework \myhwnum} \\
\myclass \\
Name: \myname \\
Due: \duedate

\begin{question}{Problem 1}
A composition of closed linear operators is not, in general, closed.

Consider the Banach space $X := C([0,1];\R)$ of continuous function from
$[0,1]$ to $\R$, under the supremum norm $\|f\| := \sup_{x \in [0,1]} |f(x)|$.

Let $\D(B) = C^1([0,1];\R) \subseteq X$ and let $B : \D(B) \to X$ be the
differentiation operator $Bf := f'$, $\forall f \in \D(B)$. It is a well-known
result that if a sequence $\{f_n\}_{n = 1}^\infty$ of differentiable functions
converges uniformly to $f$ and $\{f_n'\}_{n = 1}^\infty$ converges uniformly to
$g$, then $f \in C^1([0,1];\R)$ and $g = f'$. It follows that $B$ is closed.

Now define $A : X \to X$ by $(Af)(x) := f(0)$, $\forall f \in X, x \in [0,1]$.
$A$ is continuous and hence closed.

For all $n \in \N$, define $f_n \in \D(B)$ by
\[f_n(x) := \frac{e^{-nx}}{n}, \quad \forall x \in [0,1].\]
Then, $\|f_n\| = f_n(0) = 1/n$ and $f'(0) = -1$, $\forall n \in \N$, so that, as
$n \to \infty$, $f_n \to 0$ and $ABf_n \to -1 \neq 0 = AB(0)$. Thus, $AB$ is
not closed.
\end{question}

\begin{question}{Problem 2}
($\Rightarrow$) Suppose $A$ is closable and $(0,y) \in cl(\Gr(A))$. If $\bar A$
denotes the closure of $A$, then $(0,y) \in \Gr(\bar A)$. Since $\bar A$ is a
linear operator $(0,0) \in \Gr(\bar A)$, and it follows that $y = 0$.

($\Leftarrow$) Suppose that
\[\forall (x,y) \in cl(\Gr(A)), \quad x = 0 \Rightarrow y = 0.\]
First note that $cl(\Gr(A))$ is a linear manifold. Define
\[\D(\bar A) := \{x : (x,y) \in cl(\Gr(A)) \mbox{ for some } y \in X\}.\]
It is easily checked that $\D(\bar A)$ is a linear manifold. If
$(x,y_1),(x,y_2) \in cl(\Gr(A))$, then
\[(0,y_1 - y_2) = (x - x, y_1 - y_2) \in cl(\Gr(A)),\] and hence $y_1 = y_2$ by
the given condition. Thus, we can define an operator
$\bar A : \D(\bar A) \to Y$ by $(x, \bar Ax) \in cl(\Gr(A))$ for each
$x \in cl(\D(A))$ and have that $cl(\Gr A) = \Gr(\bar A)$, so that $\bar A$ is
closed. Thus, $A$ is closable (and $\bar A$ is the closure of $A$). \qed
\end{question}

\newpage
\begin{question}{Problem 3}
\begin{enumerate}[(a)]
\item $A$ is not closed. Define the sequence $\{x_n\}_{n = 1}^\infty$ in
$\K^{(\N)}$ by
\[x_n^{(k)} = \left\{
        \begin{array}{ll}
            k^{-3}  & \mbox{if } k \leq n   \\
            0       & \mbox{if } k > n
        \end{array}
    \right., \quad \forall n,k \in \N.
\]
Then, defining $x = (1^{-3},2^{-3},3^{-3},\dots)$, $x_n \to x$ as
$n \to \infty$. Also,
\[(Ax_n)^{(k)} = \left\{
        \begin{array}{ll}
            \sum_{i = 1}^n i^{-2}   & \mbox{if } k = 1          \\
            k^{-3}                  & \mbox{if } 1 < k \leq n   \\
            0                       & \mbox{if } k > n
        \end{array}
    \right., \quad \forall n,k \in \N,
\]
so $Ax_n \to (\pi^2/6,2^{-3},3^{-3},\dots)$ as $n \to \infty$. Since
$x \notin \K^{(\N)}$, $\Gr(A)$ is not closed. \qed

\item $A$ is not closable. Define the sequence $\{x_n\}_{n = 1}^\infty$ in
$\K^{(\N)}$ by
\[x_n^{(k)} = \left\{
        \begin{array}{ll}
            (nk)^{-1}  & \mbox{if } k \leq n   \\
            0       & \mbox{if } k > n
        \end{array}
    \right., \quad \forall n,k \in \N.
\]
Then, $\|x_n\|_p \leq \frac{H_n}{n} \to 0$ as $n \to \infty$ (where $H_n$
denotes the $n^{th}$ harmonic number). Also, noting
$\sum_{i = 1}^n k(nk)^{-1} = 1$ for all $n \in \N$,
\[(Ax_n)^{(k)} = \left\{
        \begin{array}{ll}
            1           & \mbox{if } k = 1          \\
            (nk)^{-1}   & \mbox{if } 1 < k \leq n   \\
            0           & \mbox{if } k > n
        \end{array}
    \right., \quad \forall n,k \in \N,
\]
so, $Ax_n \to (1,0,0,\dots) \neq 0$ as $n \to \infty$. By the result of Problem
2, $A$ is not closable. \qed
\end{enumerate}
\end{question}

\begin{question}{Problem 4}
As the following translation semigroup shows, it may be the case that
$T(t) = 0$ for some $t \geq 0$.

Let
\[X := \left\{ f \in C([0,1];\R) : f(1) = 0 \right\}\]
and consider the mapping $T : [0,\infty) \to \L(X;X)$ given by
\[(T(t)f)(x)
    = \left\{
        \begin{array}{ll}
        0           & \mbox{ if } x + t > 1\\
        f(x + t)    & \mbox{ else }
        \end{array}
    \right., \forall f \in X, t \geq 0, x \in [0,1]
\]

Since $T$ simply applies a left-shift, clearly $T(0) = I$ and,
$\forall s,t \geq 0$, $T(s + t) = T(s)T(t)$. Since $[0,1]$ is compact, every
$f \in X$ is uniformly continuous, and hence (since each $f(1) = 0$), $T$ is
strongly continuous, so that $T$ is a linear $C_0$-semigroup. Furthermore, it
is clear that, $\forall t \geq 1$, $T(t) = 0$.

It's probably worth noting that this is not possible with $T$ continuous in the
uniform operator topology, since $T$ is then the exponential of its generator
and is therefore invertible.
\end{question}

\newpage
\begin{question}{Problem 5}
\begin{enumerate}[(a)]
\item By definition of $T$, $\forall t \geq 0, x \in X_0$, $T(t)x \in X_0$.
Suppose that, for some $n \in \{1,\dots,N\}$, for all $x \in X_{n - 1}$,
$T(t)x \in X_{n - 1}$ for all $t \geq 0$. Let $t \geq 0, x \in X_n$. Since
$X_n \subseteq X_{n - 1}$, $T(t)x \in X_{n - 1}$. Since $Ax \in X_{n - 1}$, by
part (b) of Lemma 9.2, $AT(t)x = T(t)Ax \in X_{n - 1}$, so $T(t) \in X_n$, and
the desired result follows by induction. \qed

\item
Suppose that, for some $n \in \{1,\dots,N\}$, $T$ is continuous in the strong
operator topology on $(X_{n - 1},\|\cdot\|_{n - 1})$. If $x \in X_n, t \geq 0$,
then by part (b) of Lemma 9.2
\[\|T(t)x - x\|_n
    = \|T(t)x - x\|_{n - 1} + \|A(T(t)x - x)\|_{n - 1}
    = \|T(t)x - x\|_{n - 1} + \|T(t)Ax - Ax\|_{n - 1}
    \to 0
\]
as $t \downarrow 0$, since $Ax \in X_{n - 1}$. Thus, by induction, $T$ is
continuous in the strong operator topology on each $(X_n,\|\cdot\|_n)$.

Since $T$ is continuous in the strong operator topology on $(X_n,\|\cdot\|_n)$,
$u \in C^0([0,\infty);X_N)$.

Suppose that, for some $n \in \{1,\dots,N\}$, we have
$u \in C^{n - 1}([0,\infty);X_{N - (n - 1)})$ and
$u^{(n - 1)}(t) = T(t)A^{n - 1}x$. Then, for $t \geq 0$, $x \in X_{N - n}$.
\[\lim_{h \downarrow 0} \frac{u^{(n - 1)}(t + h) - u^{(n - 1)}(t)}{h}
    = T(t)\lim_{h \downarrow 0} \frac{T(h)A^{n - 1}x - A^{n - 1}x}{h}
    = T(t)A^nx
,\]
so $u \in C^n([0,\infty);X_{N - n})$ and $u^{(n)} = T(t)A^nx$. The desired
result follows by induction. \qed
\end{enumerate}
\end{question}

\begin{question}{Problem 6}
By the Hille-Yosida Theorem, we have, for some $M > 0,\omega \in \R$,
\begin{enumerate}[(i)]
\item $\D(A)$ is dense in $X$ and $A$ is closed.
\item $\rho(A) \supseteq (\omega,\infty)$.
\item $\displaystyle \|R(\lambda;A)^n\| \leq \frac{M}{(\lambda - \omega)^n}$,
$\forall n \in \N, \lambda > \omega$.
\end{enumerate}
Furthermore, also by the Hille-Yosida Theorem, it suffices to show
\begin{enumerate}[(i)]
\setcounter{enumi}{3}
\item $\D(A + L)$ is dense in $X$ and $A + L$ is closed.
\item $\rho(A + L) \supseteq (\omega + \|L\|,\infty)$.
\item $\displaystyle \|R(\lambda;A + L)^n\|
            \leq \frac{M}{(\lambda - (\omega + M\|L\|))^n}$,
$\forall n \in \N, \lambda > \omega$.
\end{enumerate}

Since $L$ is continuous, $\D(A + L) = \D(A)$ and $A + L$ is closed.

Also, since $L$ is bounded, it can shift the spectrum by at most $\|L\|$, so
$(\omega + \|L\|,\infty) \subseteq \rho(A + L)$.

It follows from (iii) that, for $\lambda > \omega + M\|L\|$,
\[\|LR(\lambda;A)\|
    \leq \frac{M\|L\|}{\lambda - \omega}
    < \frac{M\|L\|}{M\|L\|}
    = 1.
\]
\paragraph{Lemma 1} For $\lambda > \omega + M\|L\|$ (so
$\lambda \in \rho(A + L)$),
\[R(\lambda; A + L)
    = R(\lambda; A) \left( I - LR(\lambda; A) \right)\inv
    = \sum_{k = 0}^\infty R(\lambda; A) \left( LR(\lambda; A) \right)^k.
\]
\emph{Proof:} Since
\[
\left( \lambda I - (A + L) \right) R(\lambda;A)
    = (\lambda I - A)R(\lambda;A) - LR(\lambda; A)
    = \left( I - LR(\lambda; A) \right),
\]
\[\left( \lambda I - (A + L) \right) R(\lambda;A)
                                        \left( I - LR(\lambda; A)\right) \inv
    = I.
\]
Since $\lambda \in \rho(A + L)$, $\lambda I - (A + L)$ is bijective, and hence
its right inverse is its inverse. $\square$

Taking norms and reindexing terms in terms of powers of $\|L\|$,
\[\|R(\lambda; A + L)^n\|
    = \left\|\left[ \sum_{k = 0}^\infty R(\lambda; A) \left( LR(\lambda; A) \right)^k \right]^n
\right\|
    \leq \sum_{k = 0}^\infty \binom{n}{k} \frac{M^{k + 1}\|L\|^k}
                                                {(\lambda - \omega)^{n + k}}
\]

Then, since $\lambda > \omega + M\|L\|$, using the identity
$\frac{1}{(1 - x)^n} = \sum_{k = 0}^\infty \binom{n}{k} x^k$.
\begin{align*}
\|R(\lambda; A + L)^n\|
 &  \leq \sum_{k = 0}^\infty \binom{n}{k} \frac{M^{k + 1}\|L\|^k}
                                            {(\lambda - \omega)^{n + k}}    \\
 &  = \frac{M}{(\lambda - \omega)^n}
        \sum_{k = 0}^\infty \binom{n}{k} \left( \frac{M\|L\|}
                                {\lambda - \omega} \right)^k    \\
 &  = \frac{M}{(\lambda - \omega)^n} \cdot \left(1 - \frac{M\|L\|}
                                                {\lambda - \omega}\right)^{-n}         \\
 &  = \frac{M}{(\lambda - (\omega + M\|L\|))^n}. \qed
\end{align*}
\end{question}

\newpage
\begin{question}{Problem 7}
I wasn't able to prove this on my own. This proof is based on the proof of
Theorem 1.6 in Engel and Nagel's
\emph{A Short Course on Operator Semigroups}.

By the Principle of Uniform Boundedness, $\forall t \geq 0$,
$T(t) \in \L(X;X)$, and hence, again by the Principle of Uniform Boundedness,
$\exists M, \delta > 0$ such that $\|T(t)\| \leq M$ for all $t \in [0,\delta]$.
Thus, it suffices to show that the set
\[E := \left\{ x \in X : \lim_{t \downarrow 0} T(t)x = x \right\}\]
is dense in $X$. In particular, since $E$ is convex, it suffices to show $E$ is
weakly dense in $X$.

$\forall x \in X, r > 0$, define $x_r \in X^{**}$ by
\[x_r(x^*)
    := \frac{1}{r} \int_0^r x^*(T(t)x) \, dt, \quad \forall x^* \in X^*.
\]
Let $x \in X$ and suppose $r > 0$. Since $[0,r]$ is compact and the mapping
$s \to T(s)x$ is weakly continuous, $S := \{T(s)x : s \in [0,r]\}$ is weakly
compact. Thus, the closure of the convex hull of $S$, $cl(co(S))$, is weakly
compact. Since $x_r$ is a limit of Riemann sums, $x_r \in cl(co(S))$, and so
$x_r \in X$. Since the set $D := \{x_r : r > 0, x \in X\}$ is weakly dense in
$X$, it suffices to show $D \subseteq E$.
\begin{align*}
\|T(t)x_r - x_r\|
    & = \sup_{\|x^*\| \leq 1} \frac{1}{r}
            \left| \int_t^{t + r} x^*(T(s)x) \, ds
                - \int_0^r x^*(T(s)x) \, ds\right|  \\
    & \leq \sup_{\|x^*\| \leq 1} \frac{1}{r}
            \left( \left| \int_r^{r + t} x^*(T(s)x) \, ds\right|
                + \left|\int_0^r x^*(T(s)x) \, ds \right| \right)  \\
    & \leq \frac{2t}{r} \|x\| \sup_{0 \leq s \leq r + t} \|T(s)\|
      \to 0 \quad \mbox{ as } t \downarrow 0. \qed
\end{align*}
\end{question}

\begin{question}{Problem 8}
We first note that, $\forall u \in X, t > 0$,
$T(t)u = k_t * u$, where $k_t(x) = (\pi t)^{-1/2}e^{-x^2/t}$,
$\forall x \in \R$. Then, $\{k_t\}_{t \geq 0}$ is an approximate identity (that
is, $\forall t \geq 0$, $\int_\R k_t = 1$ and $\forall \delta > 0$,
$\int_{\R\sminus(-\delta,\delta)} k_t(y) \, dy \to 0$ as $t \to 0$). We showed
in Measure Theory that this implies $\|k_t * f - f\|_2 \to 0$ as
$t \to 0$, $\forall f \in L^2(\R)$. Thus, $T$ is continuous in the strong
operator topology. If $s,t \geq 0$, changing coordinates,
\begin{align*}
(T(s)T(t)u)(x)
 &  = (\pi st)^{-1/2} \int_{-\infty}^\infty \int_{-\infty}^\infty
        e^{-\frac{ty^2 + sz^2}{st}} u(x - y - z) \, dy \, dz    \\
 &  = \left( \frac{st}{s + t} \right)^{1/2} (\pi st)^{-1/2}
        \int_{-\infty}^\infty e^{-\frac{y^2}{s + t}} u(x - y) \, dy
    = (T(s + t)u)(x),
\end{align*}
so we have the semigroup property. From reading Engel and Nagel's \emph{A Short
Course on Operator Semigroups}, I understand that the infinitesimal generator
of $T$ is the second derivative $A : C^2(\R) \cap L^2(\R) \to X$ defined by
$Au = u''$, $\forall u \in C^2(\R) \cap L^2(\R)$, but I wasn't able to show
this. It seems to require Fourier transforms, which I'm not too familiar with.
\end{question}
\end{document}
