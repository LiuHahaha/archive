\documentclass[11pt]{article}
\usepackage{enumerate}
\usepackage{fullpage}
\usepackage{fancyhdr}
\usepackage{amsmath, amsfonts, amsthm, amssymb}
\usepackage{color}
\setlength{\parindent}{0pt}
\setlength{\parskip}{5pt plus 1pt}
\pagestyle{empty}

\def\indented#1{\list{}{}\item[]}
\let\indented=\endlist

\newcounter{questionCounter}
\newcounter{partCounter}[questionCounter]
\newenvironment{question}[2][\arabic{questionCounter}]{%
    \setcounter{partCounter}{0}%
    \vspace{.25in} \hrule \vspace{0.5em}%
        \noindent{\bf #2}%
    \vspace{0.8em} \hrule \vspace{.10in}%
    \addtocounter{questionCounter}{1}%
}{}
\renewenvironment{part}[1][\alph{partCounter}]{%
    \addtocounter{partCounter}{1}%
    \vspace{.10in}%
    \begin{indented}%
       {\bf (#1)} %
}{\end{indented}}

%%%%%%%%%%%%%%%%%%%%%%%HEADER%%%%%%%%%%%%%%%%%%%%%%%%%%%%%%
\newcommand{\myname}{Shashank Singh\footnote{sss1@andrew.cmu.edu}}
\newcommand{\myclass}{21-740 Introduction to Functional Analysis II}
\newcommand{\myhwnum}{4}
\newcommand{\duedate}{Monday, November 25, 2013}
%%%%%%%%%%%%%%%%%%%%%%%%%%%%%%%%%%%%%%%%%%%%%%%%%%%%%%%%%%%

%%%%%%%%%%%%%%%%%%%%CONTENT MACROS%%%%%%%%%%%%%%%%%%%%%%%%%
\renewcommand{\qed}{\quad \ensuremath{\blacksquare}}
\newcommand{\inv}{^{-1}}
\newcommand{\bv}{\mathbf{v}}
\newcommand{\bx}{\mathbf{x}}
\newcommand{\by}{\mathbf{y}}
\newcommand{\bff}{\mathbf{f}}
\newcommand{\bzero}{\mathbf{0}}
\newcommand{\bxi}{\boldsymbol{\xi}}
\newcommand{\boldeta}{\boldsymbol{\eta}}
\newcommand{\dist}{\operatorname{dist}} % distance from or between sets
\newcommand{\area}{\operatorname{area}} % area of a polygon
\newcommand{\Gr}{\operatorname{Gr}}     % graph of a function
\renewcommand{\sp}{\operatorname{span}} % span of a set
\newcommand{\bdry}{\operatorname{bdry}} % boundary of a set
\newcommand{\sminus}{\backslash}        % set difference
\newcommand{\N}{\mathbb{N}}             % natural numbers
\newcommand{\Z}{\mathbb{Z}}             % integers
\newcommand{\Q}{\mathbb{Q}}             % rational numbers
\newcommand{\R}{\mathbb{R}}             % real numbers
\newcommand{\C}{\mathbb{C}}             % complex numbers
\newcommand{\D}{\mathcal{D}}            % domain of an operator
\newcommand{\Cmp}{\mathcal{C}}          % space of compact linear operators\s
\newcommand{\K}{\mathbb{K}}             % underlying field of a linear space
\newcommand{\Ran}{\mathcal{R}}          % range of a linear operator
\newcommand{\Nul}{\mathcal{N}}          % null-space of a linear operator
\renewcommand{\L}{\mathcal{L}}          % space of bounded linear functions
\newcommand{\pow}[1]{\mathcal{P}\left(#1\right)}    % power set of #1
\newcommand{\e}{\varepsilon}            % \varepsilon
\newcommand{\wto}{\rightharpoonup}      % weak convergence
\newcommand{\wsto}{\stackrel{*}{\rightharpoonup}}   % weak-* convergence
\renewcommand{\Re}{\operatorname{Re}}   % real part of a complex number
\newcommand{\tT}{\widetilde{T}}         % for P3
\newcommand{\A}{\mathcal{A}}            % for P3
%%%%%%%%%%%%%%%%%%%%%%%%%%%%%%%%%%%%%%%%%%%%%%%%%%%%%%%%%%%

\begin{document}
\thispagestyle{plain}

{\Large Homework \myhwnum} \\
\myclass \\
Name: \myname \\
Due: \duedate

\begin{question}{Problem 1}
\vspace{-6mm}
\paragraph{Lemma 1} Suppose $\D(B) \subseteq X$ is dense in $X$,
$B : \D(B) \to X$ is linear and closed. Then, for any integrable
$g : [0,\tau] \to \D(B)$, $t \in [0,\tau]$, if $B \circ g$ is integrable, then
\[B\int_0^t g(s) \, ds = \int_0^t Bg(s) \, ds.\]

\emph{Proof:} Claim 1: Suppose $g$ is a simple function, i.e., for some
$k \in \N$, $c_1,\dots,c_k \in \D(B)$,
\[g = \sum_{i = 1}^k c_i \chi_{E_i},\]
where $\{E_i : i = 1,\dots,k\}$ is a partition of $[0,\tau]$ into
Lebesgue-measurable sets and $\chi_{E_i}$ is the characteristic function of
$A_i$ (we refer to $\{E_i : i = 1,\dots,k\}$ as the \emph{partition underlying
$g$}). Then,
\[B\int_0^t g(s) \, ds
    = B \sum_{i = 1}^k c_i \lambda(E_i)
    = \sum_{i = 1}^k B c_i \lambda(E_i)
    = \int_0^t B g(s) \, ds,
\]
where $\lambda$ denotes the Lebesgue measure, proving Claim 1.

General Case: Now consider a sequences of
simple functions $\{g_k\}_{k = 1}^\infty$ mapping $[0,\tau]$ to $\D(B)$ and
$\{h_k\}_{k = 1}^\infty$ mapping $[0,\tau]$ to $X$ such that
\[\sup_{t \in [0,\tau]} \|g(t) - g_k(t)\|_X \to 0
\quad \mbox{ and } \quad
\sup_{t \in [0,\tau]} \|Bg(t) - h_k(t)\|_X \to 0
\quad \mbox{ as } k \to \infty
\]
(such sequences exist because $\D(B)$ is dense in $X$ and both $g$ and
$B \circ g$ are integrable.) In particular, we can take such sequences such
that $g_k(s) = g(s)$ for some $s$ in each set of the partition underlying $g$,
and, $\forall k \in \N$, the partition of $[0,\tau]$ underlying $g_k$ is a
refinement of the partition underlying $h_k$. It follows from these constraints
that
\[\sup_{t \in [0,\tau]} \|Bg_k(t) - h_k(t)\|_X \to 0
    \quad \mbox{ as } k \to \infty,
\]
and so, by Claim 1,
\[\lim_{k \to \infty} B \int_0^t g_k(s) \, ds
    = \lim_{k \to \infty} \int_0^t Bg_k(s) \, ds
    = \lim_{k \to \infty} \int_0^t h_k(s) \, ds
    = \int_0^t Bg(s) \, ds
.\]
Since $B$ is closed, $\int g_k \to \int g$, and
$B \circ \int g_k \to \int B \circ g$, it follows that
$\int g : [0,\tau] \to \D(B)$ and $B \int g = \int B \circ g$. $\square$

We calculate
\begin{align*}
\lim_{h \to 0} \frac{1}{h} \left( \int_0^t T(t + h - s) f(s) \, ds
 - \int_0^t T(t - s) f(s) \, ds \right)
 &  = \lim_{h \to 0} \frac{1}{h} \int_0^t (T(h) - I) T(t - s) f(s) \, ds    \\
 &  = \lim_{h \to 0} \frac{(T(h) - I)}{h} \int_0^t T(t - s) f(s) \, ds      \\
 &  = A \int_0^t T(t - s) f(s) \, ds = Av(t)
\end{align*}
and
\begin{align*}
\lim_{h \to 0} \frac{1}{h} \int_t^{t + h} T(t + h - s) f(s) \, ds
 &  = \lim_{h \to 0} T(h) \frac{1}{h} \int_t^{t + h} T(t - s) f(s) \, ds    \\
 &  = \lim_{h \to 0} T(h) \frac{1}{h} \int_t^{t + h} T(s) f(t - s) \, ds    \\
 &  = I f(t) = f(t).
\end{align*}
Thus, $\dot v(t) = Av(t) + f(t), \forall t \in [0,\tau]$.
\end{question}

\begin{question}{Problem 2}
By a generation theorem from class, there exist constants $C,\omega \in \R$ and
$\delta \in (0,\pi/2)$ such that
\[\left\{\lambda \in \C
            : |\arg(\lambda - \omega)| < \frac{\pi}{2} + \delta\right\}
    \subseteq \rho(A)\]
and
\[\|R(\lambda;A)\|
    \leq \frac{C}{|\lambda - \omega|},
    \quad \forall \lambda \in \C
            \mbox{ with } |\arg(\lambda - \omega)| < \frac{\pi}{2} + \delta.
\]
Furthermore, by another part of the same theorem, it suffices to show that
\[\{\lambda \in \C : \Re(\lambda) > \omega + \|L\|\}
    \subseteq \rho(A + L)
\quad \mbox{ and } \quad
\|R(\lambda;A + L)\| \leq \frac{C}{\lambda - (\omega + C\|L\|)},\]
whenever $\Re \lambda > \omega + C\|L\|$.
Since $L$ is bounded, it can shift the spectrum by at most $\|L\|$, and so
\[
\{\lambda \in \C : \Re(\lambda) > \omega + \|L\|\}
    \subseteq \left\{\lambda \in \C
        : |\arg(\lambda - (\omega + \|L\|))| < \frac{\pi}{2} + \delta\right\}
    \subseteq \rho(A + L).
\]
It is easily checked from the definition of the resolvent operator that
\[R(\lambda; A + L)
    = \sum_{k = 0}^\infty R(\lambda; A) \left( LR(\lambda; A) \right)^k
\]
(we showed this as a lemma for Problem 6 of Assignment 3). Thus, by the
triangle inequality and the identity $\frac1{1 - x} = \sum_{i = 0}^\infty x^i$,
for $\Re \lambda - \omega > C\|L\|$,
\begin{align*}
\|R(\lambda; A + L)\|
    \leq \sum_{k = 0}^\infty \frac{C^{k + 1}\|L\|^k}
                                            {(\lambda - \omega)^{k + 1}}
 &  = \frac{C}{\lambda - \omega} \sum_{k = 0}^\infty \frac{C^k\|L\|^k}
                                                    {(\lambda - \omega)^k}  \\
 &  = \frac{C}{\lambda - \omega}
                    \left( 1 - \frac{C\|L\|}{\lambda - \omega} \right)\inv  \\
 &  = \frac{C}{\lambda - \omega}
        \left( \frac{\lambda - \omega}{\lambda - \omega - C\|L\|} \right)
    = \frac{C}{\lambda - \omega - C\|L\|}. \qed
\end{align*}
\end{question}

\vspace{-8mm}
\begin{question}{Problem 3}
We assume without loss of generality that $T$ is quasicontractive, since there
exists an equivalent norm under which $T$ is quasicontractive, and it suffices
to prove the result for an equivalent norm. In particular, suppose
$\omega \in \R$ with $\|T(t)\| \leq e^{\omega t}, \forall t \geq 0$.
Note that $\forall c \in \R$,
\[\left\| e^{cI} \right\|
    = \left\| \sum_{k = 0}^\infty \frac{(cI)^k}{k!} \right\|
    = \left\| \sum_{k = 0}^\infty \frac{c^k}{k!} I\right\|
    = \|e^cI\|.
    = e^c.
\]
Since $\|T(t)\| \leq e^{\omega t}$, there exists $\e > 0$ such
\[\frac{\|T(t)\| - 1}{h}
    \leq 1 + \frac{d}{dt} e^{\omega t} \bigg|_{t = 0}
    = \omega + 1,
\]
for all $h \in (0,\e)$. Since, $\forall t \geq 0, h \in (0,\e)$, since $T(h)$
and $I$ commute,
\begin{align*}
\left\| e^{t\A_h} \right\|
&   = \left\| \exp \left( \frac{t(T(h) - I)}{h} \right) \right\|
    = \left\| \exp \left( \frac{tT(h)}{h} \right) \right\|
                                \left\| \exp \left( -tI/h \right) \right\| \\
&   \leq \exp \left( t\|T(h)\|/h \right) \exp \left( -t/h \right)
    = \exp \left( \frac{t(\|T(h)\| - 1)}{h} \right)
    \leq e^{(\omega + 1) t}
\end{align*}
By the Fundamental Theorem of Calculus, $\forall t \geq 0, h \in (0,\e),
x \in X$
\begin{align*}
T(t) x - e^{t\A_h} x
 &  = \int_0^t \frac{d}{ds} e^{(t - s)\A_h} T(s) x \, ds    \\
 &  = \int_0^t e^{(t - s)\A_h} (A - \A_h) T(s) x \, ds      \\
 &  = \int_0^t e^{(t - s)\A_h} T(s) (Ax - \A_hx) \, ds
\end{align*}
since $A$ and each $\A_h$ commutes with $T(t)$ for all $t \geq 0$. Then,
$\forall t \geq 0, x \in X$,
\begin{align*}
\|T(t) x - e^{t\A_h} x\|
 &  \leq \int_0^t \|e^{(t - s)\A_h}\| \|T(s)\| \|(Ax - \A_hx)\| \, ds   \\
 &  \leq te^{(2\omega + 1)t} \|(Ax - \A_hx)\| \to 0,
\end{align*}
as $h \to 0$, by definition of $A$ and $\A_h$. \qed
\end{question}

\begin{question}{Problem 4}
If $x \in \D(A)$ is non-zero, then
\[
\Re [Ax,x]
    = \|x\|^2 \Re \left[ A\frac{x}{\|x\|}, \frac{x}{\|x\|} \right]
    \leq 0.
\]

Thus, $A$ is dissipative. As in the proof of Lemma 10.10, we have,
$\forall x \in X, \lambda \in \C$,
\begin{align*}
\|(\lambda I - A)x\| \|x\|
    \geq |[(\lambda I - A)x,x]|
    = \Re[(\lambda I - A)x,x]
    = \Re \lambda \|x\|^2 - \Re [Ax,x]
    \geq \Re \lambda \|x\|^2.
\end{align*}
It follows that, for $\Re \lambda > 0$, $\lambda I - A$ is
injective and, if $\lambda I - A$ is surjective, then
$R(\lambda;A) \leq \frac{1}{\Re \lambda}$.

By a generation theorem for analytic semigroups, it suffices now to show that
\[H := \{\lambda \in \C : \Re \lambda > 0\} \subseteq \rho(A),\]
for which we essentially follow the proof of Lemma 10.11.

Define $\Lambda := \rho(A) \cap H$. Since
$\rho(A)$ is open in $\C$, $\Lambda \neq \emptyset$ by assumption, and $H$ is
connected, it suffices to show that $\Lambda$ is closed in the relative
topology on $H$. Let $\{\lambda_k\}_{k = 1}^\infty$ be a sequence in $\Lambda$
converging to $\lambda \in H$. To show $\lambda \in \Lambda$, it suffices to
show $\lambda I - A$ is surjective. Let $y \in X$ and, $\forall n \in \N$, put
\[x_n := R(\lambda_n; A)y.\]
Noting that, since $1/\lambda_n \to 1/\lambda$ as $n \to \infty$, the sequence
$\{1/\lambda_n\}_{n = 1}^\infty$ is bounded,
\begin{align*}
\|x_n - x_m\|
 &  = \|R(\lambda_n;A)y - R(\lambda_m;A)y\| \\
 &  = |\lambda_n - \lambda_m|\|R(\lambda_n;A)\|\|R(\lambda_m;A)\|\|y\|  \\
 &  \leq |\lambda_n - \lambda_m|\frac{\|y\|}{\lambda_n\lambda_m}
\end{align*}
using a resolvent identity (Proposition 7.26). It follows that
$\{x_n\}_{n = 1}^\infty$ is Cauchy, and so we may put
\[x := \lim_{n \to \infty} x_n.\]
Note that each $x_n \in \D(A)$ and $Ax_n \to \lambda x - y$. Since $A$ is
closed, $x \in \D(A)$ and $y = \lambda x - Ax$.
\end{question}

\newpage
\begin{question}{Problem 5}
Since $T$ is analytic, by Proposition 11.19, for $t > 0$, $T(t) : X \to \D(A)$.
Thus, for $s,t \in [0,\tau]$ with $s < t$, $T(t - s)(f(s) - f(t)) \in \D(A)$,
and hence $w(t) \in \D(A), \forall t \in [0,\tau]$.

Let $f \in C^{0,\theta}([0,\tau];X)$ with $C > 0$ such that
\[\|f(t) - f(s)\| \leq C|t - s|^\theta, \quad \forall s,t \in [0,\tau].\]

Since $T$ is an analytic semigroup, $\exists K > 0$ such that
\[\|AT(t)\| \leq K/t, \quad \forall t \in [0,\tau],\]
and let $s,t \in [0,\tau]$ with $s < t$. Adding forms of $0$, by integral and
semigroup properties,
\begin{align*}
w(t) - w(s)
 &  = \int_0^t T(t - r)(f(r) - f(t)) \, dr
    - \int_0^s T(s - r)(f(r) - f(s)) \, dr \\
 &  = \int_0^s T(t - r)(f(r) - f(t))
    -  T(s - r)(f(r) - f(s)) \, dr
    + \int_s^t T(t - r)(f(r) - f(t)) \, dr \\
 &  = \int_0^s (T(t - s) - I)T(s - r)(f(r) - f(s)) \, dr
    + \int_s^t T(t - r)(f(r) - f(t)) \, dr  \\
 &  + \int_0^s T(t - r)(f(s) - f(t)) \, dr.
\end{align*}
We now bound the norm of $A$ applied to each of these three terms.
\begin{align*}
\left\| A \int_0^s T(t - r) (f(s) - f(t)) \, dr \right\|
 &  \leq \int_0^s \| A T(t - r) \| \| (f(s) - f(t)) \| \, dr    \\
 &  \leq \int_0^s \frac{k}{t - r} C (t - s)^\theta \, dr
    \leq kC \int_0^s (t - r)^{\theta - 1} \, dr    \\
 &  = \frac{kC}{\theta} (t^\theta - (t - s)^\theta)
\end{align*}
\vspace{-7mm}
\begin{align*}
\left\| A \int_s^t T(t - r) (f(r) - f(t)) \, dr \right\|
 &  \leq \int_s^t \| A T(t - r) \| \| (f(r) - f(t)) \| \, dr    \\
 &  \leq \int_s^t \frac{k}{t - r} C (t - r)^\theta \, dr    \\
 &  \leq kC \int_0^{t - s} t^{\theta - 1} \, dr
    = \frac{kC}{\theta} (t - s)^\theta.
\end{align*}

[I didn't have time to finish writing this part up. I'm not sure I divided the
pieces of $Aw$ correctly\dots]


Thus, we have, $\forall s,t \in [0,\tau]$,
\[\|Aw(t) -Aw(s)\| \leq \frac{kC}{\theta} |t - s|^\theta,\]
so that $Aw \in C^{0,\theta}([0,\tau];X)$. \qed
\end{question}
\end{document}
