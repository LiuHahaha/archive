\documentclass[11pt]{article}
\usepackage{enumerate}
\usepackage{fullpage}
\usepackage{fancyhdr}
\usepackage{amsmath, amsfonts, amsthm, amssymb}
\usepackage{color}
\setlength{\parindent}{0pt}
\setlength{\parskip}{5pt plus 1pt}
\pagestyle{empty}

\def\indented#1{\list{}{}\item[]}
\let\indented=\endlist

\newcounter{questionCounter}
\newcounter{partCounter}[questionCounter]
\newenvironment{question}[2][\arabic{questionCounter}]{%
    \setcounter{partCounter}{0}%
    \vspace{.25in} \hrule \vspace{0.5em}%
        \noindent{\bf #2}%
    \vspace{0.8em} \hrule \vspace{.10in}%
    \addtocounter{questionCounter}{1}%
}{}
\renewenvironment{part}[1][\alph{partCounter}]{%
    \addtocounter{partCounter}{1}%
    \vspace{.10in}%
    \begin{indented}%
       {\bf (#1)} %
}{\end{indented}}

%%%%%%%%%%%%%%%%%%%%%%%HEADER%%%%%%%%%%%%%%%%%%%%%%%%%%%%%%
\newcommand{\myname}{Shashank Singh}
\newcommand{\myandrew}{sss1@andrew.cmu.edu}
\newcommand{\myclass}{21-630 Ordinary Differential Equations}
\newcommand{\myhwnum}{1}
\newcommand{\duedate}{Wednesday, March 6, 2013}
%%%%%%%%%%%%%%%%%%%%%%%%%%%%%%%%%%%%%%%%%%%%%%%%%%%%%%%%%%%

%%%%%%%%%%%%%%%%%%%%CONTENT MACROS%%%%%%%%%%%%%%%%%%%%%%%%%
\renewcommand{\qed}{\quad $\blacksquare$}
\newcommand{\mqed}{\quad \blacksquare}
\newcommand{\inv}{^{-1}}
\newcommand{\bv}{\mathbf{v}}
\newcommand{\bx}{\mathbf{x}}
\newcommand{\by}{\mathbf{y}}
\newcommand{\bff}{\mathbf{f}}
\newcommand{\bzero}{\mathbf{0}}
\newcommand{\bxi}{\boldsymbol{\xi}}
\newcommand{\boldeta}{\boldsymbol{\eta}}
\newcommand{\dist}{\operatorname{dist}}
\newcommand{\sminus}{\backslash}
\newcommand{\N}{\mathbb{N}} % natural numbers
\newcommand{\Z}{\mathbb{Z}} % integers
\newcommand{\Q}{\mathbb{Q}} % rational numbers
\newcommand{\R}{\mathbb{R}} % real numbers
\newcommand{\pow}[1]{\mathcal{P}\left(#1\right)} % power set of #1
\newcommand{\e}{\varepsilon} % \varepsilon
\newcommand{\F}{\mathcal{F}}
\newcommand{\C}{\mathcal{C}}
%%%%%%%%%%%%%%%%%%%%%%%%%%%%%%%%%%%%%%%%%%%%%%%%%%%%%%%%%%%

\begin{document}
\thispagestyle{plain}

{\Large Midterm \myhwnum} \\
\myclass \\
Name: \myname \\
Email: \myandrew \\
Due: \duedate

%TODO
\begin{question}{Problem 1}
$\forall t \geq 0$, define $F_t : \R^4 \rightarrow \R^4$ by
\[F_t
    \begin{pmatrix}
        x_1 \\
        x_2 \\
        v_1 \\
        v_2 \\
    \end{pmatrix}
 =
    \begin{pmatrix}
        X_1(t,x,v) \\
        X_2(t,x,v) \\
        V_1(t,x,v) \\
        V_2(t,x,v) \\
    \end{pmatrix}
 =
    \begin{pmatrix}
x_1 + \int_0^t V_1(s,x,v) \, ds \\
x_2 + \int_0^t V_2(s,x,v) \, ds \\
v_1 + \int_0^t E_1(s,X(s,x,v)) \, ds \\
v_2 + \int_0^t E_2(s,X(s,x,v)) \, ds
    \end{pmatrix},
\]
so that $J(t)$ is the Jacobian of $F_t$. Thus, for $t = 0$,
\[
F_t
    \begin{pmatrix}
        x_1 \\
        x_2 \\
        v_1 \\
        v_2 \\
    \end{pmatrix}
 =
    \begin{pmatrix}
        x_1 \\
        x_2 \\
        v_1 \\
        v_2 \\
    \end{pmatrix},
\mbox{ so that }
\det(J(t)) = \det
    \begin{pmatrix}
        1 & 0 & 0 & 0 \\
        0 & 1 & 0 & 0 \\
        0 & 0 & 1 & 0 \\
        0 & 0 & 0 & 1 \\
    \end{pmatrix}
 = 1
.\]
I wasn't sure where to go from here; since $E$, and hence the system, are not
necessarily linear, we can't use the Abel/Liouville Theorem or similar
results to show that $\det(J(t))$ is constant.
\end{question}

\begin{question}{Problem 2}
Let $T = (4|x_0|)^{-2} > 0$, $B = [-2|x_0|, 2|x_0|]$,
$\C_B := \{X : [0,T] \rightarrow B : X \mbox{ is continuous}\}$, and
$\F : \C_B \rightarrow \C_B$ defined by
\[\F(X)(t) = x_0 + \int_0^{t/2}2\cos(t)X^3(s) \, ds + \int_0^t X^3(s) \, ds,
 \quad \forall X \in \C_B, t \in [0,T].\]

We show that the range of $\F$ is indeed contained in $\C_B$ and then that $\F$
is a contraction, so that, by the Contraction Mapping Theorem, $\F$ has a fixed
point (which clearly has the desired properties).

Suppose that $X \in \C_B$. Then, $\forall t \in [0,T]$,
\begin{align*}
|\F(X)(t)|
 & \leq |x_0| + \int_0^{t/2}|2\cos(t)X^3(s)| \, ds + \int_0^t |X^3(s)| \, ds \\
 & \leq |x_0| + 2\int_0^{t/2} |X^3(s)| \, ds + \int_0^t |X^3(s)| \, ds \\
 & \leq |x_0| + 2\int_0^{T/2} 8|x_0|^3 \, ds + \int_0^T 8|x_0|^3 \, ds \\
 & =    |x_0| + 8\frac{|x_0|^3}{(4|x_0|)^2} + 8\frac{|x_0|^3}{(4|x_0|)^2}
   =    |x_0| + \frac12 |x_0| + \frac12|x_0|
   =    2|x_0|,
\end{align*}
so that $\|\F(X)\| \leq 2|x_0|$ and thus $\F : \C_B \rightarrow \C_B$. Suppose
now that $X,Y \in \C_B$. Then,
$\forall t \in [0,T]$,
\begin{align*}
|\F(X)(t) - \F(Y)(t)|
 & \leq \int_0^{t/2}|2\cos(t)||X^3(s) - Y^3(s)| \, ds
   +    \int_0^t|X^3(s) - Y^3(s)| \, ds \\
 & \leq 2\int_0^{t/2}|X^3(s) - Y^3(s)| \, ds
   +    \int_0^t|X^3(s) - Y^3(s)| \, ds \\
 & =    2\int_0^{t/2}|X^2(s) + XY(s) + Y^2(s)||X(s) - Y(s)| \, ds \\
 & +    \int_0^t|X^2(s) + XY(s) + Y^2(s)||X(s) - Y(s)| \, ds
  \quad \mbox{(difference of cubes)} \\
 & \leq 2\int_0^{T/2}3|x_0|^2\|X - Y\| \, ds 
   +    \int_0^T3|x_0|^2\|X - Y\| \, ds \\
 & \leq 3\frac{|x_0|^2\|X - Y\|}{(4|x_0|)^2}
   +    3\frac{|x_0|^2\|X - Y\|}{(4|x_0|)^2}
   =    \frac{3}{16}\|X - Y\|,
\end{align*}
so that $\|\F(X) - \F(Y)\| \leq \frac{3}{16} \|X - Y\|$, and thus $\F$ is a
contraction. \qed
\end{question}

%TODO
\begin{question}{Problem 3}
Since $f$ is bounded, $\exists M_f > 0$ with $|f| \leq M_f$ on $\R$.
Note first that, for all $X : [0,T] \rightarrow \R$ satisfying
\begin{equation}
X(t)
 = x_0 + \int_0^{\min(t + \delta, T)} f(X(s)) \, ds
    \quad \forall t \in [0,T],
\label{eq:3.1}
\end{equation}
\[\left| X(t) - x_0 \right|
 \leq \int_0^{\min(t + \delta, T)} M_f \, ds
 \leq \int_0^T M_f \, ds
 = M_f T.
\]
Since $f'$ is continuous and $I := [x_0 - M_f T, x_0 + M_f T]$ is compact,
$\exists M_{f'} > 0$ with $|f'| \leq M_{f'}$ on $I$. Thus,
$\forall t \in [0,T]$, for $X,Y$ satisfying Equation (\ref{eq:3.1}), by the
Mean Value Theorem, $\exists c \in I$ with
\[|f(X(t)) - f(Y(t))| = |f'(c)||X(t) - Y(t)| \leq M_{f'}|X(t) - Y(t)|.\]
Thus,
\begin{align*}
\left| X(t) - Y(t) \right|
 & \leq \int_0^{\min(t + \delta, T)} |f(X(s)) - f(Y(s))| \, ds \\
 & \leq \int_0^{\min(t + \delta, T)} M_{f'}|X(s) - Y(s)| \, ds.
\end{align*}
This is close, but not quite sufficient to use Gronwall to get the desired
result\dots
\end{question}

\newpage
\begin{question}{Problem 4}
\begin{enumerate}[A)]
\item Since $f$ is continuous and $D := [t_0,t_1] \times [-C,C]^N$ is compact,
for some $M \in \R$, $|f| \leq M$ on $D$. Note that $C$ bounds $\{X^{(k)}\}$
uniformly on $[t_0,t_1]$. Furthermore, letting $\e > 0$, for
$\delta := \frac{\e}{M}$, $\forall k \in \N, s,t \in [t_0,t_1]$ with
$|t - s| < \delta$, using the integral equation form of $(ODE)$,
\[|X^{(k)}(t) - X^{(k)}(s)|
 \leq \int_s^t |f(u,X^{(k)}(u))| \, du
 \leq \int_s^t M \, du
 = M(t - s)
 < \e.
\]
Thus, $\{X^{(k)}\}$ is equicontinuous on $[t_0,t_1]$, so, by the Ascoli-Arzela
Theorem, some subsequence $\{X^{(n_k)}\}$ converges uniformly to some
continuous $X : [t_0,t_1] \rightarrow \R^N$.

Since $f$ is continuous, and $D := [t_0,t_1] \times [-C,C]$ is compact, $f$ is
uniformly continuous on $D$, so that the sequence
$\{t \mapsto f(t,X^{(n_k)}(t))\}$ converges uniformly to the function
$t \mapsto f(t,X(t))$, allowing us to pass the limit into the integral in the
following equality:
\begin{align*}
X(t)
 & = \lim_{k \rightarrow \infty} X^{(n_k)}(t)
   = \lim_{k \rightarrow \infty} \left( X^{(n_k)}(t_0)
   + \int_{t_0}^t f(s,X^{(n_k)}(s)) \, ds \right) \\
 & = X(t_0) + \int_{t_0}^t \lim_{k \rightarrow \infty} f(s,X^{(n_k)}(s)) \, ds
   = X(t_0) + \int_{t_0}^t f(s,X(s)) \, ds,
\end{align*}
$\forall t \in [t_0,t_1]$, so that $X$ is a solution to $(ODE)$. \qed

\item Let $\e > 0$. Since, $f$ is continuous on the compact set
$D := [t_0,t_1] \times [-C,C]$, $f$ is uniformly continuous on $D$. Thus,
$\exists \delta > 0$, such that, $\forall x_1,x_2 \in [-C,C]$
\[|x_1 - x_2| < \delta
    \quad \mbox{implies} \quad
   |f(t,x_1) - f(t,x_2)| < \frac{\e}{2(t_1 - t_0)}.
\]

From part A and the uniqueness of solutions with the same initial
conditions, we have a subsequence $\{X^{(n_k)}\}$ converging uniformly to $X$.
Thus, we can take $n \in \N$ such that,
$\forall t \in [t_0,t_1], |X^{(n)}(t) - X(t)| < \delta$, and then, since
$X^{(k)}(t_0) \rightarrow X(t_0)$ as $k \rightarrow \infty$, we can take
$k_0 \in \N$ such that, $\forall k \geq k_0$,
$|X^{(k)}(t_0) - X(t_0)| < |X^{(n)}(t_0) - X(t_0)|$.

Then, $\forall t \in [t_0,t_1]$,
\begin{align*}
|X^{(k)}(t) - X(t)|
 & \leq |X^{(k)}(t_0) - X(t_0)|
 +      \int_{t_0}^t |f(s,X^{(k)}(s)) - f(s,X(s))| \, ds \\
 & \leq |X^{(k)}(t_0) - X(t_0)|
 +      \int_{t_0}^t |f(s,X^{(n)}(s)) - f(s,X(s))| \, ds \\
 & <    \frac{\e}{2} + \int_{t_0}^t \frac{\e}{2(t_1 - t_0)}\, ds
   =    \frac{\e}{2} + \frac{\e(t - t_0)}{2(t_1 - t_0)}\, ds
   =    \frac{\e}{2} + \frac{\e(t - t_0)}{2(t_1 - t_0)}\, ds
   \leq \e,
\end{align*}
so that $X^{(k)} \rightarrow X$ uniformly as $k \rightarrow \infty$. \qed

\end{enumerate}
\end{question}
\end{document}
