\documentclass[11pt]{article}
\usepackage{enumerate}
\usepackage{fullpage}
\usepackage{fancyhdr}
\usepackage{amsmath, amsfonts, amsthm, amssymb}
\usepackage{color}
\setlength{\parindent}{0pt}
\setlength{\parskip}{5pt plus 1pt}
\pagestyle{empty}

\def\indented#1{\list{}{}\item[]}
\let\indented=\endlist

\newcounter{questionCounter}
\newcounter{partCounter}[questionCounter]
\newenvironment{question}[2][\arabic{questionCounter}]{%
    \setcounter{partCounter}{0}%
    \vspace{.25in} \hrule \vspace{0.5em}%
        \noindent{\bf #2}%
    \vspace{0.8em} \hrule \vspace{.10in}%
    \addtocounter{questionCounter}{1}%
}{}
\renewenvironment{part}[1][\alph{partCounter}]{%
    \addtocounter{partCounter}{1}%
    \vspace{.10in}%
    \begin{indented}%
       {\bf (#1)} %
}{\end{indented}}

%%%%%%%%%%%%%%%%%%%%%%%HEADER%%%%%%%%%%%%%%%%%%%%%%%%%%%%%%
\newcommand{\myname}{Shashank Singh}
\newcommand{\myandrew}{sss1@andrew.cmu.edu}
\newcommand{\myclass}{21-630 Ordinary Differential Equations}
\newcommand{\myhwnum}{7}
\newcommand{\duedate}{Wednesday, March 20, 2013}
%%%%%%%%%%%%%%%%%%%%%%%%%%%%%%%%%%%%%%%%%%%%%%%%%%%%%%%%%%%

%%%%%%%%%%%%%%%%%%%%CONTENT MACROS%%%%%%%%%%%%%%%%%%%%%%%%%
\renewcommand{\qed}{\quad $\blacksquare$}
\newcommand{\mqed}{\quad \blacksquare}
\newcommand{\inv}{^{-1}}
\newcommand{\bv}{\mathbf{v}}
\newcommand{\bx}{\mathbf{x}}
\newcommand{\by}{\mathbf{y}}
\newcommand{\bff}{\mathbf{f}}
\newcommand{\bzero}{\mathbf{0}}
\newcommand{\bxi}{\boldsymbol{\xi}}
\newcommand{\boldeta}{\boldsymbol{\eta}}
\newcommand{\dist}{\operatorname{dist}}
\newcommand{\sminus}{\backslash}
\newcommand{\N}{\mathbb{N}} % natural numbers
\newcommand{\Z}{\mathbb{Z}} % integers
\newcommand{\Q}{\mathbb{Q}} % rational numbers
\newcommand{\R}{\mathbb{R}} % real numbers
\newcommand{\pow}[1]{\mathcal{P}\left(#1\right)} % power set of #1
\newcommand{\e}{\varepsilon} % \varepsilon
\newcommand{\F}{\mathcal{F}}
\newcommand{\C}{\mathcal{C}}
%%%%%%%%%%%%%%%%%%%%%%%%%%%%%%%%%%%%%%%%%%%%%%%%%%%%%%%%%%%

\begin{document}
\thispagestyle{plain}

{\Large Homework \myhwnum} \\
\myclass \\
Name: \myname \\
Email: \myandrew \\
Due: \duedate

\begin{question}{Problem 1}
We first show that $0$ is asymptotically stable for the homogeneous
\begin{equation}
\dot X(t) = AX,
\label{eq:1.1}
\end{equation}
and then use Variation of Parameters to show that stability also holds for $Y$.

By the comment on page 90 of the notes, it suffices to show that $0$ is
asymptotically stable in the case that $A$ is in Jordan form. Let
$J_0,J_1,\dots J_k$ be the Jordan blocks of $A$ (with only $J_0$ diagonal).
Since the operator norm of a matrix is at least the operator norm of any
submatrix, $\forall t \geq 0, i \in \{0,\dots,k\}$, if $J_i$ is a
$k_i \times k_i$ block associated with generalized eigenvalue $\lambda_i$ and
$\sigma$ is the eigenvalue with the greatest real part,
\[Ce^{-\lambda t}
    \geq \left|e^{At}\right|
    \geq |e^{J_it}|
    \geq |e^{J_it}|_{\infty}
    \geq \left\{
        \begin{array}{cl}
            e^{\sigma t} : \mbox{ if } i = 0 \\
            e^{\lambda_i t} \frac{t^{k_i - 1}}{(k_i - 1)!} : \mbox{ else }
        \end{array}
    \right.
\]
Since the above inequality holds for all $t \geq 0$, $\sigma, \lambda_i < 0$.
Then, by Theorem 5.1, $0$ is asymptotically stable for equation (\ref{eq:1.1}).

Pick $\delta$ such that, for all solutions $\phi$ to equation (\ref{eq:1.1})
with initial condition $|\phi(0)| < \delta$, $\phi(t) \rightarrow \infty$ as
$t \rightarrow \infty$, and suppose $\phi$ is a solution to equation
(\ref{eq:1.1}) (with $|\phi(0) > 0|$). Then, since solutions to a homoeneous
linear system form a vector space,
$\frac{\delta \phi(t)}{2\phi(0)} \rightarrow 0$ as $t \rightarrow \infty$, so
that $\phi(t) \rightarrow 0$ as $t \rightarrow 0$ (i.e., all solutions of
equation (\ref{eq:1.1}) converge to $0$).

By Theorem 4.3 (Variation of Parameters), for any fundamental matrix solution
$\phi$ to equation (\ref{eq:1.1}), then, $\forall t \geq 0$,
\[|Y(t)|
 \leq |\phi(t)\phi\inv(0)Y(0)|
        + \int_0^t |\phi(t)\phi\inv(s)||b(s)| \, ds.
\]
I wasn't able to finish this solution, but I think that if I were to establish
a stronger statement about equation (\ref{eq:1.1}), such as that solutions
decay \emph{expontially}, then the result could easily be deduced from the
above inequality; the fact that $\phi(t) \rightarrow 0$ as
$t \rightarrow \infty$ doesn't seem quite sufficient, as the integral is still
positive and non-decreasing with $t$. \qed
\end{question}

\newpage
\begin{question}{Problem 2}
\begin{enumerate}[(A)]
\item For
$P =
    \begin{pmatrix}
        i/\omega    & -i/\omega \\
        1           & 1
    \end{pmatrix},
D =
    \begin{pmatrix}
        -i\omega    & 0 \\
        0           & i\omega
    \end{pmatrix},
\mbox{ so that }
P\inv = \frac12
    \begin{pmatrix}
        -i\omega    & 1 \\
        i\omega     & 1
    \end{pmatrix}
$, $A = PDP\inv$.

Then, since $D$ is diagonal,
\begin{align*}
e^{At}
   = Pe^{Dt}P\inv
 & = P \begin{pmatrix}
        e^{-i\omega t}    & 0 \\
        0           & e^{i\omega t}
       \end{pmatrix} P\inv \\
 & = P \begin{pmatrix}
        \cos(\omega t) - i\sin(\omega t)   & 0 \\
        0                                   & \cos(\omega t) + i\sin(\omega t)
       \end{pmatrix} P\inv \\
 & = P \left(\cos(\omega t) I + \omega\inv \sin(\omega t) D \right) P\inv \\
 & = \cos(\omega t) PIP\inv + \omega\inv \sin(\omega t) PDP\inv
   = \cos(\omega t) I + \omega\inv \sin(\omega t) A. \mqed
\end{align*}

\item We first show, by induction on $n$, that, $\forall n \in \N$,
\vspace{-0.1in}
\[A^{2n} = (-\omega^2)^n I
\quad \mbox{ and } \quad
A^{2n + 1} = (-\omega^2)^n A.\]
\vspace{-0.3in}

For $n = 0$, this clearly holds.
Suppose now that the formula holds for some $n \in \N$. Then,
\begin{align*}
A^{2(n + 1)}
 & = A^2A^{2n}
   = ((-\omega^2)I)((-\omega^2)^nI)
   = (-\omega^2)^{n + 1}I, \\
\mbox{ and so }
A^{2(n + 1) + 1}
 & = A^{2(n + 1)}A
   = (-\omega^2)^{n + 1}A,
\end{align*}
completing the induction. Using this formula for $A^k$ and the Taylor
series of sine and cosine,
\begin{align*}
e^{At}
   = \sum_{k = 0}^{\infty} \frac{t^k}{k!}A^k
 & = \sum_{k = 0}^{\infty} \frac{t^{2k}}{(2k)!}A^{2k}
   + \sum_{k = 0}^{\infty} \frac{t^{2k + 1}}{(2k + 1)!}A^{2k + 1} \\
 & = \sum_{k = 0}^{\infty} \frac{t^{2k}}{(2k)!}(-\omega^2)^k I
   + \sum_{k = 0}^{\infty}
                        \frac{t^{2k + 1}}{(2k + 1)!}(-\omega^2)^k A \\
 & = \sum_{k = 0}^{\infty} \frac{(-1)^k(\omega t)^{2k}}{(2k)!} I
   + \omega\inv \sum_{k = 0}^{\infty}
                        \frac{(-1)^k(\omega t)^{2k + 1}}{(2k + 1)!} A
   = \cos(\omega t) I + \frac{\sin(\omega t)}{\omega\inv} A. \mqed
\end{align*}
\end{enumerate}
\end{question}

\newpage
\begin{question}{Problem 3}
We first show that $0$ is stable for the homogeneous equation
\begin{equation}
\dot X(t) = AX,
\label{eq:3.1}
\end{equation}
and then use Variation of Parameters to show that stability also holds for $Y$.

By the comment on page 90 of the notes, it suffices to show that $0$ is stable
in the case that $A$ is in Jordan form. Let $J_0,J_1,\dots J_k$ be the
Jordan blocks of $A$ (with only $J_0$ diagonal). Since the operator norm of a
matrix is at least the operator norm of any submatrix, $\forall t \geq 0,
i \in \{0,\dots,k\}$, if $J_i$ is a $k_i \times k_i$ block associated with
generalized eigenvalue $\lambda_i$ and $\sigma$ is the eigenvalue with the
greatest real part,
\[B
    \geq \left|e^{At}\right|
    \geq |e^{J_it}|
    \geq |e^{J_it}|_{\infty}
    \geq \left\{
        \begin{array}{cl}
            e^{\sigma t} : \mbox{ if } i = 0 \\
            e^{\lambda_i t} \frac{t^{k_i - 1}}{(k_i - 1)!} : \mbox{ else }
        \end{array}
    \right.
\]
Since the above inequality holds for all $t \geq 0$, $\sigma \leq 0$ and each
$\lambda_i < 0$ (as $k_i \geq 2$). Then, by Theorem 5.1, $0$ is stable for
equation (\ref{eq:3.1}).

Let $\e > 0$, and define $\beta := \int_0^{\infty} |b(s)| \, ds$ (we assume the
Lebesgue integral, so that $|b|$ has finite integral). By Theorem 4.3
(Variation of Parameters), for any fundamental matrix solution $\phi$ to
equation (\ref{eq:3.1}), $\forall t \geq 0$,
\begin{align*}
|Y(t)|
 \leq |\phi(t)\phi\inv(0)Y(0)|
        + \int_0^t |\phi(t)\phi\inv(s)||F(s,Y(s))| \, ds    \\
 \leq |\phi(t)\phi\inv(0)Y(0)|
        + \int_0^t |\phi(t)\phi\inv(s)||b(s)||Y(s)| \, ds.
\end{align*}
Since $\phi(t)\phi\inv(s)$ is a solution of equation (\ref{eq:3.1}) with
initial condition $\phi(s)\phi\inv(s) = I$, $\delta\phi(t)\phi\inv(s)$ is also
a solution, with initial condition $\delta I$. Thus, since $0$ is stable for
equation (\ref{eq:3.1}), by choosing $\delta$ sufficiently small,
$|\delta\phi(t)\phi\inv(s)| \leq 1$, and thus $\phi(t)\phi\inv(s)| \leq
\delta\inv$, for all $s,t \geq 0$, so that
\[|Y(t)| \leq |Y(0)|\delta\inv + \int_0^t \delta\inv|b(s)||Y(s)| \, ds.\]

Then, by the full version of Gronwall's Inequality, $\forall t \geq 0$,
\begin{align*}
|Y(t)|
 & \leq |Y(0)|\delta\inv + \int_0^t |Y(0)|\delta\inv \delta\inv |b(s)|
            e^{\int_s^t \delta\inv |b(\tau)| \, d\tau} \, ds \\
 & \leq |Y(0)|\left( \delta\inv
                + \delta^{-2} \int_0^t |b(s)| e^{\delta\inv\beta} \right) \, ds
   \leq |Y(0)|\left( \delta\inv
                + \delta^{-2} \beta e^{\delta\inv\beta} \right)
   \leq \e,
\end{align*}
for
$|Y(0)|
  \leq \left( \delta\inv + \delta^{-2} \beta e^{\delta\inv\beta} \right)\inv$,
so that $0$ is stable for the non-homogeneous equation. \qed
\end{question}

\newpage
\begin{question}{Problem 4}
Let $\delta > 0$, and suppose $Y(0) = \delta/2$. Solving by separation of
variables gives, for some $C \in \R$,
\[\log(Y(t))
 = \int \frac{1}{Y(t)} \, dY
 = \int \frac{1}{1 + t} \, dt
 = \log(1 + t) + C,
\]
so that $Y(t) = e^C(1 + t)$, and thus, using the initial condition
$Y(0) = \delta/2$, $Y(t) = (\delta/2)(1 + t) \rightarrow \infty$ as
$t \rightarrow \infty$. It follows that $0$ is not stable. \qed
\end{question}
\end{document}
