\documentclass[11pt]{article}
\usepackage{enumerate}
\usepackage{fullpage}
\usepackage{fancyhdr}
\usepackage{amsmath, amsfonts, amsthm, amssymb}
\usepackage{color}
\setlength{\parindent}{0pt}
\setlength{\parskip}{5pt plus 1pt}
\pagestyle{empty}

\def\indented#1{\list{}{}\item[]}
\let\indented=\endlist

\newcounter{questionCounter}
\newcounter{partCounter}[questionCounter]
\newenvironment{question}[2][\arabic{questionCounter}]{%
    \setcounter{partCounter}{0}%
    \vspace{.25in} \hrule \vspace{0.5em}%
        \noindent{\bf #2}%
    \vspace{0.8em} \hrule \vspace{.10in}%
    \addtocounter{questionCounter}{1}%
}{}
\renewenvironment{part}[1][\alph{partCounter}]{%
    \addtocounter{partCounter}{1}%
    \vspace{.10in}%
    \begin{indented}%
       {\bf (#1)} %
}{\end{indented}}

%%%%%%%%%%%%%%%%%%%%%%%HEADER%%%%%%%%%%%%%%%%%%%%%%%%%%%%%%
\newcommand{\myname}{Shashank Singh}
\newcommand{\myandrew}{sss1@andrew.cmu.edu}
\newcommand{\myclass}{21-630 Ordinary Differential Equations}
\newcommand{\myhwnum}{12}
\newcommand{\duedate}{Wednesday, April 24, 2013}
%%%%%%%%%%%%%%%%%%%%%%%%%%%%%%%%%%%%%%%%%%%%%%%%%%%%%%%%%%%

%%%%%%%%%%%%%%%%%%%%CONTENT MACROS%%%%%%%%%%%%%%%%%%%%%%%%%
\renewcommand{\qed}{\quad $\blacksquare$}
\newcommand{\mqed}{\quad \blacksquare}
\newcommand{\inv}{^{-1}}
\newcommand{\bv}{\mathbf{v}}
\newcommand{\bx}{\mathbf{x}}
\newcommand{\by}{\mathbf{y}}
\newcommand{\bff}{\mathbf{f}}
\newcommand{\bzero}{\mathbf{0}}
\newcommand{\bxi}{\boldsymbol{\xi}}
\newcommand{\boldeta}{\boldsymbol{\eta}}
\newcommand{\dist}{\operatorname{dist}}
\newcommand{\sminus}{\backslash}
\newcommand{\N}{\mathbb{N}} % natural numbers
\newcommand{\Z}{\mathbb{Z}} % integers
\newcommand{\Q}{\mathbb{Q}} % rational numbers
\newcommand{\R}{\mathbb{R}} % real numbers
\newcommand{\pow}[1]{\mathcal{P}\left(#1\right)} % power set of #1
\newcommand{\e}{\varepsilon} % \varepsilon
\newcommand{\F}{\mathcal{F}}
\newcommand{\C}{\mathcal{C}}
\newcommand{\ol}{\overline}
%%%%%%%%%%%%%%%%%%%%%%%%%%%%%%%%%%%%%%%%%%%%%%%%%%%%%%%%%%%

\begin{document}
\thispagestyle{plain}

{\Large Homework \myhwnum} \\
\myclass            \\
Name: \myname       \\
Email: \myandrew    \\
Due: \duedate

\begin{question}{Problem 1}
\begin{enumerate}[a)]
\item $\ol y$ may be a critical point, and so there need not exist a
transversal $T$ whose center is $\ol y$.

\item Define $f \in C^1(\R^2,\R^2)$ in polar coordinates by
\[f(r,\theta)
  = \begin{bmatrix}
        -(r - 1)^4  \\
        (r - 1)^2 + \sin^2(\theta)
    \end{bmatrix}
.\]
There is a transversal $L$ normal to the unit circle and centered at $(0,1)$.
We showed in Problem 2 of Assignment 11 that $\Omega((1,1))$ is the unit circle
and that the solution with $X(0) = (1,1)$ goes through $L$ infinitely many
times, but that $\Omega((0,1)) = \{(-1,0)\}$, and hence, by uniqueness, the
unit circle is not the orbit of a periodic solution. \qed
\end{enumerate}
\end{question}

\begin{question}{Problem 2}
Consider the Predator-Prey model discussed in lecture, with
$a = b = c = d = 2$. $\forall k \in \N$, the solution with $X_k(0) = (2,1/k)$
is periodic, but the solution with $X(0) = (2,0)$ is not periodic.
\end{question}

\newpage
\begin{question}{Problem 3}
Note that, by the chain rule,
\[\frac{d}{dt} \left( (\dot X)^2 + X^4 \right)
    = 2\dot X \ddot X + 4X^3 \dot X
    = 2\dot X (\ddot X + 2X^3)
    = 0,
\]
and hence $(\dot X)^2 + X^4 = C$, for some constant $C \in \R$.
I wasn't able to finish this problem.
\end{question}

\begin{question}{Problem 4}
\begin{align*}
\frac{d}{dt}\left( Z - X^2 - Y^2 \right)
 & = \dot Z - 2X \dot X - 2Y \dot Y                             \\
 & = 2(X^2 + Y^2)^{3/2}(1 - Z) + 2X^2 \left( X^2 + Y^2 \right)  \\
 & - 2X\left(X\sqrt{X^2 + Y^2}(1 - Z) + X^3 - Y\right)          \\
 & - 2Y\left(Y\sqrt{X^2 + Y^2}(1 - Z) + X^2Y + X\right)         \\
 & = 2\sqrt{X^2 + Y^2}(1 - Z)(X^2 + Y^2 - (X^2 + Y^2))          \\
 & + 2X^2(X^2 + Y^2 - (X^2 + Y^2))
   + 2(XY - YX)
   = 0,
\end{align*}
Thus that $Z - X^2 - Y^2 = C$, for some constant $C \in \R$. Hence,
converting to polar coordinates, we replace $Z$ by $C + r^2$, giving
\begin{align*}
\dot r
 & = \dot X \cos\theta + \dot Y \sin\theta   \\
 & = \left( r^2\cos^2\theta(1 - C - r^2)
        + r^3\cos^4\theta - r\sin\theta\cos\theta \right)    \\
 & + \left( r^2\sin^2\theta(1 - C - r^2)
        + r^3\cos^2\theta\sin^2\theta + r\cos\theta\sin\theta \right)   \\
 & = r^2(1 - C - r^2) + r^3\cos^2\theta
   = r^2(1 - C - r^2 + r\cos^2\theta),    \\
\dot \theta
 & = \dot Y \frac{\cos\theta}{r} - \dot X \frac{\sin\theta}{r}  \\
 & = \left( r\sin\theta\cos\theta(1 - C - r^2)
        + r^2\cos^3\theta\sin\theta + \cos^2\theta \right)  \\
 & - \left( r\cos\theta\sin\theta(1 - C - r^2)
        + r^2\cos^3\theta\sin\theta - \sin^2\theta \right)
   = 1.
\end{align*}
Now observe that
\[-r^2 + 1 - C
    \leq -r^2 + r\cos\theta + 1 - C
    \leq -r^2 + r + 1 - C.
\]
If $C \in (0,1)$, since $-r^2 + 1 - C > 0$ for $r < \sqrt{1 - C}$ and
$-r^2 + r + 1 - C < 0$ for $r > s$. Thus, the annulus
\[A
    = \left[ \sqrt{1 - C}, \frac{1 + \sqrt{5 - 4C}}{2} \right] \times \R
\]
defined in polar coordinates is positively invariant. Since
$\dot \theta = 1$, solutions in $A$ are bounded away from critical points.
Since $A$ is bounded, by Poincar\'e-Bendixson, the planar system has a periodic
solution. Since such a solution exists for each $C \in (0,1)$ and (since
$Z = r^2 + C$) different values of $C$ lead to distinct solutions, the original
system has infinitely many periodic solutions. \qed
\end{question}
\end{document}
