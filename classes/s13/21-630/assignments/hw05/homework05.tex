\documentclass[11pt]{article}
\usepackage{enumerate}
\usepackage{fullpage}
\usepackage{fancyhdr}
\usepackage{amsmath, amsfonts, amsthm, amssymb}
\usepackage{color}
\setlength{\parindent}{0pt}
\setlength{\parskip}{5pt plus 1pt}
\pagestyle{empty}

\def\indented#1{\list{}{}\item[]}
\let\indented=\endlist

\newcounter{questionCounter}
\newcounter{partCounter}[questionCounter]
\newenvironment{question}[2][\arabic{questionCounter}]{%
    \setcounter{partCounter}{0}%
    \vspace{.25in} \hrule \vspace{0.5em}%
        \noindent{\bf #2}%
    \vspace{0.8em} \hrule \vspace{.10in}%
    \addtocounter{questionCounter}{1}%
}{}
\renewenvironment{part}[1][\alph{partCounter}]{%
    \addtocounter{partCounter}{1}%
    \vspace{.10in}%
    \begin{indented}%
       {\bf (#1)} %
}{\end{indented}}

%%%%%%%%%%%%%%%%%%%%%%%HEADER%%%%%%%%%%%%%%%%%%%%%%%%%%%%%%
\newcommand{\myname}{Shashank Singh}
\newcommand{\myandrew}{sss1@andrew.cmu.edu}
\newcommand{\myclass}{21-630 Ordinary Differential Equations}
\newcommand{\myhwnum}{5}
\newcommand{\duedate}{Wednesday, February 20, 2013}
\newcommand{\mycollaborators}{None}
%%%%%%%%%%%%%%%%%%%%%%%%%%%%%%%%%%%%%%%%%%%%%%%%%%%%%%%%%%%

%%%%%%%%%%%%%%%%%%%%CONTENT MACROS%%%%%%%%%%%%%%%%%%%%%%%%%
\renewcommand{\qed}{\quad $\blacksquare$}
\newcommand{\mqed}{\quad \blacksquare}
\newcommand{\inv}{^{-1}}
\newcommand{\bv}{\mathbf{v}}
\newcommand{\bx}{\mathbf{x}}
\newcommand{\by}{\mathbf{y}}
\newcommand{\bff}{\mathbf{f}}
\newcommand{\bzero}{\mathbf{0}}
\newcommand{\bxi}{\boldsymbol{\xi}}
\newcommand{\boldeta}{\boldsymbol{\eta}}
\newcommand{\dist}{\operatorname{dist}}
\newcommand{\sminus}{\backslash}
\newcommand{\N}{\mathbb{N}} % natural numbers
\newcommand{\Z}{\mathbb{Z}} % integers
\newcommand{\Q}{\mathbb{Q}} % rational numbers
\newcommand{\R}{\mathbb{R}} % real numbers
\newcommand{\pow}[1]{\mathcal{P}\left(#1\right)} % power set of #1
\newcommand{\e}{\varepsilon} % \varepsilon
\newcommand{\F}{\mathcal{F}}
\newcommand{\C}{\mathcal{C}}
%%%%%%%%%%%%%%%%%%%%%%%%%%%%%%%%%%%%%%%%%%%%%%%%%%%%%%%%%%%

\begin{document}
\thispagestyle{plain}

{\Large Homework \myhwnum} \\
\myclass \\
Name: \myname \\
Email: \myandrew \\
Due: \duedate \\
Collaborators: \mycollaborators

\begin{question}{Problem 2}
\begin{enumerate}[A)]
\item For $\lambda = 0$, the differential equation defining $X$ degenerates to
\[
    \left\{
        \begin{array}{rcl}
            \dot{X}(t,t_0,x_0,0) & = & X(t,t_0,x_0,0) \\
            X(t_0,t_0,x_0,0)     & = & x_0
        \end{array}
    \right.,
\]
whose solution is \fbox{$X(t,t_0,x_0,0) = x_0 e^{t - t_0}$.} We can then
compute \fbox{$\frac{\partial X}{\partial x_0} = e^{t - t_0},$
$\frac{\partial X}{\partial t_0} = -x_0e^{t - t_0}$.}

\item For $\lambda = 0$, the integral equation defining $X$ is
$X(t,t_0,x_0,0) = x_0 + \int_{t_0}^t X(s,t_0,x_0,0) \, ds$. Thus,
$\frac{\partial X}{\partial x_0}$ satisfies the integral equation
\[
 \frac{\partial X}{\partial x_0}
 = \frac{\partial}{\partial x_0}
                        \left(x_0 + \int_{t_0}^t X(s,t_0,x_0,0) \, ds \right)
 = 1 + \int_{t_0}^t \frac{\partial X}{\partial x_0}(s,t_0,x_0,0) \, ds,
\]
the solution to which is
\fbox{$\frac{\partial X}{\partial x_0} = e^{t - t_0}$.}
Similarly, by the Fundamental Theorem of Calculus,
\[
 \frac{\partial X}{\partial t_0}
 = \frac{\partial}{\partial t_0}
                        \left(x_0 + \int_{t_0}^t X(s,t_0,x_0,0) \, ds \right)
 = -X(t,t_0,x_0,0),
\]
so that \fbox{$\frac{\partial X}{\partial t_0} = -x_0e^{t - t_0}$.}

\item We use the shorthand $X(t,\lambda)$ to denote $X(t,t_0,x_0,\lambda)$.
$\frac{\partial X}{\partial \lambda}$ satisfies the linear integral equation
\begin{align*}
\frac{\partial}{\partial \lambda} X(t,0)
 = \left( \frac{\partial}{\partial \lambda} X(t,\lambda) \right)
                                                        \bigg|_{\lambda = 0}
 & = \frac{\partial}{\partial \lambda}
        \left( x_0 + \int_{t_0}^t X(s,\lambda)
                        + \lambda X^2(s,\lambda)
                        + \lambda^2 X^3(s,\lambda) \, ds \right)
                                                        \bigg|_{\lambda = 0} \\
 & = \int_{t_0}^t \left( \frac{\partial}{\partial \lambda}
                  \left( X(s,\lambda)
                  + \lambda X^2(s,\lambda)
                  + \lambda^2 X^3(s,\lambda) \right) \right)
                                          \bigg|_{\lambda = 0} \, ds \\
 & = \int_{t_0}^t \left( \frac{\partial}{\partial \lambda} X(s,\lambda)
          + X^2(s,\lambda)
          + \lambda \frac{\partial}{\partial \lambda} X^2(s,\lambda) \right.\\
 & + \left. 2\lambda X^3(s,\lambda) 
          + \lambda^2 \frac{\partial}{\partial \lambda} X^3(s,\lambda)
                                  \right) \bigg|_{\lambda = 0} \, ds \\
 & = \int_{t_0}^t \left( \frac{\partial}{\partial \lambda}
                        X(s,0) + X^2(s,0) \right) \, ds \\
\end{align*}
Noting that, by the result of part A),
$\int_{t_0}^t X^2(t,0) \, ds = \frac{x_0}{2}\left( e^{2(t - t_0} - 1 \right)$
then gives
\[
\mbox{\fbox{$\displaystyle \frac{\partial}{\partial \lambda} X(t,0)
 = \frac{x_0}{2}\left( e^{2(t - t_0} - 1 \right) e^{t - t_0}$.}}
\]
\end{enumerate}
\end{question}

\begin{question}{Problem 3}
$\forall (t_0,x_0)$, define $X$ as the solution to the linear initial value
problem
\[
    \left\{
        \begin{array}{rcl}
            \frac{dX}{dt}(t,t_0,x_0) & = & X(t,t_0,x_0) \\
            X(t_0,t_0,x_0)     & = & x_0
        \end{array}
    \right.,
\]
(so that, $\forall (t,t_0,x_0), X(t,t_0,x_0) = x_0e^{t - t_0}$). Then, $\forall
(t,t_0,x_0)$,
\begin{align*}
   \frac{d}{dt}(u(t,X(t,t_0,x_0)))
 & = \frac{\partial }{\partial t}(u(t,X(t,t_0,x_0)))
 + \frac{dX}{dt}(t,t_0,x_0) \frac{\partial }{\partial x}(u(t,X(t,t_0,x_0))) \\
 & = \frac{\partial }{\partial t}(u(t,X(t,t_0,x_0)))
 + X(t,t_0,x_0) \frac{\partial }{\partial x}(u(t,X(t,t_0,x_0))) \\
 & = u(t,X(t,t_0,x_0)).
\end{align*}
Thus, $\forall (t_0,x_0)$, $u$ satisfies the linear initial value problem
\[
    \left\{
        \begin{array}{rcl}
            \frac{du}{dt}(t_0,x_0) & = & u(t_0,x_0) \\
            u(0,x_0)               & = & g(x_0e^{-t_0})
        \end{array}
    \right.
\]
It follows that, $\forall (t_0,x_0)$,
\fbox{$u(t_0,x_0) = g(x_0e^{-t_0})e^{t_0}$.}
\end{question}

\end{document}
