\documentclass[11pt]{article}
\usepackage{enumerate}
\usepackage{fullpage}
\usepackage{fancyhdr}
\usepackage{amsmath, amsfonts, amsthm, amssymb}
\usepackage{color}
\setlength{\parindent}{0pt}
\setlength{\parskip}{5pt plus 1pt}
\pagestyle{empty}

\def\indented#1{\list{}{}\item[]}
\let\indented=\endlist

\newcounter{questionCounter}
\newcounter{partCounter}[questionCounter]
\newenvironment{question}[2][\arabic{questionCounter}]{%
    \setcounter{partCounter}{0}%
    \vspace{.25in} \hrule \vspace{0.5em}%
        \noindent{\bf #2}%
    \vspace{0.8em} \hrule \vspace{.10in}%
    \addtocounter{questionCounter}{1}%
}{}
\renewenvironment{part}[1][\alph{partCounter}]{%
    \addtocounter{partCounter}{1}%
    \vspace{.10in}%
    \begin{indented}%
       {\bf (#1)} %
}{\end{indented}}

%%%%%%%%%%%%%%%%%%%%%%%HEADER%%%%%%%%%%%%%%%%%%%%%%%%%%%%%%
\newcommand{\myname}{Shashank Singh}
\newcommand{\myandrew}{sss1@andrew.cmu.edu}
\newcommand{\myclass}{21-630 Ordinary Differential Equations}
\newcommand{\myhwnum}{3}
\newcommand{\duedate}{Wednesday, February 7, 2013}
%%%%%%%%%%%%%%%%%%%%%%%%%%%%%%%%%%%%%%%%%%%%%%%%%%%%%%%%%%%

%%%%%%%%%%%%%%%%%%%%CONTENT MACROS%%%%%%%%%%%%%%%%%%%%%%%%%
\renewcommand{\qed}{\quad $\blacksquare$}
\newcommand{\mqed}{\quad \blacksquare}
\newcommand{\inv}{^{-1}}
\newcommand{\bv}{\mathbf{v}}
\newcommand{\bx}{\mathbf{x}}
\newcommand{\by}{\mathbf{y}}
\newcommand{\bff}{\mathbf{f}}
\newcommand{\bzero}{\mathbf{0}}
\newcommand{\bxi}{\boldsymbol{\xi}}
\newcommand{\boldeta}{\boldsymbol{\eta}}
\newcommand{\dist}{\operatorname{dist}}
\newcommand{\sminus}{\backslash}
\newcommand{\N}{\mathbb{N}} % natural numbers
\newcommand{\Z}{\mathbb{Z}} % integers
\newcommand{\Q}{\mathbb{Q}} % rational numbers
\newcommand{\R}{\mathbb{R}} % real numbers
\newcommand{\pow}[1]{\mathcal{P}\left(#1\right)} % power set of #1
\newcommand{\e}{\varepsilon} % \varepsilon
\newcommand{\F}{\mathcal{F}}
\newcommand{\C}{\mathcal{C}}
%%%%%%%%%%%%%%%%%%%%%%%%%%%%%%%%%%%%%%%%%%%%%%%%%%%%%%%%%%%

\begin{document}
\thispagestyle{plain}

{\Large Homework \myhwnum} \\
\myclass \\
Name: \myname \\
Email: \myandrew \\
Due: \duedate

\begin{question}{Problem 1}
\begin{enumerate}[A)]
\item Since $g$ is continuous and $[0,T]$ is compact, $g$ is bounded. Since $f$
is also bounded, $\exists$ bounds $M_g, M_f \in \R$ for $|g|,|f|$,
respectively. The $X^{(n)}$ is uniformly bounded, since,
$\forall n \in \N$, $t \in [0,T]$,
\[
 X^{(n)}(t)
 = g(t) + t\int_0^tf(X^{(n)}(s - 1/n))\,ds
 \leq M_g + t\int_0^tM_f\,ds
 \leq M_g + M_fT^2.
\]

Let $\e > 0$. Since $g$ is continuous, $\exists \delta_g > 0$, such that,
$\forall s,t \in [0,T]$, if $|s - t| < \delta_g$, then $|g(s) - g(t)| < \e/2$.
By the Fundamental Theorem of Calculus and the Product Rule, $\forall n \in \N$,
\begin{align*}
 \frac{d}{dt}\left(t\int_0^tf(X^{(n)}(s - 1/n))\,ds\right)
 &= \int_0^tf(X^{(n)}(s - 1/n))\,ds + tf(X^{(n)}(t - 1/n))\\
 &\leq \int_0^tM_f\,ds + tM_f
 \leq 2M_fT.
\end{align*}
Thus the quantity differentiated above has a bounded derivative with respect to
$t$, and so, it is Lipschitz Continuous in $t$ (with Lipschitz constant
$2M_fT$).
Thus, if we let $\delta = \min\{\delta_g,\frac{\e}{4M_fT}\}$, $\forall s,t \in
[0,T]$ with $|s - t| < \delta$, $\forall n \in \N$,
\begin{align*}
 \left|X^{(n)}(s) - X^{(n)}(t)\right|
 &\leq \left|g(s) - g(t)\right|
    + \left|s\int_0^sf(X^{(n)}(\tau - 1/n))\,d\tau
    - t\int_0^tf(X^{(n)}(\tau - 1/n))\,d\tau \right|\\
 &< \e/2 + \e/2 = \e.
\end{align*}
Thus, the family $\{X^{(n)} : n \in \N\}$ is equicontinuous. Then, by the
Ascoli-Arzela Theorem, there is a subsequence $X^{(k_n)}$ converging uniformly
on $[0,T]$. \qed

\item Let $\e > 0$. Since the family $\{X^{k_n)} : n \in \N\}$ is
equicontinuous, $\exists n_0 \in \N$ such that, if $n > n_0$,
$\forall t \in [0,T]$, $\left|X^{k_n}(t - 1/k_n) - X^{k_n}(t)\right| < \e/2$.
Since the $X^{(k_n)} \rightarrow X$ uniformly as $n \rightarrow \infty$,
$\exists n_1 \in \N$ such that, if $n > n_1$, $\forall t \in [0,T]$,
$\left|X^{(k_n)}(t) - X(t)\right| < \e/2$. Taking $N := \max\{n_0,n_1\}$,
$\forall n > N$, $\forall t \in [0,T]$,
\[
 \left|X^{k_n}(t - 1/k_n) - X(t)\right|
 \leq \left|X^{k_n}(t - 1/k_n) - X^{k_n)}(t)\right|
 +    \left|X^{k_n}(t) - X(t)\right|
 < \e/2 + \e/2 = \e. \mqed
\]

\item By part B), since $f$ is continuous, and thus uniformly
continuous on any compact interval, $f(X^{(k_n)}(t - 1/k_n)) \rightarrow
f(X(t))$ uniformly on any compact interval. Thus, $\forall t \in [0,T]$,
\[
 \lim_{n \rightarrow \infty} \int_0^t f(X^{(n)}(s - 1/k_n)) \, ds
 = \int_0^t f(X(s)) \, ds,
\]
\[\mbox{so }\quad
 X(t)
 = \lim_{n \rightarrow \infty} X^{(n)}(t)
 = g(t) + t\int_0^t f(X(s)) \, ds. \mqed
\]

\end{enumerate}
\end{question}

\begin{question}{Problem 2}
\begin{enumerate}[A)]
\item Suppose $X \leq Y$. Then, since $B \geq 0$,
$\forall t \in [t_0, \infty)$,
\[
\F[X]
 = A + B \int_{t_0}^t X(s) \, ds
 \leq A + B \int_{t_e}^t Y(s) \, ds
 = \F[Y]. \mqed
\]

\item $\forall t \in [t_0,\infty)$,
\begin{align*}
\F[\overline{X}](t)
 & = A + B \int_{t_0}^t Ae^{B(s - t_0)} \, ds
   = A + A B \int_{0}^{t - t_0} e^{Bs} \, ds \\
 & = A + A B \frac{e^{Bs}}{B} \bigg|_{s = 0}^{s = t - t_0}
   = A + A \left(e^{B(t - t_0)} - 1 \right)
   = Ae^{B(t - t_0)}
   = \overline{X}(t). \mqed
\end{align*}

\item Since, as shown in part A), $\F$ is non-decreasing, it follows by
induction on $n$ that, $\forall n \in \N$, 
\[X^{(0)} \leq \F[X^{(0)}] = X^{(1)} \leq \F[X^{(1)}] = \ldots \leq 
\F[X^{(n-1)}] = X^{(n)}.\]
Moreover, since $\left\{ X^{(n)} \right\}_{n = 0}^{\infty}$ is a
non-decreasing sequence converging to $\overline{X}$, $X = X^{(0)} \leq
\overline{X}$. \qed

\end{enumerate}
\end{question}

\begin{question}{Problem 3}
Suppose $T \in (t_0,\infty)$. Since $a$ and $b$ are continuous and $[t_0,T]$ is
compact, $\exists A_T,B_T \in \R$ with $|a(t)| \leq A_T$ and $|b(t)| \leq B_T$,
$\forall t \in [t_0,T]$. Thus, $\forall t \in [t_0,T]$,
\[
 |X(t)| \leq \int_{t_0}^ta(s) + b(s)|X(s)| \, ds
        \leq \int_{t_0}^tA_T + B_T|X(s)| \, ds
        \leq A_TT + B_T\int_{t_0}^t|X(s)| \, ds.
\]
By the result of Problem 2, $\forall t \in [t_0,T], |X(t)| \leq
A_TTe^{B_T(t - t_0)}$, and so $|X|$ is bounded on $[t_0,T]$. It follows from the
Extension Theorem that $X$ can be extended beyond $T$, and so, since $T$ is
arbitrary, $X$ can be extended to $[t_0,\infty)$. \qed
\end{question}

\begin{question}{Problem 4}
If $p = 1$, then, for $\F$ as defined in Problem 2, with $A = 0, B = 1, t_0 =
0$, we have $|X(t)| \leq \F\left[ |X(t)| \right]$. Then, as shown in parts B)
and C) of Problem 2, $\forall t \in [0,\infty)$,
$|X(t)| \leq \overline{X} = Ae^{B(t - t_0)} = 0$.

Suppose now that $p \in (0,1)$. Note that any solution to the system
\[
\left\{
    \begin{array}{c}
        \frac{dX}{dt} = X^p \\
        X(0) = 0
    \end{array}
\right.
\]
satisfies the given inequality. We have already shown in class that $X :
[0,\infty) \rightarrow \R$ defined by $X(t) = (1 - p)t^{\frac{1}{1 - p}},
\forall t \in [0,\infty)$ is a solution to this equation, so $X$ need not
vanish. \qed
\end{question}

\end{document}
