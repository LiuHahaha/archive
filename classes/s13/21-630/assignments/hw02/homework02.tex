\documentclass[11pt]{article}
\usepackage{enumerate}
\usepackage{fullpage}
\usepackage{fancyhdr}
\usepackage{amsmath, amsfonts, amsthm, amssymb}
\usepackage{color}
\setlength{\parindent}{0pt}
\setlength{\parskip}{5pt plus 1pt}
\pagestyle{empty}

\def\indented#1{\list{}{}\item[]}
\let\indented=\endlist

\newcounter{questionCounter}
\newcounter{partCounter}[questionCounter]
\newenvironment{question}[2][\arabic{questionCounter}]{%
    \setcounter{partCounter}{0}%
    \vspace{.25in} \hrule \vspace{0.5em}%
        \noindent{\bf #2}%
    \vspace{0.8em} \hrule \vspace{.10in}%
    \addtocounter{questionCounter}{1}%
}{}
\renewenvironment{part}[1][\alph{partCounter}]{%
    \addtocounter{partCounter}{1}%
    \vspace{.10in}%
    \begin{indented}%
       {\bf (#1)} %
}{\end{indented}}

%%%%%%%%%%%%%%%%%%%%%%%HEADER%%%%%%%%%%%%%%%%%%%%%%%%%%%%%%
\newcommand{\myname}{Shashank Singh}
\newcommand{\myandrew}{sss1@andrew.cmu.edu}
\newcommand{\myclass}{21-630 Ordinary Differential Equations}
\newcommand{\myhwnum}{2}
\newcommand{\duedate}{Wednesday, January 30, 2013}
%%%%%%%%%%%%%%%%%%%%%%%%%%%%%%%%%%%%%%%%%%%%%%%%%%%%%%%%%%%

%%%%%%%%%%%%%%%%%%%%CONTENT MACROS%%%%%%%%%%%%%%%%%%%%%%%%%
\renewcommand{\qed}{\quad $\blacksquare$}
\newcommand{\mqed}{\quad \blacksquare}
\newcommand{\inv}{^{-1}}
\newcommand{\bv}{\mathbf{v}}
\newcommand{\bx}{\mathbf{x}}
\newcommand{\by}{\mathbf{y}}
\newcommand{\bff}{\mathbf{f}}
\newcommand{\bzero}{\mathbf{0}}
\newcommand{\bxi}{\boldsymbol{\xi}}
\newcommand{\boldeta}{\boldsymbol{\eta}}
\newcommand{\dist}{\operatorname{dist}}
\newcommand{\sminus}{\backslash}
\newcommand{\N}{\mathbb{N}} % natural numbers
\newcommand{\Z}{\mathbb{Z}} % integers
\newcommand{\Q}{\mathbb{Q}} % rational numbers
\newcommand{\R}{\mathbb{R}} % real numbers
\newcommand{\pow}[1]{\mathcal{P}\left(#1\right)} % power set of #1
\newcommand{\e}{\varepsilon} % \varepsilon
\newcommand{\F}{\mathcal{F}}
\newcommand{\C}{\mathcal{C}}
%%%%%%%%%%%%%%%%%%%%%%%%%%%%%%%%%%%%%%%%%%%%%%%%%%%%%%%%%%%

\begin{document}
\thispagestyle{plain}

{\Large Homework \myhwnum} \\
\myclass \\
Name: \myname \\
Email: \myandrew \\
Due: \duedate \\
\begin{question}{Problem 1}
\begin{enumerate}[A)]
\item For $T > 1$, let $X,Y \in \C[0,T]$ be the constant functions $1$ and $0$,
respectively. Then,
\[\|\F[X] - \F[Y]\|_{\C}
 \geq |\F[X](T) - \F[Y](T)|
 =    \left| \int_0^T 1 \, ds\right|
 =    T
 >    1
 >    C\|X - Y\|_{\C},
\]
for any $C \in (0,1)$. Thus, $\F$ is not a contraction. \qed

\item We first show, by induction on $n$, that, $\forall n \in \N,
t \in [0,T]$, \[|X^{(n + 1)}(t) - X^{(n)}(t)| \leq \|g\|\frac{t^n}{n!}.\]

For $n = 0$, $\forall t \in [0,T]$,
\begin{align*}
\left| X^{(n + 1)}(t) - X^{(n)}(t) \right|
 =    \left| X^{(1)} \right|
 =    \left| g(t) + \int_0^t 0 \, ds \right|
 \leq \|g\|
 =    \|g\|\frac{t^n}{n!}.
\end{align*}
since $t^0 = 0! = 1$.
Supposing now that the conclusion holds for some $n \in \N$,
$\forall t \in [0,T]$,
\[\left| X^{(n + 2)}(t) - X^{(n + 1)}(t) \right|
 \leq \int_0^t \left| X^{(n + 1)} - X^{(n)} \right| \, ds
 \leq \int_0^t \|g\|\frac{s^n}{n!} \, ds
 =    \|g\|\frac{t^{n + 1}}{(n + 1)!},\]
concluding the proof by induction. Thus, by the Triangle Inequality, $\forall
n,k \in \N, t \in [0,T]$,
\[|X^{(n + k)} - X^{(n)}|
 = \left| \sum_{l = n}^{n + k - 1} X^{(n + l + 1)} - X^{(n + l)} \right|
 \leq \sum_{l = n}^{n + k - 1} \left| X^{(n + l + 1)} - X^{(n + l)} \right|
 \leq \sum_{l = n}^{\infty} \|g\|\frac{t^l}{l!}.
\]
$\sum_{l = 0}^{\infty} \frac{t^l}{l!}$ converges, so $X^{(n)}$ is uniformly
Cauchy and thus uniformly convergent on $[0,T]$. \qed
\end{enumerate}
\end{question}

\newpage
\begin{question}{Problem 2}
We first show by induction on $n$ that, $\forall n \in \N$, $\forall t \in
[0,\infty)$,
\begin{align*}
X^{(2n + 1)}(t) & = t^2 \\
\mbox{and } X^{(2n + 2)}(t) & = -t^2.
\end{align*}
For $n = 0$, $\forall t \in [0,\infty)$,
\[X^{(2n + 1)}(t)
 = \int_0^t f(s,X^{(0)}(s)) \, ds
 = \int_0^t f(s,0) \, ds
 = \int_0^t 2s \, ds
 = s^2 \bigg|_{s = 0}^{s = t}
 = t^2,
\]
\[X^{(2n + 2)}(t)
 = \int_0^t f(s,X^{(1)}(s)) \, ds
 = \int_0^t f(s,s^2) \, ds
 = \int_0^t -2s \, ds
 = -s^2 \bigg|_{s = 0}^{s = t}
 = -t^2,
\]
as desired.
If we suppose now that the conclusion holds for some $n \in \N$, then,
$\forall t \in [0,\infty)$,
\[X^{(2(n + 1) + 1)}(t)
 = \int_0^t f(s,X^{(2n + 2)}(s)) \, ds
 = \int_0^t f(s,-s^2) \, ds
 = \int_0^t 2s \, ds
 = s^2 \bigg|_{s = 0}^{s = t}
 = t^2,
\]
\[X^{(2(n + 1) + 2)}(t)
 = \int_0^t f(s,X^{(2(n + 1) + 1)}(s)) \, ds
 = \int_0^t f(s,s^2) \, ds
 = \int_0^t -2s \, ds
 = -s^2 \bigg|_{s = 0}^{s = t}
 = -t^2,
\]
concluding the proof by induction.

$\{X^{(n)}\}_{n = 0}^{\infty}$ has two constant, and thus convergent,
subsequences:
$\{X^{(2n + 1)}\}_{n = 0}^{\infty} = \{t \mapsto t^2\}_{n = 0}^{\infty}$
and $\{X^{(2n + 2)}\}_{n = 0}^{\infty} = \{t \mapsto -t^2\}_{n = 0}^{\infty}$,
with limits $X,Y : [0,\infty) \rightarrow \R$ defined $\forall t \in
[0,\infty)$ by $X(t) = t^2$ and $Y(t) = -t^2$, respectively.
However, neither $X$ nor $Y$ satisfies the differential equation:
$\forall t \in (0,\infty)$,
\begin{align*}
\frac{dX}{dt}(t) & = 2t > -2t = f(t,X(t)), \\
\mbox{and } Y(t) & = -2t < 2t = f(t,Y(t)). \mqed
\end{align*}

\end{question}

\newpage
%TODO
\begin{question}{Problem 3}
\begin{enumerate}[A)]
\item Let $B = 1/4 + \|g\| < (0,1/2)$, define
$\C_B := \{X \in \C : \|X\|_{\C} \leq B\}$, and define
$\F : \C_B \rightarrow \C$ by $\F[X](t) = g(t) + \int_0^1 X^2(s) \, ds$,
$\forall X \in t \in [0,1]$. We show first that the image
$\F[\C_B] \subseteq \C_B$ and then that $\F$ is a contraction on $\C_B$. Then,
the existence of $X \in \C_b$ with the desired property (i.e., being a fixed
point of $\F$) follows from the Contraction Mapping Theorem (noting that any
limit of a sequence of functions bounded by $B$ is itself bounded by $B$, so
that $\C_B$ is closed).

Suppose $X \in \C_b$. Clearly, $\F[X] \in \C$, so it suffices to show that
$\|\F[X]\|_{\C} \leq B$:
\begin{align*}
\|\F[X]\|_{\C}
 & =    \sup_{t \in [0,1]} \left| g(t)  + \int_0^1 X^2(s) \, ds \right| \\
 & \leq \sup_{t \in [0,1]} \left| g(t) \right|
   +    \int_0^1 \left| X^2(s) \right| \, ds \\
 & =    \|g(t)\| + \int_0^1 \|X\|_{\C}^2  \, ds
   \leq \|g\| + B^2 \leq B,
\end{align*}
since $B^2 < 1/4$. We now show that $\F$ is a contraction. Suppose
$X,Y \in \C_B$. Then,
\begin{align*}
\|\F[X] - \F[Y]\|_{\C}
 & = \left| \int_0^1 X^2(s) - Y^2(s) \, ds \right|
   \leq \int_0^1 \left| (X(s) + Y(s))(X(s) - Y(s)) \right| \, ds \\
 & \leq \int_0^1 \|X + Y\|\|X - Y\| \, ds
   \leq \|X + Y\|\|X - Y\| \, ds \\
 & \leq \left( \|X\| + \|Y\| \right) \|X - Y\| \, ds \leq 2B \|X - Y\|,
\end{align*}
so that, since $2B \in (0,1)$, $\F$ is a contraction, as desired. \qed

%TODO
\item Note that, $\forall t \in [0,1]$, \fbox{$X(t) = g(t) + c$,} where
$c = \int_0^1 X^2(s) \, ds$ does not vary with $t$. Thus,
\[c
 = \int_0^1 X^2(s) \, ds
 = \int_0^1 (g(t) + c)^2 \, ds
 = k_2 + 2ck_1 + c^2,
\]
where $k_1 = \int_0^1 g(s) \, ds$ and $k_2 = \int_0^1 g^2(s) \, ds$ are
constants. The quadratic formula then gives
\[\mbox{\fbox{$\displaystyle
 c
 = \frac{1}{2} - k_1 \pm \sqrt{k_1^2 - k_1 + \frac{1}{4} - k_2}$.}}
\]
Note that, since $k_2 \leq k_1^2$ and $k_1 < 1/4$,
$0 < k_1^2 - k_1 + \frac{1}{4} - k_2$, and thus the possible values of $c$
give rise to \fbox{two distinct real solutions.}
\end{enumerate}
\end{question}

\newpage
\begin{question}{Problem 4}
\begin{enumerate}[A)]
\item $\forall n \in \N$, $X^{(n)}(0) = 0 < 1$ and, $\forall t \in (0,1]$,
$X^{(n)}(t)
 =    \frac{t^2}{t^2 + (1 - nt)^2}
 \leq \frac{t^2}{t^2}
 = 1$,
so $X^{(n)}$ is uniformly bounded on $[0,1]$. \qed

\item Suppose, for sake of contradiction, that $X^{(n)}$ is equicontinuous on
$[0,1]$, so that, for $\e = 1/2$, $\exists \delta > 0$ such that,
$\forall n \in \N, s,t \in [0,1]$ with $|t - s| < \delta$,
$|X^{(n)}(t) - X^{(n)}(s)| < \e$. Then, however, for $s = 0$,
$t = \left( 2\lceil \delta\inv \rceil \right)\inv \in (0,\delta)$, $n = 1/t \in
\N$, $|t - s| < \delta$, but
\[|X^{(n)}(t) - X^{(n)}(s)|
 = \left| \frac{t^2}{t^2 + (1 - nt)^2} - 0 \right|
 = \left| \frac{t^2}{t^2} \right|
 = 1 \geq 1/2 = \e,
\]
which is a contradiction. \qed
\end{enumerate}
\end{question}

\begin{question}{Problem 5}
Let $\e > 0$ be given, and choose $\delta := \left( \frac{\e}{3000} \right)^3$.
Then, $\forall n \in \N$, since $X^{(n)}$ is continuously differentiable,
$\forall t,s \in [0,1]$ with $|t - s| < \delta$ (without loss of generality,
$s \leq t$),
\begin{align*}
\left| X^{(n)}(t) - X^{(n)}(s) \right|
 & =    \left| \int_s^t \frac{dX^{(n)}}{dx}(x) \, dx \right|
 & \mbox{(Fundamental Theorem of Calculus)} \\
 & \leq \int_s^t \left| \frac{dX^{(n)}}{dx}(x) \right| \, dx
 & \mbox{(Triangle Inequality)} \\
 & \leq \int_s^t 1000x^{-2/3} \, dx
 & \mbox{(given bound)} \\
 & =    3000x^{1/3} \bigg|_{x = s}^{x = t} \\
 & \leq 3000(t - s)^{1/3}
 & \mbox{(concavity of cube root)} \\
 &  <   3000\delta^{1/3}
   = \e. \mqed
\end{align*}
\end{question}
\end{document}
