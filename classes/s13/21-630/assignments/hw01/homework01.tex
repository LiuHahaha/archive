\documentclass[11pt]{article}
\usepackage{enumerate}
\usepackage{fullpage}
\usepackage{fancyhdr}
\usepackage{amsmath, amsfonts, amsthm, amssymb}
\usepackage{color}
\setlength{\parindent}{0pt}
\setlength{\parskip}{5pt plus 1pt}
\pagestyle{empty}

\def\indented#1{\list{}{}\item[]}
\let\indented=\endlist

\newcounter{questionCounter}
\newcounter{partCounter}[questionCounter]
\newenvironment{question}[2][\arabic{questionCounter}]{%
    \setcounter{partCounter}{0}%
    \vspace{.25in} \hrule \vspace{0.5em}%
        \noindent{\bf #2}%
    \vspace{0.8em} \hrule \vspace{.10in}%
    \addtocounter{questionCounter}{1}%
}{}
\renewenvironment{part}[1][\alph{partCounter}]{%
    \addtocounter{partCounter}{1}%
    \vspace{.10in}%
    \begin{indented}%
       {\bf (#1)} %
}{\end{indented}}

%%%%%%%%%%%%%%%%%%%%%%%HEADER%%%%%%%%%%%%%%%%%%%%%%%%%%%%%%
\newcommand{\myname}{Shashank Singh}
\newcommand{\myandrew}{sss1@andrew.cmu.edu}
\newcommand{\myclass}{21-630 Ordinary Differential Equations}
\newcommand{\myhwnum}{1}
\newcommand{\duedate}{Wednesday, January 23, 2013}
%%%%%%%%%%%%%%%%%%%%%%%%%%%%%%%%%%%%%%%%%%%%%%%%%%%%%%%%%%%

%%%%%%%%%%%%%%%%%%%%CONTENT MACROS%%%%%%%%%%%%%%%%%%%%%%%%%
\renewcommand{\qed}{\quad $\blacksquare$}
\newcommand{\mqed}{\quad \blacksquare}
\newcommand{\inv}{^{-1}}
\newcommand{\bv}{\mathbf{v}}
\newcommand{\bx}{\mathbf{x}}
\newcommand{\by}{\mathbf{y}}
\newcommand{\bff}{\mathbf{f}}
\newcommand{\bzero}{\mathbf{0}}
\newcommand{\bxi}{\boldsymbol{\xi}}
\newcommand{\boldeta}{\boldsymbol{\eta}}
\newcommand{\dist}{\operatorname{dist}}
\newcommand{\sminus}{\backslash}
\newcommand{\N}{\mathbb{N}} % natural numbers
\newcommand{\Z}{\mathbb{Z}} % integers
\newcommand{\Q}{\mathbb{Q}} % rational numbers
\newcommand{\R}{\mathbb{R}} % real numbers
\newcommand{\pow}[1]{\mathcal{P}\left(#1\right)} % power set of #1
\newcommand{\e}{\varepsilon} % \varepsilon
\newcommand{\F}{\mathcal{F}}
%%%%%%%%%%%%%%%%%%%%%%%%%%%%%%%%%%%%%%%%%%%%%%%%%%%%%%%%%%%

\begin{document}
\thispagestyle{plain}

{\Large Homework \myhwnum} \\
\myclass \\
Name: \myname \\
Email: \myandrew \\
Due: \duedate \\
\begin{question}{Problem 1}
If $p > 2$, then
\[f_n(t)
 = \frac{t^p}{1 + nt^2}
 = \frac{1}{t^{-p} + nt^{2 - p}}
 \geq \frac{1}{nt^{2 - p}}.
\]
Since $2 - p < 0$, $\forall n \in \N$, $\exists t \in [0,\infty)$ such that
$t^{2 - p} < 1/n$, so that $f_n(t) > 1$, and thus $f_n$ does not converge
uniformly to $0$.

If $p = 2$, then
\[f_n(t) = \frac{t^2}{1 + nt^2} = \frac{1}{t^{-2} + n} \leq \frac{1}{n},\]
so that $f_n$ clearly converges uniformly to zero.

I wasn't able to show the case $0 < p < 2$.
\end{question}

\begin{question}{Problem 2}
Consider the infinite family of functions
\[\F := \left\{X : \R \rightarrow \R \left|
 X(t) = \left\{ \begin{array}{lc}
       0 & \mbox{ if } t \leq c \\
       ((1 - p)(t - c))^{\frac{1}{1 - p}} & \mbox{ if } c < t <
\frac{1}{1 - p} + c \\
       t + 1 - \left( \frac{1}{1 - p} + c\right) & \mbox{ if } \frac{1}{1 - p} + c \leq t
     \end{array} \right. \right., \mbox{ for some } c \in [0,\infty) \right\}.
\]
Suppose $X \in \F$. \\
If $X(t) \leq 0$, then $t \leq c$, so $\frac{dX}{dt}(t) = 0$, as desired. \\
If $0 < X(t) < 1$, then $c < t < \frac{1}{1 - p} + c$, so
$\frac{dX}{dt}(t) = ((1 - p)(t - c))^{\frac{p}{1 - p}} = X^p$, as desired. \\
If $1 \leq X(t)$, $\frac{1}{1 - p} + c \leq t$, so $\frac{dX}{dt}(t) = 1$,
as desired. \\
Finally, $X(0) = 0$.

Thus, $\F$ is an infinite family of solutions to the given initial value
problem. \qed
\end{question}

\newpage
\begin{question}{Problem 3}
Suppose, for sake of contradiction, that $f$ satisfies a Lipschitz condition in
$x$ on $D$, so that, $\exists C > 0$ such that, $\forall (t,x),(t,y) \in D$,
\[|f(t,x) - f(t,y)| \leq C|x - y|.\]
Then, for $y = 0$, $x = e^{-(C + 1)} \in (0,e\inv]$,
\[|x\ln(x)| = |f(t,x) - f(t,y)| \leq C|x - y| = C|x|,\]
implying $C + 1 = |\ln(x)| \leq C$, which is a contradiction. \qed

By the given fact, $\forall \alpha \in (0,1)$, $\exists C_{1 - \alpha}$ such
that, $\forall x \in (0,1)$, $|\ln(x)| \leq C_{1 - \alpha}x^{\alpha - 1}$.
Multiplying both sides by $x$ gives $|x\ln(x)| \leq C_{1 - \alpha}x^{\alpha}$.
I wasn't able to get further in showing the Holder condition.
\end{question}

\begin{question}{Problem 4}
Suppose, for sake of contradiction, that $f$ satisfies a Lipschitz condition in
$x$ on $D$, so that $\exists C > 0$ such that, $\forall (t,x),(t,y) \in D$,
\[|f(t,x) - f(t,y)| \leq C|x - y|.\]
Then, for $t = 1/C, x = t^2, y = 0$,
\[4/C = |4t| = |f(t,x) - f(t,y)| \leq C|x - y| = C|t^2| = 1/C,\]
which is impossible, since $C > 0$.

$\forall t \in [0,\infty)$, if $0 \leq x \leq y \leq t^2$, then,
\[|f(x) - f(y)|
 = 4\left|\frac{x - y}{t}\right|
 \leq 4\left|\frac{x - y}{\sqrt{xy}}\right|
 \leq 4\left|\sqrt{x} - \sqrt{y}\right|
 \leq C_1\sqrt{x - y}, \mbox{ for $C_1 = 4$.}
\]

$\forall t \in [0,\infty)$, if $x \leq 0 \leq t^2 \leq y$, then
$|f(t,x) - f(t,y)| = 4|t| \leq C_2\sqrt{x - y}$, for $C_2 = 4$.

$\forall t \in [0,\infty)$, if $x, y \leq 0$ or $t^2 \leq x, y$, then
$|f(t,x) - f(t,y)| = 0 \leq C_3\sqrt{x - y}$, for $C_3 = 1$.

$\forall t \in [0,\infty)$, if $x \leq 0 \leq y \leq t^2$, then
$|f(t,x) - f(t,y)| = 4|y/t| \leq 4\sqrt{y} \leq C_4\sqrt{y - x}$, for $C_4 = 4$.

$\forall t \in [0,\infty)$, if $0 \leq x \leq t^2 \leq y$, then
$|f(t,x) - f(t,y)| = 4|t - x/t| \leq C_5\sqrt{y - x}$, for $C_5 = 4$.

It follows that $f$ satisfies a Holder condition in $x$ on $D$, with exponent
$\alpha = 1/2$ and constant $C = \max\{C_1,C_2,C_3,C_4, C_5\} = 4$.
\qed
\end{question}

\newpage
\begin{question}{Problem 5}
$\forall n \in \N$, define $f_n : \R \rightarrow \R$ by
\[f_n(x)
  = \sum_{k = 1}^{n} \frac{x^k\sin(e^{kx})}{k^k},
  \forall x \in \R.
\]
Since continuity is defined pointwise, it suffices to show that, $\forall B >
0$, $f$ is continuous when restricted to $[-B,B]$. To do this, it suffices to
show that the sequence $\{f_n\}_{n = 1}^{\infty}$ of continuous functions
converges uniformly to $f$ when restricted to $[-B,B]$, $\forall B > 0$.
$\forall m \in \N, x \in [-B,B]$,
\begin{align*}
|f(x) - f_{n - 1}(x)|
 & = \left| \sum_{k = m}^{\infty} \frac{x^k\sin(e^{kx})}{k^k} \right| \\
 & \leq \sum_{k = m}^{\infty} \left| \frac{x^k\sin(e^{kx})}{k^k} \right|
   \leq \sum_{k = m}^{\infty} \left| \frac{B}{k} \right|^k \\
 & \leq \sum_{k = m}^{\infty} \left| \frac{B}{B + 1} \right|^k
 & \mbox{(assuming $m > |x|$, since we will take $m \rightarrow \infty$)} \\
 & =    \frac{    \left| \frac{B}{B + 1} \right|^m}
             {1 - \left| \frac{B}{B + 1} \right|}
   \rightarrow 0,
 & \mbox{(geometric series)}
\end{align*}
as $m \rightarrow \infty$, so that $\{f_n\}_{n = 1}^{\infty}$ indeed converges
uniformly to $f$. \qed
\end{question}

\end{document}
