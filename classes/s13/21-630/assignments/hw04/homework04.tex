\documentclass[11pt]{article}
\usepackage{enumerate}
\usepackage{fullpage}
\usepackage{fancyhdr}
\usepackage{amsmath, amsfonts, amsthm, amssymb}
\usepackage{color}
\setlength{\parindent}{0pt}
\setlength{\parskip}{5pt plus 1pt}
\pagestyle{empty}

\def\indented#1{\list{}{}\item[]}
\let\indented=\endlist

\newcounter{questionCounter}
\newcounter{partCounter}[questionCounter]
\newenvironment{question}[2][\arabic{questionCounter}]{%
    \setcounter{partCounter}{0}%
    \vspace{.25in} \hrule \vspace{0.5em}%
        \noindent{\bf #2}%
    \vspace{0.8em} \hrule \vspace{.10in}%
    \addtocounter{questionCounter}{1}%
}{}
\renewenvironment{part}[1][\alph{partCounter}]{%
    \addtocounter{partCounter}{1}%
    \vspace{.10in}%
    \begin{indented}%
       {\bf (#1)} %
}{\end{indented}}

%%%%%%%%%%%%%%%%%%%%%%%HEADER%%%%%%%%%%%%%%%%%%%%%%%%%%%%%%
\newcommand{\myname}{Shashank Singh}
\newcommand{\myandrew}{sss1@andrew.cmu.edu}
\newcommand{\myclass}{21-630 Ordinary Differential Equations}
\newcommand{\myhwnum}{4}
\newcommand{\duedate}{Wednesday, February 13, 2013}
\newcommand{\mycollaborators}{None}
%%%%%%%%%%%%%%%%%%%%%%%%%%%%%%%%%%%%%%%%%%%%%%%%%%%%%%%%%%%

%%%%%%%%%%%%%%%%%%%%CONTENT MACROS%%%%%%%%%%%%%%%%%%%%%%%%%
\renewcommand{\qed}{\quad $\blacksquare$}
\newcommand{\mqed}{\quad \blacksquare}
\newcommand{\inv}{^{-1}}
\newcommand{\bv}{\mathbf{v}}
\newcommand{\bx}{\mathbf{x}}
\newcommand{\by}{\mathbf{y}}
\newcommand{\bff}{\mathbf{f}}
\newcommand{\bzero}{\mathbf{0}}
\newcommand{\bxi}{\boldsymbol{\xi}}
\newcommand{\boldeta}{\boldsymbol{\eta}}
\newcommand{\dist}{\operatorname{dist}}
\newcommand{\sminus}{\backslash}
\newcommand{\N}{\mathbb{N}} % natural numbers
\newcommand{\Z}{\mathbb{Z}} % integers
\newcommand{\Q}{\mathbb{Q}} % rational numbers
\newcommand{\R}{\mathbb{R}} % real numbers
\newcommand{\pow}[1]{\mathcal{P}\left(#1\right)} % power set of #1
\newcommand{\e}{\varepsilon} % \varepsilon
\newcommand{\F}{\mathcal{F}}
\newcommand{\C}{\mathcal{C}}
%%%%%%%%%%%%%%%%%%%%%%%%%%%%%%%%%%%%%%%%%%%%%%%%%%%%%%%%%%%

\begin{document}
\thispagestyle{plain}

{\Large Homework \myhwnum} \\
\myclass \\
Name: \myname \\
Email: \myandrew \\
Due: \duedate \\
Collaborators: \mycollaborators

\begin{question}{Problem 1}
I wasn't able to solve this problem. I did notice that the solution to the
equation
\[Y(t) = A + \int_{t_0}^t b(s)\sqrt{Y(s)} \, ds,\]
has the form of the desired bound:
\[Y(t) = \left( \sqrt{A} + \int _{t_0}^t b(s) \, ds \right)^2.\]
I tried working along the lines of the proof given in class of Gronwall's
Inequality with this solution in mind, but wasn't able to get the desired
result.
\end{question}

\begin{question}{Problem 2}
\begin{enumerate}[A)]
\item We suppose that $R$ is continuous. It follows, as discussed in class,
that, since $R(0) = 1 > 0$, $R$ is positive and differentiable on $[0,\infty)$.
Differentiating the given equation gives, $\forall t \in [0,\infty)$,
$\frac{dR(t)}{dt} = \frac{1}{R(t)}$. Separation of variables and then
integration give, $\forall t \in [0,\infty)$,
\[
 \frac{R^2(t)}{2} = t + C,
\]
for some constant $C \in \R$, and so, since $R(0) = 1$,
\fbox{$R(t) = \sqrt{2t + 1}$.}

\item The function $X: [0,\infty) \rightarrow \R$ defined
$\forall t \in [0,\infty)$ by
\[X(t) =
    \left\{
        \begin{array}{cl}
            0.1         & : t \in [0,1.01)      \\
            10(t - 1)   & : t \in [1.01,2)      \\
            10          & : t \in [2,\infty)    \\
        \end{array}
    \right..
\]
is continuous and positive on $[0,\infty)$.
$\forall t \in [0,1.01)$, $X(t) \leq 1 \leq 1 + \int_0^t \frac{1}{X(s)} \, ds$,
and $\forall t \in [1.01,\infty)$,
\[
  X(t)
  \leq 10
  \leq 1 + \int_0^1 10 \, ds
  \leq 1 + \int_0^1 \frac{1}{X(s)} \, ds
  \leq 1 + \int_0^t \frac{1}{X(s)} \, ds.
\]
Furthermore, $X(2) = 10 > \sqrt{5} = R(2)$. Thus, $X$ is a counterexample. \qed

\end{enumerate}
\end{question}

\begin{question}{Problem 3}
By orthogonality and then by linearity of $R$,
\begin{equation}
 f(t,Rx)
 = F(t,|Rx|)Rx
 = F(t,|x|)Rx
 = R(F(t,|x|)x)
 = R f(t,x).
\label{eq:3.1}
\end{equation}

Since solutions to the given system are unique, it suffices to show that
\begin{align*}
\frac{d(RX)}{dt} & = f(t,RX(t,t_0,x_0)) \\
RX(t_0,t_0,x_0) & = Rx_0.
\end{align*}
The latter equation trivially holds. Furthermore, for all $t$, by
(\ref{eq:3.1}) and the fact that $R$ is linear,
\[
  f(t,RX(t,t_0,x_0))
 = R f(t,X(t,t_0,x_0)).
 = R \frac{dX}{dt}
 = \frac{d(RX)}{dt}. \mqed
\]
\end{question}

\begin{question}{Problem 4}
Suppose $f(t,x) = x^2(1 - x)$, $\forall x \in \R$. Since $f$ satisfies a
Lipschitz condition in $x$ on any bounded subset of the domain, any solution
$X$ to the given system is unique. However, for any $x_0 \in \R$, if $x_0 \leq
0$, then $\lim_{t \rightarrow +\infty} X(t,t_0,x_0) = 0$, whereas, if $x_0 > 0$,
then $\lim_{t \rightarrow +\infty} X(t,t_0,x_0) = 1$. Thus, the function $x_0
\mapsto \lim_{t \rightarrow +\infty} X(t,t_0,x_0)$ is discontinuous, and so the
given statement is false. \qed
\end{question}

\end{document}
