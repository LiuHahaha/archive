\documentclass[11pt]{article}
\usepackage{enumerate}
\usepackage{fullpage}
\usepackage{fancyhdr}
\usepackage{amsmath, amsfonts, amsthm, amssymb}
\usepackage{color}
\setlength{\parindent}{0pt}
\setlength{\parskip}{5pt plus 1pt}
\pagestyle{empty}

\def\indented#1{\list{}{}\item[]}
\let\indented=\endlist

\newcounter{questionCounter}
\newcounter{partCounter}[questionCounter]
\newenvironment{question}[2][\arabic{questionCounter}]{%
    \setcounter{partCounter}{0}%
    \vspace{.25in} \hrule \vspace{0.5em}%
        \noindent{\bf #2}%
    \vspace{0.8em} \hrule \vspace{.10in}%
    \addtocounter{questionCounter}{1}%
}{}
\renewenvironment{part}[1][\alph{partCounter}]{%
    \addtocounter{partCounter}{1}%
    \vspace{.10in}%
    \begin{indented}%
       {\bf (#1)} %
}{\end{indented}}

%%%%%%%%%%%%%%%%%%%%%%%HEADER%%%%%%%%%%%%%%%%%%%%%%%%%%%%%%
\newcommand{\myname}{Shashank Singh}
\newcommand{\myandrew}{sss1@andrew.cmu.edu}
\newcommand{\myclass}{21-630 Ordinary Differential Equations}
\newcommand{\myhwnum}{9}
\newcommand{\duedate}{Wednesday, April 3, 2013}
\newcommand{\mycollaborators}{Kenneth Lu (krlu)}
%%%%%%%%%%%%%%%%%%%%%%%%%%%%%%%%%%%%%%%%%%%%%%%%%%%%%%%%%%%

%%%%%%%%%%%%%%%%%%%%CONTENT MACROS%%%%%%%%%%%%%%%%%%%%%%%%%
\renewcommand{\qed}{\quad $\blacksquare$}
\newcommand{\mqed}{\quad \blacksquare}
\newcommand{\inv}{^{-1}}
\newcommand{\bv}{\mathbf{v}}
\newcommand{\bx}{\mathbf{x}}
\newcommand{\by}{\mathbf{y}}
\newcommand{\bff}{\mathbf{f}}
\newcommand{\bzero}{\mathbf{0}}
\newcommand{\bxi}{\boldsymbol{\xi}}
\newcommand{\boldeta}{\boldsymbol{\eta}}
\newcommand{\dist}{\operatorname{dist}}
\newcommand{\sminus}{\backslash}
\newcommand{\N}{\mathbb{N}} % natural numbers
\newcommand{\Z}{\mathbb{Z}} % integers
\newcommand{\Q}{\mathbb{Q}} % rational numbers
\newcommand{\R}{\mathbb{R}} % real numbers
\newcommand{\pow}[1]{\mathcal{P}\left(#1\right)} % power set of #1
\newcommand{\e}{\varepsilon} % \varepsilon
\newcommand{\F}{\mathcal{F}}
\newcommand{\C}{\mathcal{C}}
%%%%%%%%%%%%%%%%%%%%%%%%%%%%%%%%%%%%%%%%%%%%%%%%%%%%%%%%%%%

\begin{document}
\thispagestyle{plain}

{\Large Homework \myhwnum} \\
\myclass            \\
Name: \myname       \\
Email: \myandrew    \\
Due: \duedate       \\
Collaborators: \mycollaborators

\begin{question}{Problem 1}
From the given bounds on $v$, we have
\[\frac{d}{dt} v(t,X(t))
    = D_*v(t,X(t))
    \leq -C_3 |X(t)|^2
    \leq -\frac{C_3}{C_2} v(t,X(t)).
\]
Define $R$ by $R(t_0) = v(t_0,X(t_0))$ and
$\dot R(t) = -\frac{C_3}{C_2} v(t,X(t))$, and note
$v(t) \leq R(t), \forall t \geq t_0$. Thus,
$\dot R(t) \leq -\frac{C_3}{C_2} R(t)$, and it follows that
\[\frac{d}{dt} \left( e^{\frac{C_3}{C_2} (t - t_0)} R(t) \right)
    = e^{\frac{C_3}{C_2} (t - t_0)} (\dot R(t) + \frac{C_3}{C_2} R(t))
    \geq 0,
\]
so that $e^{\frac{C_3}{C_2} (t - t_0)} R(t) \geq R(t_0) = v(t_0,X(t_0))$.
We want, now, to say that, $\forall t \geq t_0$,
\[C_1 |X(t)|^2
    \leq v(t,X(t))
    \leq v(t_0,X(t_0)) e^{-\frac{C_3}{C_2} (t - t_0)} 
    \leq C_2|X(t_0)|^2 e^{-\frac{C_3}{C_2} (t - t_0)}
\]
which can be re-written as the desired inequality, but the second inequality
isn't quite what the above derivation gives us. \qed
\end{question}

\begin{question}{Problem 2}
Define $v : \R^3 \rightarrow \R$ for all $(x_1,x_2,x_3) \in \R^3$ by
$v(x_1,x_2,x_3) = \frac12x_1^2 + x_2^2 + \frac12x_3^2$.
Clearly, $v$ is positive definite. For the first system, it can be checked that
\[D_*v(x_1,x_2,x_3)
    =
        \begin{bmatrix}
            x_1     \\
            2x_2    \\
            x_3     \\
        \end{bmatrix}
    \cdot
        \begin{bmatrix}
            -2x_2 + x_2x_3  \\
            x_1 - x_1x_3    \\
            x_1x_2          \\
        \end{bmatrix}
    = 0,
\]
and so $D_*v$ is negative semi-definite. It follows from Theorem 5.4 that the
zero solution is \fbox{stable.}

For the second system, it can be checked that
\[D_*v(x_1,x_2,x_3)
    =
        \begin{bmatrix}
            x_1     \\
            2x_2    \\
            x_3     \\
        \end{bmatrix}
    \cdot
        \begin{bmatrix}
            -2x_2 + x_2x_3 - x_1^3  \\
            x_1 - x_1x_3 - x_2^3    \\
            x_1x_2 - x_3^3          \\
        \end{bmatrix}
    = -(x_1^4 + x_2^4 + x_3^4),
\]
and so $D_*v$ is negative definite. Since $v$ does not depend on time,
it follows from Theorem 5.5 that the zero solution is \fbox{asymptotically
stable.}
\end{question}

\begin{question}{Problem 3}
We first compute
\[D_*v(x,y)
    =   \begin{bmatrix}
            2x  \\
            2y
    \end{bmatrix}
  \cdot \begin{bmatrix}
            -x - y^2    \\
            -y - x^2
        \end{bmatrix}
    =   -2(x^2 + x^2y + xy^2 + y^2).
\]

Suppose $C \in (0,2)$. Re-writing $D_*v$ in polar coordinates gives
\[D_*v(x,y)
    =   -2(r^2 + r^3(\cos^2(\theta)\sin(\theta) + \sin^2(\theta)\cos(\theta)),
\]
and, in particular,
\[\frac{d}{d\theta} D_*v(x,y) = -2r^3(\cos^3(\theta) - \sin^3(\theta)),\]
so that $D_*v$ is maximized only when $\cos(\theta) = \sin(\theta)$ (so
$x = y$). Under this constraint we want
\[D_*v(x,y) + 2C_2x^2 = x^2(2(C_2 + 1) + 2x) \leq 0\]
which is satisfied if we choose \fbox{$C_2 := 2(1 - \sqrt{C/2})$} (since
$C < 2$, $C_2 > 0$).

For $C = 2$, $(-1,-1) \in \{(x,y) : x^2 + y^2 \leq C\}$ and
\[D_*v(-1,-1) = -2(1 + -1 + -1 + 1) = 0 > -2C_2 = -C_2 ((-1)^2 + (-1)^2),\]
for all $C_2 > 0$, and so no $C_2$ with the desired properties can exist. \qed
\end{question}

\begin{question}{Problem 4}
The critical points of the system are \fbox{$(0,0)$ and $(1,1)$.}
\[Df\big|_{(0,0)}
    =   \begin{bmatrix}
            1 - Y   & -X        \\
            Y       & X - 1     \\
        \end{bmatrix} \bigg|_{(0,0)}
    =   \begin{bmatrix}
            1   & 0     \\
            0   & -1    \\
        \end{bmatrix},
\]
which has as an eigenvalue $1 > 0$, and so, by Theorem 5.3, \fbox{$(0,0)$ is an
unstable critical point.}

To analyze the stability of $(1,1)$, define $v : \R^2 \rightarrow \R$ for all
 $(x,y) \in \R^2$ by
\[v(x,y) := e^{-2} - xe^{-x}ye^{-y}.\]
Then, $v(1,1) = 0$, and
\[D_*v(x,y)
    =   \begin{bmatrix}
            ye^{-x - y}(1 - x) \\
            xe^{-x - y}(1 - y)
    \end{bmatrix}
  \cdot \begin{bmatrix}
            x(1 - y) \\
            y(x - 1)
        \end{bmatrix}
    =   xye^{-x - y}(1 - y)((1 - x) + (x - 1))
    =   0,
\]
and so $D_*v$ is negative semidefinite. Using the Taylor series for the
exponential, for $\e_1,\e_2$ small,
\[v(x,y)
    = (1 - (1 + \e_1)(1 + \e_2)(1 - \e_1 + o(\e_1))(1 - \e_2 + o(\e_2)))e^{-2},
\]
and so, for $(\e_1,\e_2) \neq (0,0)$ since
\[(1 - (1 + \e_1)(1 + \e_2)(1 - \e_1)(1 - \e_2))e^{-2},
    = (1 - (1 - \e_1^2)(1 - \e_2^2))e^{-2}
    > 0
\]
in some ball around $(1,1)$, $v(x,y) > 0$ (except at $(1,1)$ itself), and hence
$v$ is positive definite. Thus, by Theorem 5.4, $(1,1)$ is a stable critical
point.
\end{question}
\end{document}
