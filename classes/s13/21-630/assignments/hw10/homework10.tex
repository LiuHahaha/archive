\documentclass[11pt]{article}
\usepackage{enumerate}
\usepackage{fullpage}
\usepackage{fancyhdr}
\usepackage{amsmath, amsfonts, amsthm, amssymb}
\usepackage{color}
\setlength{\parindent}{0pt}
\setlength{\parskip}{5pt plus 1pt}
\pagestyle{empty}

\def\indented#1{\list{}{}\item[]}
\let\indented=\endlist

\newcounter{questionCounter}
\newcounter{partCounter}[questionCounter]
\newenvironment{question}[2][\arabic{questionCounter}]{%
    \setcounter{partCounter}{0}%
    \vspace{.25in} \hrule \vspace{0.5em}%
        \noindent{\bf #2}%
    \vspace{0.8em} \hrule \vspace{.10in}%
    \addtocounter{questionCounter}{1}%
}{}
\renewenvironment{part}[1][\alph{partCounter}]{%
    \addtocounter{partCounter}{1}%
    \vspace{.10in}%
    \begin{indented}%
       {\bf (#1)} %
}{\end{indented}}

%%%%%%%%%%%%%%%%%%%%%%%HEADER%%%%%%%%%%%%%%%%%%%%%%%%%%%%%%
\newcommand{\myname}{Shashank Singh}
\newcommand{\myandrew}{sss1@andrew.cmu.edu}
\newcommand{\myclass}{21-630 Ordinary Differential Equations}
\newcommand{\myhwnum}{10}
\newcommand{\duedate}{Wednesday, April 10, 2013}
%%%%%%%%%%%%%%%%%%%%%%%%%%%%%%%%%%%%%%%%%%%%%%%%%%%%%%%%%%%

%%%%%%%%%%%%%%%%%%%%CONTENT MACROS%%%%%%%%%%%%%%%%%%%%%%%%%
\renewcommand{\qed}{\quad $\blacksquare$}
\newcommand{\mqed}{\quad \blacksquare}
\newcommand{\inv}{^{-1}}
\newcommand{\bv}{\mathbf{v}}
\newcommand{\bx}{\mathbf{x}}
\newcommand{\by}{\mathbf{y}}
\newcommand{\bff}{\mathbf{f}}
\newcommand{\bzero}{\mathbf{0}}
\newcommand{\bxi}{\boldsymbol{\xi}}
\newcommand{\boldeta}{\boldsymbol{\eta}}
\newcommand{\dist}{\operatorname{dist}}
\newcommand{\sminus}{\backslash}
\newcommand{\N}{\mathbb{N}} % natural numbers
\newcommand{\Z}{\mathbb{Z}} % integers
\newcommand{\Q}{\mathbb{Q}} % rational numbers
\newcommand{\R}{\mathbb{R}} % real numbers
\newcommand{\pow}[1]{\mathcal{P}\left(#1\right)} % power set of #1
\newcommand{\e}{\varepsilon} % \varepsilon
\newcommand{\F}{\mathcal{F}}
\newcommand{\C}{\mathcal{C}}
\newcommand{\ol}{\overline}
%%%%%%%%%%%%%%%%%%%%%%%%%%%%%%%%%%%%%%%%%%%%%%%%%%%%%%%%%%%

\begin{document}
\thispagestyle{plain}

{\Large Homework \myhwnum} \\
\myclass            \\
Name: \myname       \\
Email: \myandrew    \\
Due: \duedate

\begin{question}{Problem 1}
Note that $(0,0)$ is a critial point, and so any solution with initial
condition at $(0,0)$ will be constant. Hence, we assume the initial condition
is not $(0,0)$.

We first calculate
\begin{align*}
\dot r
    & = \dot X \cos \theta + \dot Y \sin \theta \\
    & = (1 - r^2) r (\cos^2 \theta - \sin \theta \cos \theta
        + \cos \theta \sin \theta + \sin^2 \theta) \\
    & = (1 - r^2) r \\
\dot \theta
    & = \dot Y r\inv \cos \theta - \dot X r\inv \sin \theta \\
    & = (1 - r^2) r (\cos^2 \theta + \sin \theta \cos \theta
        - \cos \theta \sin \theta + \sin^2 \theta) \\
    & = (1 - r^2) = \dot r / r.
\end{align*}
Thus, $\forall t \geq t_0$,
\[\theta(t)
    = \theta(t_0) + \int_{t_0}^t \dot r / r \, dt
    = \theta(t_0) + \int_{t_0}^t \frac{d}{dt} \ln(r(t)) \, dt
    = \theta(t_0) + \ln(r(t)) - \ln(r(t_0)).
\]
By choosing $\theta(t_0)$ appropriately, we ensure that $r(t_0) \geq 0$, for
any initial conditions. Thus, it is clear from the above (autonomous) equation
for $\dot r$ that \[r(t) \rightarrow 1 \mbox{ as } t \rightarrow \infty.\] Then,
since $\log(r(t)) \rightarrow 0$ as $t \rightarrow \infty$, from the above
equation for $\theta$,
\[\theta(t) \rightarrow \theta(t_0) - \log(r(t_0)) \mbox{ as }
    t \rightarrow \infty.\]
It follows that
\[\mbox{\fbox{$\displaystyle
    \Omega(r(t_0),\theta(t_0))
    = \{(1,\theta(t_0) - \log(r(t_0)))\}
$.}}\]
\end{question}

\newpage
\begin{question}{Problem 2}
Suppose $X(t) \in C^+(X(0))$, and define, $\forall k \in \N$, $t_k := t + kT$.
Since $T$ is a period of $X$, a trivial induction argument shows that
$0 \leq t_k \rightarrow \infty$ and $X(t_k) \rightarrow X(t)$ as
$k \rightarrow \infty$. Hence, $X(t) \in \Omega(X(0))$.

Suppose $\ol x \in \Omega(X(0))$, so that there is a sequence
$0 \leq t_k \rightarrow \infty$ with $X(t_k) \rightarrow \ol x$ as
$k \rightarrow \infty$. Since $T$ is a period of $X$, a trivial induction
argument shows that $C^+(X(0)) = \{X(t) : t \in [0,T]\}$. Therefore,
$C^+(X(0))$ is the image of the compact set $[0,T]$ under the continuous
function $X$, and so $C^+(X(0))$ is compact. Hence, since each
$X(t_k) \in C^+(X(0))$, $\ol x \in C^+(X(0))$. \qed
\end{question}

\begin{question}{Problem 3}
We first calculate
\begin{align*}
D_*w(x,y)
 &  =   \begin{bmatrix}
            2x  \\
            2y  \\
        \end{bmatrix}
  \cdot \begin{bmatrix}
            yg(x,y) - x(x - y)^2  \\
           -xg(x,y) - y(x - y)^2  \\
        \end{bmatrix}   \\
 &  = 2xy(g(x,y) - g(x,y)) - 2(x^2 + y^2)(x - y)^2  \\
 &  = - 2(x^2 + y^2)(x - y)^2.
\end{align*}
It is clear, then, that
$Z := \{(x,y) : D_*w(x,y) = 0\} = \{(x,y) \in \R^2 : x = y\}$.

Fix $\eta > 0$. From the choice of $w$, it is clear that $H_{\eta}$ is the
circle of radius $\sqrt{\eta}$ centered at the origin.

By Theorem 5.8, it suffices to show that the largest positively invariant
subset $M$ of $H_{\eta} \cap Z$ is the singleton $\{(0,0)\}$. Since $(0,0)$ is
a critical point, $(0,0) \in M$.

Suppose $X(0) = Y(0) \neq 0$. Since $g(X(0),Y(0)) \neq 0$, without loss of
generality,
\[\frac{d(X - Y)}{dt}(0)
    = 2X(0)g(X,Y) > 0.
\]
By continuity of this derivative, $\exists \e, \delta > 0$ such that,
$\forall t \in [0,\delta]$, $\frac{d(X - Y)}{dt}(t) > \e$. Hence,
\[(X - Y)(\delta) \geq \int_0^{\delta} \e \, dt = \delta\e > 0.\]
and hence $(X(\delta),Y(\delta)) \notin M$. By definition of positive
invariance, $M = \{(0,0)\}$. \qed
\end{question}

\newpage
%TODO
\begin{question}{Problem 4}
\begin{enumerate}[A)]
\item By definition of $v$ and $u$, the given system can be written as
\begin{align*}
\frac{dv}{dt} = - \left( v + u + \frac{\partial P}{\partial x} \right)  \\
\frac{du}{dt} = - \left( v + u + \frac{\partial P}{\partial y} \right)
\end{align*}
Thus,
\begin{align*}
D_*w(x,y,v,u)
    =   \begin{bmatrix}
            \frac{\partial P}{\partial x}   \\
            \frac{\partial P}{\partial y}   \\
            v                               \\
            u
        \end{bmatrix}
  \cdot \begin{bmatrix}
            v               \\
            u               \\
            \frac{dv}{dt}   \\
            \frac{du}{dt}   \\
        \end{bmatrix}
 &  =   v\frac{\partial p}{\partial x} + u\frac{\partial p}{\partial y}
    -   v\left( v + u + \frac{\partial P}{\partial x} \right)
    -   u\left( v + u + \frac{\partial P}{\partial y} \right)   \\
 &  =   -(v^2 + 2vu + u^2) = \mbox{\fbox{$-(v + u)^2$}} \leq 0.
\end{align*}
 
\item The origin is asymptotically stable. Let $M$ be as in Theorem 5.8 (for an
arbitrary $\eta > 0$). Since the origin is a critical point, it is in $M$.
Then, by Theorem 5.8, it suffices to show that any initial condition aside from
the origin causes the solution to leave $M$.

Since $D_*w = 0$ in $M$, $v + u = 0$ in $M$. Thus, in $M$,
\begin{align*}
\frac{d^2x}{dt^2} & = - \frac{\partial P}{\partial x} = -4x^3   \\
\frac{d^2y}{dt^2} & = - \frac{\partial P}{\partial y} = -2y.
\end{align*}
This system has no (non-zero) solution preserving
$V(t) + U(t) = 0, \forall t \geq t_0$. Thus solutions with initial conditions
not at the origin leave $M$. \qed


\item Letting $M$ be as in part (b), we have, for solutions lying in $M$,
\begin{align*}
\frac{d^2x}{dt^2} & = - \frac{\partial P}{\partial x} = -4x^3   \\
\frac{d^2y}{dt^2} & = - \frac{\partial P}{\partial y} = -4y^3.
\end{align*}
If, for any $\delta > 0$, we choose the initial condition $X(t_0) = \delta$,
$Y(t_0) = -\delta$, $V(t_0) = U(t_0) = 0$, then, since the negation of any
solution to each of the above equations is a solution, by uniqueness, the
solution will preserve $V(t) + U(t) = 0$. However, the solution is periodic
rather than converging to the origin. \qed
\end{enumerate}
\end{question}
\end{document}
