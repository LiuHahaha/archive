\documentclass[11pt]{article}
\usepackage{enumerate}
\usepackage{fullpage}
\usepackage{fancyhdr}
\usepackage{amsmath, amsfonts, amsthm, amssymb}
\usepackage{color}
\setlength{\parindent}{0pt}
\setlength{\parskip}{5pt plus 1pt}
\pagestyle{empty}

\def\indented#1{\list{}{}\item[]}
\let\indented=\endlist

\newcounter{questionCounter}
\newcounter{partCounter}[questionCounter]
\newenvironment{question}[2][\arabic{questionCounter}]{%
    \setcounter{partCounter}{0}%
    \vspace{.25in} \hrule \vspace{0.5em}%
        \noindent{\bf #2}%
    \vspace{0.8em} \hrule \vspace{.10in}%
    \addtocounter{questionCounter}{1}%
}{}
\renewenvironment{part}[1][\alph{partCounter}]{%
    \addtocounter{partCounter}{1}%
    \vspace{.10in}%
    \begin{indented}%
       {\bf (#1)} %
}{\end{indented}}

%%%%%%%%%%%%%%%%%%%%%%%HEADER%%%%%%%%%%%%%%%%%%%%%%%%%%%%%%
\newcommand{\myname}{Shashank Singh}
\newcommand{\myandrew}{sss1@andrew.cmu.edu}
\newcommand{\myclass}{21-640 Introduction to Functional Analysis}
\newcommand{\myhwnum}{7}
\newcommand{\duedate}{Wednesday, May 7, 2013}
%%%%%%%%%%%%%%%%%%%%%%%%%%%%%%%%%%%%%%%%%%%%%%%%%%%%%%%%%%%

%%%%%%%%%%%%%%%%%%%%CONTENT MACROS%%%%%%%%%%%%%%%%%%%%%%%%%
\renewcommand{\qed}{\quad $\blacksquare$}
\newcommand{\mqed}{\quad \blacksquare}
\newcommand{\inv}{^{-1}}
\newcommand{\bv}{\mathbf{v}}
\newcommand{\bx}{\mathbf{x}}
\newcommand{\by}{\mathbf{y}}
\newcommand{\bff}{\mathbf{f}}
\newcommand{\bzero}{\mathbf{0}}
\newcommand{\bxi}{\boldsymbol{\xi}}
\newcommand{\boldeta}{\boldsymbol{\eta}}
\newcommand{\dist}{\operatorname{dist}}
\newcommand{\area}{\operatorname{area}}
\newcommand{\vspan}{\operatorname{span}}
\newcommand{\Gr}{\operatorname{Gr}} % graph of a function
\renewcommand{\sp}{\operatorname{span}} % span of a set
\newcommand{\sminus}{\backslash}
\newcommand{\N}{\mathbb{N}} % natural numbers
\newcommand{\Z}{\mathbb{Z}} % integers
\newcommand{\Q}{\mathbb{Q}} % rational numbers
\newcommand{\R}{\mathbb{R}} % real numbers
\newcommand{\C}{\mathcal{C}} % compact functions
\newcommand{\K}{\mathbb{K}} % underlying field of a linear space
\newcommand{\Ran}{\mathcal{R}} % range of a linear operator
\newcommand{\Nul}{\mathcal{N}} % null-space of a linear operator
\renewcommand{\L}{\mathcal{L}} % bounded linear functions
\newcommand{\pow}[1]{\mathcal{P}\left(#1\right)} % power set of #1
\newcommand{\e}{\varepsilon} % \varepsilon
\newcommand{\wto}{\rightharpoonup} % weak convergence
\newcommand{\wsto}{\stackrel{*}{\rightharpoonup}} % weak-* convergence
%%%%%%%%%%%%%%%%%%%%%%%%%%%%%%%%%%%%%%%%%%%%%%%%%%%%%%%%%%%

\begin{document}
\thispagestyle{plain}

{\Large Homework \myhwnum} \\
\myclass \\
Name: \myname \\
Email: \myandrew \\
Due: \duedate

\begin{question}{Problem 1}
Suppose $x \in (C + K)^C$. Then, $\forall k \in K$, $x \notin C + k$, and so,
since $C + k$ is closed (by continuity of $+$), there exist neighborhoods
$U_k$ and $V_k$ with $x \in U_k$, $k \in V_k$, and
$U_k \cap (C + V_k) = \emptyset$ (take neighborhoods $U$ of $x$ and $V$ of $0$
with $U \cap C = \emptyset$ and $V - V \subseteq U - x$, and choose
$U_k := V + x, V_k := V + k$).

Since $K$ is compact, it has a finite cover
$\{V_{k_1},\dots,V_{k_n}\} \subseteq \{V_k : k \in K\}$.
Thus, for $U := \bigcap_{i = 1} U_{k_i}$,
\[(C + K) \cap U
    \subseteq \left( C + \bigcup_{i = 1}^n V_{k_i} \right) \cap U
    = \emptyset,
.\]
$x \in U$, and $U$ is open.
Therefore, $(C + K)^C$ is open, and so $C + K$ is closed. \qed
\end{question}

\begin{question}{Problem 2}
Since $X$ is locally convex, it has a separating family of seminorms; in
particular, there is a seminorm $p : X \to \R$ with $p(x) \neq 0$. We can then
(uniquely) define a linear functional $f_0 : \vspan(x) \to \R$ by
$f_0(x) = p(x) \neq 0$, noting $|f(y)| \leq p(y)$, $\forall y \in \vspan(x)$.
Then, by the Hahn-Banach Theorem, there is an extension $f : X \to \R$ of $f_0$
with $|f| \leq p$. Thus, since $p$ is continuous on $X$ and $p(0) = 0$, it is
immediate that $f$ is continuous at $0$, and so, by Proposition 13.3, $f$ is
continuous on $X$. \qed
\end{question}

\newpage
\begin{question}{Problem 3}
\begin{enumerate}[(a)]
\item Suppose $E \subseteq X$ is $\tau$ bounded, so that, by Theorem 14.2,
$\forall x \in X$, the function $f \mapsto |f(x)|$ is bounded on $E$. Then,
we can define $b : [0,1] \to \R$ by
\[b(x) := \sup_{f \in E} |f(x)|, \quad \forall x \in [0,1].\]
Note that since each $f$ is continuous, $\forall \alpha \in \R$ the set
\[\{x \in [0,1] : b(x) > \alpha\}
    = \bigcup_{f \in E} \{x \in [0,1] : |f(x)| > \alpha\}\]
is a union of open sets and is therefore open, so that $b$ is lower
semi-continuous (thanks to Jimmy Murphy for pointing this out to me).
Since the function $g := x \mapsto \frac{|x|}{1 + |x|}$ is continuous and
non-decreasing, $g \circ b$ is also lower semi-continuous.

Semi-continuous functions are clearly Borel measurable and hence Lebesgue
measurable. Also, $g \leq 1$ on $\R$, so, by the Lebesgue's
Dominated Convergence Theorem, $g \circ b$ is integrable on $[0,1]$.

By construction of $\rho$ and the fact that $\rho$-balls give a local base of
$\sigma$ at $0$, $E$ is $\sigma$-bounded if and only if, $\forall \e > 0$,
$\exists t_\e > 0$ such that $\forall t > t_\e, f \in E$,
$\rho(f/t,0) < \e$. Since
\[\rho(f/t,0)
    = \int_0^1 \frac{|f(x)/t|}{1 + |f(x)/t|} \, dx
    = \int_0^1 \frac{|f(x)|}{t + |f(x)|} \, dx
    \leq \int_0^1 \frac{|b(x)|}{t + |b(x)|} \, dx,
\]
and, again by the Dominated Convergence Theorem, the last integral approaches
$0$ as $t \to \infty$, for every $\e > 0$, such a $t_\e$ must exist. \qed

\item 
By Theorem 14.2 and the definition of $p_x$, the family
\[\{\{f \in X : p_x(f) = |f(x)| < 1/n_x, \forall x \in S\}
    : S \subseteq [0,1] \mbox{ finite, }
        n_x \in \N \mbox{ for each } x \in S\}\]
of sets is a local base of $(X,\tau)$ at $0$. Thus, letting $\e = 1/4$, to show
that $I$ is not continuous, it suffices, for an arbitrary finite set
$S \subseteq [0,1]$ to construct a function $f$ with $f = 0$ on $S$ and
\[\rho(f,0)
    = \int_0^1 \frac{|f(x)|}{1 + |f(x)|} \, dx \geq \e.\]
Although we do not explicitly give the construction, on can construct such a
function $f$ by fixing $f = 0$ on $S$ and $f = 1$ on a set of measure at least
$1/2$ (and positive distance from $S$), and then interpolating to make $f$
continuous. \qed
\end{enumerate}
\end{question}

\newpage
\begin{question}{Problem 4}
Suppose $E$ is topologically bounded, and let $\{\alpha_n\}_{n = 1}^\infty$ and
$\{x_n\}_{n = 1}^\infty$ be sequences in $\K$ and $X$, respectively, with
$\alpha_n \to 0$ as $n \to \infty$. Suppose $U \subseteq X$ is a neighborhood
of $0$. Since $E$ is topologically bounded, $\exists t_0 > 0$ such that
$E \subseteq tU, \forall t > t_0$. Since $\alpha_n \to 0$, $\exists N \in \N$
with $\alpha_n < 1/t, \forall n > N$. since each $x_n \in tU$, $\forall n > N$,
$\alpha_nx_n \in U$. Therefore, $\alpha_nx_n \to 0$ as $n \to \infty$.

Suppose, on the other hand, that $E$ is not topologically bounded, so that
there is a neighborhood $V$ of $0$ such that, $\forall n \in \N, \exists t_n > n$
with $E \not\subseteq t_nV$. Then, $\forall n \in \N$, for
$\alpha_n = \frac{1}{t_n}$, $\exists x_n \in E$ with $\alpha_nx_n \notin V$, and
hence $\alpha_nx_n \not\to 0$ as $n \to \infty$. \qed
\end{question}

%TODO
\begin{question}{Problem 5}
\begin{enumerate}[(a)]
\item $\forall x,y,z \in X$,
$\rho(x + z,y + z) = F((x + z) - (y + z)) = F(x - y) = \rho(x,y)$.

\item Let $r > 0, B := \{x \in X : \rho(x,0) < r\}$, $\alpha \in \K$ with
$|\alpha| \leq 1$, and $x \in B$.

By part (d) of Lemma 5.32, since each $V_n$ is convex, absorbing, and balanced,
\begin{align*}
\rho(\alpha x,0)
    = F(\alpha x)
    & = \max \left\{ \frac1n\min\{1,p_n(\alpha x)\} : n \in \N \right\}     \\
    & = \max \left\{ \frac1n\min\{1,|\alpha| p_n(x)\} : n \in \N \right\}   \\
    & \leq \max\left\{\frac1n\min\{1,p_n(x)\} : n \in \N\right\}
      = F(x)
      = \rho(x,0)
      < r.
\end{align*}
Thus, $\alpha x \in B$, and so $B$ is balanced. \qed

\item Let $r > 0, B := \{x \in X : \rho(x,0) < r\}$, $x,y \in B$, and
$t \in (0,1)$. Then, $\forall n \in \N$,
\[p_n(tx + (1 - t)y) \leq p_n(tx) + p_n((1 - t)y) = tp_n(x) + (1 - t)p_n(y),\]
by Lemma 5.32, since each $V_n$ is convex and absorbing.
Since $tp_n(x), (1 - t)p_n(y) \geq 0$,
\[\min\{1,tp_n(x) + (1 - t)p_n(y)\}
    \leq \min\{1,tp_n(x)\} + \min\{1,(1 - t)p_n(y)\}.
\]
Thus,
\begin{align*}
\rho(tx + (1 - t)y,0)
    & = \max \left\{ \frac1n\min\{1,p_n(tx + (1 - t)y)\} : n \in \N \right\}     \\
    & \leq \max \left\{ \frac1n\min\{1,tp_n(x) + (1 - t)p_n(y)\} : n \in \N \right\}   \\
    & \leq t\max \left\{ \frac1n\min\{1,p_n(x)\} : n \in \N \right\}   \\
    & + (1 - t)\max \left\{ \frac1n\min\{1,p_n(y)\} : n \in \N \right\}   
     = t\rho(x,0) + (1 - t)\rho(y,0)
      < r.
\end{align*}
Thus, $tx + (1 - t)y \in B$, and so $B$ is convex. \qed

%TODO
\item Let $\tau$ denote the topology of $X$. If $r \in (0,1]$, then the
$\rho$-ball of radius $R$ centered at $0$ is
\begin{align*}
\{x \in X : F(x) < r\}
    & = \{x \in X : \max\{\min\{1,p_n(x)\}/n : n \in \N\} < r\}   \\
    & = \bigcap_{n \in \N} \{x \in X : \min\{1,p_n(x)\} < nr\}  \\
    & = \bigcap_{n = 1}^{\lceil r \rceil} \{x \in X : p_n(x) < nr\}
      = \bigcap_{n = 1}^{\lceil r \rceil} V(p_n,nr).
\end{align*}
By Theorems 14.1 and 14.2, then, the topology induced on $X$ by $\rho$ is no
finer than $\tau$.

I didn't have time to figure out how to show the converse.
\end{enumerate}
\end{question}
\end{document}
