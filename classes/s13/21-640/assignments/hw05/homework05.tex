\documentclass[11pt]{article}
\usepackage{enumerate}
\usepackage{fullpage}
\usepackage{fancyhdr}
\usepackage{amsmath, amsfonts, amsthm, amssymb}
\usepackage{color}
\setlength{\parindent}{0pt}
\setlength{\parskip}{5pt plus 1pt}
\pagestyle{empty}

\def\indented#1{\list{}{}\item[]}
\let\indented=\endlist

\newcounter{questionCounter}
\newcounter{partCounter}[questionCounter]
\newenvironment{question}[2][\arabic{questionCounter}]{%
    \setcounter{partCounter}{0}%
    \vspace{.25in} \hrule \vspace{0.5em}%
        \noindent{\bf #2}%
    \vspace{0.8em} \hrule \vspace{.10in}%
    \addtocounter{questionCounter}{1}%
}{}
\renewenvironment{part}[1][\alph{partCounter}]{%
    \addtocounter{partCounter}{1}%
    \vspace{.10in}%
    \begin{indented}%
       {\bf (#1)} %
}{\end{indented}}

%%%%%%%%%%%%%%%%%%%%%%%HEADER%%%%%%%%%%%%%%%%%%%%%%%%%%%%%%
\newcommand{\myname}{Shashank Singh}
\newcommand{\myandrew}{sss1@andrew.cmu.edu}
\newcommand{\myclass}{21-640 Introduction to Functional Analysis}
\newcommand{\myhwnum}{5}
\newcommand{\duedate}{Wednesday, April 10, 2013}
%%%%%%%%%%%%%%%%%%%%%%%%%%%%%%%%%%%%%%%%%%%%%%%%%%%%%%%%%%%

%%%%%%%%%%%%%%%%%%%%CONTENT MACROS%%%%%%%%%%%%%%%%%%%%%%%%%
\renewcommand{\qed}{\quad $\blacksquare$}
\newcommand{\mqed}{\quad \blacksquare}
\newcommand{\inv}{^{-1}}
\newcommand{\bv}{\mathbf{v}}
\newcommand{\bx}{\mathbf{x}}
\newcommand{\by}{\mathbf{y}}
\newcommand{\bff}{\mathbf{f}}
\newcommand{\bzero}{\mathbf{0}}
\newcommand{\bxi}{\boldsymbol{\xi}}
\newcommand{\boldeta}{\boldsymbol{\eta}}
\newcommand{\dist}{\operatorname{dist}}
\newcommand{\area}{\operatorname{area}}
\newcommand{\Gr}{\operatorname{Gr}} % graph of a function
\renewcommand{\sp}{\operatorname{span}} % span of a set
\newcommand{\sminus}{\backslash}
\newcommand{\N}{\mathbb{N}} % natural numbers
\newcommand{\Z}{\mathbb{Z}} % integers
\newcommand{\Q}{\mathbb{Q}} % rational numbers
\newcommand{\R}{\mathbb{R}} % real numbers
\newcommand{\K}{\mathbb{K}} % underlying field of a linear space
\newcommand{\Ran}{\mathcal{R}} % range of a linear operator
\newcommand{\Nul}{\mathcal{N}} % null-space of a linear operator
\renewcommand{\L}{\mathcal{L}} % bounded linear functions
\newcommand{\pow}[1]{\mathcal{P}\left(#1\right)} % power set of #1
\newcommand{\e}{\varepsilon} % \varepsilon
%%%%%%%%%%%%%%%%%%%%%%%%%%%%%%%%%%%%%%%%%%%%%%%%%%%%%%%%%%%

\begin{document}
\thispagestyle{plain}

{\Large Homework \myhwnum} \\
\myclass \\
Name: \myname \\
Email: \myandrew \\
Due: \duedate

\begin{question}{Problem 1}
We prove the contrapositive statement.

Suppose $\liminf_{n \rightarrow \infty} f(x_n) < f(x)$. Then, there exist
$\e > 0$ and a subsequence $\{x_{n_k}\}_{k = 1}^{\infty}$ such that each
$f(x_{n_k}) \leq f(x) - \e$. Since $f$ is lower semi-continuous,
$B := \{y : f(y) \leq f(x) - \e\}$ is closed. Furthermore, $B$ is convex,
since, if $y_1,y_2 \in B, t \in (0,1)$, then
\[f(ty_1 + (1 - t)y_2) \leq tf(y_1) + (1 - t)f(y_2) \leq f(x) - \e.\]
By Theorem 8.12, since $x \notin B$,
$x_{n_k} \not\rightarrow x$ as $k \rightarrow \infty$. But then
$x_n \not\rightarrow \infty$ as $n \rightarrow \infty$. \qed
\end{question}

\begin{question}{Problem 2}
Since $f$ is proper, $\exists x_0 \in X$ with $f(x_0) < \infty$. Define
$B := \{x \in X : f(x) \leq f(x_0)\}$, and let $m := \inf f[B]$
(\emph{a priori}, $m$ may be $-\infty$). Note that it suffices to show $f$
achieves $m$.

Since $m$ is an infimum, there is a sequence $\{x_n\}_{n = 1}^{\infty}$ with
each $x_n \in B$ and $f(x_n) \rightarrow m$ as $n \rightarrow \infty$.

Since $f$ is coercive, $B$ is bounded, and so, by Theorem 8.1,
$\{x_n\}_{n = 1}^{\infty}$ has a subsequence $\{x_{n_k}\}_{k = 1}^{\infty}$
converging weakly to some $x \in X$. Since $f(x_{n_k}) \rightarrow m$ as
$k \rightarrow \infty$, by the result of problem 1,
\[m
    = \liminf_{k \rightarrow \infty} f(x_{n_k})
    \geq f(x)
\mqed.\]
\vspace{-0.2in}
\end{question}

%TODO
\begin{question}{Problem 5}
Linearity of $T$ is clear, since each coordinate of $Tx$ is a sum of
coordinates of $x$. Define
\[M
 := \sup \left\{ \sum_{n = 1}^{\infty} |a_{mn}| : m \in \N \right\}
    \in \R
.\]

If $x \in c_0$ with $\|x\|_{\infty} = 1$, then
\[\|Tx\|
 =      \sup_{m \in \N} \left| \sum_{n = 1}^{\infty} a_{mn} x_n \right|
 \leq   \sup_{m \in \N} \sum_{n = 1}^{\infty} |a_{mn}|
 = M,
\]
and hence $T \in \L(c_0,c_0)$.
%TODO: show $T : X \rightarrow X$

Define $L$ for all $n \in \N, y^* \in l^1$ by
\[(Ly^*)_n = \sum_{m = 1}^{\infty} a_{mn} y^*_m.\]
We first check that $L \in \L(l^1,l^1)$. Since $L$ is clearly linear, it
suffices to observe that, if $y^* \in l^1$,
\[\|Ly^*\|_1
    = \sum_{n = 1}^{\infty} \left| \sum_{m = 1}^{\infty} a_{mn} y^*_m \right|
    \leq \sum_{m = 1}^{\infty} \sum_{n = 1}^{\infty} | a_{mn} y^*_m |
    \leq \sum_{m = 1}^{\infty} |y^*_m| \sum_{n = 1}^{\infty} | a_{mn} |
    \leq M \sum_{m = 1}^{\infty} |y^*_m|
    = M \|y^*\|_1
\]
(we can switch the order of summation, since each term
is non-negative).

Then, by the usual identification of $l^1$ with $c_0^*$, if
$y^* \in l^1, x \in c_0$,
\begin{align*}
\langle Ly^*,x \rangle
    = \sum_{n = 1}^{\infty} \left( Ly^* \right)_n x_n
    = \sum_{n = 1}^{\infty} \left( \sum_{m = 1}^{\infty} a_{mn} \right) x_n
\end{align*}
Hence, $L = T^*$. \qed
\end{question}

\begin{question}{Problem 7}
We already mentioned in Remark 10.6 that
$Z \subseteq \left( ^\perp Z \right)^\perp$.

Let $X = (l^1,\|\cdot\|_1)$, and let $Z = c_0$. Identifying the dual of $X$
with $l^\infty$ in the usual way, we note $Z \subseteq X^*$, and, since
$(c_0,\|\cdot\|_\infty)$ is complete, $Z$ is closed in $X^*$. If $x \in X$ is
non-zero, then (noting $l^1 \subseteq c_0$), $\langle x, x \rangle \neq 0$, and
hence $x \notin ^\perp Z$. Thus, $^\perp Z = \{0\}$, and so
$\left( ^\perp Z \right)^\perp = l^\infty$. But $c_0 \subsetneq l^\infty$. \qed
\end{question}

%TODO
\begin{question}{Problem 8}
For first step of this proof (showing $cl(\Ran(T))$ contains a ball in $Y$), I
roughly followed the proof of Theorem 4.13 in Rudin's
\emph{Functional Analysis}.

If $y_0 \in Y \sminus cl(T[B_1(0)])$, then, by Theorem 8.13 and the fact that
$T[B_1(0)]$ is balanced, there exists $y^* \in Y^*$ such that
$\|y^*(y)\| \leq \|y^*(y_0)\|, \forall y \in T[B_1(0)]$. Note that
$\forall x \in X$,
\[\|\langle T^*y^*, x \rangle\|
    = \|\langle y^*, Tx \rangle\|
    < \|y^*(y_0)\|.
\]

Thus, $\|T^*y^*\| \leq \|y^*(y_0)\|$, and so
\[\|y_0\|
    \geq \frac{\|y_0\|\|T^*y^*\|}{\|y^*(y_0)\|}
    \geq \frac{\|y_0\| c\|y^*\|}{\|y^*(y_0)\|}
    \geq \frac{c\|y^*(y_0)\|}{\|y^*(y_0)\|}
    = c.
\]
Thus, if $\|y\| \leq c$, then $y \in cl(T[B_1(0)])$, and so
$B_{c/2}(0) \subseteq T[B_1(0)]$.

Then, by a proof identical to the proof of Lemma 4.2 (note that, in Lemma 4.2,
surjectivity is only used to cite Lemma 4.1, whose result we already have), we
have $B_{c/2}(0) \subseteq T[B_2(0)]$.

It now follows from linearity that, $\forall x \in X$,
$B_{c/2}(Tx) \subseteq \Ran(T)$. Thus, it suffices to show that $\Ran(T)$ is dense
in $Y$. Since $S^*$ is clearly injective, $\Nul(S^*) = \{0\}$, and hence
\[cl(\Ran(T)) = ^\perp \left( \Nul(S^* \right) = Y. \mqed\]
\end{question}
\end{document}
