\documentclass[11pt]{article}
\usepackage{enumerate}
\usepackage{fullpage}
\usepackage{fancyhdr}
\usepackage{amsmath, amsfonts, amsthm, amssymb}
\usepackage{color}
\setlength{\parindent}{0pt}
\setlength{\parskip}{5pt plus 1pt}
\pagestyle{empty}

\def\indented#1{\list{}{}\item[]}
\let\indented=\endlist

\newcounter{questionCounter}
\newcounter{partCounter}[questionCounter]
\newenvironment{question}[2][\arabic{questionCounter}]{%
    \setcounter{partCounter}{0}%
    \vspace{.25in} \hrule \vspace{0.5em}%
        \noindent{\bf #2}%
    \vspace{0.8em} \hrule \vspace{.10in}%
    \addtocounter{questionCounter}{1}%
}{}
\renewenvironment{part}[1][\alph{partCounter}]{%
    \addtocounter{partCounter}{1}%
    \vspace{.10in}%
    \begin{indented}%
       {\bf (#1)} %
}{\end{indented}}

%%%%%%%%%%%%%%%%%%%%%%%HEADER%%%%%%%%%%%%%%%%%%%%%%%%%%%%%%
\newcommand{\myname}{Shashank Singh}
\newcommand{\myandrew}{sss1@andrew.cmu.edu}
\newcommand{\myclass}{21-640 Introduction to Functional Analysis}
\newcommand{\myhwnum}{6}
\newcommand{\duedate}{Wednesday, April 17, 2013}
%%%%%%%%%%%%%%%%%%%%%%%%%%%%%%%%%%%%%%%%%%%%%%%%%%%%%%%%%%%

%%%%%%%%%%%%%%%%%%%%CONTENT MACROS%%%%%%%%%%%%%%%%%%%%%%%%%
\renewcommand{\qed}{\quad $\blacksquare$}
\newcommand{\mqed}{\quad \blacksquare}
\newcommand{\inv}{^{-1}}
\newcommand{\bv}{\mathbf{v}}
\newcommand{\bx}{\mathbf{x}}
\newcommand{\by}{\mathbf{y}}
\newcommand{\bff}{\mathbf{f}}
\newcommand{\bzero}{\mathbf{0}}
\newcommand{\bxi}{\boldsymbol{\xi}}
\newcommand{\boldeta}{\boldsymbol{\eta}}
\newcommand{\dist}{\operatorname{dist}}
\newcommand{\area}{\operatorname{area}}
\newcommand{\Gr}{\operatorname{Gr}} % graph of a function
\renewcommand{\sp}{\operatorname{span}} % span of a set
\newcommand{\sminus}{\backslash}
\newcommand{\N}{\mathbb{N}} % natural numbers
\newcommand{\Z}{\mathbb{Z}} % integers
\newcommand{\Q}{\mathbb{Q}} % rational numbers
\newcommand{\R}{\mathbb{R}} % real numbers
\newcommand{\C}{\mathcal{C}} % compact functions
\newcommand{\K}{\mathbb{K}} % underlying field of a linear space
\newcommand{\Ran}{\mathcal{R}} % range of a linear operator
\newcommand{\Nul}{\mathcal{N}} % null-space of a linear operator
\renewcommand{\L}{\mathcal{L}} % bounded linear functions
\newcommand{\pow}[1]{\mathcal{P}\left(#1\right)} % power set of #1
\newcommand{\e}{\varepsilon} % \varepsilon
\newcommand{\wto}{\rightharpoonup} % weak convergence
\newcommand{\wsto}{\stackrel{*}{\rightharpoonup}} % weak-* convergence
%%%%%%%%%%%%%%%%%%%%%%%%%%%%%%%%%%%%%%%%%%%%%%%%%%%%%%%%%%%

\begin{document}
\thispagestyle{plain}

{\Large Homework \myhwnum} \\
\myclass \\
Name: \myname \\
Email: \myandrew \\
Due: \duedate

\begin{question}{Problem 1}
Suppose $\alpha_n \to 0$ as $n \to \infty$. Consider the sequence
$\{T_k\}_{k = 1}^\infty$ of operators in $\L(l^2;l^2)$ defined for all
$x \in l^2$ by
\[(T_kx)_n
    = \left\{
        \begin{array}{cl}
            \alpha_nx_n &   : \mbox{ if } n \leq k    \\
            0           &   : \mbox{ else }
        \end{array}
    \right..
\]
Suppose $\e > 0$. Then, for $n_0$ sufficiently large, $\forall n \geq n_0$,
$\alpha_n < \epsilon$, and hence, $\forall x \in l^2$ with $\|x\|_2 = 1$,
\[\|(T_n - T)x\|_2 \leq \e\|x\|_2 = \e,\]
so $T_n \to T$ in the uniform operator topology. Thus, by Proposition 11.8
and Theorem 12.11, since each $T_n$ has finite rank and is thus compact, $T$ is
compact. Thus, $\alpha_n \to 0$ is sufficient for $T$ compact.

Suppose $\alpha_n \not\to 0$ as $n \to \infty$. Then, since,
\[\|Te^{(n)} - Te^{(m)}\|_2 = \alpha_n^2 + \alpha_m^2,
    \quad \forall n,m \in \N,\]
$\{Te^{(n)}\}$ has no Cauchy subsequence and hence no convergent subsequence.
Thus, $\alpha_n \to 0$ is also necessary for $T$ compact. \qed
\end{question}

\begin{question}{Problem 2}
I solved this problem with a hint from Jimmy Murphy suggesting that I design
components of $Tx$ to be `local averages' of components of $x$.

To construct a counterexample $T$, we make use of the triangle numbers defined
by
\[t_1 = 0, t_{k + 1} = t_k + k, \forall k \in \N.\]

Define $T : l^2 \to l^2$ for all $x \in l^2, n \in \N$ by
\[(Tx)_n
   := \frac1k \sum_{n = t_k}^{t_{k + 1} - 1} x_n,
        \quad \mbox{for} \quad t_k \leq n \leq t_{k + 1} - 1,
\]
\[(\mbox{so } \quad Tx = \left( x_1, \frac{x_2 + x_3}{2},
        \frac{x_2 + x_3}{2}, \frac{x_4 + x_5 + x_6}{3},
          \frac{x_4 + x_5 + x_6}{3}, \frac{x_4 + x_5 + x_6}{3}, \dots \right)
  ).
\]
$T$ is linear, since as each component is a linear combination of components of
$x$. To see that $T$ is continuous (and $\|T\| \leq 1$), it suffices observe
that, by the Cauchy-Schwarz inequality,
$\forall x \in l^2, k \in \N$,
\[\sum_{n = t_k}^{t_{k + 1} - 1} \left( (Tx)_n \right)^2
    = k \left( \sum_{n = t_k}^{t_{k + 1} - 1} \frac{x_n}{k} \right)^2
    \leq k \left( \sum_{n = t_k}^{t_{k + 1} - 1} x_n^2 \right)
        \left( \sum_{n = t_k}^{t_{k + 1} - 1} \frac{1}{k^2} \right)
    = \sum_{n = t_k}^{t_{k + 1} - 1} x_n^2.
\]
Thus, $T \in \L(l^2;l^2)$. Also, $\forall n \in \N$ with
$t_k \leq n \leq t_{k + 1} - 1$, $\|Te^{(n)}\|_2 = \frac{1}{k^2}$, so
$Te^{(n)} \to 0$ strongly.

Now consider $\{x^{(n)}\}_{n = 1}^\infty$ in $l^2$ defined by
\[x^{(n)}_i
    := \left\{
        \begin{array}{cl}
            \frac{1}{\sqrt n} & : \mbox{ if } t_n \leq i \leq t_{n + 1} - 1 \\
            0                 & : \mbox{ else }
        \end{array}
    \right..
\]
$\{x^{(n)}\}_{n = 1}^\infty$ is bounded, since the $n$ non-zero components of
$x^{(n)}$ are identically $\frac{1}{\sqrt n}$, giving
\[\|x^{(n)}\| = n\left( \frac{1}{\sqrt n} \right)^2 = 1.\]
Furthermore, $\forall n \in \N$, $Tx^{(n)} = x^{(n)}$, and so, for
$n, m \in \N$,
\[\|Tx^{(n)} - Tx^{(m)}\| = n\left( \frac{1}{\sqrt n} \right)^2
    + m\left( \frac{1}{\sqrt m} \right)^2 = 2,
\]
and so $\{Tx^{(n)}\}_{n = 1}^\infty$ has no convergent subsequence.
Consequently, $T$ is not compact. \qed
\end{question}

\begin{question}{Problem 3}
Let $L,R \in \L(l^2;l^2)$ denote the shift operators defined by
\[Lx = (x_2,x_3,\dots), \quad Rx = (0,x_1,x_2,\dots), \quad
    \forall x \in l^2.\]
$\forall n \in \N$, put $T_n = L^n$ and $T = 0$. $\forall x \in l^2$,
\[\|T_nx\|_2
    = \sum_{k = n + 1}^\infty x_k^2
    \to 0
\]
as $n \to \infty$, and so $T_n \to T$ in the strong operator topology. An easy
induction argument using $(L^{n - 1}L)^* = L^*(L^{n - 1})^*$ shows
$T_n^* = R^n$. Thus, $\|T_n^*x\|_2 = \|x\|_2$, so $T_n^* \not\to 0 = T^*$
as $n \to \infty$. \qed
\end{question}

\begin{question}{Problem 4}
Assume $X$ is complete and $Y$ is weakly sequentially complete (e.g., by
Theorem 8.5, if $Y$ is reflexive).

For all $x \in X$, the
condition that $\{y^*(T_nx)\}_{n = 1}^\infty$ is convergent for all
$y^* \in Y^*$ is equivalent to $\{T_nx\}_{n = 1}^\infty$ being weakly
convergent.
Thus, we can define $T : X \to Y$ by assigning $Tx$ to be the weak limit of
$\{T_nx\}_{n = 1}^\infty$ (which is in $Y$ by weak sequential completeness).
Since the weak limit operator is linear, $T$ is linear. Since $X$ is complete,
by the Principle of Uniform Boundedness, $T$ is bounded, and hence
$T \in \L(X;Y)$. \qed
\end{question}

\newpage
\begin{question}{Problem 5}
\vspace{-0.2in}
\begin{enumerate}[(a)]
\item Let $X = l^2, Y = l^1$, and suppose $T \in \L(X;Y)$. If
$\{x_n\}_{n = 1}^\infty$ is a bounded sequence in $X$, by Theorem
8.1, $\{x_n\}_{n = 1}^\infty$ has a weakly convergent subsequence
$\{x_{n_k}\}_{k = 1}^\infty$. Since continuous linear operators respect
weak convergence, $\{Tx_{n_k}\}_{k = 1}^\infty$ is weakly convergent, and
hence, since, in $l^1$, weak convergence is equivalent to strong convergence,
$\{Tx_{n_k}\}_{k = 1}^\infty$ is a convergent subsequence of
$\{Tx_n\}_{n = 1}^\infty$. Thus, $T \in C(X;Y)$. Since
$\C(X;Y) \subseteq \L(X;Y)$, $\C(X;Y) = \L(X;Y)$. \qed

\item I wasn't able to solve this problem.
\end{enumerate}
\vspace{-0.2in}
\end{question}

\begin{question}{Problem 6}
\vspace{-0.2in}
\begin{enumerate}[(a)]
\item $T$ is not surjective. If it were, by the Open Mapping Theorem,
$T[B_1(0)]$ would be open and hence contain a non-empty ball $B$. Since $Y$ is
infinite dimensional, $B$ would contain a sequence with no convergent
subsequence, contradicting the compactness of $T$. \qed

\item Let $X = (l^{\infty}, \|\cdot\|_\infty)$, and let
$Y = (l^\infty,\|\cdot\|)$ where $\|x\| := \sup_{n \in \N} x_n/n$, and let
$I \in \L(X;Y)$ be the identity (which is continuous, since clearly $\|\cdot\|$
is bounded by $\|\cdot\|_\infty$, and note that $I$ is surjective.

Now consider the sequence $\{I_k\}_{k = 1}^\infty$ in $\L(X;Y)$ defined,
$\forall x \in l^\infty, n \in \N$ by
\[(I_kx)_n
    = \left\{
        \begin{array}{cl}
            \alpha_nx_n &   : \mbox{ if } n \leq k    \\
            0           &   : \mbox{ else }
        \end{array}
    \right..
\]
$\forall x \in l^2$ with $\|x\| = 1$,
$\|(I_k - I)x\|_2 \leq \frac1k\|x\| = \frac1k$,
and so $I_k \to I$ in the uniform operator topology. Thus, by Proposition 11.8
and Theorem 12.11, since each $I_n$ has finite rank and is thus compact, $I$ is
compact. \qed
\end{enumerate}
\vspace{-0.2in}
\end{question}

\begin{question}{Problem 7}
\vspace{-0.15in}
\begin{enumerate}[(a)]
\item Suppose $T \in \L(l^2;l^2)$. As shown in the solution to part (a) of
Problem 5, $T$ is compact. Then, by part (a) of Problem 6, $T$ is not
surjective. \qed

\item By the identification of $(c_0)^*$ with $l^1$ and $(l^2)^*$ with
$(l^2)^*$, if $L \in \L(c_0;l^2)$, then, as shown in the solution to part (a)
of Problem 5, $L^* : l^2 \to l^1$ would be compact. Then, by Theorem 11.15,
$T$ would be compact, and hence, by part (a) of Problem 6, $T$ is not
surjective.
\end{enumerate}
\vspace{-0.2in}
\end{question}

\begin{question}{Problem 8}
If $x = 0$, then, as $n \to \infty$, $\|x_n\| \to \|x\|$ immediately implies
$x_n \to x$. Thus, we assume $x \neq 0$. Then, since $\|x_n\| \to \|x\|$ as
$n \to \infty$, by considering only $n$ sufficiently large, we may assume,
without loss of generality, that each $x_n \neq 0$, and we may therefore define
\[z := \frac{x}{\|x\|} \quad \mbox{ and } z_n := \frac{x_n}{\|x_n\|},
    \quad \forall n \in \N.
\]
It suffices to show $z_n \to z$ strongly, and so, since each
$\|z_n\| = \|z\| = 1$, by uniform convexity, it suffices to show that
$\|z_n + z\| \to 2$. By part (iii) of Theorem 7.15, since $z_n + z \wto 2z$ weakly,
\[2
    = \|2z\|
    \leq \liminf_{n \to \infty} \|z_n + z\|
    \leq \limsup_{n \to \infty} \|z_n + z\|
    \leq \limsup_{n \to \infty} \|z_n\| + \|z\|
    = \|z\| + \|z\|
    = 2.
\]
\vspace{-0.3in}
\end{question}

\begin{question}{Problem 9}
\begin{enumerate}[(a)]
\item
Let $x,y \in X$ with $x \neq y$ and $\|x\| = \|y\| = 1$, and let $t \in (0,1)$.
Since $\|tx + (1 - t)y\| \leq t\|x\| + (1 - t)\|y\| = 1$, it suffices to show
that $\|tx + (1 - t)y\| \neq 1$. If it were the case that
$\|\frac12x + \frac12y\| = 1$, then $\|x + y\| = 2$, and hence, by uniform
convexity (using the constant sequences
$\{x\}_{i = 1}^\infty, \{y\}_{i = 1}^\infty$), $x = y$. Thus, it suffices to
show that, if $\|tx + (1 - t)y\| = 1$, then $\|\frac12x + \frac12y\| = 1$.

The case $t = 1/2$ is trivial, and the case $t \in (1/2,1)$ follows by
switching $x$ and $y$, so we may assume $t \in (0,1/2)$.
Then, $tx + (1 - t)y$ is a convex combination of $x$ and $\frac12x + \frac12y$,
and so, for some $t_2 \in (0,1)$,
\[1
    = \|tx + (1 - t)y\|
    = \|t_2x + (1 - t_2)(\frac12x + \frac12y)\|
    \leq t_2 + (1 - t_2)\|(\frac12x + \frac12y)\|,
\]
and so $1 \leq \|(\frac12x + \frac12y)\| \leq 1$. \qed

\item Let $\displaystyle X' := \prod_{i = 1}^\infty \R^2$,
let
\vspace{-0.2in}
\[X := \left\{ x \in X'
        \, : \, \sum_{i = 1}^\infty \|(x_i,y_i)\|_k < \infty \right\},\]
(where $\|\cdot\|_k$ denotes the usual $k$-norm on $\R^2$) and, define
$\|\cdot\| : X \to \R$ for all $x \in X$ by
\[\|x\| = \sum_{i = 1}^\infty \|(x_i,y_i)\|_k.\]
The proof that $(X,\|\cdot\|)$ is a Banach space is essentially identical to
the proof for $l^1$.

Suppose $x, y \in X$, $\|x\| = \|y\| = 1$ and $x \neq y$ (say $x_n \neq y_n$).
Then, since each $(R^2,\|\cdot\|_k)$ is strictly convex,
$\|tx_n + (1 - t)y_n\| < \frac{\|x_n\|_n + \|y_n\|_n}{2}$ (and each
$\|tx_k + (1 - t)y_k\| \leq \frac{\|x_k\|_k + \|y_k\|_k}{2}$), and so 
\[\|tx + (1 - t)y\|
    = \sum_{k = 1}^\infty \|tx_k + (1 - t)y_k\|_k
    < \sum_{k = 1}^\infty \frac{\|x_i\|_k + \|y_i\|_k}{2}
    = \frac12(\|x\| + \|y\|)
    = 1.
\]
Thus, $X$ is strictly convex. However, suppose
$\forall n \in \N, x^{(n)}$ and $y^{(n)}$ have $x^{(n)}_n = (1,0),
y^{(n)}_n = (0,1)$, and $x^{(n)}_i = y^{(n)}_i = 0$ for $i \neq n$. Then, each
$\|x^{(n)}\| = \|y^{(n)}\| = 1$ and, as $n \to \infty$,
\[\|x^{(n)} + y^{(n)}\| = 2^{\frac{1}{1 + 1/n}} \to 2\]
but
\[\|x^{(n)} - y^{(n)}\| = 2^{\frac{1}{1 + 1/n}} \to 2.\]
Therefore, $X$ is not uniformly convex. The proof that $X$ is separable is
similar to the proof of separability for $l^1$. I'm not quite sure about
reflexivity\dots
\end{enumerate}
\end{question}
\end{document}
