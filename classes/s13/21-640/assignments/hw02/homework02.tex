\documentclass[11pt]{article}
\usepackage{enumerate}
\usepackage{fullpage}
\usepackage{fancyhdr}
\usepackage{amsmath, amsfonts, amsthm, amssymb}
\usepackage{color}
\setlength{\parindent}{0pt}
\setlength{\parskip}{5pt plus 1pt}
\pagestyle{empty}

\def\indented#1{\list{}{}\item[]}
\let\indented=\endlist

\newcounter{questionCounter}
\newcounter{partCounter}[questionCounter]
\newenvironment{question}[2][\arabic{questionCounter}]{%
    \setcounter{partCounter}{0}%
    \vspace{.25in} \hrule \vspace{0.5em}%
        \noindent{\bf #2}%
    \vspace{0.8em} \hrule \vspace{.10in}%
    \addtocounter{questionCounter}{1}%
}{}
\renewenvironment{part}[1][\alph{partCounter}]{%
    \addtocounter{partCounter}{1}%
    \vspace{.10in}%
    \begin{indented}%
       {\bf (#1)} %
}{\end{indented}}

%%%%%%%%%%%%%%%%%%%%%%%HEADER%%%%%%%%%%%%%%%%%%%%%%%%%%%%%%
\newcommand{\myname}{Shashank Singh}
\newcommand{\myandrew}{sss1@andrew.cmu.edu}
\newcommand{\myclass}{21-640 Introduction to Functional Analysis}
\newcommand{\myhwnum}{2}
\newcommand{\duedate}{Tuesday, February 5, 2013}
%%%%%%%%%%%%%%%%%%%%%%%%%%%%%%%%%%%%%%%%%%%%%%%%%%%%%%%%%%%

%%%%%%%%%%%%%%%%%%%%CONTENT MACROS%%%%%%%%%%%%%%%%%%%%%%%%%
\renewcommand{\qed}{\quad $\blacksquare$}
\newcommand{\mqed}{\quad \blacksquare}
\newcommand{\inv}{^{-1}}
\newcommand{\bv}{\mathbf{v}}
\newcommand{\bx}{\mathbf{x}}
\newcommand{\by}{\mathbf{y}}
\newcommand{\bff}{\mathbf{f}}
\newcommand{\bzero}{\mathbf{0}}
\newcommand{\bxi}{\boldsymbol{\xi}}
\newcommand{\boldeta}{\boldsymbol{\eta}}
\newcommand{\dist}{\operatorname{dist}}
\newcommand{\area}{\operatorname{area}}
\newcommand{\sminus}{\backslash}
\newcommand{\N}{\mathbb{N}} % natural numbers
\newcommand{\Z}{\mathbb{Z}} % integers
\newcommand{\Q}{\mathbb{Q}} % rational numbers
\newcommand{\R}{\mathbb{R}} % real numbers
\newcommand{\pow}[1]{\mathcal{P}\left(#1\right)} % power set of #1
\newcommand{\e}{\varepsilon} % \varepsilon
%%%%%%%%%%%%%%%%%%%%%%%%%%%%%%%%%%%%%%%%%%%%%%%%%%%%%%%%%%%

\begin{document}
\thispagestyle{plain}

{\Large Homework \myhwnum} \\
\myclass \\
Name: \myname \\
Email: \myandrew \\
Due: \duedate \\

\begin{question}{Problem 2}
Assuming the Axiom of Choice, let $(x_i | i \in I)$ be a Hamel basis of unit
vectors for $X$ (given a Hamel basis, we can divide each vector by its norm to
create such a basis).
Let $J \subseteq I$ be countably infinite, and let $\sigma: \N \rightarrow J$
be a bijection. Define $L: X \rightarrow \mathbb{K}$, by 
\[
L x = \sum_{i \in \N}i\alpha_{\sigma(i)},
\quad \forall x = \sum_{i \in \N}\alpha_{\sigma(i)}x_{\sigma(i)} \in 
span(x_i | i \in J)\]
(noting at most finitely many $\alpha_{\sigma(i)} \neq 0$,
so $L$ is well-defined), and $L x = 0$ for all other $x \in X$.
Note that $L$ is clearly linear. However, $L$ is unbounded on the unit ball,
since, for any $n \in \N$, $L x_{\sigma(n)} = n$. By Proposition 2.8, $L$ is
discontinuous. \qed
\end{question}

\begin{question}{Problem 4}
Assuming the Axiom of Choice, let $(x_i | i \in I)$ be a Hamel basis for $X$.
Since each $x \in X$ can be written uniquely in the form
\[
 x = \sum_{i \in J_x} \alpha_i x_i,
\]
where $J_x \subseteq I$ is finite, and each $\alpha_i$ is non-zero, we can
define the norms
\[
 \|x\|_1 := \sum_{i \in J_x} |\alpha_i|
 \quad\mbox{ and }\quad
 \|x\|_{\infty} := \max\{\alpha_i : i \in J_x\}.
\]
Let $J \subseteq I$ be countably infinite, and let $\sigma : \N \rightarrow J$
be a bijection.
For each $n \in \N$, define $x_n := \sum_{i = 1}^{n} x_{\sigma(i)}$.
Then, $\forall n \in \N$, $\|x_n\|_1 = n$, but $\|x_n\|_{\infty} = 1$.
Thus, there is no $M \in \N$ with $\|x\|_1 \leq M\|x\|_{\infty}$, so that
$\|\cdot\|_1$ and $\|\cdot\|_{\infty}$ are not equivalent norms. \qed
\end{question}

\begin{question}{Problem 6}
Suppose, for sake of contradiction, that $L$ is discontinuous. Then, by
Proposition 2.8, $L$ is unbounded on the unit ball, so that $\exists f$ in the
unit ball with $Lf > Lg$, where $g = 1$ is the constant function that is
identically $1$ on $[0,1]$. Since $f$ is in the unit ball (and thus is bounded
above by $g$, $g - f \in K$. However, since $L$ is linear, $L(g - f) = Lg - Lf
< 0$, which is a contradiction. \qed
\end{question}

\newpage
\begin{question}{Problem 9}
($\Rightarrow$) Suppose first that $X$ is complete, and let
$x = \{x_n\}_{n = 1}^{\infty}$ be an absolutely summable sequence in $X$.
It suffices to show that $x$ is Cauchy. For any $n,k \in \N$, by the Triangle
Inequality,
\[
\|x_{n+k} - x_n\| \leq \sum_{i = n}^{\infty}\|x_{n + 1} - x_n \| \leq 2\sum_{i
= n}^{\infty}\|x_n\| \rightarrow 0
\]
as $n \rightarrow \infty$, since $x$ is absolutely summable, and a series
converges only if its tail goes to zero. Thus, $x$ is Cauchy. 

($\Leftarrow$) Suppose now that every absolutely summable sequence in $X$ is
summable, and let $x = \{x_n\}_{n = 1}^{\infty}$ be a Cauchy sequence in $X$.
It suffices to show that $x$ converges to some $y \in X$.

Since $x$ is Cauchy, $\forall k \in \N, \exists n \in \N$ such that for every
$m \in \N, \|x_{n + m} - x_n\| < 2^{-k}$. Thus, we can take a subsequence
$\{x_{n_k}\}_{k = 1}^{\infty}$ of $x$ such that, for each $k \in \N$, $\|x_{n_{k
+ 1}} - x_{n_k}\| < 2^{-k}$. Thus, the sequence
$\{x_{n_{k + 1}} - x_{n_k}\}_{k = 1}^{\infty}$ is absolutely summable, and so,
it is summable. However, $\forall k \in \N$, $\sum_{i = 1}^k x_{n_{i + 1}} -
x_{n_i} = x_{n_{k + 1}} - x_{n_1}$, so that, since the limit of this sum is in
$X$, $y := x_{n_1} + \sum_{i = 1}^{\infty} x_{n_{i + 1}} - x_{n_i} =
\lim_{k \rightarrow \infty}x_{n_k} \in X$.  

It remains only to state that any Cauchy sequence with a convergent
subsequence is itself convergent, although we show this explicitly here.
Let $\e > 0$. Since, $x$ is Cauchy and $\{x_{n_k}\}_{k = 1}^{\infty}$
converges to $y$, $\exists n \in \N$ such that, $\forall m \in \N$, $\|x_n -
x_{n + m}\| < \e/2$, and $\|y - x_n\| < \e/2$. Then, $\forall m \in \N$,
\[
 \|y - x_{n + m}\| 
 \leq \|y - x_n\| + \|x_n - x_{n + m}\|
 < \e/2 + \e/2
 = \e.
\]
Therefore, $x$ converges to $y$. \qed
\end{question}

\begin{question}{Problem 11}
Let $X = c$, the space of convergent real sequences, equipped with the norm
$\|\cdot\| = \|\cdot\|_{\infty}$. As done in the proof that the unit ball in any infinite
dimensional non-linear space is not compact, using Riesz's Lemma and induction,
we can construct a sequence $\{x_n\}_{n = 1}^{\infty}$ in $c$ such that each
$\|x_n\| \leq 1$ and, $\forall n, m \in \N$ distinct $\|x_n - x_m\| > \frac12$.
For each $n \in \N$, let $B_n := B_{\frac12} (x_n)$ be the ball of radius
$\frac12$ centered at $x_n$. Note that, by construction, the $B_n$'s are
disjoint. Define, for each $n \in \N$, $f_n : X \rightarrow
\R$ by
\[
 f_n(x) = \left\{
    \begin{array}{cl}
    1 - 2\|x - x_n\| & : \mbox{if $x \in B_n$}, \\
    0 & : \mbox{otherwise}
    \end{array}
\right.,
\]
$\forall x \in X$. Define $f : X \rightarrow \R$ by $f(x) := \sum_{n =
1}^{\infty}nf_n(x)$, $\forall x \in X$.

$\forall n \in \N$, define $S_n := X \sminus \bigcup_{i = n + 1}^{\infty}
B_n$. On each $S_n$, $f$ is a finite sum of continuous functions, and is thus
continuous. Thus, since $X = \bigcup_{n = 1}^{\infty}S_n$, $f$ is
continuous on $X$.

However, $\forall n \in \N$, $f(x_n) = n$, so that $f$ is unbounded on
$B_1(0)$. \qed
\end{question}

\newpage
\begin{question}{Problem 13}
It is immediate, from the definition of continuity, that, if a linear function
$T$ is continuous, then $\mathcal{N}(T)$ is closed.

However, the converse is not true. Suppose, for example, that $X$ is a linear
space, and $\|\cdot\|_1$ and $\|\cdot\|_2$ are two norms on $X$ that are not
equivalent (we have already shown the existence of such norms for some spaces;
see Problem 4). Then, by Remark 2.14, the identity map $Id : (X,\|\cdot\|_1)
\rightarrow (X,\|\cdot\|_2)$ is discontinuous, but $Id$ is clearly linear, and
$\mathcal{N}(Id) = \{0\}$ is closed. \qed
\end{question}

\begin{question}{Problem 14}
I wasn't able to complete this problem.
\end{question}
\end{document}
