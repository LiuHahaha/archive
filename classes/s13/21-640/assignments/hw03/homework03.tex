\documentclass[11pt]{article}
\usepackage{enumerate}
\usepackage{fullpage}
\usepackage{fancyhdr}
\usepackage{amsmath, amsfonts, amsthm, amssymb}
\usepackage{color}
\setlength{\parindent}{0pt}
\setlength{\parskip}{5pt plus 1pt}
\pagestyle{empty}

\def\indented#1{\list{}{}\item[]}
\let\indented=\endlist

\newcounter{questionCounter}
\newcounter{partCounter}[questionCounter]
\newenvironment{question}[2][\arabic{questionCounter}]{%
    \setcounter{partCounter}{0}%
    \vspace{.25in} \hrule \vspace{0.5em}%
        \noindent{\bf #2}%
    \vspace{0.8em} \hrule \vspace{.10in}%
    \addtocounter{questionCounter}{1}%
}{}
\renewenvironment{part}[1][\alph{partCounter}]{%
    \addtocounter{partCounter}{1}%
    \vspace{.10in}%
    \begin{indented}%
       {\bf (#1)} %
}{\end{indented}}

%%%%%%%%%%%%%%%%%%%%%%%HEADER%%%%%%%%%%%%%%%%%%%%%%%%%%%%%%
\newcommand{\myname}{Shashank Singh}
\newcommand{\myandrew}{sss1@andrew.cmu.edu}
\newcommand{\myclass}{21-640 Introduction to Functional Analysis}
\newcommand{\myhwnum}{3}
\newcommand{\duedate}{Friday, February 15, 2013}
\newcommand{\mycollaborators}{Evan Cavallo (ecavallo)}
%%%%%%%%%%%%%%%%%%%%%%%%%%%%%%%%%%%%%%%%%%%%%%%%%%%%%%%%%%%

%%%%%%%%%%%%%%%%%%%%CONTENT MACROS%%%%%%%%%%%%%%%%%%%%%%%%%
\renewcommand{\qed}{\quad $\blacksquare$}
\newcommand{\mqed}{\quad \blacksquare}
\newcommand{\inv}{^{-1}}
\newcommand{\bv}{\mathbf{v}}
\newcommand{\bx}{\mathbf{x}}
\newcommand{\by}{\mathbf{y}}
\newcommand{\bff}{\mathbf{f}}
\newcommand{\bzero}{\mathbf{0}}
\newcommand{\bxi}{\boldsymbol{\xi}}
\newcommand{\boldeta}{\boldsymbol{\eta}}
\newcommand{\dist}{\operatorname{dist}}
\newcommand{\area}{\operatorname{area}}
\newcommand{\Gr}{\operatorname{Gr}} % graph of a function
\newcommand{\sminus}{\backslash}
\newcommand{\N}{\mathbb{N}} % natural numbers
\newcommand{\Z}{\mathbb{Z}} % integers
\newcommand{\Q}{\mathbb{Q}} % rational numbers
\newcommand{\R}{\mathbb{R}} % real numbers
\newcommand{\K}{\mathbb{K}} % real numbers
\newcommand{\Ran}{\mathcal{R}} % range of a linear operator
\newcommand{\pow}[1]{\mathcal{P}\left(#1\right)} % power set of #1
\newcommand{\e}{\varepsilon} % \varepsilon
%%%%%%%%%%%%%%%%%%%%%%%%%%%%%%%%%%%%%%%%%%%%%%%%%%%%%%%%%%%

\begin{document}
\thispagestyle{plain}

{\Large Homework \myhwnum} \\
\myclass \\
Name: \myname \\
Email: \myandrew \\
Due: \duedate \\
Collaborators: \mycollaborators

\begin{question}{Problem 3}
We suppose $\Ran(T)$ is not of the first category in $Y$, and show that
$\Ran(T) = Y$. Since $\Ran(T)$ is a linear manifold, it suffices to show that
$\Ran(T)$ contains some ball, say, of radius $\delta > 0$, in $Y$, since, then,
$\Ran(T)$ contains a Hamel basis of $Y$ (normalized to have each element's norm
less than $\delta$). Didn't have time to finish writing this one.
\end{question}

\begin{question}{Problem 4}
By Remark 2.14, the identity function $I : (X,\|\cdot\|_1) \rightarrow
(X,\|\cdot\|_2)$ is discontinuous. Then, by the Closed Graph Theorem, the graph
$\Gr(I) = \{(x,x) : x \in X\}$ is not closed, so that there is a sequence
$\{x_n\}_{n = 1}^{\infty}$ in $X$ with $(x_n,x_n) \rightarrow (y,z)$ as
$n \rightarrow \infty$ and $y \neq z$. From the definition of the product norm
on $X^2$, it follows that $\|x_n - y\| \rightarrow 0$ and
$\|x_n - z\| \rightarrow 0$ as $n \rightarrow \infty$. \qed
\end{question}

\begin{question}{Problem 6}
Let $X = Y = c_0$ (equipped with the norm $\|\cdot\| = \|\cdot\|_{\infty}$, so
that $X$ and $Y$ are Banach spaces), and define $T: X \rightarrow Y$ such that,
for all sequences $\left\{ x_n \right\}_{n = 1}^{\infty} \in c_0$,
$T\left( \left\{ x_n \right\}_{n = 1}^{\infty} \right)
 = \left\{ n\inv x_n \right\}_{n = 1}^{\infty}$. Clearly $T$ is linear and
injective, and, since $\sup \{\|Tx\| : x \in c_0, \|x\| = 1\} = 1$, $T$ is
continuous.

$\forall k \in \N$, since the sequence whose first $k$ terms are $1$ and whose
remaining terms are $0$ is in $c_0$, the sequence
$S_k = \{x_n\}_{n = 1}^{\infty}$ with $x_n = n\inv$ for $n \leq k$ and
$x_n = 0$ otherwise is in $c_0$, and furthermore $\|S_k - \{1\}_{n =
1}^{\infty}\| = n\inv \rightarrow 0\|$ as $k \rightarrow \infty$. However,
since the constant sequence $\{1\}_{n = 1}^{\infty} \notin c_0$, so
$\{n\inv\}_{n = 1}^{\infty} \notin T[X]$, and thus $T[X]$ is not closed. \qed
\end{question}

\begin{question}{Problem 7}
Since $V$ is continuous, the graph $\Gr(V)$ is closed, so that
$V[Y] \times Y = \{(Vy,y) : y \in Y\}$ is closed in $Z \times Y$. Thus, by
definition of the product norm, $V[Y]$ is closed in $Z$, and thus, since $V$ is
linear, $V[Y]$ is a Banach space with $V : Y \rightarrow V[Y]$ bijective. Then,
by the Bounded Inverse Theorem, $V\inv : V[Y] \rightarrow Y$ is continuous.
Then, since $U = V\inv \circ T$, $U$ is continuous. \qed
\end{question}

\newpage
\begin{question}{Problem 8}
Since $X \subseteq Y$, $\forall x_n \in \R^{\N}$, $\|x_n\|_Y \rightarrow 0$ as
$n \rightarrow \infty$ implies $\|x_n\|_X \rightarrow 0$ as $n \rightarrow
\infty$. Suppose some sequence $(x_n,x_n) \rightarrow (x,y)$ in $(X,
\|\cdot\|_X) \times (X, \|\cdot\|_Y)$. Then, since $\|x_n - x\| \rightarrow 0$
a $n \rightarrow \infty$, $\|x_n - x\|_X \rightarrow 0$ as $n \rightarrow
\infty$. It follows that $x = y$, so that the graph $\Gr(I)$ of the identity
$I : (X,\|\cdot\|_X) \rightarrow (X,\|\cdot\|_Y)$ is closed.

By the Closed Graph Theorem, then, $I$ is continuous. It follows, by Remark
2.14 that $\|\cdot\|_X$ and $\|\cdot\|_Y$ are equivalent norms on $X$, implying
the desired result.
\end{question}

\begin{question}{Problem 9}
Suppose $Y$ is a Banach space over $\K$, and let
$L : X \rightarrow \K$ be discontinuous and linear (we showed the
existence of such mappings in Problem 2 of Assignment 2). Since $L$ is linear,
the graph $X := \Gr(L)$ is a normed linear space (under the product norm). Let
$\pi : X \rightarrow \K$ be the projection mapping $(x,Tx) \mapsto Tx$, noting
that, by definiton of the product topology, projections are continuous. Then,
$f : Y \rightarrow X$ defined by $f(x) = (x,Tx)$ is discontinuous, since
otherwise $L = \pi \circ f$ would be continuous.

Since $f$ is bijective, define $T = f\inv$. Since $f$ is discontinuous, it is
immediate from the topological definition of continuity that $T$ is not open.
However, since $T$ is the just projection of $X$ into $Y$, $T$ is linear,
continuous, and surjective, as desired. \qed
\end{question}

\begin{question}{Problem 10}
Since $|a_n| \rightarrow \infty$ as $n \rightarrow \infty$, we can take a
subsequence $\left\{ a_{n_k} \right\}_{k = 1}^{\infty}$ with $a_{n_k} > 4^k$,
$\forall k \in \N$. Define $g : \R \rightarrow \R$ to be the $2\pi$-periodic
function defined by
\[g(x) = \sum_{k = 1}^{\infty} 2^{-k} \sin(n_k x),
\quad \forall x \in \R.\]

It is easy to show that the sequence of continuous functions
$\sum_{k = 1}^n 2^{-k} \sin(n_k x) \rightarrow g$ uniformly on $\R$ as
$n \rightarrow \infty$, so that $g$ is continuous. Note also that,
$\forall n,m \in \N$,
\[\int_0^{2\pi} \sin(nx) \sin(mx) \, dx
 =  \left\{
        \begin{array}{cl}
            \pi - \frac{\sin(4 \pi n)}{4n} \geq 1   & : n = m               \\
            0                                       & : \mbox{otherwise}
        \end{array}
    \right..
\]
Therefore, $\forall k \in \N$,
\[
      a_{n_k} \int_0^{2\pi} g(x) \sin(n_k x) \, dx
 \geq 4^k \sum_{i = 1}^{\infty}
                \int_0^{2\pi} 2^{-i} \sin(n_i x) \sin(n_k x) \, dx
 \geq 4^k \left( 2^{-k} \right)
 =    2^k \rightarrow \infty
\]
as $k \rightarrow \infty$ (where we use the Bounded Convergence Theorem move
the summation outside the integral). Thus, since it has an unbounded
subsequence, the desired sequence is unbounded. \qed
\end{question}
\end{document}
