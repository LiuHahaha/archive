\documentclass[11pt]{article}
\usepackage{enumerate}
\usepackage{fullpage}
\usepackage{fancyhdr}
\usepackage{amsmath, amsfonts, amsthm, amssymb}
\usepackage{color}
\setlength{\parindent}{0pt}
\setlength{\parskip}{5pt plus 1pt}
\pagestyle{empty}

\def\indented#1{\list{}{}\item[]}
\let\indented=\endlist

\newcounter{questionCounter}
\newcounter{partCounter}[questionCounter]
\newenvironment{question}[2][\arabic{questionCounter}]{%
    \setcounter{partCounter}{0}%
    \vspace{.25in} \hrule \vspace{0.5em}%
        \noindent{\bf #2}%
    \vspace{0.8em} \hrule \vspace{.10in}%
    \addtocounter{questionCounter}{1}%
}{}
\renewenvironment{part}[1][\alph{partCounter}]{%
    \addtocounter{partCounter}{1}%
    \vspace{.10in}%
    \begin{indented}%
       {\bf (#1)} %
}{\end{indented}}

%%%%%%%%%%%%%%%%%%%%%%%HEADER%%%%%%%%%%%%%%%%%%%%%%%%%%%%%%
\newcommand{\myname}{Shashank Singh}
\newcommand{\myandrew}{sss1@andrew.cmu.edu}
\newcommand{\myclass}{21-640 Introduction to Functional Analysis}
\newcommand{\myhwnum}{1}
\newcommand{\duedate}{Friday, March 29, 2013}
%%%%%%%%%%%%%%%%%%%%%%%%%%%%%%%%%%%%%%%%%%%%%%%%%%%%%%%%%%%

%%%%%%%%%%%%%%%%%%%%CONTENT MACROS%%%%%%%%%%%%%%%%%%%%%%%%%
\renewcommand{\qed}{\quad $\blacksquare$}
\newcommand{\mqed}{\quad \blacksquare}
\newcommand{\inv}{^{-1}}
\newcommand{\bv}{\mathbf{v}}
\newcommand{\bx}{\mathbf{x}}
\newcommand{\by}{\mathbf{y}}
\newcommand{\bff}{\mathbf{f}}
\newcommand{\bzero}{\mathbf{0}}
\newcommand{\bxi}{\boldsymbol{\xi}}
\newcommand{\boldeta}{\boldsymbol{\eta}}
\newcommand{\dist}{\operatorname{dist}}
\newcommand{\area}{\operatorname{area}}
\newcommand{\Gr}{\operatorname{Gr}} % graph of a function
\renewcommand{\sp}{\operatorname{span}} % span of a set
\newcommand{\sminus}{\backslash}
\newcommand{\N}{\mathbb{N}} % natural numbers
\newcommand{\Z}{\mathbb{Z}} % integers
\newcommand{\Q}{\mathbb{Q}} % rational numbers
\newcommand{\R}{\mathbb{R}} % real numbers
\newcommand{\K}{\mathbb{K}} % underlying field of a linear space
\newcommand{\Ran}{\mathcal{R}} % range of a linear operator
\newcommand{\Nul}{\mathcal{N}} % null-space of a linear operator
\renewcommand{\L}{\mathcal{L}} % bounded linear functions
\newcommand{\pow}[1]{\mathcal{P}\left(#1\right)} % power set of #1
\newcommand{\e}{\varepsilon} % \varepsilon
%%%%%%%%%%%%%%%%%%%%%%%%%%%%%%%%%%%%%%%%%%%%%%%%%%%%%%%%%%%

\begin{document}
\thispagestyle{plain}

{\Large Midterm \myhwnum} \\
\myclass \\
Name: \myname \\
Email: \myandrew \\
Due: \duedate

\begin{question}{Problem 1}
We show $\L(l^2,l^2)$ is not separable by finding an uncountable subset
with discrete induced topology.

Let $\pow{\N}$ denote the set of subsets of $\N$. $\forall K \in \pow{N}$,
define $T_K : l^2 \rightarrow l^2$ by
\[(T_K(x))_i =
    \left\{
        \begin{array}{cl}
            x_i & : \mbox{ if } i \in K \\
            0   & : \mbox{ else }
        \end{array}
    \right.,
\forall i \in \N, x \in l^2.\]
Let $S := \{T_K : K \in \pow{\N} \sminus \{\emptyset\}\}$. Clearly, $S$ is
uncountable. $\forall T \in S$, linearity is clear, and, furthermore,
$\forall x \in l^2$, $\|T(x)\|_2 \leq \|x\|_2$, so that $T$ is bounded. Thus,
$S \subseteq \L(l^2,l^2)$. Also, for each $T \in S$, it is easy to
find $x \in l^2$ with $T_K(x) = x$, and so $\|T_K\| \geq 1$.

If $K_1,K_2 \in \pow{\N}$ are distinct, then, since
$K_3 := (K_1 \sminus K_2) \sminus (K_2 \sminus K_1) \neq \emptyset$,
$\|T_{K_1} - T_{K_2}\| = \|T_{K_3}\| \geq 1$. Hence, for $T,R \in S$ distinct,
$B_{\frac12}(T) \cap B_{\frac12}(S) = \emptyset$.

If there were a dense set $A \subseteq \L(l^2,l^2)$, then, $\forall T \in S$,
$B_{\frac12}(T) \cap A \neq \emptyset$. However, a countable set cannot have
non-empty intersection with uncountably many disjoint sets, so $A$ is
uncountable. \qed
\end{question}

\begin{question}{Problem 2}
We prove the contrapositive statement. Suppose $T$ is discontinuous and
hence unbounded on $B_1(0)$. Then, there is a sequence
$\{x_n\}_{n = 1}^{\infty}$ such that, $\forall n \in \N$,
$\|T(x_n)\| \geq n^2$. $\forall n \in \N$,
$\left\| \frac{x_n}{n} \right\| \leq n$, so that, as $n \rightarrow \infty$,
$x_n \rightarrow 0$, but
$\left \|T\left( \frac{x_n}{n} \right) \right\| \geq n \rightarrow \infty$.
\qed
\vspace{-0.1in}
\end{question}

\begin{question}{Problem 3}
If $x_1,x_2 \in X$ with $x_1 = y_1 + z_1, x_2 = y_2 + z_2$, $y_1, y_2 \in Y$,
$z_1,z_2 \in Z$, then, $\forall a,b \in \K$,
\vspace{-0.1in}
\[ax_1 + bx_2 = ay_1 + by_2 + az_1 + bz_2.\]
Since $Y$ is a linear manifold, $ay_1 + by_2 \in Y$ so that
\vspace{-0.1in}
\[T(ax_1 + bx_2) = ay_1 + by_2 = aT(x_1) + bT(x_2).\]
Thus, $T$ is linear. Since $\forall x \in X, T(x) \in Y$, $T^2 = T$. Thus, by
the result of Problem 5, to show that $T$ is continuous, it suffices to show
that $\Nul(T)$ and $\Ran(T)$ are closed.

Since $x = 0 + x$, it follows from the uniqueness assumption that $T(x) = 0$ if
and only if $x \in Z$, and so $\Nul(T) = Z$, which is closed. Also, clearly,
$\Ran(T) = Y$, which is closed.

Since $T$ is linear and continuous, $T \in \L(X,X)$. The proof that
$L \in \L(X,X)$ is identical. \qed
\end{question}

\begin{question}{Problem 4}
\begin{enumerate}[(a)]
\item By Theorem 7.15, $\|x_n\|$ is bounded by some $B \in \R$. Since
$\|x_n^* - x^*\|_*, \|x^*(x_n) - x^*(x)\|\rightarrow 0$,
\[\|x_n^*(x_n) - x^*(x)\|
    \leq \|x_n^*(x_n) - x^*(x_n)\| + \|x^*(x_n) - x^*(x)\|
    \leq \|x_n^* - x^*\|_*B + \|x^*(x_n) - x^*(x)\|
    \rightarrow 0
\]
as $n \rightarrow 0$. \qed

\item By Theorem 7.24, $x_n^*$ is bounded by some $B \in \R$. Since
$\|x_n^*(x) - x_n^*(x)\|, \|x_n - x\| \rightarrow 0$,
\[\|x_n^*(x_n) - x^*(x)\|
    \leq \|x_n^*(x_n) - x_n^*(x)\| + \|x_n^*(x) - x^*(x)\|
    \leq B\|x_n - x\| + \|x_n^*(x) - x^*(x)\|
    \rightarrow 0
\]
as $n \rightarrow 0$. \qed
\end{enumerate}
\end{question}

\begin{question}{Problem 5}
We first note that, since $T^2 = T$, $\forall y \in \Ran(T), y = T(y)$.

($\Rightarrow$) Suppose $T$ is continuous. Since the singleton $\{0\}$ is
closed, $\Nul(T) = T\inv[\{0\}]$ is closed.

Suppose $\exists y_n \in \Ran(T)$ with
$\lim_{n \rightarrow \infty} y_n = y \in X$. Then,
\[y
    = \lim_{n \rightarrow \infty} y_n
    = \lim_{n \rightarrow \infty} T(y_n)
    = T\left(\lim_{n \rightarrow \infty} y_n\right)
    = T(y) \in \Ran(T),
\]
since $T$ is continuous. Thus, $\Ran(T)$ is closed. \qed

($\Leftarrow$) Suppose that $\Nul(T),\Ran(T)$ are closed. $\forall n \in \N$,
let $(x_n,y_n) \in \Gr(T)$ with $(x_n,y_n) \rightarrow (x,y)$ as
$n \rightarrow \infty$. Since $\Ran(T)$ is closed,
$y = \lim_{n \rightarrow \infty} y_n \in \Ran(T)$, so that $y = T(y)$. Since
each $y_n = T(y_n)$, $x_n - y_n \in \Nul(T)$, so that, since $\Nul(T)$ is
closed, $x - y = \lim_{n \rightarrow \infty} (x_n - y_n) \in \Nul(T)$, and so
$T(x) = T(y) = y$. Thus, $(x,y) \in \Gr(T)$, so $\Gr(T)$ is closed. Then, by
the Closed Graph Theorem, $T$ is continuous. \qed
\end{question}

\begin{question}{Problem 6}
\begin{enumerate}[(a)]
\item Since $|||x||| \leq K\|x\|$, $\forall x \in X$,
$\L((X,|||\cdot|||),\K) \subseteq \L((X,\|\cdot\|),\K)$, and, similarly,
$\L((X^*,|||\cdot|||_*),\K) \subseteq \L((X^*,\|\cdot\|_*),\K)$, where $X^*$ is
defined in terms of the associated norm. Thus, since the canonical embedding
$J$ of $X$ into $X^{**}$ under $\|\cdot\|$, it is also a surjection under
$|||\cdot|||$. Hence, $(X,|||\cdot|||)$ is also reflexive, and thus complete.
It follows from Corollary 3.24 that $\|\cdot\|$ and $|||\cdot|||$ are in fact
equivalent norms on $X$.

Thus, since $Z$ is closed under the topology induced by $\|\cdot\|$, it is also
closed under the topology $|||\cdot|||$. Then, since any closed subspace of a
complete metric space is itself complete, $(Z,\rho)$ is complete. \qed

\item I wasn't able to come up with a counterexample for this one. Assuming a
counterexample lies in one of the spaces we've discussed, I did make the
following observations that should narrow the space of possible counterexamples
to very few possibilites:
\begin{enumerate}
\item The only non-reflexive Banach spaces we've discussed are
$l^1,c_0,c,l^{\infty}$.
\item Of these spaces, $c_0,c,l^{\infty}$ have only one well-defined norm
($\|\cdot\|_{\infty}$, up to scaling), so the counterexample should be in
$l^1$.
\item In order for $(l^1,\|\cdot\|)$ to be a Banach space,
$\|\cdot\| = \|\cdot\|_1$.
\item The only norm that $\|\cdot\|_1$ bounds by a constant multiple is
$\|\cdot\|_{\infty}$, so $|||\cdot||| = \|\cdot\|_{\infty}$.
\end{enumerate}
I wasn't able to find a counterexample beyond this though\dots
\end{enumerate}
\end{question}

\begin{question}{Problem 7}
Let $X := (C[0,1],\|\cdot\|_{\infty})$ (so that $X$ is a Banach space), and
define, $\forall n \in \N$,
\[
K_n := K \cap B_n,
\quad \mbox{ where } \quad
K := \left\{ f \in X :
        f(0) = 0 \mbox{ and }
        \int_0^1 f(x) \, dx \geq 1 \right\}
\]
and $B_n = \{f \in X : \|f\|_{\infty} \leq 1 + 1/n\}$. $B_n$ is clearly
bounded, convex and closed.

I showed in my solution to Problem 9 on Assignment 4 that $K$ is
convex and closed, so that, as the intersection of two convex and closed sets,
each $K_n$ is also convex and closed. I also demonstrated a family
$\{f_n\} \in K$ with each $f_n \in B_n$, so that $f_n \in K_n$. Thus,
$\{K_n\}_{n = 1}^{\infty}$ satisfies condition (i).

I also showed that, $\forall f \in K$, $\|f\|_{\infty} > 1$, so that
\[\bigcap_{n = 1}^{\infty} K_n
    = K \cap \bigcap_{n = 1}^{\infty} B_n
    = K \cap \{f \in X : \|f\|_{\infty} \leq 1\}
    = \emptyset,
\]
and so $\{K_n\}_{n = 1}^{\infty}$ satisfies condition (iii).

Finally, $\forall n \in \N$, since
$K_{n + 1} = K \cap B_{n + 1} \subseteq K \cap B_n = K_n$,
$\{K_n\}_{n = 1}^{\infty}$ satisfies condition (ii). \qed

In principle, $K$ could be any set satisfying the properties in Problem 9 on
Assignment 4.
\vspace{-0.1in}
\end{question}

\begin{question}{Problem 8}
We construct a closed subspace $Z \subseteq X$ such that the restriction of $T$
to $Z$ bijection into $Y$, allowing us to use the Bounded Inverse Theorem to
obtain the desired result.

By Proposition 1.21, we can construct a Hamel basis $(x_i | i \in I)$ for $X$
such that, for some $J \subseteq I$, $(x_i | i \in J)$ is a Hamel basis for
$\Nul(T)$. $\forall i \in I$, let $\alpha_i$ denote the projection onto $x_i$.

Define
\[Z
    := \{x \in X \, | \, \forall j \in J, \alpha_j(x) = 0\}
    = \bigcap_{j \in J} \Nul(\alpha_j)
.\]
By definition of the product topology, each $\alpha_i$ is continuous. Then,
since projections are continuous, each $\Nul(\alpha_i)$ is closed, so that $Z$
is a closed linear manifold in $X$, and hence $Z$ is a Banach space.

Let $T_Z : Z \rightarrow Y$ denote the restriction of $T$ to $Z$. It follows
from the construction of $J$ and $Z$ that $Z \cap \Nul(T) = \{0\}$, so that
$T_Z$ is injective.

Let $x \in X$, and choose finite sets $J_x \subseteq J,
I_x \subseteq I \sminus J$ by
\[x = \sum_{j \in J_x} \alpha_j(x) x_j + \sum_{i \in I_x} \alpha_i(x) x_i
\quad \mbox{ and } \quad
\alpha_i(x) \neq 0, \forall i \in J_x \cup I_x.\]
Then,
\[T(x)
    = T\left( \sum_{j \in J_x} \alpha_j(x) x_j
    + \sum_{i \in I_x} \alpha_i(x) x_i \right)
    = \sum_{j \in J_x} \alpha_j(x) T(x_j)
    + T\left( \sum_{i \in I_x} \alpha_i(x) x_i \right)
    = T(x'),
\]
for $x' := \sum_{i \in I_x} \alpha_i(x) x_i$. Furthermore, $x' \in Z$, and it
follows that $T_Z$ is surjective.

By the Bounded Inverse Theorem, $T_Z$ has a bounded linear inverse $T_Z\inv$.
Thus, given a convergent sequence $\{y_n\}_{n = 1}^{\infty}$ in $Y$ for
$x_n := T_Z\inv(y_n)$, by continuity of $T_Z$,
$\{x_n\}_{n = 1}^{\infty}$ is convergent, and, furthermore,
\[\forall n \in \N, \mbox{ we have } y_n = T(x_n) \mbox{ and }
\|x_n\| \leq \left\| T_Z\inv \right\|_*\|y_n\|. \mqed\]
\end{question}

\begin{question}{Problem 9}
($\Rightarrow$) If $p^K$ is continuous, then, since $p^K(0) = 0$, for some
$\delta > 0$, $p^K < 1$ on $B_{\delta}(0)$. If $x \in B_{\delta}(0)$, then
by definition of $p^K$, $\exists s \in (0,1)$ with $s\inv x \in K$. Then, since
$K$ is convex and $0 \in K$, $x = s(s\inv x) + (1 - s)(0) \in K$, and so
$B_{\delta}(0) \subseteq K$. \qed

($\Leftarrow$) Suppose $B_{\delta}(0) \subseteq K$, for some $\delta > 0$. Note
that, $\forall x \in X$, since $\frac{\delta x}{2\|x\|} \in B_{\delta}(0)$, by
definition of $p^K$, $p^K(x) \leq 2\delta\inv\|x\|$. By part (c) of Lemma 5.32,
$\forall x, h \in X$,
\begin{align*}
p^K(x + h)
 &  \leq p^K(x) + p^K(h)
    \leq p^K(x) + 2\delta\inv \|h\| \\
\mbox{ and } 
p^K(x)
    =    p^K(x + h - h)
 &  \leq p^K(x + h) + p^K(-h)
    \leq p^K(x + h) + 2\delta\inv \|h\|.
\end{align*}
Thus, $|p^K(x + h) - p^K(x)| \leq 2\delta\inv \|h\| \rightarrow 0$ as
$h \rightarrow 0$, and so $p^K$ is (Lipschitz) continuous at $x$. \qed
\end{question}

\begin{question}{Problem 10}
We disprove the given statement.

Let $x_n^* = x^{(n)}$ and $x^* = x$, as defined in part (b) of Example 7.23 of
the notes. As shown in the example, $x_n^* \stackrel{*}{\rightharpoonup} x^*$
(weakly*), but $x_n^*$ does not converge weakly to $x^*$ as
$n \rightarrow \infty$, so that there exists $x^{**} \in X^{**}$ such that
$x^{**}(x_n^*)$ does not converge to $x^{**}(x^*)$ as $n \rightarrow \infty$.
\qed
\end{question}
\end{document}
