\documentclass[11pt]{article}
\usepackage{enumerate}
\usepackage{fullpage}
\usepackage{amsmath, amsfonts, amsthm, amssymb}
\usepackage{color}
\pagestyle{empty}

\def\indented#1{\list{}{}\item[]}
\let\indented=\endlist

%%%%%%%%%%%%%%%%%%%%%%%HEADER%%%%%%%%%%%%%%%%%%%%%%%%%%%%%%
\newcommand{\myname}{Shashank Singh}
\newcommand{\myandrew}{sss1@andrew.cmu.edu}
\newcommand{\mydate}{Thursday, August 22, 2013}
%%%%%%%%%%%%%%%%%%%%%%%%%%%%%%%%%%%%%%%%%%%%%%%%%%%%%%%%%%%

%%%%%%%%%%%%%%%%%%%%CONTENT MACROS%%%%%%%%%%%%%%%%%%%%%%%%%
\renewcommand{\qed}{\quad $\blacksquare$}           % QED square (text mode)
\newcommand{\mqed}{\quad \blacksquare}              % QED square (math mode)
\newcommand{\inv}{^{-1}}                            % inverse operator
\newcommand{\sminus}{\backslash}                    % set minus
\newcommand{\N}{\mathbb{N}}                         % natural numbers
\newcommand{\Z}{\mathbb{Z}}                         % integers
\newcommand{\Q}{\mathbb{Q}}                         % rational numbers
\newcommand{\R}{\mathbb{R}}                         % real numbers
\newcommand{\K}{\mathbb{K}}                         % underlying field
\renewcommand{\P}{\mathcal{P}}                      % power set
\newcommand{\e}{\varepsilon}                        % \varepsilon
\newcommand{\B}{\mathcal{B}}                        % local base
%%%%%%%%%%%%%%%%%%%%%%%%%%%%%%%%%%%%%%%%%%%%%%%%%%%%%%%%%%%

\begin{document}
\thispagestyle{plain}

\begin{center}
{\Large Some Comments on the Lecture Notes for 21-640} \\
\myname \\
\myandrew \\
\mydate
\end{center}

{\bf Week 3}

Page 10, Second to last paragraph   \\
    - should read `Open Mapping Theorem', rather than `Banach-Steinhaus'.   \\

{\bf Week 12}

Page 7, Proof of Lemma 12.24       \\
    - $F$ should map $(x,y)$ rather than $x + y$ to $x + y$.                \\

{\bf Week 13}

Page 5, Proof of Theorem 13.10     \\
    - This may have been intentional, but the proof appears to stop abruptly
      before actually showing
        \[\forall x \in X, \|x\| = 0 \Rightarrow x = 0.\]

{\bf Week 14}

Page 1, Theorem 14.1               \\
    - minor typo in first sentence - topological has an extra 'v'.  \\
    - A (probably minor) exception - What if $\B = \{\{0\}\}$
    (by translation invariance, the topology would have to be
    discrete)? I think this should be possible on a TVS. However,
    $\B$ would be a local base of balanced and convex, but not
    absorbing, sets.    \\

Page 1, Proof of Theorem 14.1               \\
    - minor typo in second sentence - a ':' is missing the family
    $\{p^V : V \in \B\}$.        \\

Page 1, Theorem 14.2               \\
    - typo in the last sentence:
        \[\forall p \in \P, p \mbox{ is toplogically on } E\]
should be
        \[\forall p \in \P, p \mbox{ is bounded on } E.\]

Page 4, Paragraph 2 \\
    - minor typo in last sentence - 'equicontinuous' is missing the last 'u'\\

Page 4, Definition 14.11 \\
    - minor typo in part (c)- there is an extraneous `it'  \\

Page 4, Theorem 14.14 \\
    - minor typo in first sentence - `linear amppings' should read `linear mappings'  \\

\newpage
Page 5, Paragraph 2 \\
    - minor typo in first sentence - `separating' is missing the second `a' \\
    - (I think) the fact that $\{p_x : x^* \in X^*\}$ is separating depends on
        a Hahn-Banach argument; if so, it might be clearer to mention this. \\

Page 5, Remark 14.18 \\
    - typo in second sentence - $X$ should be $X^*$.    \\

Pages 9-10, Theorems 14.27-14.29 and Proposition 14.30 are mislabelled Theorems
4.27-4.29 and Proposition 4.30, respectively.   \\

Page 10, Definition 14.31\\
    - typo in part (c) - $\forall x,y \in \K$ should read $\forall x,y \in X$.
    \\

Page 12, Paragraph 1\\
    - minor typo in second sentence - missing close parenthesis. \\
    - minor typo in third sentence - missing period.    \\
\end{document}
