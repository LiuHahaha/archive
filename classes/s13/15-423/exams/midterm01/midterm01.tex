\documentclass[11pt]{article}
\usepackage{enumerate}
\usepackage{fullpage}
\usepackage{fancyhdr}
\usepackage{amsmath, amsfonts, amsthm, amssymb}
\usepackage{color}
\usepackage[]{graphicx}
\setlength{\parindent}{0pt}
\setlength{\parskip}{5pt plus 1pt}
\pagestyle{empty}

\def\indented#1{\list{}{}\item[]}
\let\indented=\endlist

\newcounter{questionCounter}
\newcounter{partCounter}[questionCounter]
\newenvironment{question}[2][\arabic{questionCounter}]{%
    \setcounter{partCounter}{0}%
    \vspace{.25in} \hrule \vspace{0.5em}%
        \noindent{\bf #2}%
    \vspace{0.8em} \hrule \vspace{.10in}%
    \addtocounter{questionCounter}{1}%
}{}
\renewenvironment{part}[1][\alph{partCounter}]{%
    \addtocounter{partCounter}{1}%
    \vspace{.10in}%
    \begin{indented}%
       {\bf (#1)} %
}{\end{indented}}

%%%%%%%%%%%%%%%%%%%%%%%HEADER%%%%%%%%%%%%%%%%%%%%%%%%%%%%%%
\newcommand{\myname}{Shashank Singh}
\newcommand{\myandrew}{sss1@andrew.cmu.edu}
\newcommand{\myclass}{15-423 Digital Signal Processing for CS}
\newcommand{\myhwnum}{1}
\newcommand{\duedate}{After Spring Break}
%%%%%%%%%%%%%%%%%%%%%%%%%%%%%%%%%%%%%%%%%%%%%%%%%%%%%%%%%%%

%%%%%%%%%%%%%%%%%%%%CONTENT MACROS%%%%%%%%%%%%%%%%%%%%%%%%%
\renewcommand{\qed}{\quad $\blacksquare$}
\newcommand{\mqed}{\quad \blacksquare}
\newcommand{\inv}{^{-1}}
\newcommand{\N}{\mathbb{N}} % natural numbers
\newcommand{\Z}{\mathbb{Z}} % natural numbers
\newcommand{\Q}{\mathbb{Q}} % rational numbers
\newcommand{\R}{\mathbb{R}} % real numbers
\newcommand{\sminus}{\backslash} % asymmetric set difference
\newcommand{\I}{\mathcal{I}}
\newcommand{\E}{\mathcal{E}v} % even part of a signal
\renewcommand{\O}{\mathcal{O}d} % odd part of a signal
\newcommand{\F}{\mathcal{F}} % Fourier transform
\newcommand{\lcm}{\operatorname{lcm}} % least common mutiple
%%%%%%%%%%%%%%%%%%%%%%%%%%%%%%%%%%%%%%%%%%%%%%%%%%%%%%%%%%%

\begin{document}
\thispagestyle{plain}

{\Large Midterm \myhwnum} \\
\myclass \\
Name: \myname \\
Email: \myandrew \\
Due: \duedate

\section{Signals}
\begin{enumerate}[a.]
\item In all senses of boundedness, $a^nu[n]$ is bounded if and only if
$|a| \in [0,1)$, in which case $1$ bounds the signal, $\frac{1}{1 - a}$ bounds
the signal's integral, and $\frac{1}{1 - a^2}$ bounds the integral of the
signal's energy.

\item Since, $\forall t \in \R$, $u(t) + u(-t) = 1$,
\[\E(x(t))
 = \frac12(\cos(2\pi t)u(t) + \cos(2\pi t)u(-t))
 = \frac12 \cos(2\pi t),\]
so $\E(x(t))$ is periodic with period $1$.
\[\O(x(t))
 = \frac12(\cos(2\pi t)u(t) - \cos(2\pi t)u(-t))
 = \cos(2\pi t)\left( u(t) - \frac12\right),\]
so $\O(x(t))$ is not periodic (although it is periodic with period $1$ if
restricted to $t < 0$ or $t \geq 0$).
\end{enumerate}

%TODO
\section{Systems}
For boundedness, I wasn't sure which definition to use, so I answered in terms
of $1$-, $2$-, and $\infty$-norms, corresponding to the integral, square
integral (energy), and suprememum, respectively, of the input and output.
\begin{enumerate}[a.]
\item The system is memoryless and causal ($y(t)$ depends only on $x(t)$),
non-linear (if $x_1(0) = x_2(0) = 0$, then
$e^{x_1(0) + x_2(0)} = 1 < 2 = e^{x_1(0)} + e^{x_2(0)}$), time-invariant,
unstable in $1$- and $2$-norms, and stable in the $\infty$-norm.

\item The system is causal but not memoryless ($y[n]$ depends on $x[n]$ and
$x[n - 1]$), non-linear (if $x_1[0] = x_2[1] = 0, x_1[1] = x_2[0] = 1$, then
$(x_1[1] + x_2[1])(x_1[0] + x_2[0]) = 1 > 0 = x_1[1]x_1[0] + x_2[1]x_2[0]$),
time-invariant, and stable in all three norms.

\item The system is neither causal nor memoryless ($y(-2)$ depends on $x(-1)$
and $y(2)$ depends on $x(1)$), but is linear (each coordinate is a linear
combination of coordinates of the input), not time-invariant, and stable with
repsect to all three norms.

\item The system is neither causal nor memoryless ($y[1]$ depends on $x[2]$ and
$y[-1]$ depends on $x[-2]$), but is linear, not time-invariant, and stable with
respect to all three norms.

\item The system is memoryless but not causal, linear, not time-invariant, and
stable with respect to all three norms.
\end{enumerate}

\newpage
\section{Discrete Time Fourier Analysis}

{\bf Problem III}
\begin{enumerate}[a.]
\item The period of the cosine term is $10$. The period of the sine term is
$20$. Since the summed signal repeats precisely when both
the sine term and the cosine term repeat, the period is $\lcm(10,20) = 20$.

\item Since $\cos(0.4 n)$ is not periodic, it has no Fourier series.
\end{enumerate}

{\bf Problem V}
\begin{enumerate}[a.]
\item $b$ must divide $a$.

\item Define, $\forall n \in \N$,
\[y[n] = \left\{
\begin{array}{cl}
    x\left[ \frac{an}{b} \right]    : \mbox{ if } n = k\frac{b}{a}, k \in \N \\
    0                                 \mbox{ else }.
\end{array}
\right.
\]

\item Let $I$ denote the unit pulse train of minimum period $N \in \N$ such
that $N\frac{a}{b} \in \N$. Then, since $y = x \cdot I$, by the Convolution
Theorem, $Y = X * \mathcal{I}$, where $\mathcal{I}$ denotes the Fourier
transform of $I$ (which is itself an impulse train).
\end{enumerate}


{\bf Problem VI}
\begin{enumerate}[a.]
\item 
\[\F\{x[-n]\}
    = \sum_{n = -\infty}^\infty x[-n] e^{i\Omega n}
    = \sum_{n = -\infty}^\infty x[n] e^{i(-\Omega) n}
    = X(-\Omega). \mqed
\]
\item
\[ -x(t)
    = x^*(t)
    = \left( \F\inv \left\{ X(\Omega) \right\} \right)^*)
    = \int_{-\infty}^\infty X^*(\Omega) e^{-j\Omega t}
    = \int_{-\infty}^\infty X^*(-\Omega) e^{j\Omega t}
    = \F\inv \left\{ X^*(-\Omega) \right\}.
\]
Thus, by linearity of the Fourier Transform, since $x$ 
\[\F\left\{ x^*(t) \right\} = X^*(-\Omega) = -X(-\Omega). \mqed\]
\item By part a., if $x$ is symmetric, then
\[X(-\Omega)
    = \F\{x[-n]\}
    = \F\{x[n]\}
    = X(\Omega),
\]
and so $X$ is also symmetric. Also, as shown in part b. (without needing
$x$ imaginary)
\[X(-\Omega)
    = X(\Omega)
    = \F\left\{ x[n] \right\}
    = \F\left\{ x^*[n] \right\}
    = X^*(-\Omega),
\]
and so $X$ is real. \qed
\end{enumerate}
\end{document}
