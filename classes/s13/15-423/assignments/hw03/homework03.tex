\documentclass[11pt]{article}
\usepackage{enumerate}
\usepackage{fullpage}
\usepackage{fancyhdr}
\usepackage{amsmath, amsfonts, amsthm, amssymb}
\usepackage{color}
\usepackage[]{graphicx}
\setlength{\parindent}{0pt}
\setlength{\parskip}{5pt plus 1pt}
\pagestyle{empty}

\def\indented#1{\list{}{}\item[]}
\let\indented=\endlist

\newcounter{questionCounter}
\newcounter{partCounter}[questionCounter]
\newenvironment{question}[2][\arabic{questionCounter}]{%
    \setcounter{partCounter}{0}%
    \vspace{.25in} \hrule \vspace{0.5em}%
        \noindent{\bf #2}%
    \vspace{0.8em} \hrule \vspace{.10in}%
    \addtocounter{questionCounter}{1}%
}{}
\renewenvironment{part}[1][\alph{partCounter}]{%
    \addtocounter{partCounter}{1}%
    \vspace{.10in}%
    \begin{indented}%
       {\bf (#1)} %
}{\end{indented}}

%%%%%%%%%%%%%%%%%%%%%%%HEADER%%%%%%%%%%%%%%%%%%%%%%%%%%%%%%
\newcommand{\myname}{Shashank Singh}
\newcommand{\myandrew}{sss1@andrew.cmu.edu}
\newcommand{\myclass}{15-423 Digital Signal Processing for CS}
\newcommand{\myhwnum}{3}
\newcommand{\duedate}{Friday, March 8, 2013}
%%%%%%%%%%%%%%%%%%%%%%%%%%%%%%%%%%%%%%%%%%%%%%%%%%%%%%%%%%%

%%%%%%%%%%%%%%%%%%%%CONTENT MACROS%%%%%%%%%%%%%%%%%%%%%%%%%
\renewcommand{\qed}{\quad $\blacksquare$}
\newcommand{\mqed}{\quad \blacksquare}
\newcommand{\inv}{^{-1}}
\newcommand{\argmax}{\operatornamewithlimits{argmax}}
\newcommand{\argmin}{\operatornamewithlimits{argmin}}
\newcommand{\N}{\mathbb{N}} % natural numbers
\newcommand{\Z}{\mathbb{Z}} % natural numbers
\newcommand{\Q}{\mathbb{Q}} % rational numbers
\newcommand{\R}{\mathbb{R}} % real numbers
\newcommand{\sminus}{\backslash} % asymmetric set difference
\newcommand{\e}{\varepsilon} % \varepsilon
\newcommand{\I}{\mathcal{I}}
\newcommand{\sinc}{\operatorname{sinc}}
%%%%%%%%%%%%%%%%%%%%%%%%%%%%%%%%%%%%%%%%%%%%%%%%%%%%%%%%%%%

\begin{document}
\thispagestyle{plain}

{\Large Homework \myhwnum} \\
\myclass \\
Name: \myname \\
Email: \myandrew \\
Due: \duedate

\begin{enumerate}[I.]
\item Square Wave \\
Since the square wave is anti-symmetric, \fbox{$a_n = 0$, $\forall n \in \N$.}
$\forall$ positive $n \in \N$,
\begin{align*}
b_n
   = \frac2T \int_0^T x(t) \sin\left( \frac{2n\pi t}{T} \right) \; dt
 & = \frac4T \int_0^{T/2} \sin\left( \frac{2n\pi t}{T} \right) \; dt    \\
 & = -\frac{2}{\pi n} \cos\left( \frac{2n\pi t}{T} \right)
        \bigg|_{t = 0}^{t = T/2}                                        \\
 & = \frac{2}{\pi n} \left( 1 - \cos\left( n\pi \right) \right)
   = \mbox{\fbox{$\displaystyle
        \left\{
            \begin{array}{cl}
                \frac{2}{\pi n} & : \mbox{ if $n$ is odd} \\
                0               & : \mbox{ else}
            \end{array}
        \right.$.}}
\end{align*}

\item Triangle Wave \\
Since the triangle wave is anti-symmetric,
\fbox{$a_n = 0$, $\forall n \in \N$.} $\forall$ positive $n \in \N$, since the
triangle wave is the integral of the square wave, dividing the series from
problem I. by $\frac{2\pi n}{T}$ gives
\begin{align*}
\mbox{\fbox{$\displaystyle b_n = \frac{T}{\pi^2 n^2}$.}}
\end{align*}

%TODO
\item Amplitude-Modulated Sine Wave \\
Since the modulated sine wave is symmetric,
\fbox{$b_n = 0$, $\forall$ positive $n \in \N$.}
By a trigonometric identity, $\forall t \in \R$,
\[sin(\omega t)\sin(\omega_1 t)
 = \frac12(\cos((\omega - \omega_1) t) + \cos((\omega - \omega_1) t)).
\]
Thus, \fbox{$a_{\omega - \omega_n} = a_{\omega + \omega_n} = \frac12$,} and,
for all other \fbox{$n \in \N$, $a_n = 0$.}

\item Gaussian \\
By definition of the Fourier Transform, $\forall \Omega \in \R$,
\begin{align*}
F(\Omega)
 & :=   \frac{1}{\sqrt{2\pi}} \int_{\R} \exp\left( -t^2 \right)
            \cdot \exp\left( -2\pi it\Omega \right) \, dt               \\
 & =    \frac{\exp \left( (\pi i \Omega)^2 \right)}{\sqrt{2\pi}}
            \int_{\R} \exp\left( -t^2 \right)
                \cdot \exp\left( -2\pi it\Omega \right)
                    \cdot \exp\left( -(\pi i \Omega)^2 \right) \, dt    \\
 & =    \frac{\exp \left( -\pi^2 \Omega^2 \right)}{\sqrt{2\pi}}
            \int_{\R} \exp\left( -(t + \pi i \Omega)^2 \right) \, dt
   =    \mbox{\fbox{$\displaystyle
            \frac{\exp \left( -\pi^2 \Omega^2 \right)}{\sqrt{2}}
        $,}}
\end{align*}
since
$\displaystyle
   \int_{\R} \exp\left( -(t + \pi i \Omega)^2 \right) \, dt
 = \int_{\R} \exp\left( -t^2 \right) \, dt
 = \sqrt{\pi}
$.

\item Triangle \\
Since the triangle is symmetric, its Fourier transform is given by
\begin{align*}
X(\Omega)
 & = 2\int_0^{\infty} x(t) \cos(2\pi\Omega t) \; dt \\
 & = 2\int_0^{T/2} x(t) \cos(2\pi\Omega t) \; dt    \\
 & = \int_0^{T/2} T \cos(2\pi\Omega t) \; dt
        - 2\int_0^{T/2} t \cos(2 \pi\Omega t) \; dt    \\
 & = \frac{t}{2\pi\Omega} \sin(2\pi\Omega t) \bigg|_{t = 0}^{t = T/2}
        - 2\frac{2\pi\Omega t \sin(2\pi\Omega t) + \cos(2\pi\Omega t)}
            {4\pi^2\Omega^2} \bigg|_{t = 0}^{t = T/2} \\
 & = -\frac{\cos(2\pi\Omega t)} {2\pi^2\Omega^2} \bigg|_{t = 0}^{t = T/2}
   = \frac{1 - \cos(\pi\Omega T)} {2\pi^2\Omega^2}
   = \mbox{\fbox{$(T\sinc(\pi\Omega))^2$,}}
\end{align*}
where the last equality follows from a trigonometric inequality.

\item Triangle Wave Again \\
Since the triangle wave $y(t)$ is the convolution over $t$ of $x(t)$ (as
defined in part V.) and the impulse train
$\sum_{i \in \Z} \delta\left( t - iT \right)$, whose Fourier transform is
$\frac1T\sum_{i \in \Z} \delta\left( \Omega - \frac{i}{T} \right)$, by the
Convolution Theorem,
\begin{align*}
Y(\Omega)
   = X(\Omega)
        \cdot \frac1T\sum_{i \in \Z} \delta\left( \Omega - \frac{i}{T} \right)
 & = T\sinc^2(\pi\Omega)
        \cdot \sum_{i \in \Z} \delta\left( \Omega - \frac{i}{T} \right) \\
 & = \mbox{\fbox{$\displaystyle
      \left\{
          \begin{array}{cl}
              T\sinc^2(\pi\Omega)   & : \mbox{ if $\Omega = iT, i \in \Z$} \\
              0                     & : \mbox{ else}
          \end{array}
      \right.$.}}
\end{align*}
\end{enumerate}
\end{document}
