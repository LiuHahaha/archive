\documentclass[11pt]{article}
\usepackage{enumerate}
\usepackage{fullpage}
\usepackage{fancyhdr}
\usepackage{amsmath, amsfonts, amsthm, amssymb}
\usepackage{color}
\setlength{\parindent}{0pt}
\setlength{\parskip}{5pt plus 1pt}
\pagestyle{empty}

\def\indented#1{\list{}{}\item[]}
\let\indented=\endlist

\newcounter{questionCounter}
\newcounter{partCounter}[questionCounter]
\newenvironment{question}[2][\arabic{questionCounter}]{%
    \setcounter{partCounter}{0}%
    \vspace{.25in} \hrule \vspace{0.5em}%
        \noindent{\bf #2}%
    \vspace{0.8em} \hrule \vspace{.10in}%
    \addtocounter{questionCounter}{1}%
}{}
\renewenvironment{part}[1][\alph{partCounter}]{%
    \addtocounter{partCounter}{1}%
    \vspace{.10in}%
    \begin{indented}%
       {\bf (#1)} %
}{\end{indented}}

%%%%%%%%%%%%%%%%%%%%%%%HEADER%%%%%%%%%%%%%%%%%%%%%%%%%%%%%%
\newcommand{\myname}{Shashank Singh}
\newcommand{\myandrew}{sss1@andrew.cmu.edu}
\newcommand{\myclass}{15-423 Digital Signal Processing for CS}
\newcommand{\myhwnum}{1}
\newcommand{\duedate}{Monday, February 4, 2013}
%%%%%%%%%%%%%%%%%%%%%%%%%%%%%%%%%%%%%%%%%%%%%%%%%%%%%%%%%%%

%%%%%%%%%%%%%%%%%%%%CONTENT MACROS%%%%%%%%%%%%%%%%%%%%%%%%%
\renewcommand{\qed}{\quad $\blacksquare$}
\newcommand{\mqed}{\quad \blacksquare}
\newcommand{\inv}{^{-1}}
\newcommand{\argmax}{\operatornamewithlimits{argmax}}
\newcommand{\argmin}{\operatornamewithlimits{argmin}}
\newcommand{\N}{\mathbb{N}} % natural numbers
\newcommand{\Z}{\mathbb{Z}} % natural numbers
\newcommand{\Q}{\mathbb{Q}} % rational numbers
\newcommand{\R}{\mathbb{R}} % real numbers
\newcommand{\sminus}{\backslash} % asymmetric set difference
\newcommand{\e}{\varepsilon} % \varepsilon
\newcommand{\I}{\mathcal{I}}

% probability related macros
\newcommand{\pr}[1]{\mathsf{P}\left( #1 \right)} % probability of event #1
\newcommand{\E}[2]{\operatornamewithlimits{\mathbb{E}}_{#2}\left[ #1 \right]} % expected value of #1 over #2
\newcommand{\Bern}[1]{\operatorname{Bernoulli}\left( #1 \right)} % Bernoulli distribution of parameter p
\newcommand{\giv}{\, | \,} % \pr{A \giv B} probability of A given B
%%%%%%%%%%%%%%%%%%%%%%%%%%%%%%%%%%%%%%%%%%%%%%%%%%%%%%%%%%%

\begin{document}
\thispagestyle{plain}

{\Large Homework \myhwnum} \\
\myclass \\
Name: \myname \\
Email: \myandrew \\
Due: \duedate

\section{Composing signals}
\begin{enumerate}
\item The following signal is a pulse train of period $N_1 + N_2$ and duty
cycle $\displaystyle \frac{N_1}{N_1 + N_2}$:
\[s_1[n]
 := \sum_{k \in \Z} p[n + k(N_1 + N_2)],
 \mbox{ where $p[n] := u[n] - u[n - N_1]$ is a pulse of length $N_1$.}
\]
 
\item The following signal is an exponential signal beginning at $n = n_0$ and
tapering as $\alpha^{n - n_0}$:
\[s_2[n]
 := u[n - n_0]e_{\alpha}[n - n_0].
\]
 
\item The following signal is a sinusoid of zero phase and amplitude decaying
as $\alpha^n$:
\[s_3[n]
 := \sin[n]e_{\alpha}[-n].
\]
 
\item The following signal is a complex exponential of frequency $\omega$ and
zero phase, ending at $n = n_0$:
\[s_4[n]
 := e_{\omega,0}[n]u[-n + n_0].
\]
 
\item The following signal is an impulse train of phase $-1$ and period $4$:
\[s_5[n]
 := \sum_{k \in \Z} \delta[n - (4k + 1)].
\]
$s_5$ can also be written in terms of $s_1$ (see problem $1$), with $N_1 = 1$,
$N_2 = 3$:
\[s_5[n]
 = s_1[n - 1].
 \]
\end{enumerate}

\newpage
\section{System properties}
\begin{enumerate}
\item The system is causal ($y[n]$ depends only on $x[n - 4]$), is not
memoryless ($y[n]$ must remember $x[n - 4]$), is linear ($y$ is a linear
combination of terms of $x$), and is shift-invariant ($n$ appears only as an
argument to $x$).

The system is invertible if and only if $\alpha$ is non-zero (the mapping
$x[n] \mapsto \frac{y[n + 4]}{\alpha}$ is an inverse).
 
\item
The system is not causal ($y[n]$ anticipates $x[n + 1]$), is memoryless ($y[n]$
depends only on $x[n]$ and $x[n + 1]$), is linear ($y[n]$ is a linear
combination of terms of $x$), and is shift-invariant ($n$ appears only as an
argument to $x$).
 
\item
It can be shown by induction on $n$ that, $\forall n \in \N,\;y[n] = \sum_{i =
0}^{\infty} \beta^i(\alpha_0 x[n - i] + \alpha_1 x[n - i - 1])$. Thus, the 
system is causal ($y[n]$ depends only on $x[n], x[n - 1]$, and $y[n - 1]$),
is not memoryless ($y[n]$ depends on $x[n - 1]$), is linear ($y[n]$ is a linear
combination of terms of $x$), and is shift-invariant ($n$ appears only as an
argument to $x$).
 
\item 
The system is causal ($y[n]$ depends only on previous terms), is not memoryless
(if $K > 0$, $y[n]$ depends on $x[n - 1]$), is linear ($y[n]$ is a linear
combination of terms of $x$), and is not shift-invariant (e.g., if $x$ is the
DC signal $s[n]$, and $K = 0$, then $y[n] = n$, which is clearly not
shift-invariant).

\item
The system is causal ($y[n]$ depends only on previous terms), is not memoryless
(if $K > 0$, $y[n]$ depends on $x[n - 1]$), is linear ($y[n]$ is a linear
combination of terms of $x$), and is shift-invariant ($n$ appears only as an
argument to $x$).


\item The system is causal and memoryless ($y[n]$ depends only on $x[n]$), is
non-linear (if $x_1$ and $x_2$ are the DC signal $s[n]$, then the sum of their
quartics, the constant signal of amplitude $2$, is not the quartic of their
sum, constant signal of amplitude $16$), and is shift-invariant ($n$ appears
only as an argument to $x$).

\end{enumerate}
\end{document}
