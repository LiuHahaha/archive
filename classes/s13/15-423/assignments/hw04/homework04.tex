\documentclass[11pt]{article}
\usepackage{enumerate}
\usepackage{fullpage}
\usepackage{fancyhdr}
\usepackage{amsmath, amsfonts, amsthm, amssymb}
\usepackage{color}
\usepackage[]{graphicx}
\setlength{\parindent}{0pt}
\setlength{\parskip}{5pt plus 1pt}
\pagestyle{empty}

\def\indented#1{\list{}{}\item[]}
\let\indented=\endlist

\newcounter{questionCounter}
\newcounter{partCounter}[questionCounter]
\newenvironment{question}[2][\arabic{questionCounter}]{%
    \setcounter{partCounter}{0}%
    \vspace{.25in} \hrule \vspace{0.5em}%
        \noindent{\bf #2}%
    \vspace{0.8em} \hrule \vspace{.10in}%
    \addtocounter{questionCounter}{1}%
}{}
\renewenvironment{part}[1][\alph{partCounter}]{%
    \addtocounter{partCounter}{1}%
    \vspace{.10in}%
    \begin{indented}%
       {\bf (#1)} %
}{\end{indented}}

%%%%%%%%%%%%%%%%%%%%%%%HEADER%%%%%%%%%%%%%%%%%%%%%%%%%%%%%%
\newcommand{\myname}{Shashank Singh}
\newcommand{\myandrew}{sss1@andrew.cmu.edu}
\newcommand{\myclass}{15-423 Digital Signal Processing for CS}
\newcommand{\myhwnum}{4}
\newcommand{\duedate}{Sunday, April 28, 2013}
%%%%%%%%%%%%%%%%%%%%%%%%%%%%%%%%%%%%%%%%%%%%%%%%%%%%%%%%%%%

%%%%%%%%%%%%%%%%%%%%CONTENT MACROS%%%%%%%%%%%%%%%%%%%%%%%%%
\renewcommand{\qed}{\quad $\blacksquare$}
\newcommand{\mqed}{\quad \blacksquare}
\newcommand{\inv}{^{-1}}
\newcommand{\argmax}{\operatornamewithlimits{argmax}}
\newcommand{\argmin}{\operatornamewithlimits{argmin}}
\newcommand{\N}{\mathbb{N}}             % natural numbers
\newcommand{\Q}{\mathbb{Q}}             % rational numbers
\newcommand{\R}{\mathbb{R}}             % real numbers
\newcommand{\sminus}{\backslash}        % asymmetric set difference
\newcommand{\e}{\varepsilon}            % \varepsilon
\newcommand{\I}{\mathcal{I}}
\newcommand{\sinc}{\operatorname{sinc}} % sinc function
\newcommand{\Z}{\mathcal{Z}}            % Z-transform
%%%%%%%%%%%%%%%%%%%%%%%%%%%%%%%%%%%%%%%%%%%%%%%%%%%%%%%%%%%

\begin{document}
\thispagestyle{plain}

{\Large Homework \myhwnum} \\
\myclass \\
Name: \myname \\
Email: \myandrew \\
Due: \duedate

\begin{enumerate}
\item
\begin{enumerate}
\item By Linearity and the Time-Shifting property of the Z-transform,
\[Y(z)
    = b_0X(z) + b_1z^{-1}X(z) + b_2z^{-2}X(z)
    = \mbox{\fbox{$(b_0 + b_1z^{-1} + b_2z^{-2})X(z)$.}}
\]

\item By Linearity and the Differentiation property of the Z-transform,
\[Y(z) = \mbox{\fbox{$\displaystyle -z\frac{dX(z)}{dz} - 5X(z)$.}}\]
\end{enumerate}

\item The snap of a whip should closely approximate a delta function. Thus,
assuming effect of the concert hall on the music was time-invariant (pretty
reasonable) and linear (probably still reasonable), given a recording of the
violinist, we can perform the same transformation on the sound as did the
concert hall, by writing the signal as a linear combination of delta functions
and then replacing the delta functions with the recording of the whip.

\item Since convolutions in the time domain correspond to products in the
frequency domain, we can simply convolve the signal with system corresponding
to $H_1(Z)H_2(Z)$, rather than convolving it separately with the signal
corresponding to $H_1(Z)$ and the signal corresponding to $H_2(Z)$.

\item
\begin{enumerate}
\item Decomposing into partial fractions, we note
\[\frac{z(2z - a - b)}{(z - a)(z - b)}
    = \frac{z}{z - a} + \frac{z}{z - b}.
    = \frac{1}{1 - az\inv} + \frac{1}{1 - bz\inv}.
\]
Then, by linearity of the inverse Z-transform,
\[x[n]
    = \Z\inv\left\{ \frac{z}{z - a} \right\}
    + \Z\left\{ \frac{z}{z - b} \right\}
    = \mbox{\fbox{$a^n u[n] + b^n u[n]$.}}
\]

\item Since $\Z\inv\{1\} = \delta[n]$, by linearity and the time-shifting
property of Z-transform,
\[x[n]
    = \mbox{\fbox{$\delta[n] + 2\delta[n - 1] + 5\delta[n - 2] + 7\delta[n - 3]
    + \delta[n - 5]$.}}
\]
\end{enumerate}

\item By Parseval's Theorem, the sum of squared values of the original signal
is equal to the integral of the square of the Z-transform (over a contour
dependent on the region of convergence). Thus, we could just approximate this
integral and compare it to the threshold.

\item Since convolutions in the time domain correspond to products in the
frequency domain, we can invert the effects of the systems in Problem 3 by
convolving with the signal corresponding to $\frac{1}{H_1(Z)H_2(Z)}$.

\item
\begin{enumerate}
\item The original time-domain signal must have bandwidth no greater than
$\frac12\min\{Y,X\}$.

\item The signal must have no frequencies greater than $Y$.

\item The signal should first be low-pass filtered to remove any frequencies
greater than $Y$, and then all but $1$ in $\alpha$ of the samples should be
removed from the signal.

\end{enumerate}

\end{enumerate}
\end{document}
