\documentclass{article} % For LaTeX2e
\usepackage{amsmath, amsfonts, amsthm, amssymb}
\usepackage{nips14submit_e,times}
\usepackage{hyperref}
\usepackage{url}
\usepackage{graphicx}
\usepackage{mathrsfs}
\usepackage{fullpage}

% For bibliography
\usepackage[round]{natbib}
\renewcommand{\bibsection}{}

\title{11-745 Final Report}

\author{
Shashank Singh \\
Statistics \& Machine Learning Departments \\
Carnegie Mellon University \\
Pittsburgh, PA 15213 \\
\texttt{sss1@andrew.cmu.edu}
}

\newcommand{\fix}{\marginpar{FIX}}
\newcommand{\new}{\marginpar{NEW}}

%%%%%%%%%%%%%%%%%%%%CONTENT MACROS%%%%%%%%%%%%%%%%%%%%%%%%%
\renewcommand{\qed}{\quad \ensuremath{\blacksquare}}    % QED blacksquare
\newcommand{\inv}{^{-1}}                            % inverse operator
\newcommand{\sminus}{\backslash}                    % set minus
\newcommand{\N}{\mathbb{N}}                         % natural numbers
\newcommand{\R}{\mathbb{R}}                         % real numbers
\newcommand{\pow}{\mathcal{P}}                      % power set
\newcommand{\Se}{\mathcal{S}}                       % partition
\newcommand{\e}{\varepsilon}                        % \varepsilon
\newcommand{\X}{\mathcal{X}}                        % X domain
\newcommand{\Y}{\mathcal{Y}}                        % Y domain
\newcommand{\Z}{\mathcal{Z}}                        % Z domain
\newcommand{\A}{\mathcal{A}}                        % sub-domain
\newcommand{\E}{\mathbb{E}}                         % expected value
\newcommand{\V}{\mathbb{V}}                         % variance
\newcommand{\pr}{\mathbb{P}}                        % probability
\newcommand{\vi}{{\vec{i}}}                         % multi-index vector i
\newcommand{\dist}{\operatorname{dist}}             % distance operator
\newcommand{\acro}[1]{\textsc{\MakeLowercase{#1}}}
\renewcommand{\hat}{\widehat}
\renewcommand{\tilde}{\widetilde}
\newcommand{\Rc}{\mathfrak{R}}
\renewcommand{\H}{\mathcal{H}}
%%%%%%%%%%%%%%%%%%%%%%%%%%%%%%%%%%%%%%%%%%%%%%%%%%%%%%%%%%%

\newcommand\ind{\protect\mathpalette{\protect\independenT}{\perp}}
\def\independenT#1#2{\mathrel{\rlap{$#1#2$}\mkern2mu{#1#2}}}

\nipsfinalcopy % Uncomment for camera-ready version

\begin{document}

\maketitle

Most of the semester covered chapters from \citet{mohri12foundations}, as
discussed in Sections \ref{sec:PAC} to \ref{sec:reg} below. The main
theoretical ideas are PAC learning and Rademacher complexity, presented in
Chapters 2 and 3, while the remainder of the book is devoted largely to
applications of these ideas to give generalization bounds various learning
problems and algorithms. Thus, we discuss these in detail (in Sections
\ref{sec:PAC} and \ref{sec:Rademacher}) and discuss later sections relatively
more summarily.

\section{PAC Learning}
\label{sec:PAC}
\subsection{The PAC learning model}
Let $h$ be a hypothesis and let $c$ be a target concept. Then, for an
underlying distribution $D$ the \emph{generalization error} or \emph{risk} of
$h$ is
\[R(h)
    = \pr_{x \sim D}\left[ h(x) \neq c(x) \right]
    = \E_{x \sim D}\left[ 1_{h(x) \neq c(x)} \right],\]
where $1_E$ denotes the indicator function of an event $E$.
Given a sample $x_1,\dots,x_n$, the \emph{empirical error} is
\[\hat R(h) = \frac1n \sum_{i = 1}^n 1_{h(x_i) \neq c(x_i)}.\]
Note that
\begin{equation}
\E[\hat R(h)] = R(h)
\label{eq:exp_emp_risk}
\end{equation}
Moreover, we will show finite-sample guarantees for the convergence
$\hat R(h) \to R(h)$ as $x \to \infty$. \\

A concept class $C$ is \emph{PAC-learnable} if $\exists$ and algorithm $\A$ and
a polynomial function $poly(\cdot,\cdot,\cdot,\cdot)$ such that,
$\forall \e, \delta > 0$, distributions $D$ on $\X$, and $c \in C$,
\[\pr_{S = (x_1,\dots,x_n) \sim D^n}\left[ R(h_S) \leq \e \right]
    \geq 1 - \delta\]
whenever the sample size $n \geq poly(1/\e,1/\delta,n,size(c))$
\footnote{$size(c)$ denotes the computational memory overhead of $c$.}
If $\A$ runs in $poly(1/\e,1/\delta,n,size(c))$, then $C$ is \emph{efficiently
PAC-learnable}, and $\A$ is a PAC-learning algorithm for $C$. \\

It is now worth noting the following three points:
\begin{itemize}
\item The PAC framework is distribution-free: no assumption is made about $D$.
\item The training and test samples are drawn from identical distributions;
this is typically necessary for generalization to occur.
\item PAC deals with learnability for a concept class $C$, not a single
(unknown) concept $c$.
\end{itemize}

\subsection{Guarantees for finite hypothesis sets: the consistent case}
A hypothesis is \emph{consistent} with a sample $S$ if $\hat R(h_S) = 0$
We begin with guarantees in the case of a finite hypothesis class $H$ and an
algorithm that returns consistent hypothesis:\\

{\bf Theorem 2.1 - Learning Bounds in the finite $H$, consistent case:}
Consider a finite hypothesis class $H$ and let $\A$ be an algorithm such that,
for any concept $c \in H$ and i.i.d. sample $S$, $\A$ returns a consistent
hypothesis $h_S$. Then, $\forall \e, \delta > 0$,
\[\pr_{S = (x_1,\dots,x_n) \sim D^n}\left[ R(h_S) \leq \e \right]
    \geq 1 - \delta\]
whenever
\[n \geq \frac1\e \left( \log|H| + \log \delta \right).\]
Said another way, $\forall \e, \delta > 0$, with probability at least
$1 - \delta$,
\[R(h_S) \leq \frac1m \left( \log|H| + \log \delta \right).\]

This subsection concludes with two examples concerning PAC-learnability of
boolean formulae and a brief discussion of the fact that a universal concept
class is not PAC-learnable.

\subsection{Guarantees for finite hypothesis sets: the inconsistent case}
In general, there may be no hypothesis in $H$ consistent with the training
sample; this is in fact common in practice, when the concept class may be far
larger than the hypothesis class. However, we can provide guarantees even when
$\hat R(h)$ is small but nonzero. To do so, we recall a corollary of
Hoeffding's Inequality (and observation (\ref{eq:exp_emp_risk})):\\

{\bf Corollary 2.1 - Corollary of Hoeffding's Inequality:} Let $\e > 0$, and
let $S$ denote an i.i.d. size $n$ sample. Then, $\forall h : \X \to \{0,1\}$,
\[\pr_{S \sim D^n}\left[ |\hat R(h) - R(h)| \geq \e \right]
    \leq 2\exp(-2n\e^2).\]
Said another way, with probability at least $1 - \delta$,
\[R(h) \leq \hat R(h) + \sqrt{\frac{\log(2/\delta)}{2n}}.\]

The corollary bounds the generalization error of any particular hypothesis,
but, in order to bound the error of a learning algorithm, we again need uniform
bounds over all hypotheses in $\H$, as provided by the following theorem:

{\bf Theorem 2.2 - Learning Bounds in the finite $H$, inconsistent case:}
Consider a finite hypothesis class $\H$. Then, $\forall \delta > 0$, with
probability at least $1 - \delta$, $\forall h \in \H$,
\[R(h) \leq \hat R(h) + \sqrt{\frac{\log|H| + \log(2/\delta)}{2n}}.\]

It is now worth noting that this result suggests that the size of the
hypothesis class $\H$ is a trade-off between empirical risk (which may be
smaller for a larger hypothesis class) and the error of $\hat R(h)$
approximating $R(h)$. Furthermore, for two hypothesis classes of equal
empirical risk, the smaller class has better generalization error; this can be
compared to Occam's Razor.

The remainder of the chapter generalizes to the agnostic case, which is
analyzed in terms of the Bayes error rate, and then discusses model selection,
in terms of empirical risk minimization (ERM), which performs poorly due to
overfitting, and structural risk minimization (SRM), which performs better by
considering a sequence of hypothesis classes of increasing complexity, but
is computationally very expensive, and finally briefly mentions regularization.

\section{Rademacher Complexity and VC Dimension}
\label{sec:Rademacher}
The discussion of the previous chapter was restricted to the case of finite
hypothesis classes $H$. The goal of this chapter is to generalize to the case
of an infinite hypothesis class. In particular, the main idea is to reduce the
infinite case to the analysis of a finite case. One approach uses the notion of
\emph{Rademacher complexity}, which leads to high-quality (sometimes
data-dependent) guarantees with simple proofs based on McDiarmid's Inequality.
Since computing Rademacher complexity is often NP-hard, we will also introduce
two purely combinatorial notions, the \emph{growth function} and the
\emph{VC-dimension}, the latter of which is often easier to bound or estimate.
We also present lower bounds based on VC-dimension.

\subsection{Rademacher Complexity}
Rademacher complexity measures the richness of a function class in terms of how
well it can fit random noise. \\

A \emph{Rademacher random variable} is $\sigma = (\sigma_1,\dots,\sigma_n)^T$
with $\sigma_i$'s independent and uniform over $\{-1,1\}$. The \emph{Empirical
Rademacher complexity} of a family $G$ of functions $f : Z \to [a,b]$ with
respect to a sample $S = (z_1,\dots,z_n) \subseteq Z$ is
\[\hat\Rc_S(G)
    := \E_\sigma\left[ \sup_G \frac1n \sum_{i = 1}^m \sigma_i g(z_i)\right]
    = \E_\sigma\left[ \sup_{g \in G} \frac{\sigma \cdot g_S}{m} \right],
\]
where $g_S = (g(z_1),\dots,g(z_n))^T$; i.e., $\hat\Rc_S(G)$ measures
how well $G$ can correlate $S$ to random noise. If $D$ is the distribution from
which $S$ is drawn, then the \emph{Rademacher complexity} of $G$ is the
expected empirical Rademacher complexity
\[\Rc_n(G) := \E_{S \sim D^n}\left[ \hat\Rc_S(G) \right].\]

We now present a generalization bound in terms of Rademacher complexity: \\

{\bf Theorem 3.1 (Rademacher Complexity Generalization Bound):} Consider a
family $G$ of functions mapping $Z$ to $[0,1]$. For any $\delta > 0$, with
probability at least $1 - \delta$,
$\forall g \in G$,
\[\E[g(z)]
    \leq \frac1n \sum_{i = 1}^n g(z_i)
    + 2\Rc_m(G)
    + \sqrt{\frac{\log(1/\delta)}{2n}}.\]
and
\[\E[g(z)]
    \leq \frac1n \sum_{i = 1}^n g(z_i)
    + 2\Rc_S(G)
    + 3\sqrt{\frac{\log(1/\delta)}{2n}}.\]

Another generalization bound, this one potentially data-dependent, in terms of
Rademacher complexity is then provided for the case of binary classification.
However, computing this bound for real data is computationally hard for many
hypothesis sets. The next subsections will relate Rademacher complexity to
complexity measures that are easier to compute.

\subsection{Growth Functions}
The \emph{growth function} $\Pi_H : \N \to \N$ of a hypothesis class $H$
defined by
\[\Pi_H(m) = \max_{\{x_1,\dots,x_m\} \subseteq \X}
    \left| \left\{
        \left( h(x_1),\dots,h(x_m) \right) : h \in H
    \right\} \right|, \quad \forall m \in \N\]
is the number of distinct ways in $H$ can classify $m$ points. Note that this
notion of complexity is distribution-free. Massart's Lemma upper bounds
$\Rc_m(H)$ in terms of $\Pi_H(m)$. \\

{\bf Theorem 3.3 (Massart's Lemma):} For $A \subseteq \R^m$ finite,
$r := \max \{\|x\|_2 : x \in A\}$,
\[\E_\sigma \left[ \sup_{x \in A} \frac{\sigma \cdot x}{m} \right]
    \leq \frac{r\sqrt{2 \log |A|}}{m},\]
where $\sigma \in \R^m$ is a Rademacher random variable.

By definition of the Rademacher complexity and the growth function, Massart's
Lemma implies
\begin{equation}
\Rc_m(H) \leq \sqrt{\frac{2 \log \Pi_H(m)}{m}}
\label{ineq:rademacher_and_growth}
\end{equation}
for any set of classifiers $H$. Combining (\ref{ineq:rademacher_and_growth})
with the Rademacher generalization bound (Theorem 3.1) gives
\begin{equation}
R(h) \leq \hat R(h) + \sqrt{\frac{2 \log \Pi_H(m)}{m}}
                    + \sqrt{\frac{\log (1/\delta)}{2m}},
\label{ineq:growth_gen_bound}
\end{equation}
for any classifier $h \in H$, with probability at least $1 - \delta$
($\delta > 0$).

\subsection{VC-Dimension}
The chapter then continues to define VC-dimension $CVdim(H)$ in the usual
manner, and presents some basic examples (intervals and sine functions on $\R$,
and hyperplanes axis-aligned rectangles, and convex polygons on $\R^2$).
VC-dimension is also distribution free and can be much easier to calculate for
important hypothesis classes than Rademacher complexity or the growth function,
though it sometimes results in looser bounds.

Sauer's Lemma presents an important relationship between the two combinatorial
notions of complexity (the growth function and VC-dimension): \\

{\bf Theorem 3.5 (Sauer's Lemma):} For a hypothesis class $H$ with
$VCdim(H) = d$ and all $m \in \N$,
\[\Pi_H(n) \leq \sum_{i = 0}^d \binom{m}{i}.\]
The proof proceeds by induction on $m + d$ and carefully counting the growth
function of a union of hypothesis classes. Sauer's lemma is important in part
because of the following corollary: \\

{\bf Corollary 3.3 (VC-Dimension Generalization Bound):} For a hypothesis class
$H$ with $VCdim(H) = d$ and all $m \geq d$,
\[\Pi_H(m) \leq \left( \frac{em}{d}^d \right) = O(m^d).\]
Combining this with the growth function generalization bound
(\ref{ineq:rademacher_and_growth}) gives
\begin{equation}
R(h) \leq \hat R(h) + \sqrt{\frac{2 d \log \frac{em}{d}}{m}}
                    + \sqrt{\frac{\log (1/\delta)}{2m}},
\label{ineq:growth_gen_bound2}
\end{equation}
for any classifier $h \in H$, with probability at least $1 - \delta$
($\delta > 0$). \\

The final section of this chapter proves lower bounds on generalization error
in terms of the VC-dimension of the hypothesis class (regardless of the
algorithm used), in both the realizable and non-realizable cases.

\subsection{Kernel Methods}
The book then discusses kernel methods, which give a standard, almost
formulaic, way of extending many algorithms for $\R^n$-valued predictors (in
particular, those depending only on the matrix of inner products between
predictor samples) to arbitrary data types or other, more complicated measures
of similarity between the predictor samples (to extract non-linear behavior,
for example).

The chapter discusses PSD kernels, RKHSs, empirical kernel maps (embeddings),
and the representer theorem. It then gives a bound on the Rademacher complexity
of kernel-based hypothesis classes in terms of the distances between points,
and, as a corollary, bounds the generalization error of large-margin
kernel-based hypotheses.

Finally, the chapter explores a variety of kernels for different settings and
data formats, such as sequence kernels and weighted transducers for text (or
otherwise sequential, data, and rational kernels.

\section{Regression}
\label{sec:reg}

This chapter first discusses extending VC theory from classification to
regression. To do this, we essentially reduce regression (i.e., the case of
$\R$-valued labels) to classification (i.e., the case of $\{-1,1\}$-valued
labels) via the signum function. Equivalently, we consider a threshold $T$
below which points are considered to be labeled $-1$ and above which points are
considered to be labeled $1$. We then consider hypothesis classes whose
decision boundaries are the graphs of functions of the predictor variables. In
place of the VC-dimension, we use the pseudo-dimension, defined for a family of
regression hypothesis as the VC-dimension of the corresponding family of
graphs.
Using this definition, it is straightforward to prove generalization bounds
similar to Theorem 3.1, as well as to compute the pseudo-dimension of some
simple parametric families, such as hyperplanes.

The chapter then goes on to discuss several algorithms for regression,
including linear regression, kernel ridge regression (KRR), support vector
regression (SVR), and the lasso. Linear regression, while simple, typically
performs poorly. KRR, as the name suggests, kernelizes ridge regression,
allowing one to fit nonlinear regression functions. Its main downside is
computational; evaluating the kernel function for all pairs of data points is
tremendously slow for large data sets. SVR is more efficient and encourages
data-sparsity (i.e., as solution depending only on the most informative (or
erroneous) data points. Finally, the lasso cannot be kernelized, and can hence
fit only linear regression functions, but can be computed efficiently (via LAR)
and encourages feature-sparsity. Rademacher-type generalization bounds are
proven for KRR, SVR, and lasso.

Finally, group-norm regression algorithms (an extension of the lasso
encouraging joint sparsity of predefined groups of variables) are mentioned and
two algorithms (the Widrow-Hoff algorithm and online dual SVR) for online
regression are discussed. Both the Widrow-Hoff algorithm and online dual SVR
essentially amount to running online gradient descent to minimize some
objective function.

\section{Discriminative vs. Generative Classifiers}
Finally, we discussed a classic paper, \citet{ng02discVsGen}. The gist of this
paper is that, while generative classifiers (e.g., naive Bayes), tend to have
higher asymptotic error than similar discriminative classifiers (e.g., logistic
regression), generative classifiers tend to approach their asymptotic error
more quickly. Thus, there is a windows of small sample sizes for which
generative classifiers generally out perform discriminative classifiers, which
perform better asymptotically. Intuitively, this occurs because generative
classifiers make stronger assumptions about the data-generating process,
allowing them to fit their models more quickly, but preventing their model from
being able to fit certain populations as well.

In addition to displaying this phenomenon in several UCI data sets,
\citet{ng02discVsGen} give several theoretical results. In particular, suppose
$(h_{Gen},h_{Dis})$ is a generative-discriminative pair of classifiers with
asymptotic (population) versions $h_{Gen,\infty}$ and $h_{Dis,\infty}$. Then,
when classifying $m$ samples in $\R^n$, letting $\e(h)$ denote the error rate
of any hypothesis $h$,
\begin{enumerate}
\item if the family of hypotheses under consideration has finite VC dimension,
$\e(h_{Dis,\infty}) \leq \e(h_{Gen,\infty})$
\item if $h_{Dis}$ is logistic regression, then, with high probability,
\[\e(h_{Dis})
    \leq \e(h_{Dis,\infty})
        + O\left( \sqrt{\frac{n}{m} \log\left( \frac{m}{n} \right)} \right).\]
\item under mild assumptions on the data, then, with high probability,
\[\e(h_{Gen})
    \leq \e(h_{Gen,\infty})
        + G\left( O\left( \sqrt{\frac{1}{m} \log(n)} \right) \right),\]
where $G$ denotes a sort of margin probability.
\end{enumerate}
Thus, generative classifiers can guarantee superior performance in the
high-dimensional, small data (large $n$, small $m$) regime.

\subsubsection*{References}
\setlength{\bibsep}{0.0pt}
{
%\small
\bibliographystyle{plainnat}
\bibliography{biblio}
}

\end{document}
