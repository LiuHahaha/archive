\documentclass{article} % For LaTeX2e
\usepackage{amsmath, amsfonts, amsthm, amssymb}
\usepackage{nips14submit_e,times}
\usepackage{hyperref}
\usepackage{url}

% For bibliography
\usepackage{natbib}
\renewcommand{\bibsection}{}

\title{Information Theoretical Estimators for Time Series}

\author{
Shashank Singh \\
Statistics \& Machine Learning Departments \\
Carnegie Mellon University \\
Pittsburgh, PA 15213 \\
\texttt{sss1@andrew.cmu.edu}
}

\newcommand{\fix}{\marginpar{FIX}}
\newcommand{\new}{\marginpar{NEW}}

%%%%%%%%%%%%%%%%%%%%CONTENT MACROS%%%%%%%%%%%%%%%%%%%%%%%%%
\renewcommand{\qed}{\quad \ensuremath{\blacksquare}}    % QED blacksquare
\newcommand{\inv}{^{-1}}                            % inverse operator
\newcommand{\sminus}{\backslash}                    % set minus
\newcommand{\N}{\mathbb{N}}                         % natural numbers
\newcommand{\R}{\mathbb{R}}                         % real numbers
\newcommand{\pow}{\mathcal{P}}                      % power set
\newcommand{\Se}{\mathcal{S}}                       % partition
\newcommand{\e}{\varepsilon}                        % \varepsilon
\newcommand{\X}{\mathcal{X}}                        % X domain
\newcommand{\Y}{\mathcal{Y}}                        % Y domain
\newcommand{\Z}{\mathcal{Z}}                        % Z domain
\newcommand{\A}{\mathcal{A}}                        % sub-domain
\newcommand{\E}{\mathbb{E}}                         % expected value
\newcommand{\V}{\mathbb{V}}                         % variance
\newcommand{\pr}{\mathbb{P}}                        % probability
\newcommand{\cpest}{\widehat{p}_h}                  % clipped estimated p_n
\newcommand{\cqest}{\widehat{q}_h}                  % clipped estimated q_n
\newcommand{\pest}{\widetilde{p}_h}                 % estimated p_n
\newcommand{\qest}{\widetilde{q}_h}                 % estimated q_n
\newcommand{\vx}{\vec{x}}                           % vector x
\newcommand{\vy}{\vec{y}}                           % vector y
\newcommand{\vz}{\vec{z}}                           % vector z
\newcommand{\vv}{\vec{v}}                           % vector v
\newcommand{\vu}{\vec{u}}                           % vector u
\newcommand{\vi}{{\vec{i}}}                         % multi-index vector i
\newcommand{\dist}{\operatorname{dist}}             % distance operator
\newcommand{\C}{\mathcal{C}}                        % center region of [0,1]^d
\newcommand{\B}{\mathcal{B}}                        % border region of [0,1]^d
\newcommand{\acro}[1]{\textsc{\MakeLowercase{#1}}}
%%%%%%%%%%%%%%%%%%%%%%%%%%%%%%%%%%%%%%%%%%%%%%%%%%%%%%%%%%%

\nipsfinalcopy % Uncomment for camera-ready version

\begin{document}

\vspace{-5mm}

\maketitle

\vspace{-5mm}

Information Theoretical Functionals (ITFs) such as entropy, mutual information,
divergence, and their conditional versions, are important nonparametric
measures of variation, dependence, and distance between random variables. Over
the past few years, much work as gone into designing and characterizing
estimators for ITFs, using IID samples from each of the relevant variables.

However, many data sources, such as neurophysiological, economic, and
environmental data, consist of time series, where samples of each variable are
not independent. Often we would like to perform analyses for time series
similar to those for IID data. For example, to study functional connectivity in
fMRI data, it might be desirable to learn a graphical model over voxels via the
PC algorithm. Thus, the broad {\bf goal of our project} is to understand how
methods using ITFs and their estimators can be adapted for time series data.
A few obvious approaches might be to:
\begin{enumerate}
\item model the time series data as IID, and use ITFs and their estimators as
usual.
\item attempt to decorrelate the time series, and then use ITFs and their
estimators as usual.
\item propose new ITFs accounting for time series dependence to capture
properties we want.
\end{enumerate}

Approach 1 will likely perform poorly; for example, mutual information
estimators designed for IID data will be heavily biased when applied to time
series because they fail to account for autocorrelation. Approach 2 attempts to
correct for this, and is {\bf one the ideas we propose} to explore.

An example of Approach 3 that has been studied previously is the notion of 
transfer entropy (actually a case of conditional mutual information), defined
for general time series $\{Y_i\}_{i = 1}^n, \{Y_i\}_{i = 1}^n$ as
\[T_{X \to Y}
    = H(Y_n | \{Y_i\}_{i = 1}^{n - 1}) - H(Y_n | \{X_i\}_{i = 1}^{n - 1}, \{Y_i\}_{i = 1}^{n - 1})
    = I(Y_t; \{X_i\}_{i = 1}^{n - 1} | \{Y_i\}_{i = 1}^{n - 1}).
\]
Transfer entropy, the mutual information between the last point of $Y$ and the
previous values of $X$ given the previous values of $X$, roughly measures the
predictive power of $X$ on $Y$ after accounting for autocorrelation. Thus,
this can be used as a dependence measure among time series. On the other hand,
the dimension of variables involved in transfer entropy increases with the
length of the time series, making estimation difficult without some sort of
independence (e.g., Markov or mixing) assumptions. Thus, {\bf we also propose}
to study the problem of estimating transfer entropy, and perhaps other similar
ITFs designed for time series, under such assumptions 

We intend to study these approaches both {\bf theoretically}, using techniques
such as those in \cite{kandasamy2014influence}, \cite{moon14ensemble}, and
\cite{singh14densityfunc}, and {\bf empirically}, using synthetic data as well 
as an fMRI dataset used in \cite{Clute14OHBM}.

\section{Ideas}

Another possibility might be to generalize the notion of cross-correlation to a
notion of cross-information, which would\dots This wouldn't share the
computational benefits of cross-correlation, but\dots

\subsubsection*{References}
\setlength{\bibsep}{0.0pt}
{
%\small
\bibliographystyle{plain}
\bibliography{biblio}
}

\end{document}
