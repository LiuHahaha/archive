% LaTeX file for resume or CV
% This file uses the resume document class (res.cls)

\documentclass{res} 
\setlength{\textheight}{9.5in} % increase text height to fit on 1-page 
\usepackage{enumerate}
\usepackage{fullpage}
\usepackage{fancyhdr}
\usepackage{amsmath, amsfonts, amsthm, amssymb}
\newcommand{\ba}{\mathbf{a}}
\newcommand{\bb}{\mathbf{b}}
\newcommand{\bc}{\mathbf{c}}
\newcommand{\bd}{\mathbf{d}}
\newcommand{\bu}{\mathbf{u}}
\newcommand{\bv}{\mathbf{v}}
\newcommand{\bx}{\mathbf{x}}
\newcommand{\by}{\mathbf{y}}
\newcommand{\bchi}{\boldsymbol{\chi}}
\newcommand{\bzero}{\mathbf{0}}

\begin{document} 

\name{Skew-Symmetric Matrices in Linear Systems of Differential Equations\\[12pt]}     % the \\[12pt] adds a blank
				                     % line after name      

\address{       Shashank Singh\\
                86-712 Computational Neuroscience of Vision\\
                September 9, 2012
}
\address{}
\begin{resume}

Consider an equation of the form
\begin{equation}
\frac{dx}{dt} = a x,
\end{equation}
where $a \in \mathbb{R}$.

Rewriting equation (1) and then integrating gives
\begin{align*}
\frac1x \, dx & = a \, dt \\
\ln(x)        & = at + C \\
x             & = e^{at + C} = e^Ce^{at}
\end{align*}
(we can ``absorb'' the integration constants from both sides into $C$).

Suppose now that we have $n$ equations with $n$ independent variables:
\begin{align*}
\frac{dx_1}{dt} & = a_{1,1}x_1 + a_{1,2}x_2 + \ldots + a_{1,n}x_n \\
\frac{dx_1}{dt} & = a_{2,1}x_1 + a_{2,2}x_2 + \ldots + a_{2,n}x_n \\
                &                     \vdots                      \\
\frac{dx_1}{dt} & = a_{n,1}x_1 + a_{n,2}x_2 + \ldots + a_{n,n}x_n.
\end{align*}
We can write this more compactly as
\begin{equation}
\frac{d\bx}{dt} = A \bx,
\end{equation}

where
\[A = \begin{bmatrix}
        a_{1,1} & a_{1,2} & \ldots & a_{1,n} \\
        a_{2,1} & a_{2,2} & \ldots & a_{2,n} \\
        \vdots  & \vdots  & \vdots & \vdots  \\
        a_{2,1} & a_{2,2} & \ldots & a_{2,n} \\
      \end{bmatrix}
\quad \mbox{ and } \quad
\bx = \begin{bmatrix}
        x_1    \\
        x_2    \\
        \vdots \\
        x_n    \\
      \end{bmatrix}
\]
Inspired by the solution to the $1$ variable equation, we'll assume that the
solution is an exponential. Since we want a vector solution, we just multiply
by a vector $\bchi$, so that we assume a solution is of the form
$\bx = \bchi e^{rt}$, for some $r \in \mathbb{R}$. Plugging this into equation
(2) gives
\[r \bchi e^{rt} = A \bchi e^{rt}.\]
Since $e^{rt} \neq 0$,
\[(A - rI)\bchi = \bzero.\]
Notice that $r$ and $\bchi$ solve this equation if and only if $\bchi$ is an
eigenvector of $A$ and $r$ is the associated eigenvalue.

For reasons that I won't go into here, the solutions of an $n \times n$ system
of linear differential equations form an $n$-dimensional vector space, so that
we can summarize the solution set as a linear combination of a basis set of
solutions:
\[\bx = \sum_{k = 1}^{n} c_k \bchi_k e^{r_k t},\]
where $\bchi_k$ is the $i^{th}$ eigenvector and $r_k$ is the associated
eigenvalue. (This gets more complicated when $A$ there are fewer than $n$
linearly independent eigenvalues, but this doesn't happen when $A$ is
skew-symmetric, which is what we care about here.)

Now what happens when some eigenvalues are in $\mathbb{C}$ but not in
$\mathbb{R}$? Then, each $r_k = a_k + b_ki$, where $a,b \in \mathbb{R}$.
Thus,
\begin{equation}
\bx = \sum_{k = 1}^{n} c_k \bchi_k e^{a_k t}e^{b_k i t}.
\end{equation}
Now it's clear that $a_k$, the real part of the each eigenvalue acts as a
dilaiton factor. But how do we interpret this complex component? We use Euler's
identity:
\[e^{it} = \cos(t) + i\sin(t).\]
(if you haven't seen this before, the proof is one line; examine the Taylor
expansions of sine, cosine, and the exponential function.)
Rewriting equation (3) using Euler's Identity gives us
\[\bx
 = \sum_{k = 1}^{n}
            c_k \bchi_k e^{a_k t}\left(\cos(b_k t) + i\sin(b_k t)\right).\]
Splitting up $\bchi_k = \bc_k + i\bd_k$ into its real and complex components and
doing some algebra shows that, for
\[\bu_k = e^{a_kt}(\bc_k \cos(b_k t) - \bd_k \sin(b_k t)\]
\[\bv_k = e^{a_kt}(\bc_k \sin(b_k t) + \bd_k \cos(b_k t)\]
(since the complex eigenvectors of $A$ come in complex conjugate pairs and
$\bu_k$ and $\bv_k$ are the same for each pair, we're replacing each pair with the
pair $\{\bu_k,\bv_k\}$, which can be shown to be linearly independent, so that we
still have a basis.)

As it happens, when $A$ is skew-symmetric, \emph{all} (nonzero) eigenvalues of
$A$ are \emph{purely} complex; that is, each $a_k = 0$. Therefore, we get the
solution
\[\bx
 = \sum_{k = 1}^{n} c_k \bchi_k \left(\cos(b_k t) + i\sin(b_k t)\right),\]
which is purely oscillatory.

Thus, if we fit a skew-symmetric matrix $M_{skew}$ to the equation
\[\bx^{\prime} = M_{skew} \bx,\]
we get a matrix that captures the oscillatory motion of the dynamical system.


\end{resume}
\end{document}
