\documentclass[11pt]{article}
\usepackage{enumerate}
\usepackage{fullpage}
\usepackage{fancyhdr}
\usepackage{amsmath, amsfonts, amsthm, amssymb}
\usepackage{color}
\setlength{\parindent}{0pt}
\setlength{\parskip}{5pt plus 1pt}
\pagestyle{empty}

\def\indented#1{\list{}{}\item[]}
\let\indented=\endlist

\newcounter{questionCounter}
\newcounter{partCounter}[questionCounter]
\newenvironment{question}[2][\arabic{questionCounter}]{%
    \setcounter{partCounter}{0}%
    \vspace{.25in} \hrule \vspace{0.5em}%
        \noindent{\bf #2}%
    \vspace{0.8em} \hrule \vspace{.10in}%
    \addtocounter{questionCounter}{1}%
}{}
\renewenvironment{part}[1][\alph{partCounter}]{%
    \addtocounter{partCounter}{1}%
    \vspace{.10in}%
    \begin{indented}%
       {\bf (#1)} %
}{\end{indented}}

%%%%%%%%%%%%%%%%%%%%%%%HEADER%%%%%%%%%%%%%%%%%%%%%%%%%%%%%%
\newcommand{\myname}{Shashank Singh}
\newcommand{\myandrew}{sss1@andrew.cmu.edu}
\newcommand{\myclass}{21-651 General Topology}
\newcommand{\myhwnum}{5}
\newcommand{\duedate}{Monday, November 5, 2012}
%%%%%%%%%%%%%%%%%%%%%%%%%%%%%%%%%%%%%%%%%%%%%%%%%%%%%%%%%%%

%%%%%%%%%%%%%%%%%%%%CONTENT MACROS%%%%%%%%%%%%%%%%%%%%%%%%%
\renewcommand{\qed}{\quad $\blacksquare$}
\newcommand{\mqed}{\quad \blacksquare}
\newcommand{\inv}{^{-1}}
\newcommand{\bx}{\mathbf{x}}
\newcommand{\by}{\mathbf{y}}
\newcommand{\bff}{\mathbf{f}}
\newcommand{\bzero}{\mathbf{0}}
\newcommand{\bxi}{\boldsymbol{\xi}}
\newcommand{\boldeta}{\boldsymbol{\eta}}
\newcommand{\B}{\mathcal{B}}
\newcommand{\sminus}{\backslash}
\newcommand{\N}{\mathbb{N}} % natural numbers
\newcommand{\Z}{\mathbb{Z}} % integers
\newcommand{\Q}{\mathbb{Q}} % rational numbers
\newcommand{\R}{\mathbb{R}} % real numbers
\newcommand{\pow}[1]{\mathcal{P}\left(#1\right)} % power set of #1
\newcommand{\epi}[1]{\operatorname{epi} #1 } % epigraph of #1
%%%%%%%%%%%%%%%%%%%%%%%%%%%%%%%%%%%%%%%%%%%%%%%%%%%%%%%%%%%

\begin{document}
\thispagestyle{plain}

{\Large Homework \myhwnum} \\
\myclass \\
Name: \myname \\
Email: \myandrew \\
Due: \duedate

The following characterization of lower semi-continuity is used in the proofs
of problems 4. and 6.

{\bf Lemma 1:}
$f : X \rightarrow \R$ is lower semi-continuous if and only if,
$\forall x_0 \in X$, $\forall \epsilon > 0$, $\forall x$ is some neighborhood
$U$ of $x_0$ in $X$, $f(x) \geq f(x_0) - \epsilon$.

{\bf Proof of Lemma 1:} $(\Rightarrow)$ Suppose that $f$ is lower
semi-continuous, and let $x_0 \in X, \epsilon > 0$. Let
$U = (f(x_0) - \epsilon, \infty)$, so that, since $f$ is lower
semi-continuous, $V := f\inv(U)$ is open. Then, $V$ is a neighborhood of $x_0$
in $X$ such that $f(V) \subseteq (f(x_0) - \epsilon, \infty)$, so that,
$\forall x \in V$, $f(x) \geq f(x_0) - \epsilon$. \qed

$(\Leftarrow)$ Suppose that, $\forall x_0 \in X, \epsilon > 0$, there is a
neighborhood $W$ of $x_0$ in $X$ such that, $\forall x \in W$,
$f(x) \geq f(x_0) - \epsilon$.
Let $a \in \R$, and, for $V := (a,\infty)$, let
$x_0 \in U := f\inv(V)$. Since, $V$ is open, for some $\epsilon > 0$,
$(f(x_0) - \epsilon, \infty) \subseteq (a,\infty)$. Thus, there is a
neighborhood $W$ of $x_0$ in $X$ such that, $\forall x \in W$,
$f(x) \geq f(x_0) - \epsilon > a$, so that $f(W) \subseteq (a,\infty)$, and
thus $W \subseteq U$. Since every point in $U$ is contained in a neighborhood
in $U$, $U$ is open, so that $f$ is lower semi-continuous. \qed

\begin{question}{Problem 1}
Let $I := C(X,[0,1])$, and let $e$ be the function on $X$ defined by
\[e(x) (f) := f(x), \forall f \in I\]
($e$ is, in some sense, a restriction of the evaluation function used in
Stone-Cech compatification). Note that, since, $\forall x \in X$,
$e(x) : I \rightarrow [0,1]$, $e(X) \subseteq [0,1]^I$. $e$ is a
homeomorphism between $X$ and $e(X)$. Since $(X,\tau)$ is Tikhonov,
the proof is identical to the proof that the evaluation function used in
Stone-Cech compactification is homeomorphism (as used in the proof of
Theorem 142). \qed
\end{question}

\begin{question}{Problem 2}
\begin{enumerate}[(i)]
\item $(\Rightarrow)$ Suppose, $f : X \rightarrow [0,1]$ is continuous, with
$f = 0$ on $C$, and $f > 0$ on $X \sminus C$. $\forall i \in \N$, define
$U_i := f\inv([0,1/i))$. Since, $\forall i \in \N$, $(-1,1/i)$ is open in the
standard topology on $\R$, $[0,1/i)$ is open in the topology induced on
$[0,1]$, so that, since $f$ is continuous, $U_i \in \tau$. Then,
\[
\bigcap_{i = 1}^{\infty} U_i
 = \{x \in X : \forall i \in \N, f(x) \in [0,1/i)\}
 = f\inv(\{0\})
 = C,
\]
so that $C$ is a $G_{\delta}$ set. \qed

$(\Leftarrow)$ Suppose $C$ is a $G_{\delta}$, so that
$C = \bigcap_{i = 1}^{\infty} U_i$ for $U_i \in \tau$. Note that,
$\forall i \in \N$, $X \sminus U_i$ is closed and disjoint from $C$, so that,
since $(X,\tau)$ is normal, by Urysohn's Lemma, there exists a continuous
function $f_i : X \rightarrow [0,1]$ with $f_i = 0$ on $C$ and $f_i = 1$ on
$X \sminus U_i$.

Define $f : X \rightarrow [0,1]$ by
$f(x) = \sum_{i = 1}^{\infty} \frac{f_i(x)}{2^i}$. (Note that, since,
$\forall x \in X, i \in \N$, $f_i(x) \in [0,1]$,
\[0
 \leq f(x)
 =    \sum_{i = 1}^{\infty} \frac{f_i(x)}{2^i}
 \leq \sum_{i = 1}^{\infty} \frac{1}     {2^i}
 \leq 1,\]
so that $f$ has range in $[0,1]$.) If $x \in C$, then, $\forall i \in \N$,
$f_i(x) = 0$, so that $f(x) = 0$. If $x \in X \sminus C$, then, since
$C = \bigcap_{i = 1}^{\infty} U_i$, for some $i \in \N$,
$x \in X \sminus U_i$. Thus, $f_i(x)/2^i > 0$, so that
$f(x) \geq f_i(x)/2^i > 0$. Finally, since $f$ is the limit of the uniformly
convergent sequence $\{\sum_{i = 1}^n \frac{f_i}{2^i}\}_{n = 1}^{\infty}$ of
continuous functions, $f$ is continuous, so $f$ has the desired properties.
\qed

\item $(\Rightarrow)$ Suppose $\exists f : X \rightarrow [0,1]$ continuous
such that $f = 1$ on $C_1$, $f = 0$ on $C_2$, and $0 < f < 1$ on
$X \sminus (C_1 \cup C_2)$. If follows from the result of part (i) that $C_2$
is a $G_{\delta}$ set. Since $g : = (1 - f) : X \rightarrow [0,1]$ is
continuous with $g = 0$ on $C_1$ and $g > 0$ on $X \sminus C_1$, by the result
of part (i), $C_1$ is a $G_{\delta}$ set. \qed

$(\Leftarrow)$ Suppose, on the other hand, that $C_1,C_2$ are $G_{\delta}$
sets. By the result of part (i), $\exists f_1,f_2 : X \rightarrow [0,1]$ with
$f_1\inv(\{0\}) = C_1$ and $f_2\inv(\{0\}) = C_2$. Let
$f : X \rightarrow [0,1]$ be defined by
\[f(x) = \frac{f_1(x)}{f_1(x) + f_2(x)}, \forall x \in X.\]
Then, $f(x) = 0$ if $x \in C_1$, $f(x) = 1$ if $x \in C_2$, and
$0 < f(x) < 1$ if $x \in X \sminus (C_1 \cup C_2)$ (noting that $f$ is
everywhere well-defined because $C_1 \cap C_2 = \emptyset$), and $f$ is
continuous, as the quotient of continuous functions with a nonzero
denominator. \qed
\end{enumerate}
\end{question}

\begin{question}{Problem 3}
Suppose, for sake of contradiction, that there exist two distinct such
extensions $g_1,g_2 : \overline{E} \rightarrow Z$. Then,
$\exists x \in \overline{E}$ with $g_1(x)$ distinct from $g_2(x)$.
Then, since $Z$ is Hausdorff, $\exists U_1,U_2 \in \tau_Z$ such
that $g_1(x) \in U_1, g_2(x) \in U_2$ and $U_1 \cap U_2 = \emptyset$. Since
$g_1$ and $g_2$ are continuous,
$V_1 := g_1\inv(U_1)$, $V_2 := g_2\inv(U_2) \in \tau_Y$, so
$V_1 \cap V_2 \in \tau_Y$. Thus, $V_1 \cap V_2$ is a neighborhood of $x$,
so that, since $x \in \overline{E}$, $\exists y \in V_1 \cap V_2 \cap E$.
Then, $g_1(y) = f(y) = g_2(y)$, contradicting the fact that
$U_1 \cap U_2 = \emptyset$. \qed
\end{question}

\newpage
\begin{question}{Problem 4}
Suppose $f$ is lower semi-continuous, suppose that some convergent
sequence $x_n \rightarrow a$ in $X$ as $n \rightarrow \infty$, and let
$\epsilon > 0$. Since $f$ is lower semi-continuous, there is a neighborhood
$U$ of $a$ such that, $\forall x \in U$, $f(x) \geq f(a) - \epsilon$. Since
$x_n \rightarrow a$ as $n \rightarrow \infty$, $\exists n_0 \in \N$ such that,
$\forall n \in \N$ with $n \geq n_0$, $x_n \in U$, so that
$\inf_{n\geq n_0}f(x_n) \geq f(a)$. Since this is the case every
$\epsilon > 0$, $\liminf_{n\rightarrow \infty}f(x_n) \geq f(a)$, and thus,
$f$ is sequentially lower semi-continuous at every $a \in X$. \qed

Let $\tau_{coco}$ be the co-countable topology on $\R$, and let $f$ be the
identity function on $\R$ (mapping into the standard topology). It is easy to
see that a function is sequentially continuous if and only if it is both upper
and lower semi-continuous (since a sequence $x_n \rightarrow a$ as
$n \rightarrow \infty$ if and only if
$\liminf_{n \rightarrow \infty} x_n = a = \limsup_{n \rightarrow \infty} x_n$.
Since $f$ is
sequentially continuous (as sequences converge in $\tau_{coco}$ only if they
are eventually constant), $f$ is sequentially lower and upper semi-continuous.
However, since $f$ is not continuous ($(0,1)$ is not the complement of a
countable set), either $f$ or $-f$ must not be lower semi-continuous (in
particular, $f$ is not lower semi-continuous).
\end{question}

\begin{question}{Problem 5}
Suppose, for sake of contradiction, that $f$ is sequentially lower
semi-continuous but not lower semi-continuous. Since $f$ is not lower
semi-continuous, for some $a \in \R$, $S := f\inv(a,\infty)$ is not open, so
that some $x \in S$ is not an interior point of $S$. Let
$\B = \{B_i : i \in \N\}$ be a local base of $X$ at $a$, with
$B_{i + 1} \subseteq B_i$ (we can construct such a decreasing local base of
$X$ at $a$ by taking a countable base $\{C_i : i \in \N\}$ and replacing each
$C_i$ with $B_i := \bigcap_{j = 1}^i C_i$, which is a finite intersection of
open sets and is thus open). Since $x$ is not an interior point of $S$,
$\forall n \in \N$, $\exists x_n \in B_n \sminus U$, so that
$x_n \rightarrow x$ as $n \rightarrow \infty$. However, since no $f(x_n)$ is
in $(a,\infty)$, $f(x_n) \leq a < f(x)$, so that
$\liminf_{n \rightarrow \infty} f(x_n) \leq a < f(x)$, contradicting the fact
that $f$ is sequentially lower semi-continuous. \qed
\end{question}

\begin{question}{Problem 6}
Let $\tau_S$ denote the standard topology on $\R$, and let
$U := (X \times \R) \sminus (\epi{f})$. It suffices, then, to show that $f$ is
lower semi-continuous if and only if $U$ is open in the product topology
$\tau \times \tau_S$ on $X \times \R$.

$(\Rightarrow)$ Suppose $f$ is lower semi-continuous, and suppose
$(x_0,y_0) \in U$. Let $\epsilon > 0$, so that, by lower semi-continuity of
$f$, $\exists U_1 \in \tau$ with $x \in U_1$ such that, $\forall x \in U_1$,
$f(x) \geq f(x_0) - \epsilon$. Thus, for $U_2 := (-\infty,f(x_0) - \epsilon)$,
$U_1 \times U_2 \subseteq U$. Furthermore, since $U_1 \in \tau$ and
$U_2 \in \tau_S$, $U_1 \times U_2 \in \tau \times \tau_S$. Since,
$\forall (x,y) \in U$, $(x,y)$ has a neighborhood contained in $U$, $U$ is open. \qed

$(\Leftarrow)$ Suppose $U \in \tau \times \tau_S$, let $x_0 \in X$, and let
$\epsilon > 0$. Since $X \times \R$ is the product of finitely many
topological spaces, $\{U_1 \times U_2 : U_1 \in \tau, U_2 \in \tau_S\}$ gives
a base for $\tau \times \tau_S$. Thus, since
$(x_0,f(x_0) - \epsilon) \in U$, $\exists U_1 \in \tau, U_2 \in \tau_S$ such
that $(x_0,f(x_0) - \epsilon) \in U_1 \times U_2 \subseteq U$. Then, by
definition of $U$, $\forall x \in U_1$, $f(x) \geq f(x_0) - \epsilon$.
Therefore, $f$ is lower semi-continuous at all $x_0 \in X$. \qed
\end{question}
\end{document}
