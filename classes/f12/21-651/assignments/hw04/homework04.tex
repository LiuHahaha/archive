\documentclass[11pt]{article}
\usepackage{enumerate}
\usepackage{fullpage}
\usepackage{fancyhdr}
\usepackage{amsmath, amsfonts, amsthm, amssymb}
\usepackage{color}
\setlength{\parindent}{0pt}
\setlength{\parskip}{5pt plus 1pt}
\pagestyle{empty}

\def\indented#1{\list{}{}\item[]}
\let\indented=\endlist

\newcounter{questionCounter}
\newcounter{partCounter}[questionCounter]
\newenvironment{question}[2][\arabic{questionCounter}]{%
    \setcounter{partCounter}{0}%
    \vspace{.25in} \hrule \vspace{0.5em}%
        \noindent{\bf #2}%
    \vspace{0.8em} \hrule \vspace{.10in}%
    \addtocounter{questionCounter}{1}%
}{}
\renewenvironment{part}[1][\alph{partCounter}]{%
    \addtocounter{partCounter}{1}%
    \vspace{.10in}%
    \begin{indented}%
       {\bf (#1)} %
}{\end{indented}}

%%%%%%%%%%%%%%%%%%%%%%%HEADER%%%%%%%%%%%%%%%%%%%%%%%%%%%%%%
\newcommand{\myname}{Shashank Singh}
\newcommand{\myandrew}{sss1@andrew.cmu.edu}
\newcommand{\myclass}{21-651 General Topology}
\newcommand{\myhwnum}{4}
\newcommand{\duedate}{Wednesday, October 24, 2012}
%%%%%%%%%%%%%%%%%%%%%%%%%%%%%%%%%%%%%%%%%%%%%%%%%%%%%%%%%%%

%%%%%%%%%%%%%%%%%%%%CONTENT MACROS%%%%%%%%%%%%%%%%%%%%%%%%%
\renewcommand{\qed}{\quad $\blacksquare$}
\newcommand{\mqed}{\quad \blacksquare}
\newcommand{\inv}{^{-1}}
\newcommand{\bx}{\mathbf{x}}
\newcommand{\by}{\mathbf{y}}
\newcommand{\bff}{\mathbf{f}}
\newcommand{\bzero}{\mathbf{0}}
\newcommand{\bxi}{\boldsymbol{\xi}}
\newcommand{\boldeta}{\boldsymbol{\eta}}
\newcommand{\B}{\mathcal{B}}
\newcommand{\sminus}{\backslash}
\newcommand{\N}{\mathbb{N}} % natural numbers
\newcommand{\Z}{\mathbb{Z}} % integers
\newcommand{\Q}{\mathbb{Q}} % rational numbers
\newcommand{\R}{\mathbb{R}} % real numbers
\newcommand{\pow}[1]{\mathcal{P}\left(#1\right)} % power set of #1
%%%%%%%%%%%%%%%%%%%%%%%%%%%%%%%%%%%%%%%%%%%%%%%%%%%%%%%%%%%

\begin{document}
\thispagestyle{plain}

{\Large Homework \myhwnum} \\
\myclass \\
Name: \myname \\
Email: \myandrew \\
Due: \duedate \\
\begin{question}{Problem 1}
Let
\[\mathcal{U}
    := \left\{\prod_{i \in \N} U_i : U_i = (1/3,1] \mbox{ or } U_i = [0,1/3)\right\}.
\]
Suppose $x_1,x_2,\ldots \in [0,1]$. Then, if $\forall i \in \N$,
$U_i := (1/3,1]$ if $x_i > 1/3$ and $U_i = [0,1/3)$ otherwise,
$\{x_i\}_{i \in \N} \in \prod_{i \in \N} U_i \in \mathcal{U}$. Since
$(1/3,1]$ and $[0,1/3)$ are both open in the topology induced on $[0,1]$ by
the standard topology on $\R$, $\mathcal{U}$ is a subset of the box topology.
Thus, $\mathcal{U}$ is an open cover of $[0,1]^{\N}$.
Suppose $U := \prod_{i \in \N} U_i \in \mathcal{U}$. $\forall i \in \N$ define
$x_i := 1$ if $U_i = (1/3,1]$ and $x_i := 0$ otherwise. Then, $U$ is the
unique element of $\mathcal{U}$ with $\{x_i\}_{i \in \N} \in U$. Thus, no
proper subset of $\mathcal{U}$ is a cover of $[0,1]^{\N}$.

Clearly, $\mathcal{U}$ is not finite (the function
$f: \mathcal{U} \rightarrow \pow{N}$ including $i$ if and only if
$U_i = (1/3,1]$ is a bijection).
Thus, $\mathcal{U}$ an open cover of $[0,1]^{\N}$ with no finite
subcover, so that $[0,1]^{\N}$ cannot be compact. \qed
\end{question}

\begin{question}{Problem 2}
Since, by definition of $E$,
$\displaystyle
    E \subseteq \prod_{\alpha \in \Lambda} E_{\alpha}$,
$\displaystyle \overline{E}
    \subseteq \overline{\prod_{\alpha \in \Lambda} E_{\alpha}}$.

Suppose $f \in \prod_{\alpha \in \Lambda} E_{\alpha}$, and let $U$ be a
neighborhood of $f$. By Lemma 126, there is some basis element
\[B = \prod_{\alpha \in \Lambda} U_{\alpha} \subseteq U,\]
where $f \in B$, $U_{\alpha} \in \tau_{\alpha}$ and $U_{\alpha} = E_{\alpha}$.
Let $h$ be defined by $h(\alpha) = f(\alpha)$ when $\alpha \in \Lambda_0$, and
$h(\alpha) = g(\alpha)$ when $\alpha \in \Lambda \sminus \Lambda_0$. By
definition of $B$, $h \in B$. Since $f,g \in \prod_{\alpha \in \Lambda} E_{\alpha}$,
$h \in \prod_{\alpha \in \Lambda} E_{\alpha}$, so that, since $\Lambda_0$ is finite, $h \in E$.

Since any neighborhood of $f$ contains elements of $E$, $f \in \overline{E}$.
Thus,
$\displaystyle \prod_{\alpha \in \Lambda} E_{\alpha} \subseteq \overline{E}$,
so that
$\displaystyle \overline{\prod_{\alpha \in \Lambda} E_{\alpha}}
    =         \overline{E}$.

Thus,
\[\overline{E} = \overline{\prod_{\alpha \in \Lambda} E_{\alpha}}.\mqed\]
\end{question}

\newpage
\begin{question}{Problem 3}
\begin{enumerate}[(i)]
\item 
Let $\{(X_{\alpha},\tau_{\alpha})\}_{\alpha \in \Lambda}$ be a nonempty
collection of nonempty Hausdorff topological spaces, and let $\tau$ be the
product topology on $X := \prod_{\alpha \in \Lambda} X_{\alpha}$. Suppose
$x,y \in X$ are distinct. Then, $\exists \alpha \in \Lambda$ such that
$x(\alpha),y(\alpha)$ are distinct. Since $X_{\alpha}$ is Hausdorff, there
exists disjoint $U_x,U_y \in \tau_{\alpha}$ with
$x(\alpha) \in U_x, y(\alpha) \in U_y$. Then, By definition of the product
topology, $\pi_{\alpha}\inv(U_x),\pi_{\alpha}\inv(U_y) \in \tau$. By
definition of the projection function, $x \in \pi_{\alpha}\inv(U_x)$ and
$y \in \pi_{\alpha}\inv(U_y)$, and, since $U_x \cap U_y = \emptyset$,
$\pi_{\alpha}\inv(U_x) \cap \pi_{\alpha}\inv(U_y) = \emptyset$. Therefore,
$(X,\tau)$ is Hausdorff. \qed


\item Let $\{(X_{\alpha},\tau_{\alpha})\}_{\alpha \in \Lambda}$ be a nonempty
collection of nonempty regular topological spaces, and let $\tau$ be the
product topology on $X := \prod_{\alpha \in \Lambda} X_{\alpha}$. Suppose
$C$ is closed in $\tau$, $x \in X \sminus C$. Since $X \sminus C$ is open,
by Lemma 126, there is some some basis element
\[\bigcup_{\alpha \in \Lambda_0} \pi_{\alpha}\inv(U_{\alpha})
 \subseteq X \sminus C,
\]
containing $x$, where $\Lambda_0 \subseteq \Lambda$ is finite and
$U_{\alpha} \in \tau_{\alpha}$. $\forall \alpha \in \Lambda_0$, since
$(X_{\alpha},\tau_{\alpha})$ regular and $X_{\alpha} \sminus U_{\alpha}$ is a
closed set not containing $x(\alpha)$, there exist disjoint
$V_{\alpha},W_{\alpha} \in \tau_{\alpha}$ such that
$X_{\alpha} \sminus U_{\alpha} \subseteq V_{\alpha}$ and $x \in W_{\alpha}$.
Since
$W_{\alpha} \subseteq X_{\alpha} \sminus V_{\alpha} \subseteq U_{\alpha}$ and the
latter is closed, it follows from Proposition 26 that
$\overline{W_{\alpha}}
    \subseteq X_{\alpha} \sminus V_{\alpha}
    \subseteq U_{\alpha}$.
Thus, by part (ii) of Lemma 135,
\[x
    \in \prod_{\alpha \in \Lambda} W_{\alpha}
    \subseteq \overline{\prod_{\alpha \in \Lambda} W_{\alpha}}
    \subseteq \prod_{\alpha \in \Lambda} U_{\alpha}
    \subseteq X \sminus C
\]
(if $\alpha \in \Lambda \sminus \Lambda_0$, define
$W_{\alpha} = U_{\alpha} = X_{\alpha}$). Thus, by Lemma 126,
for $V := X \sminus \overline{\prod_{\alpha \in \Lambda} W_{\alpha}}$,
$W := \prod_{\alpha \in \Lambda} W_{\alpha}$, $V,W \in \tau$, $C \subseteq V$,
$x \in W$, and $V \cap W = \emptyset$. Therefore, $(X,\tau)$ is regular. \qed

\item Let $A := \{(x,-x) : x \in \R\} \subseteq \R^2$. If
$x \in \R^2 \sminus A$, then, clearly $\exists a,b,c,d \in \R$ such that
$x \in [a,b) \times [c,d) \subseteq \R^2 \sminus A$. Thus, since such
half-open rectangles form a base of the Sorgenfrey plane topology $\tau_2$,
$\R^2 \sminus A$ is open, and thus $A$ is closed.

$\forall x \in \R$,
$A \cap [x,x + 1) \times [-x,-x + 1) = \{(x,-x)\}$, so that, since
$[x,x + 1) \times [-x,-x + 1)$ is open in $\tau_2$, the topology induced on
$A$ by $\tau_2$ is discrete.

Clearly, $A$ is equipotent to $\R$ (the function $x \mapsto (x,-x)$ is a
bijection).

Since $\Q^2$ is countable and dense in $(\R,\tau_2)$ (clearly, any non-empty
half-open rectangle contains a point with rational coordinates), $(\R,\tau_2)$
is separable.

Thus, by Lemma 90, $(\R^2,\tau_2)$ is not normal. \qed

\end{enumerate}
\end{question}

\newpage
\begin{question}{Problem 4}
$\forall i,n \in \N$, define $U_{i,n} := (-n,n)$. Then, by definition of the
box topology, $\forall n \in \N$, $\prod_{i \in \N} U_{i,n} \in \tau_B$ (the
box topology on $\R^{\N}$).
Futhermore, $U = \bigcup_{n = 1}^{\infty} \prod_{i \in \N} U_{i,n}$
(since $f \in U$ is bounded if and only if it is bounded by some $n \in \N$),
so that, since $\tau_B$ is closed under arbitrary unions, $U$ is open.

Suppose $f \in \overline{U}$. Let $V := \prod_{i \in \N} (f(i) - 1,f(i) + 1)$.
Since, $\forall i \in \N$, $(f(i) - 1,f(i) + 1)$ is open in $\R$, $V$ is open
in the box topology on $\R^{\N}$. Since $f \in \overline{U}$ and $V$ is a
neighborhood of $f$, $\exists g \in V \cap U$, so that $|g|$ is bounded by
some $b \in \R$. By definition of $V$, then, $|f|$ is bounded above by
$|b| + 1$. Therefore, $f \in U$. Thus, $\overline{U} \subseteq U$, so that by
Proposition 26, $U$ is closed.
\end{question}

\begin{question}{Problem 5}
Since $\gamma$ is bounded, $\gamma(\R)$ is bounded, so that
$\overline{\gamma(\R)}$ is bounded. Since it is a closed and bounded subset of
$\R^2$, $\overline{\gamma(\R)}$ is compact.

Since $\gamma$ is injective, and,
by definition surjective onto $\gamma(\R)$, $\gamma$ is bijective.
Since any open subset $U$ of $\R$ is a union of open intervals and the
topology on $\gamma(\R)$ must be closed under arbitrary unions,
$(\gamma\inv)\inv(U) = \gamma(U)$ is open in $\gamma(\R)$, so that
$\gamma\inv$ is continuous. Thus, since $\gamma$ itself is continuous, by the
result of Exercise 107, $\gamma$ is a homeomorphism from $\R$ to $\gamma(\R)$.

Since $\overline{\gamma(\R)}$ is compact and $\R$ is homeomorphic to
$\gamma(\R)$, which is, by definition, a dense in $\overline{\gamma(\R)}$,
$\overline{\gamma(\R)}$ is a compactification of $\R$. \qed

Let $\gamma: \R \rightarrow \gamma(\R) \subseteq \R^2$ be defined
$\forall x \in \R$ by
\[\gamma(x) :=
    \left\{
        \begin{array}{cl}
            \left(
                r\cos\theta,
                r\sin\theta
            \right) & x \geq 0 \\
            \left(
                -r\cos\theta,
                r\sin\theta
            \right) & x <    0
        \end{array}
    \right.,
    \quad \mbox{where} \quad
r := \frac{|x|}{|x| + 1},
    \quad
\theta := x.
\]
Since 
\[\sqrt{(r\cos\theta)^2 + (r\sin\theta)^2}
    = r\sqrt{(\cos^2\theta + \sin^2\theta)}
    = r\sqrt1
    = r \in (0,1],
\]
$\gamma$ is bounded (with respect to the Euclidean metric). Clearly, $\gamma$
is continuous, since it is piecewise continuous and the pieces agree when
$x = 0$. If, for some $x_1,x_2 \in \R$, $r(x_1) = r(x_2)$, then, since $r$ is
even, strictly increasing on $(0,\infty)$, and strictly decreasing
$(-\infty,0)$, either $x_1 = x_2$, or $x_1 = -x_2$ (and $x_1$ is non-zero). In
the later case, however $\gamma(x_1) = -\gamma(x_2)$. Thus, $\gamma$ is
injective. The inverse of $\gamma$ on $\gamma(\R)$ is the function
$(r\cos\theta,r\sin\theta) \mapsto \theta$, which is also clearly continuous.
Thus, by the above result, $X := \overline{\gamma(\R)}$ under the topology
induced by the standard topology on $\R^2$ is a compactification of $\R$.

Note that
\[X
 = \gamma(\R) \cup (\gamma(\R))^{\prime}
 = \gamma(\R) \cup \{(\cos\theta,\sin\theta) : \theta \in \R\}
\]
Thus, we can extend $\gamma\inv$ continuously to $X \sminus \gamma(\R)$ by
defining
\[\gamma\inv(\cos\theta,\sin\theta) = \theta.\]
Then, $\cos \circ \gamma\inv$ is a composition of continuous functions on $X$,
so that $\cos$ can be extended continuous to $X$. \qed
\end{question}

\begin{question}{Problem 6}
Since $[0,1]$ is compact, it follows from Tikhonov's theorem that $[0,1]^{\N}$
is compact.

Since $[0,1]$ is Hausdorff, it follows from the result of part (i) of problem
3 that $[0,1]^{\N}$ is Hausdorff, and thus $T_1$.

Thus, it remains only to show that $[0,1]^{\N}$ is first countable. Let
$x = x_1,x_2,\ldots \in [0,1]^{\N}$. Since $[0,1]$ is first countable,
$\forall n \in \N$, there exists some countable local base
$\beta_n = \{U_{n,1},U_{n,2},\ldots\}$ at $x_n$.
Let $\beta := \{\bigcap_{\pi_{n}\inv(U_{n,k}) : k,n \in \N}$. Since there is
a clear bijection from $\beta$ to $\N^2$, $\beta$ is countable. Suppose $U$ is
a neighborhood of $x$ in $[0,1]^{\N}$. It is clear from Lemma 126 that $U$
contains some element of $\beta$, and the $\beta$ contains only open sets of
$[0,1]^{\N}$, so that $\beta$ is a countable local base of $[0,1]^{\N}$ at
$x$. Thus, $[0,1]^{\N}$ is first countable.

By Lemma 104, since $[0,1]^{\N}$ is compact, $T_1$, and first countable,
$[0,1]^{\N}$ is sequentially compact. \qed
\end{question}
\end{document}
