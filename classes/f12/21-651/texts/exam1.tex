\documentclass[11pt]{article}
\usepackage{enumerate}
\usepackage{fullpage}
\usepackage{fancyhdr}
\usepackage[margin=0.5in]{geometry}
\usepackage{amsmath, amsfonts, amsthm, amssymb}
\usepackage{color}
\setlength{\parindent}{0pt}
\setlength{\parskip}{5pt plus 1pt}
\pagestyle{empty}

\def\indented#1{\list{}{}\item[]}
\let\indented=\endlist

\newcounter{questionCounter}
\newcounter{partCounter}[questionCounter]
\newenvironment{question}[2][\arabic{questionCounter}]{%
    \setcounter{partCounter}{0}%
    \vspace{.25in} \hrule \vspace{0.5em}%
        \noindent{\bf #2}%
    \vspace{0.8em} \hrule \vspace{.10in}%
    \addtocounter{questionCounter}{1}%
}{}
\renewenvironment{part}[1][\alph{partCounter}]{%
    \addtocounter{partCounter}{1}%
    \vspace{.10in}%
    \begin{indented}%
       {\bf (#1)} %
}{\end{indented}}

%%%%%%%%%%%%%%%%%%%%%%%HEADER%%%%%%%%%%%%%%%%%%%%%%%%%%%%%%
\newcommand{\myname}{Shashank Singh}
\newcommand{\myandrew}{sss1@andrew.cmu.edu}
\newcommand{\myclass}{21-651 General Topology}
\newcommand{\myhwnum}{3}
\newcommand{\duedate}{Wednesday, October 10, 2012}
%%%%%%%%%%%%%%%%%%%%%%%%%%%%%%%%%%%%%%%%%%%%%%%%%%%%%%%%%%%

%%%%%%%%%%%%%%%%%%%%CONTENT MACROS%%%%%%%%%%%%%%%%%%%%%%%%%
\renewcommand{\qed}{\quad $\blacksquare$}
\newcommand{\mqed}{\quad \blacksquare}
\newcommand{\inv}{^{-1}}
\newcommand{\bx}{\mathbf{x}}
\newcommand{\by}{\mathbf{y}}
\newcommand{\bff}{\mathbf{f}}
\newcommand{\bzero}{\mathbf{0}}
\newcommand{\bxi}{\boldsymbol{\xi}}
\newcommand{\boldeta}{\boldsymbol{\eta}}
\newcommand{\B}{\mathcal{B}}
\newcommand{\sminus}{\backslash}
\newcommand{\N}{\mathbb{N}} % natural numbers
\newcommand{\Q}{\mathbb{Q}} % rational numbers
\newcommand{\R}{\mathbb{R}} % real numbers
\newcommand{\pow}[1]{\mathcal{P}\left(#1\right)} % power set of #1
\newcommand{\e}{\varepsilon} % varepsilon
%%%%%%%%%%%%%%%%%%%%%%%%%%%%%%%%%%%%%%%%%%%%%%%%%%%%%%%%%%%

\begin{document}
\thispagestyle{plain}

{\Large Final Exam Note Sheet} \\
\myclass \\
Name: \myname \\
{\bf Finite and Infinite Sets, Orderings:} \\
{\bf Axiom 12: (Axiom of Choice)} For any family of nonempty disjoint sets
$\mathcal{A}$, $\exists C \subseteq \cup \mathcal{A}$ with $|C \cap A| = 1$,
$\forall A \in \mathcal{A}$. \\
{\bf Lemma 13: (Zorn's Lemma)} $(X,\preceq)$ partially ordered set, if every
totally ordered subset of $X$ has a maximal element, then $X$ has an upper
bound. \\
{\bf Topological Spaces:}
and $\tau = \{A \subseteq \R : \forall x \in A, \exists \epsilon > 0,
[x,x + \epsilon) \subseteq A\}$ is the \emph{Sorgenfrey} topology (line). \\
$x \in X$ is an \emph{isolated point} iff $\{x\} \in \tau$. \\
{\bf Basis of Topology:} \\
{\bf Definition 19: (Neighborhood, Basis of a Topology)} $U \in \tau$ is a
neighborhood of $x \in X$ iff $x \in U$. $U \in \tau$ is a neighborhood of
$E \subseteq X$ iff $E \subseteq U$.
A family $\beta \subseteq \pow{X}$ is a
base for the topology $\tau$ iff $\tau$ is the set of arbitrary unions of
elements of $\beta$.
Neighborhoods $\beta_x \subseteq \tau$ of $x$ form a \emph{local} base at
$x \in X$ iff, $\forall U \in \tau$,
$\exists B \in \beta_x$ with $B \subseteq U$. \\
{\bf Proposition 21:} $\beta \subseteq \pow{X}$ is the base of a topology iff
$\emptyset \in \beta$, $X = \cup \beta$, and every finite intersection of
elements in $\beta$ is a union of elements in $\beta$. \\
{\bf Interior and Closure:} \\
{\bf Proposition 24:} $E^\circ \subseteq E$ is open, and is the union of open
subsets of $E$. $E \in \tau$ iff $E = E^\circ$. $(E^\circ)^\circ = E^\circ$.
\\
{\bf Proposition 26:} $(X,d)$ metric space, $E \subseteq X$.
$E \subseteq \overline{E}$, $\overline{E}$ closed. $\overline{E} =
\bigcap_{E \subseteq B} B$. $E$ closed iff $E = \overline{E}$.
$\overline{(\overline{E}}) = \overline{E}$. \\
{\bf Metric Spaces:} \\
A pseudometric is a metric except that it need not differentiate points. \\
If $(X,d)$ is a pesudometric space, $\B := \{B(x,r) : x \in X, r \geq 0\}$ is
the base of a topology on $X$. \\
{\bf Quotient Metric Spaces:} Considering equivalent those points not
differentiated by a pseudometric gives a metric space.
If $(X,d_i)$ is a pseudometric space, $d(x,y) := \sup_i d_i(x,y)$ is a
pseudometric. \\
{\bf Infinite Sums:} \\
The infinite sum $\sum_{x \in X} f(x) := \sup\{\sum_{x \in Y} f(x)
 : Y \subseteq X$ is finite$\}.$ \\
Young's, H\"older's, and Minkowski's inequalities. \\ %TODO
{\bf Sequences:} \\
{\bf Lemma 53:} If $(X,\tau)$ is first countable with $E \subseteq X$, then
$\forall p \in \overline{E}$, there is a sequence in $E$ converging to $p$. \\
{\bf Separability:} \\
$E \subseteq X$ is dense in $X$ iff $\overline{E} = X$ (in a metric space, iff
all $p \in X$ are within $\epsilon$ of some $x \in E$). \\
Every second countable space is separable. \\
Separability is \emph{not} hereditary (it is in metric space, where separable
$\Leftrightarrow$ second countable). \\
{\bf Connectedness:} \\
$(X,\tau)$ is connected iff the only clopen sets are $\emptyset$ and $X$. \\
If $(X,d)$ metric space, $E \subseteq X$ is not connected iff disjoint
$\exists U_1,U_2 \in \tau$ covering $E$ with $E \cap U_1,E \cap U_2$ nonempty.
\\
If $E \subseteq X$ is connected, then $\overline{E}$ is connected. \\
$\gamma: [a,b] \rightarrow X$ continuous iff sequentially continuous.
$E \subseteq X$ pathwise connected iff $\exists$ continuous path between
all $x,y \in E$. \\
The connected components of a closed set are closed. \\
{\bf Continuity:} \\
The $\max$ of two continuous functions is continuous. \\
Continuity always implies sequential continuity. In first countable spaces,
the converse also holds. \\
Continuous functions preserve connectedness (and compactness). \\
{\bf Axioms of Separation:} \\
$T_3$ iff $T_0$ and regular. $T_{3\frac{1}{2}}$ {\bf Tikhonov} iff $T_0$ and
completely regular. $T_4$ iff $T_1$ and normal. \\
If $T_1$, then $\{x\}$ is closed, $\forall x \in X$. \\
$T_3 \Rightarrow T_2$. $T_0$ and normal does not imply $T_4$. \\
Counterexamples: $\tau_l,\tau_r$ are $T_0$ but not $T_1$. Sorgenfrey line is
normal, but Sorgenfrey plane is not. \\
{\bf Compactness:} \\
$E \subseteq X$ is relatively compact (precompact) iff $\overline{E}$ is
compact. \\
$(X,\tau)$ $T_2$ (Hausdorff), $K \subseteq X$ compact, $x_0 \in X \sminus K$.
Then, $\exists U_1,U_2 \in \tau$ with $K \subseteq K$, $x_0 \in V$. \\
$X$ compact, $C \subseteq X$ closed implies $C$ compact. \\
If $X$ Hausdorff, then $K \subseteq X$ compact implies $K$ closed, and $X$
compact implies $X$ regular and normal (recall that normal does not always imply
regular). \\
The trivial topology has sets compact but not closed. \\
$(X,d)$ compact metric space implies $X$ separable, bounded (if not
separable, $\exists$ countable $E \subseteq X$ of isolated points).
$\{E_{\alpha}\}$ has finite-$\cap$ property iff any finite subfamily
has nonempty intersection. $X$ compact iff any family of closed sets with
finite-$\cap$ property has nonempty intersection.\\
$X$ compact implies every infinite subset of $X$ has a limit point in $X$. \\
$X$ compact, $T_1$, and first countable implies $X$ sequentially compact. \\
A bijection $f: X \rightarrow Y$ is a homeomorphism iff
$\tau_y = \{f(U) : U \in \tau\}$. \\
$f : X \rightarrow Y$ continuous bijection, $X$ compact, $Y$ $T_2$ implies $f$
homeomorphism.

{\bf Lower Semicontinuity:} $f\inv((a,\infty)) \in \tau$ ($\forall x_0 \in X$,
$\exists U \in \tau$ with $x_0 \in U$, $f(x) \geq f(x_0) + \e$). \\
{\bf Upper Semicontinuity:} $f\inv((a,\infty)) \in \tau$ ($\forall x_0 \in X$,
$\exists U \in \tau$ with $x_0 \in U$, $f(x) \geq f(x_0) + \e$).

{\bf Urysohn's Lemma:} $(X,\tau)$ is normal if and only if $(X,\tau)$ is
completely normal. \\
{\bf Lemma 155:} $(X,\tau)$ loc. compact, Hausdorff $\Rightarrow$
$\forall U \in \tau$, $K \subseteq U$ compact, $\exists V \in \tau$,
$K \subseteq V \subseteq \overline{V} \subseteq U$, $\overline{V}$ compact. \\
{\bf Tietze's Extension Theorem:}  $(X,\tau)$ is normal if and only if,
$\forall C$ closed, $f : C \rightarrow \R$ continuous, $\exists F : X
\rightarrow \R$ continuous with $F = f$ on $C$. Furthermore, any bound on $f$
bounds $F$.

{\bf Partitions of Unity:} \\
{\bf Definition 168:}
$\mathcal{F}$ \emph{point finite} iff, $\forall x \in X$, $x$ is in only
finitely many $U \in F$. \\
$\mathcal{F}$ \emph{locally finite} iff, $\forall x \in X$, $x$ has a
neighborhood intersecting only finitely many $U \in F$. \\
$\mathcal{F}$ \emph{$\sigma$-locally finite} iff $\mathcal{F} =
\bigcup_{n = 1}^{\infty} \mathcal{F}_n$, where each $\mathcal{F}_n$ is locally
finite. \\
{\bf Theorem 171:} $(X,\tau)$ separable and $\mathcal{F} \subseteq \tau$ covers
$X$ $\Rightarrow$ $\exists$ locally finite partition of unity subordinated to
$\mathcal{F}$.
{\bf Theorem 172:} $(X,\tau)$ normal and $\mathcal{F} \subseteq \tau$ point
finite cover of $X$ $\Rightarrow$ $\exists$ partition of unity subordinated to
$\mathcal{F}$.

{\bf Metrization:} \\
{\bf Urysohn Metrization Theorem:} $(X,\tau)$ is metrizable and separable
$\Leftrightarrow$ $(X,\tau)$ is $T_4$ and second countable. \\
{\bf Nagata-Smirnov Metrization Theorem:} $(X,\tau)$ is metrizable
$\Leftrightarrow$ $(X,\tau)$ is $T_4$ and has a $\sigma$-locally finite base.

{\bf Baire Spaces:} \\
{\bf Definition 189:} $(X,\tau)$ is a \emph{Baire space} iff, for any countable
family $\mathcal{U} \subseteq \tau$ of dense sets, $\cap U$ is dense. \\
{\bf Definition 190:} $E \subseteq X$ is \emph{nowhere dense} iff
$\overline{E^\circ} = \emptyset$ and \emph{meager} iff it is a countable union
of nowhere dense sets. \\
{\bf Exercise 191:} $(X,\tau)$ is Baire iff, $\forall E \subseteq X$ meager,
$E^\circ = \emptyset$. If $E$ is meager, $X \sminus M$ is dense. \\
{\bf Baire Category Theorem:} Any complete metric space is Baire. \\
{\bf Corollary 194:} In any complete metric space, a countable intersection of
dense $G_{\delta}$ sets is itself a dense $G_{\delta}$ set. \\
{\bf Second Baire Category Theorem:} All locally compact Hausdorff spaces are
Baire.

{\bf Uniform Continuity:} \\
{\bf Definition 202:} $\omega : [0,\infty) \rightarrow [0,\infty]$ is a modulus
of continuity for $f : X \rightarrow Y$ iff $\omega(0) = 0$,
$\lim_{s \rightarrow 0} \omega(s) = 0$, and, $\forall x,y \in X,
d_Y(f(x),f(y)) \leq \omega(d_X(x,y))$. \\
{\bf Remark 205:} Lipschitz continuity is strictly stronger than H{o}lder
continuity. \\
{\bf Proposition 206:} $(X,d_X)$ metric space, $(Y,d_Y)$ complete metric space,
$f : E \subseteq X \rightarrow Y$ uniformly continuous. Then, $f$ can be
extended uniquely to $\overline{E}$.

{\bf Fixed Point Theorems:} \\
{\bf Banach Contraction Principle:} If $(X,d)$ is a metric space, and
$f : X \rightarrow X$ is a contraction ($\exists c \in (0,1)$ such that,
$\forall x,y \in X$, $d(f(x),f(y)) \leq cd(x,y)$), then $f$ has a unique fixed
point. \\
{\bf Brouwer Fixed Point Theorem:} If $E \subseteq \R^n$ is homeomorphic to
$\overline{B(\bzero,1)}$ and $f : E \rightarrow E$, then $f$ has a fixed
point.

{\bf Definition 212:} $(X,d)$ is \emph{totally bounded} iff, any cover of $X$ by
balls of radius $\epsilon$ has a finite subcover. $E \subseteq X$ is
\emph{bounded} iff $\sup \{d(x,y) : x,y, \in E\} < \infty$.

In a metric space, compactness, sequential compactness, and (complete +
bounded) are equivalent.

{\bf Stone Theorem:} If $(X,d)$ is a compact metric space and $\mathcal{F}
\subseteq C(X)$, if $\mathcal{F}$ separates points, contains all constant
functions, and is closed under addition, multiplication, and scalar
multiplication, the $\mathcal{F}$ is dense in $X$.

{\bf Ascoli-Arzela:} $(X,d)$ is a separable and $\mathcal{F} \subseteq
C(X)$. If $\mathcal{F}$ is pointwise bounded and equicontinuous, then every
sequence in $\mathcal{F}$ has a subsequence converging uniformly on every
compact $E \subseteq X$ to a a continuous function.
\end{document}
