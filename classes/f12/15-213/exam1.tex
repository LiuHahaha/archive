\documentclass[11pt]{article}
\usepackage{enumerate}
\usepackage{fullpage}
\usepackage{fancyhdr}
\usepackage[margin=0.8in]{geometry}
\usepackage{amsmath, amsfonts, amsthm, amssymb}
\usepackage{color}
\usepackage{titlesec}
\titleformat{\section}{\bfseries}{\thesection}{1em}{}
\setlength{\parindent}{0pt}
\setlength{\parskip}{5pt plus 1pt}
\pagestyle{empty}

\def\indented#1{\list{}{}\item[]}
\let\indented=\endlist

\newcounter{questionCounter}
\newcounter{partCounter}[questionCounter]
\newenvironment{question}[2][\arabic{questionCounter}]{%
    \setcounter{partCounter}{0}%
    \vspace{.25in} \hrule \vspace{0.5em}%
        \noindent{\bf #2}%
    \vspace{0.8em} \hrule \vspace{.10in}%
    \addtocounter{questionCounter}{1}%
}{}
\renewenvironment{part}[1][\alph{partCounter}]{%
    \addtocounter{partCounter}{1}%
    \vspace{.10in}%
    \begin{indented}%
       {\bf (#1)} %
}{\end{indented}}

%%%%%%%%%%%%%%%%%%%%%%%HEADER%%%%%%%%%%%%%%%%%%%%%%%%%%%%%%
\newcommand{\myname}{Shashank Singh}
\newcommand{\myandrew}{sss1@andrew.cmu.edu}
\newcommand{\myclass}{15-213 Introduction to Computer Systems}
\newcommand{\myhwnum}{3}
\newcommand{\duedate}{Wednesday, October 10, 2012}
%%%%%%%%%%%%%%%%%%%%%%%%%%%%%%%%%%%%%%%%%%%%%%%%%%%%%%%%%%%

%%%%%%%%%%%%%%%%%%%%CONTENT MACROS%%%%%%%%%%%%%%%%%%%%%%%%%
\renewcommand{\qed}{\quad $\blacksquare$}
\newcommand{\mqed}{\quad \blacksquare}
\newcommand{\inv}{^{-1}}
\newcommand{\bx}{\mathbf{x}}
\newcommand{\by}{\mathbf{y}}
\newcommand{\bff}{\mathbf{f}}
\newcommand{\bzero}{\mathbf{0}}
\newcommand{\bxi}{\boldsymbol{\xi}}
\newcommand{\boldeta}{\boldsymbol{\eta}}
\newcommand{\B}{\mathcal{B}}
\newcommand{\sminus}{\backslash}
\newcommand{\N}{\mathbb{N}} % natural numbers
\newcommand{\Q}{\mathbb{Q}} % rational numbers
\newcommand{\R}{\mathbb{R}} % real numbers
\newcommand{\pow}[1]{\mathcal{P}\left(#1\right)} % power set of #1
%%%%%%%%%%%%%%%%%%%%%%%%%%%%%%%%%%%%%%%%%%%%%%%%%%%%%%%%%%%

\begin{document}
\thispagestyle{plain}

{\Large Midterm Note Sheet} \\
\myclass \\
Name: \myname \\

\section{Data Representation}
\begin{tabular}{|r|c|c|c|}
\hline
Type        & Size (32/64-bit) & Assembly suffix & Alignment (L32/L64/W32/W64) \\
\hline
char        &  1               & b               & 1                           \\
\hline
short       &  2               & w               & 2                           \\
\hline
int         &  4               & l               & 4                           \\
\hline
long        &  4/8             & q               & 4/8/4/8                     \\
\hline
long long   &  8               & q               & 4/8/8/8                     \\
\hline
void *      &  4/8             & q               & 4/8/4/8                     \\
\hline
float       &  4               & s               & 4                           \\
\hline
double      &  8               & d               & 4/8/8/8                     \\
\hline
long double &  12/16/10/8      & d               & 4/16/10/8                   \\
\hline
\end{tabular} \\
On 32-bit (x86) Linux, chars aligned to 1, shorts aligned to 2, all other data types
aligned to 4. \\
On 64-bit (x86-64) Linux all $k$ byte data is aligned to multiples of $k$. \\
On Windows, all $k$ byte data is aligned to multiples of $k$. \\
When figuring out struct element alignment, biggest to smallest always works.
All x86 systems (including x86-64 systems) use little endian format.
\section{Assembly Language}
\texttt{const(b,l,step\_size) = const + b + (step\_size * l)}. \\
\begin{tabular}{|c|c|c|c|c|c|}
\hline
64-bit & 32-bit & 16-bit & High-order 8-bit & Low-order 8-bit & Purpose      \\
\hline
\%rax  & \%eax  & \%ax   & \%ah             & \%al            & Return value \\
\hline
\%rbx  & \%ebx  & \%bx   & \%bh             & \%bl            & Callee Saved \\
\hline
\%rcx  & \%ecx  & \%cx   & \%ch             & \%cl            & Argument 4   \\
\hline
\%rdx  & \%edx  & \%dx   & \%dh             & \%dl            & Argument 3   \\
\hline
\%rsi  & \%esi  & \%si   &                  & \%sil           & Argument 2   \\
\hline
\%rdi  & \%edi  & \%di   &                  & \%dil           & Argument 1   \\
\hline
\%rbp  & \%ebp  & \%bp   &                  & \%bpl           & Callee Saved \\
\hline
\%rsp  & \%esp  & \%sp   &                  & \%spl           & Stack Pointer\\
\hline
\%r8   & \%r8d  & \%r8w  &                  & \%r8b           & Argument 5   \\
\hline
\%r9   & \%r9d  & \%r9w  &                  & \%r9b           & Argument 6   \\
\hline
\%r10  & \%r10d & \%r10w &                  & \%r10b          & Caller Saved \\
\hline
\%r11  & \%r11d & \%r11w &                  & \%r11b          & Caller Saved \\
\hline
\%r12  & \%r12d & \%r12w &                  & \%r12b          & Callee Saved \\
\hline
\%r13  & \%r13d & \%r13w &                  & \%r13b          & Callee Saved \\
\hline
\%r14  & \%r14d & \%r14w &                  & \%r14b          & Callee Saved \\
\hline
\%r15  & \%r15d & \%r15w &                  & \%r15b          & Callee Saved \\
\hline
\end{tabular}

\begin{tabular}{|c|c|c|}
Flag & cmp a, b                     & test a,b              \\
ZF   & $b - a == 0$                 & $a \& b == 0$         \\
SF   & $b - a <  0$                 & sign bit of $a \& b$  \\
OF   & signed overflow in $b - a$   & 0                     \\
CF   & unsigned overflow in $b - a$ & 0                     \\
\end{tabular}<++>

In the below table, $\beta,\gamma$ are either b (1 byte), w (2 bytes),
l (4 bytes), or q (8 bytes). \\
\begin{tabular}{|c|c|}
\hline
Instrucion             & Description                                                        \\
\hline
mov $\beta$       S, D & D $\leftarrow$ S                                                   \\
movs$\beta\gamma$ S, D & D $\leftarrow$ S with sign extension                               \\
movz$\beta\gamma$ S, D & D $\leftarrow$ S with zero extension                               \\
\hline
push$\beta$       S    & push value of S into stack at \%esp; decrement \%esp by sizeof(S)  \\
push$\beta$          D & pop value from stack at \%esp into S; increment \%esp by sizeof(S) \\
\hline
lea $\beta$       S, D & D $\leftarrow$ \&S                                                 \\
\hline
inc $\beta$          D & D $\leftarrow$ D + 1                                               \\
dec $\beta$          D & D $\leftarrow$ D - 1                                               \\
neg $\beta$          D & D $\leftarrow$ -D                                                  \\
not $\beta$          D & D $\leftarrow \sim$ D                                              \\
\hline
add $\beta$       S, D & D $\leftarrow$ D + S                                               \\
sub $\beta$       S, D & D $\leftarrow$ D - S                                               \\
imul$\beta$       S, D & D $\leftarrow$ D $\times$ S                                        \\
xor $\beta$       S, D & D $\leftarrow$ D \^ S                                              \\
or  $\beta$       S, D & D $\leftarrow$ D | S                                               \\
and $\beta$       S, D & D $\leftarrow$ D \& S                                              \\
\hline
sal $\beta$          D & D $\leftarrow$ D $<<$ k                                            \\
shl $\beta$          D & D $\leftarrow$ D $<<$ k (identical to sal)                         \\
sar $\beta$          D & D $\leftarrow$ D $>>$ k (arithmetic; fills with sign)              \\
shr $\beta$          D & D $\leftarrow$ D $>>$ k (logical; fills with zeros)                \\
\hline
cmp $\beta$       a, b & See flags set below                                                \\
test$\beta$       a, b & See flags set below                                                \\
\hline
\end{tabular}
\begin{tabular}{|c|}
\vdots          \\
\hline
arg3            \\
\hline
arg2            \\
\hline
arg1            \\
\hline
return address  \\
\hline
old \%ebp       \\
\hline
local variables \\
\hline
\vdots          \\
\hline
arg3            \\
\hline
arg2            \\
\hline
arg1            \\
\hline
return address  \\
\hline
old \%ebp       \\
\hline
local variables \\
\hline
\vdots          \\
\end{tabular}
\section{Casting}
\begin{verbatim}
unsigned int a = 0xFFFFFFFF;
signed short b = -1;
x = (b > a);
\end{verbatim}
Extended b to int first, then cast b to signed, so
\texttt{b -> 0xFFFFFFFF}.
\section{Caching}
Address = 
\begin{tabular}{|c|c|c|}
\hline
$t$ tag bits & $s$ set bits & $b$ block bits \\
\hline
\end{tabular} \\
$S = 2^s$ is the number of sets
$E$ (associativitiy) is the number of lines (blocks) per set.   \\
$B = 2^b$ (block size) is the number of bytes per block         \\
The total cache capacity in bytes is $C = S \times E \times B$  \\
A Direct-Mapped cache has $E = 1$. \\
A Fully Associative cache has $S = 1, E = C/B$. \\
Array accesses are row-major. \\
\end{document}
