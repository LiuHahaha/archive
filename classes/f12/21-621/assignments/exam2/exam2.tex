\documentclass[11pt]{article}
\usepackage{enumerate}
\usepackage{fullpage}
\usepackage{fancyhdr}
\usepackage{amsmath, amsfonts, amsthm, amssymb}
\usepackage{color}
\setlength{\parindent}{0pt}
\setlength{\parskip}{5pt plus 1pt}
\pagestyle{empty}

\def\indented#1{\list{}{}\item[]}
\let\indented=\endlist

\newcounter{questionCounter}
\newcounter{partCounter}[questionCounter]
\newenvironment{question}[2][\arabic{questionCounter}]{%
    \setcounter{partCounter}{0}%
    \vspace{.25in} \hrule \vspace{0.5em}%
        \noindent{\bf #2}%
    \vspace{0.8em} \hrule \vspace{.10in}%
    \addtocounter{questionCounter}{1}%
}{}
\renewenvironment{part}[1][\alph{partCounter}]{%
    \addtocounter{partCounter}{1}%
    \vspace{.10in}%
    \begin{indented}%
       {\bf (#1)} %
}{\end{indented}}

%%%%%%%%%%%%%%%%%%%%%%%HEADER%%%%%%%%%%%%%%%%%%%%%%%%%%%%%%
\newcommand{\myname}{Shashank Singh}
\newcommand{\myandrew}{sss1@andrew.cmu.edu}
\newcommand{\myclass}{21-621 Introduction to Lebesgue Integration}
\newcommand{\myhwnum}{2}
\newcommand{\duedate}{Friday, December 14, 2012}
\newcommand{\problemlist}{Page 93, Problems 19, 22, Page 260, Problem 6}
%%%%%%%%%%%%%%%%%%%%%%%%%%%%%%%%%%%%%%%%%%%%%%%%%%%%%%%%%%%

%%%%%%%%%%%%%%%%%%%%CONTENT MACROS%%%%%%%%%%%%%%%%%%%%%%%%%
\renewcommand{\qed}{\quad $\blacksquare$}
\newcommand{\mqed}{\quad \blacksquare}
\newcommand{\inv}{^{-1}}
\newcommand{\bx}{\mathbf{x}}
\newcommand{\by}{\mathbf{y}}
\newcommand{\bff}{\mathbf{f}}
\newcommand{\bzero}{\mathbf{0}}
\newcommand{\bxi}{\boldsymbol{\xi}}
\newcommand{\boldeta}{\boldsymbol{\eta}}
\newcommand{\sminus}{\backslash}
\newcommand{\N}{\mathbb{N}} % natural numbers
\newcommand{\Q}{\mathbb{Q}} % rational numbers
\newcommand{\R}{\mathbb{R}} % real numbers
\newcommand{\B}{\mathcal{B}}
\renewcommand{\'}{^{\prime}}
%%%%%%%%%%%%%%%%%%%%%%%%%%%%%%%%%%%%%%%%%%%%%%%%%%%%%%%%%%%

\begin{document}
\thispagestyle{plain}

{\Large Exam \myhwnum} \\
\myclass \\
Name: \myname \\
Email: \myandrew \\
Due: \duedate \\
\problemlist
\begin{question}{Page 93, Problem 19}
By Fubini's Theorem,
\[
\int_0^{\infty} m(E_{\alpha}) \, d\alpha
 = \int_0^{\infty} \int_{\R^d} \chi_{E_{\alpha}}  \, dx \, d\alpha
 = \int_{\R^d} \int_0^{\infty} \chi_{E_{\alpha}} \, d\alpha \, dx
 = \int_{\R^d} \int_0^{|f(x)|} 1                 \, d\alpha \, dx
 = \int_{\R^d} |f(x)|                                       \, dx
.\mqed
\]
\end{question}

\begin{question}{Page 93, Problem 22}
By Translation Invariance of the Lebesgue Integral over $\R^d$,
\begin{align*}
\hat{f}(\xi)
 & = \int_{\R^d} f\left( x - \frac{\xi}{2\|\xi\|^2} \right) e^{-2\pi i \left( x -
\frac{\xi}{2\|\xi\|^2} \right) \cdot \xi} \, dx \\
 & = \int_{\R^d} f\left( x - \frac{\xi}{2\|\xi\|^2} \right) e^{-2\pi ix \cdot
\xi} e^{-2\pi i\frac{\xi \cdot \xi}{2\|\xi\|^2}} \, dx \\
 & = \int_{\R^d} f\left( x - \frac{\xi}{2\|\xi\|^2} \right) e^{-2\pi ix \cdot
\xi} e^{-\pi i} \, dx \\
 & = - \int_{\R^d} f\left( x - \frac{\xi}{2\|\xi\|^2} \right) e^{-2\pi ix \cdot
\xi} \, dx. & (\mbox{since } e^{-\pi i} = -1)
\end{align*}

Adding $\hat{f}(\xi)$, dividing by $2$, and taking absolute values on both
sides gives, by the Triangle Inequality and then by Monotonicity of the
Lebesgue Integral,
\begin{align*}
|\hat{f}(\xi)|
 & =    \frac12 \left| \int_{\R^d} f(x) e^{-2\pi ix \cdot \xi} \, dx
                  - \int_{\R^d} f\left( x - \frac{\xi}{2\|\xi\|^2} \right)
                                        e^{-2\pi ix \cdot \xi} \, dx \right| \\
 & =    \frac12 \left| \int_{\R^d}
            \left( f(x) - f\left( x - \frac{\xi}{2\|\xi\|^2} \right) \right)
                                        e^{-2\pi ix \cdot \xi} \, dx \right|
         & \mbox{Linearity} \\
 & \leq \frac12 \int_{\R^d} \left| 
            \left( f(x) - f\left( x - \frac{\xi}{2\|\xi\|^2} \right) \right)
                                        e^{-2\pi ix \cdot \xi} \right| \, dx
         & \mbox{Triangle Inequality} \\
 & \leq \frac12 \int_{\R^d} \left| 
            f(x) - f\left( x - \frac{\xi}{2\|\xi\|^2} \right)
                                                               \right| \, dx
         & \mbox{Monotonicity} \\
 & = \frac12 \left\| f(x) - f\left( x - \frac{\xi}{2\|\xi\|^2} \right)
                                                        \right\|_{L^2(\R^d)}
   \rightarrow 0
\end{align*}
as $\xi \rightarrow \infty$, by Propositon 2.5 (since
$\frac{\xi}{2\|\xi\|^2} \rightarrow 0$). \qed
\end{question}

%TODO
\begin{question}{Page 260, Problem 6}
\begin{enumerate}[(a)]
\item Define $E_{\alpha} := |f|\inv((\frac{\alpha}{2},\infty))$, $F_{\alpha} :=
(f^*)\inv((\alpha,\infty))$.
By regularity of the Lebesgue Measure, it suffices to show that, for any
compact $K \subseteq F_{\alpha}$, $m(K) \leq \frac{2A}{\alpha}
\int_{E_{\alpha}} |f|$.

By definition of $f^*$, $\forall x \in F_{\alpha}$, $\exists$ a ball $B_x$ with
$x \in B_x$ and $m(B_x) \leq \frac{1}{\alpha} \int_{B_x} |f|$.

Since $K$ is compact, there exists a finite subcover $\B$ of $K$ of such balls.
Then, by the version of the Vitali Covering Lemma shown in class, there is a
subset $\B\' \subseteq \B$ with
\[m(K)
 \leq m\left( \bigcup_{B \in \B} B \right)
 \leq A \sum_{B \in \B\'} m(B).\]

Then, by additivity of the Lebesgue Integral over sets and the definition of
$E_{\alpha}$, $\forall B \in \B\'$,
\[
m(B)
 \leq \frac{1}{\alpha} \left( \int_{B \sminus E_{\alpha}} |f|
                                     + \int_{B \cap E_{\alpha}} |f| \right)
 \leq \frac{1}{\alpha} \left( \frac{\alpha}{2} m(B)
                                     + \int_{B \cap E_{\alpha}} |f| \right).
\]
Isolating $m(B)$ then gives $m(B) \leq \frac{2}{\alpha}
\int_{B \cap E_{\alpha}} |f|$. Thus, since the balls in $\B\'$ are disjoint,
\[
m(K)
 \leq A \sum_{B \in \B\'} m(B)
 \leq A\sum_{B \in \B\'} \frac{2}{\alpha} \int_{B \cap E_{\alpha}} |f|
 \leq \frac{2A}{\alpha} \int_{E_{\alpha}} |f|. \mqed
\]

\item By Fubini's Theorem,
\begin{align*}
\int_{\R^d} (f^*)^2
 & = \int_{\R^d} \int_{(f^*)^2}^{\infty} 1
   = \int_{\R^d} \int_0^{\infty} \chi_{((f^*)^2)\inv((y,\infty))} \, dy \\
 & = \int_0^{\infty} \int_{\R^d} \chi_{((f^*)^2)\inv((y,\infty))} \, dy \\
 & = \int_0^{\infty} m(E_{\sqrt{y}}) \, dy
   = 2\int_0^{\infty} \alpha m(E_{\alpha}) \, d\alpha,
\end{align*}
where the last equality follows from changing variables to $\alpha = \sqrt{y}$.
\qed

\item Combining the results of parts (a) and (b) gives
\[\|f^*\|_{L^2(\R^d)}\int_{\R^d} (f^*)^2
 \leq 2\int_0^{\infty} 2A \int_{|f|\inv(\frac{\alpha}{2},\infty)} |f|
                                                                    \, d\alpha
 \leq 4A \int_{\R} f^2
 = C \|f\|_{L^2(\R^d)},
\]
where $C = 4A$. Since $f \in L^2$, it follows immediately that $f^* \in L^2$.
\qed
\end{enumerate}
\end{question}
\end{document}
