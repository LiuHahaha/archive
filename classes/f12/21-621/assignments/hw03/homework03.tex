\documentclass[11pt]{article}
\usepackage{enumerate}
\usepackage{fullpage}
\usepackage{fancyhdr}
\usepackage{amsmath, amsfonts, amsthm, amssymb}
\usepackage{color}
\setlength{\parindent}{0pt}
\setlength{\parskip}{5pt plus 1pt}
\pagestyle{empty}

\def\indented#1{\list{}{}\item[]}
\let\indented=\endlist

\newcounter{questionCounter}
\newcounter{partCounter}[questionCounter]
\newenvironment{question}[2][\arabic{questionCounter}]{%
    \setcounter{partCounter}{0}%
    \vspace{.25in} \hrule \vspace{0.5em}%
        \noindent{\bf #2}%
    \vspace{0.8em} \hrule \vspace{.10in}%
    \addtocounter{questionCounter}{1}%
}{}
\renewenvironment{part}[1][\alph{partCounter}]{%
    \addtocounter{partCounter}{1}%
    \vspace{.10in}%
    \begin{indented}%
       {\bf (#1)} %
}{\end{indented}}

%%%%%%%%%%%%%%%%%%%%%%%HEADER%%%%%%%%%%%%%%%%%%%%%%%%%%%%%%%%%%%%%%%%%%
\newcommand{\myname}{Shashank Singh}
\newcommand{\myandrew}{sss1@andrew.cmu.edu}
\newcommand{\myclass}{21-621 Introduction to Lebesgue Integration}
\newcommand{\myhwnum}{3}
\newcommand{\duedate}{Wednesday, November 28, 2012}
\newcommand{\problemlist}{Chapter 2, Page 89, Exercises 1,2,4,6,8,9}
%%%%%%%%%%%%%%%%%%%%%%%%%%%%%%%%%%%%%%%%%%%%%%%%%%%%%%%%%%%%%%%%%%%%%%%

%%%%%%%%%%%%%%%%%%%%CONTENT MACROS%%%%%%%%%%%%%%%%%%%%%%%%%%%%%%%%%%%%%
\renewcommand{\qed}{\quad $\blacksquare$}
\newcommand{\mqed}{\quad \blacksquare}
\newcommand{\inv}{^{-1}}
\newcommand{\bx}{\mathbf{x}}
\newcommand{\by}{\mathbf{y}}
\newcommand{\bff}{\mathbf{f}}
\newcommand{\bzero}{\mathbf{0}}
\newcommand{\bxi}{\boldsymbol{\xi}}
\newcommand{\boldeta}{\boldsymbol{\eta}}
\newcommand{\sminus}{\backslash}
\newcommand{\N}{\mathbb{N}} % natural numbers
\newcommand{\Q}{\mathbb{Q}} % rational numbers
\newcommand{\R}{\mathbb{R}} % real numbers
\newcommand{\pow}[1]{\mathcal{P}\left(#1\right)} % power set of #1
\newcommand{\symdif}{\; \triangle \;} % symmetric set difference
\newcommand{\diam}[1]{\operatorname{diam} \left( #1 \right)} % diameter of set #1
%%%%%%%%%%%%%%%%%%%%%%%%%%%%%%%%%%%%%%%%%%%%%%%%%%%%%%%%%%%%%%%%%%%%%%%

\begin{document}
\thispagestyle{plain}

{\Large Homework \myhwnum} \\
\myclass \\
Name: \myname \\
Email: \myandrew \\
Due: \duedate \\
\problemlist
\begin{question}{Exercise 1}
\newcommand{\F}{\mathcal{F}}
Let $\F := \{F_1,F_2,\ldots,F_n\}$. Since
$|\pow{\F}| = 2^n$, we can enumerate the subsets $\F_0,\F_1,\ldots,\F_N$, of
$\pow{\F}$, with $N = 2^n - 1$, and, without loss of generality,
$\F_0 = \emptyset$. Then, for $i \in \{1,2,\ldots,N\}$, define
\[F^*_i
 := \left( \bigcap_{F \in \F_i} F \right)
        \left\sminus \left( \bigcup_{F \notin \F_i} F \right) \right.
\]
(so that $F^*_i$ is the set of elements in those sets belonging to $\F_i$ and
not in sets not belonging to $\F_i$).

By construction, for distinct $i,j \in \{1,2,\ldots,N\}$, there is some $F_k$
such that $F^*_i \subseteq F_k$ and $F^*_j \subseteq F^c_k$ (or vice versa),
so that $F^*_i$ and $F^*_j$ are disjoint, and thus the collection $\{F^*_j\}$
is disjoint, as desired.

For any $k \in \{1,2,\ldots,n\}$, $\forall x \in F_k$,
$\exists j \in \{1,2,\ldots,N\}$ such that $\F^*_j = \{F \in \F : x \in F\}$,
since there is an $F^*_j$ for every nonempty subset of $\pow{\F}$ (and $x$
must be in \emph{some} $F_i$). Furthermore, by construction of $F^*_j$,
$F^*_j \subseteq F_k$ (since, otherwise, $F^*_j \subseteq F^c_j$, which cannot
be the case, since $x \in F^*_j$). Thus,
$F_k \subseteq \bigcup_{F^*_j \subseteq F_k} F^*_j$. That
$F_k \supseteq \bigcup_{F^*_j \subseteq F_k} F^*_j$ is immediate from the
definition of the latter set.

Since $\bigcup_{j = 1}^{N} F^*_j \subseteq \bigcup_{i = 1}^n F_i$, it follows
that $\bigcup_{j = 1}^{N} F^*_j = \bigcup_{i = 1}^n F_i$, as desired. \qed
\end{question}

\begin{question}{Exercise 2}
It follows from the density of continuous, compactly supported functions in
$L^1(\R^d)$ that there is a continuous function $g \in L^1(\R^d)$ with compact
support $G \subseteq \R^d$, such that
$\|g - f\|_{L^1} < \frac{\epsilon}{4} < \frac{\epsilon}{3}$.

Let $f_{\delta}$ and $g_{\delta}$ be the functions taking $x \in \R^d$ to
$f(\delta x)$ and $g(\delta x)$, respectively.

Then, $\forall \delta > 0$, by dilation invariance of the Lebesgue Integral,
\[
\|f_{\delta} - g_{\delta}\|_{L^1}
 = \int_{\R^d} |f(\delta x) - g(\delta x)| \; dx
 = \frac{1}{\delta^d} \int_{\R^d} |f(x) - g(x)| \; dx
 = \frac{1}{\delta^d} \|g - f\|_{L^1}
 < \frac{\epsilon}{4\delta^d}.
\]

Thus, for some $\delta_1 \in (0,1)$, $\forall \delta > \delta_1$,
$\|f_{\delta} - g_{\delta}\|_{L^1} < \frac{\epsilon}{3}$.

Let $G^{\prime} := G \cup \frac{G}{\delta}$ ($\frac{G}{\delta}$ is the
dilation of $G$ by $\frac{1}{\delta}$). Since $g$ is continuous and
$G^{\prime}$ is compact, $g$ is uniformly continuous on $G^{\prime}$. Thus,
$\exists \delta_0 > 0$ such that, whenever $\|x_1 - x_2\| < \delta_0$,
$|g(x_1) - g(x_2)| < \frac{\epsilon}{3m(G^{\prime})}$
($m(G^{\prime}) < +\infty$, since $G^{\prime}$ is compact).
Since $G^{\prime}$ is compact, $R := \diam{G^{\prime}} < +\infty$. Then, for
$\delta \in \left( 1 - \frac{\delta_0}{R}, 1 + \frac{\delta_0}{R} \right)$,
$\forall x \in G^{\prime}$,
since
$\|x - \delta x\|
 = |1 - \delta|\|x\|
 < |1 - \delta| R
 < \frac{\delta_0}{R}R
 = \delta_0$,
\[
|g_\delta(x) - g(x)|
 = |g(\delta x) - g(x)|
 < \frac{\epsilon}{3m(G^{\prime})},
\]
and thus, since $g_\delta(x) = g(x) = 0$ outside of $G^{\prime}$,
\[
\|g_\delta - g\|_{L^1}
 = \int_{\R^d} |g_{\delta}(x) - g(x)| \; dx
 = \int_{G^{\prime}} |g_{\delta}(x) - g(x)| \; dx
 < \int_{G^{\prime}} \frac{\epsilon}{3m(G^{\prime})} \; dx
 = \frac{\epsilon}{3}.
\]
Therefore, by the triangle inequality,
$\forall \delta \in (\delta_1,\infty) \cap
\left( 1 - \frac{\delta_0}{R}, 1 + \frac{\delta_0}{R} \right)$
\[\|f_{\delta} - f\|_{L^1}
 \leq \|f_{\delta} - g_{\delta}\|_{L^1}
    + \|g_{\delta} - g\|_{L^1}
    + \|g - f\|_{L^1}
 < \frac{\epsilon}{3} + \frac{\epsilon}{3} + \frac{\epsilon}{3}
 = \epsilon.
\]
Thus, $f_{\delta}$ converges to $f$ in the $L^1$-norm as
$\delta \rightarrow 1$. \qed
\end{question}

\begin{question}{Exercise 4}
Define $h: \R^2 \rightarrow \R$ be defined $\forall x,t \in [0,b]$ by
\[
h(x,t)
 := \left\{
        \begin{array}{cl}
            \frac{f(t)}{t}  & : 0 < x \leq t \leq b \\
            0               & : \mbox{otherwise}
        \end{array}
    \right..
\]
As the quotient of two continuous functions on a measurable set and a constant
function elsewhere, $h$ is measurable, so that, by Fubini's Theorem, since $g$
is identical on $[0,b]$ to the function
$x \in \R \mapsto \int_{\R} h(x,t) \; dt$, $g$ is measurable, and, moreover,
\[
\int_0^b g(x) \; dx
   = \int_0^b \left( \int_x^b \frac{f(t)}{t} \; dt \right) \; dx
   = \int_{\R^2} h(x,t)
   = \int_0^b \left( \int_0^t \frac{f(t)}{t} \; dx \right) \; dt
   = \int_0^b f(t)\; dt.
\]
It follows that, since $f$ is integrable, $g$ is integrable. \qed
\end{question}

\begin{question}{Exercise 6}
\begin{enumerate}[(a)]
\item Let $f : \R \rightarrow \R$ be defined such that, $\forall x \in \R$,
\[
    f(x)
 := \left \{
        \begin{array}{cl}
            2^n & : x \in \left[ n, n + \frac{1}{2^{2n}} \right], n \in \N\\
            0   & : \mbox{otherwise}
        \end{array}
    \right.
\]
Then,
\[
   \int_{\R} f(x) \; dx
 = \sum_{i = 0}^{\infty} \int_n^{n + \frac{1}{2^{2n}}} f(x) \; dx
 = \sum_{i = 0}^{\infty} \frac{1}{2^n}
 = 1
 < \infty,
\]
so that $f$ is integrable on $\R$. However, $\forall n_0 \in \N$, $y \in r$,
$\exists n > n_0$ such that $2^n > y$. Thus,
$\limsup_{x \rightarrow \infty} f(x) = \infty$, as desired. \qed

Note: I just realized the question requires that $f$ be continuous. I don't
have time to write up the correction properly, but $f$ should be modified such
that its graph is is an isosceles triangle of height $2^n$ on each interval
$\left[ n, n + \frac{1}{2^{2n}} \right]$, and $f = 0$ elsewhere. Then, $f$
will be continuous and $\int_{\R} f(x) \; dx = \frac12$, so $f$ is integrable,
but $\limsup_{x \rightarrow \infty} f(x) = \infty$, for the same reason as
above.

\item If we show the result in the case $f \geq 0$, the general case follows,
since $f^+,f^- \geq 0$.

Then, if we suppose, for sake of contradiction, that
the conclusion fails, $\lim_{x \rightarrow \infty} f(x) > 0$ (the case
$\lim_{x \rightarrow -\infty} f(x) > 0$ is identical). Thus,
$\exists \epsilon > 0$ such that, $\forall x_0 \in \R$,
$\exists x \in (x_0,\infty)$ with $f(x) > \epsilon$. Since $f$ is uniformly
continuous, $\exists \delta > 0$ such that, $\forall x,y \in \R$, if
$|x - y| < \delta$, then $|f(x) - f(y)| < \frac{\epsilon}{2}$. It follows that
there are infinitely many disjoint intervals of the form
$(x - \delta, x + \delta)$ on which $f \geq \frac{\epsilon}{2}$, so that
$m(f\inv((\frac{\epsilon}{2}, \infty))) = \infty$. However, since $f$ is
integrable so that $\frac{2}{\epsilon} \int_{\R} f(x) \; dx < \infty$, this
contradicts the Tchebychev Inequality (Exercise 9). \qed
\end{enumerate}
\end{question}

\begin{question}{Exercise 8}
Let $\epsilon > 0$. By part (ii) of Proposition 1.12 (absolute continuity of
the Lebesgue integral), $\exists \delta > 0$ such that, whenever
$E \subseteq \R$ has $m(E) < \delta$, $\int_E f < \epsilon$. Then, for
$x,y \in \R$ (without loss of generality, with $x \leq y$), if
$|y - x| < \delta$, then, since $m([x,y]) < \delta$,
\[
|F(y) - F(x)|
 =    \left| \int_{-\infty}^y f - \int_{-\infty}^x f \right|
 =    \left| \int_x^y f \right|
 \leq \int_x^y \left| f \right|
 <    \epsilon.
\]
Thus, $F$ is uniformly continuous. \qed
\end{question}

\begin{question}{Exercise 9}
Let $f_{\alpha}: \R \rightarrow \R$ be the function defined $\forall x \in \R$
by
\[
f_{\alpha}(x)
 := \left\{
        \begin{array}{cl}
            \alpha & : x \in E_{\alpha} \\
            0    & : \mbox{otherwise}
        \end{array}
    \right.
\]
Since $f$ is measurable, $E_{\alpha} = f\inv((\alpha,\infty))$ is measurable.
Furthermore, $f_{\alpha}$ is simple, so that
\[
\int_{\R} f_{\alpha} = \alpha m(E_{\alpha}) + 0 \cdot m(\R \sminus E_{\alpha})
    \quad \mbox{ and thus } \quad
\frac{1}{\alpha} \int_{\R} f_{\alpha} = m(E_{\alpha})
\]
Thus, since $f_{\alpha} \leq f$ on $\R$, by monotonicity of the Lebesgue
integral (and since $\alpha > 0$),
\[
m(E_{\alpha})
 \leq \frac{1}{\alpha} \int_{\R} f_{\alpha}
 \leq \frac{1}{\alpha} \int_{\R} f. \mqed
\]
\end{question}
\end{document}
