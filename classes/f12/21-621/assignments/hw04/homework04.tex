\documentclass[11pt]{article}
\usepackage{enumerate}
\usepackage{fullpage}
\usepackage{fancyhdr}
\usepackage{amsmath, amsfonts, amsthm, amssymb}
\usepackage{color}
\setlength{\parindent}{0pt}
\setlength{\parskip}{5pt plus 1pt}
\pagestyle{empty}

\def\indented#1{\list{}{}\item[]}
\let\indented=\endlist

\newcounter{questionCounter}
\newcounter{partCounter}[questionCounter]
\newenvironment{question}[2][\arabic{questionCounter}]{%
    \setcounter{partCounter}{0}%
    \vspace{.25in} \hrule \vspace{0.5em}%
        \noindent{\bf #2}%
    \vspace{0.8em} \hrule \vspace{.10in}%
    \addtocounter{questionCounter}{1}%
}{}
\renewenvironment{part}[1][\alph{partCounter}]{%
    \addtocounter{partCounter}{1}%
    \vspace{.10in}%
    \begin{indented}%
       {\bf (#1)} %
}{\end{indented}}

%%%%%%%%%%%%%%%%%%%%%%%HEADER%%%%%%%%%%%%%%%%%%%%%%%%%%%%%%
\newcommand{\myname}{Shashank Singh}
\newcommand{\myandrew}{sss1@andrew.cmu.edu}
\newcommand{\myclass}{21-621 Introduction to Lebesgue Integration}
\newcommand{\myhwnum}{4}
\newcommand{\duedate}{Wednesday, December 5, 2012}
\newcommand{\problemlist}{Chapter 2, Page 91, Problems 11, 15, 21}
%%%%%%%%%%%%%%%%%%%%%%%%%%%%%%%%%%%%%%%%%%%%%%%%%%%%%%%%%%%

%%%%%%%%%%%%%%%%%%%%CONTENT MACROS%%%%%%%%%%%%%%%%%%%%%%%%%
\renewcommand{\qed}{\quad $\blacksquare$}
\newcommand{\mqed}{\quad \blacksquare}
\newcommand{\inv}{^{-1}}
\newcommand{\bx}{\mathbf{x}}
\newcommand{\by}{\mathbf{y}}
\newcommand{\bff}{\mathbf{f}}
\newcommand{\bzero}{\mathbf{0}}
\newcommand{\bxi}{\boldsymbol{\xi}}
\newcommand{\boldeta}{\boldsymbol{\eta}}
\newcommand{\sminus}{\backslash}
\newcommand{\N}{\mathbb{N}} % natural numbers
\newcommand{\Q}{\mathbb{Q}} % rational numbers
\newcommand{\R}{\mathbb{R}} % real numbers
%%%%%%%%%%%%%%%%%%%%%%%%%%%%%%%%%%%%%%%%%%%%%%%%%%%%%%%%%%%

\begin{document}
\thispagestyle{plain}

{\Large Homework \myhwnum} \\
\myclass \\
Name: \myname \\
Email: \myandrew \\
%Due: \duedate \\
\problemlist \\
\begin{question}{Problem 11}
Let $f : \R^d \rightarrow \R$ be integrable, with $\int_E f \geq 0$ on every
measurable $E \subseteq \R^d$, and suppose, for sake of contradiction that,
that $E^- := f\inv((-\infty, 0))$ has $m(E^-) > 0$. Then, if,
$\forall n \in \N$, we define $E_n := f\inv((-\infty, -1/n))$, by countable
additivity of the Lebesgue measure,
\[
0
 < m(E^-)
 = m\left( \bigcup_{n = 1}^{\infty} E_n \right)
 \leq \sum_{n = 1}^{\infty} m(E_n)
\]
(since $f$ is measurable, each $E_n$ is measurable).
Thus, for some $n > 0$, $m(E_n) > 0$. Therefore,
\[
\int_{E_n} f
 \leq \int_{E_n} -\frac{1}{n}
 = -\frac{m(E_n)}{n}
 < 0,
\]
contradicting the fact that $f$ has non-negative integral over all measurable
sets. \qed

A nearly identical proof shows that, if $f$ has non-positive integral over
all measurable sets, then, $f \leq 0$ almost everywhere. It follows that, if
$f$ has zero integral on all measurable sets, $f \leq 0$ almost everywhere,
and $f \geq 0$ almost everywhere, so that $f = 0$ almost everywhere. \qed
\end{question}

\begin{question}{Problem 15}
$\forall n \in \N$, since $f$ is continuous on
$\left( \frac{1}{n}, 1 \right)$, since Riemann and Lebesgue integrals of $f$
agree on $\left( \frac{1}{n}, 1 \right)$,
\[\lim_{a \rightarrow 0} \int_a^1 f
 = \lim_{a \rightarrow 0} \int_a^1 \frac{1}{\sqrt{x}} \, dx
 = \lim_{a \rightarrow 0} 2\sqrt{x} \, \Big|_{x = a}^{x = 1} = 2 - 2\sqrt{a}
 = 2.
\]
Thus, by the Lebesgue Monotone Convergence Theorem, 
\[\int_{\R} f
 = \int_0^1 \lim_{n \rightarrow \infty} f\chi_{(\frac{1}{n}, 1]}
 = \lim_{n \rightarrow \infty} \int_\frac{1}{n}^1 f
 = 2.
\]
Then, again, by the Lebesgue Monotone Convergence Theorem, since translating a
function does not change its integral over $\R$,
\[\int_{\R} F
 = \sum_{n = 1}^{\infty} \int_{\R} \frac{f(x - r_n)}{2^n} \, dx.
 = \sum_{n = 1}^{\infty} \frac{1}{2^n} \int_{\R} f(x - r_n) \, dx.
 = 2 \sum_{n = 1}^{\infty} \frac{1}{2^n}
 = 2
 < \infty,
\]
so that $F$ is integrable, and thus that $F$ is finite almost everywhere, so
that the series by which $F$ is defined converges almost everywhere. \qed

Suppose $\tilde{F} = F$ almost everywhere, let $I$ be an interval, and let
$M \in \R$. Since $\Q$ is dense in $\R$, $r_n \in \Q \cap I$, for some
$n \in \N$. Note that $f > 2^nM$ on the interval
$I_M := \left( r_n, r_n + \frac{1}{2^nM} \right)$. Since $I \cap I_M$ is an
nonempty open interval, $m(I \cap I_M) > 0$, so that, since $\tilde{F} = F$
almost everywhere, $\exists x \in I \cap I_M$ with
$\tilde{F}(x) = F(x) \geq \frac{f(x)}{2^n} > M$. Thus, $\tilde{F}$ is
unbounded on $I$. \qed
\end{question}

\begin{question}{Problem 21}
The following lemma is used in proofs of parts (b) and (d) of this problem:

{\bf Lemma 1:} If $f, g \in L^1(\R^d)$, then
\[
\int_{\R^d} \int_{\R^d} |f(x - y)g(y)| \, dy \, dx
   = \|f\|_{L^1(\R^d)} \|g\|_{L^1(\R^d)}.
\]


{\bf Proof:} By Fubini's Theorem and linearity
of the integral,
\begin{align*}
\int_{\R^d} \int_{\R^d} |f(x - y)g(y)| \, dy \, dx
 & = \int_{\R^d} \int_{\R^d} |f(x - y)g(y)| \, dx \, dy, \\
 & = \int_{\R^d} |g(y)| \int_{\R^d} |f(x - y)| \, dx \, dy, \\
 & = \int_{\R^d} |g(y) \|f\|_{L^1(\R^d)} \, dy \\
 & = \|f\|_{L^1(\R^d)} \int_{\R^d} |g(y)| \, dy
   = \|f\|_{L^1(\R^d)} \|g\|_{L^1(\R^d)}.
\end{align*}
The third equality holds because translating a function does not change its
$L^1$ norm. \qed

\begin{enumerate}[(a)]
\item The function $(x,y) \in \R^{2d} \mapsto x - y$ is measurable, 
so that, since any composition or product of measurable functions is
measurable, the function $(x,y) \in \R^{2d} \mapsto f(x - y) g(y)$ is
measurable. \qed

\item By Fubini's Theorem and Lemma 1, if
$\|f\|_{L^1(\R^d)}\|,\|g\|_{L^1(\R^d)}\| < \infty$, then
\[
\int_{\R^{2d}} |f(x - y)g(y)| \, dx \, dy
 = \int_{\R^d} \int_{\R^d} |f(x - y)g(y)| \, dy \, dx
 = \|f\|_{L^1(\R^d)} \|g\|_{L^1(\R^d)} < \infty. \mqed
\]

\item This follows immediately from part (b) and Fubini's Theorem. \qed

\item By the Triangle Inequality and then by Lemma 1,
\[
\|f*g\|_{L^1(\R^d)}
 =    \int_{\R^d} \left|\int_{\R^d} f(x - y)g(y) \, dy \right| \, dx
 \leq \int_{\R^d} \int_{\R^d} |f(x - y)g(y)| \, dy \, dx,
 = \|f\|_{L^1(\R^d)} \|g\|_{L^1(\R^d)}.
 \]
with equality when $f, g \geq 0$. \qed

\item Since $f$ is integrable, it follows from Proposition 4.1 that $\hat{f}$
is continuous and bounded. \qed

\end{enumerate}
\end{question}
\end{document}
