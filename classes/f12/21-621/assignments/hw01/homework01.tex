\documentclass[11pt]{article}
\usepackage{enumerate}
\usepackage{fullpage}
\usepackage{fancyhdr}
\usepackage{amsmath, amsfonts, amsthm, amssymb}
\usepackage{color}
\setlength{\parindent}{0pt}
\setlength{\parskip}{5pt plus 1pt}
\pagestyle{empty}

\def\indented#1{\list{}{}\item[]}
\let\indented=\endlist

\newcounter{questionCounter}
\newcounter{partCounter}[questionCounter]
\newenvironment{question}[2][\arabic{questionCounter}]{%
    \setcounter{partCounter}{0}%
    \vspace{.25in} \hrule \vspace{0.5em}%
        \noindent{\bf #2}%
    \vspace{0.8em} \hrule \vspace{.10in}%
    \addtocounter{questionCounter}{1}%
}{}
\renewenvironment{part}[1][\alph{partCounter}]{%
    \addtocounter{partCounter}{1}%
    \vspace{.10in}%
    \begin{indented}%
       {\bf (#1)} %
}{\end{indented}}

%%%%%%%%%%%%%%%%%%%%%%%HEADER%%%%%%%%%%%%%%%%%%%%%%%%%%%%%%
\newcommand{\myname}{Shashank Singh}
\newcommand{\myandrew}{sss1@andrew.cmu.edu}
\newcommand{\myclass}{21-621 Introduction to Lebesgue Integration}
\newcommand{\myhwnum}{1}
\newcommand{\duedate}{Friday, November 2, 2012}
\newcommand{\problemlist}{Chapter 1, Problems 1,2,3}
%%%%%%%%%%%%%%%%%%%%%%%%%%%%%%%%%%%%%%%%%%%%%%%%%%%%%%%%%%%

%%%%%%%%%%%%%%%%%%%%CONTENT MACROS%%%%%%%%%%%%%%%%%%%%%%%%%
\renewcommand{\qed}{\quad $\blacksquare$}
\newcommand{\mqed}{\quad \blacksquare}
\newcommand{\inv}{^{-1}}
\newcommand{\bx}{\mathbf{x}}
\newcommand{\by}{\mathbf{y}}
\newcommand{\bff}{\mathbf{f}}
\newcommand{\bzero}{\mathbf{0}}
\newcommand{\bxi}{\boldsymbol{\xi}}
\newcommand{\boldeta}{\boldsymbol{\eta}}
\newcommand{\sminus}{\backslash}
\newcommand{\N}{\mathbb{N}} % natural numbers
\newcommand{\Q}{\mathbb{Q}} % rational numbers
\newcommand{\R}{\mathbb{R}} % real numbers
\newcommand{\C}{\mathcal{C}} % Cantor set
%%%%%%%%%%%%%%%%%%%%%%%%%%%%%%%%%%%%%%%%%%%%%%%%%%%%%%%%%%%

\begin{document}
\thispagestyle{plain}

{\Large Homework \myhwnum} \\
\myclass \\
Name: \myname \\
Email: \myandrew \\
%Due: \duedate \\
\problemlist
\begin{question}{Chapter 1, Problem 1}
Let $x,y \in \C$ be distinct, so that $|x - y| > 0$. Then, $\exists k \in \N$
such that $|x - y| > 1/3^k$. Since $C_k$ is a union of intervals of length
$1/3^k$ separated by intervals of length at least $1/3^k$,
no interval containing both $x$ and $y$ is contained in $C_k$; consequently,
since $\C \subseteq C_k$, no such interval is contained in $\C$.
Since the only connected subsets of $\R$ are intervals, no connected subset of
$\C$ contains both $x$ and $y$. Therefore, $\C$ is totally disconnected. \qed

Let $x \in \C$, and let $U$ be a neighborhood of $x$, so that, for some
$r > 0$, $B := (x - r,x + r) \subseteq U$. Then, $\exists k \in \N$ such that
$r > 1/3^k$. Note that, since $\C \subseteq C_k$, $x \in C_k$. Thus, since
$C_k$ is a union of intervals of length $1/3^k$, there is some such intervals
$I$ with $x \in I$, so that $I \subseteq B$.

Let $y$ be an endpoint of $I$ distinct from $x$. Since, $\forall n \geq k$,
$C_n$ is constructed from $C_k$ by removing only \emph{middle} thirds of
intervals in $C_k,C_{k + 1},\ldots,C_{n - 1}$, $y \in C_n$. Thus, since
$\{C_n\}$ is a decreasing sequence (with respect to inclusion),
$y \in \bigcap_{i = 1}^{\infty} C_i = \C$.

Since every neighborhood $U$ of any point $x \in \C$ contains a point in $\C$
distinct from $x$, $\C$ contains no isolated points and is therefore perfect.
\qed
\end{question}

\begin{question}{Chapter 1, Problem 2}
\begin{enumerate}[(a)]
\item We first show by induction in $n$ that, for any positive $n \in \N$,
$C_n$ can be characterized as
\[
C_n = \{x \in [0,1] : \mbox{ $x$ has a ternary expansion in which the first
$n$ trits are $0$'s or $2$'s }\}
\]
(trits are counted after the radix). For $n = 0$, this is trivial, since there
are no trits to consider, and $C_0 = [0,1]$.

Suppose that, for some $n \in \N$, $C_n$ obeys the above characterization.
Suppose $k \in \N$, with $k < 3^n$. The closed middle third $I$ of the
interval $J := [k/3^n,(k + 1)/3^n]$ clearly consists of those $x \in I$ with
a ternary expansion in which the $n^{th}$ trit is a $1$. The only points in
$I$ with ternary expansions differing in the $n^{th}$ digit are its endpoints,
$(3k + 1)/3^{n + 1}$ and $(3k + 2)/3^{n + 1}$, which can be alternatively
written without $1$'s in the $n^{th}$ trit.

Thus, when we remove the open middle third of $J$ in constructing $C_{n + 1}$
(noting that each connected component of $C_n$ has the form of $J$), we remove
precisely those points without ternary representations in which the $n^{th}$
trit is a $0$ or $2$. Thus, by the inductive hypothesis, $C_{n + 1}$ obeys the
above characterization, proving the claim.

Since $\C = \bigcap_{i = 1}^{\infty} C_i$, it is clear from this
characterization of $C_n$ that
\[\C = \{x \in [0,1] : \mbox{ $x$ has a ternary expansion in which every trit
is $0$ or $2$ }\}. \mqed\]

\item To show that $F$ is well-defined it suffices to show that the ternary
expansion of $x$ containing only $0$'s and $2$'s is unique when it exists.
Indeed, if we assume, for sake of contradiction, that this is not the case,
letting $k$ is the index of the first trit in which two such ternary
representations $0.c_1c_2\ldots$ and $0.d_1d_2\ldots$ of $x$ differ (without
loss of generality, $c_k = 0$ and $d_k = 2$), then,
$\sum_{i = k + 1}^{\infty} \frac{c_i}{3^i} \geq 2/3^k$. However,
\[\sum_{i = k + 1}^{\infty} \frac{c_i}{3^i}
 \leq \sum_{i = k + 1}^{\infty} \frac{2}{3^i}
 \leq 1/3^k,
\]
which is a contradiction.

(It may also be considered necessary to show that $F$ takes only real values;
i.e., that the series defining $F$ converges. This is clear from the fact
that, $\forall x \in [0,1]$, $0 \leq b_k \leq 1$, so that $|F(x)|$ is bounded
by $\sum_{i = 1}^{\infty} \frac{1}{2^i}$, which converges.) \qed

Since we show directly in part (c) that $F$ is surjective, to show that $F$ is
continuous, it is sufficient to show that $F$ is nondecreasing.
If $x,y \in \C$ with $x < y$, then there exists a smallest $k$ such that
$x_i = 0$ and $y_i = 2$, where $0.x_1x_2\ldots$ and $0.y_1y_2\ldots$ are the
ternary expansions (not containing $1$'s) of $x$ and $y$, respectively. Thus,
since
\[\sum_{i = k + 1}^{\infty} \frac{x_i/2}{2^i}
 \leq \sum_{i = k + 1}^{\infty} \frac{1}{2^i}
 < \frac{1}{2^k},
\]
\[F(x)
 =    \sum_{i = 1}^{\infty} \frac{x_i/2}{2^i}
 \leq \sum_{i = 1}^{k - 1}  \frac{x_i/2}{2^i} + \frac{1}{2^k}
 =    \sum_{i = 1}^{k}      \frac{y_i/2}{2^i}
 \leq y,
\]
so $F$ is nondecreasing and thus continuous. \qed

If $x = 0$, then $a_k = 0$, $\forall k \in \N$, so that $b_k = 0$, and
thus $F(x) = 0$. \qed

If $x = 1$, then $a_k = 2$, $\forall k \in \N$, so that $b_k = 1$, and thus
$\displaystyle F(x) = \sum_{i = 1}^{\infty} \frac{1}{2^i} = 1$. \qed

\item Suppose $y \in [0,1]$. Then, $y$ has a binary expansion
$0.b_1b_2\ldots$. If $x$ is the number whose ternary expansion (not containing
$1$'s) is $0.a_1a_2\ldots$ (where, $\forall i \in \N$, $a_i = 2b_i$),
\[F(x) = \sum_{i = 1}^{\infty} \frac{b_i}{2^i} = y,\]
by definition of the binary expansion. Since each $b_i$ is $0$ or $1$, each
$a_i$ is $0$ or $2$. Therefore, by the result of part (a), $x \in \C$.
Therefore, $F$ is surjective onto $[0,1]$. \qed

\item There doesn't seem to be any actual question here.
\end{enumerate}
\end{question}

\newpage
\begin{question}{Chapter 1, Problem 3}
\begin{enumerate}[(a)]
\item $\forall k \in \N$, let $C_k$ denote the set created after $k$
iterations of the given procedure (so that $C_0 = [0,1]$,
$C_1 = [0,\frac{1 - \xi}{2}] \cup [1 - \frac{1 - \xi}{2}, 1]$, and so on).

$\forall k \in \N$, the construction of $C_k$ removes $2^k$ intervals of
length 
$\left( \frac{1 - \xi}{2} \right)^k\xi$, the total length removed from
$C_k$ is
\[2^k \left( \frac{1 - \xi}{2} \right)^k \xi
 = (1 - \xi)^k \xi.
\]
Then, the total length removes in constructing $C_k$ from $[0,1]$ is (since
$\mu_*$ is finitely additive over disjoint intervals)
\[
C_*([0,1] \sminus C_k)
 = \sum_{i = 1}^k (1 - \xi)^i\xi
 = \xi \sum_{i = 1}^k (1 - \xi)^i
 = \xi \frac{1 - (1 - \xi)^{k + 1}}{1 - (1 - \xi)}
 = 1 - (1 - \xi)^{k + 1}.
\]
Thus, by monotonicity of $\mu_*$ (since
$[0,1] \sminus C_{\xi} \supseteq [0,1] \sminus C_k$),
\[1
 = \mu_*([0,1])
 = \mu_*([0,1] \sminus C_{\xi})
 \geq \mu_*([0,1] \sminus C_k)
 = 1 - (1 - \xi)^{k + 1}
,\]
so that, since $\lim_{k \rightarrow \infty} (1 - (1 - \xi)^{k + 1}) = 1$,
$\mu([0,1] \sminus C_{\xi}) = 1$. \qed

\item $\forall k \in \N$, let $C_k$ be defined as in part (a). We first show
by induction on $k$ that $C_k$ is the union of $2^k$ pairwise disjoint
intervals, each of length $\left( \frac{1 - \xi}{2} \right)^k$.
For $k = 0$, this is clear, since $C_k = [0,1]$, which has length $1$.
Suppose that, for some $k \in \N$, $C_k$ is the union of pairwise disjoint
intervals, each of length $\left( \frac{1 - \xi}{2} \right)^k$.

Since, in constructing $C_{k + 1}$ from $C_k$,
we remove intervals from the center of each interval, each interval in
$C_k$ corresponds clearly to $2$ intervals in $C_{k + 1}$, so that $C_{k + 1}$
is composed of $2 \cdot 2^k = 2^{k + 1}$ (clearly disjoint) intervals.

Since
we remove from each interval of length
$\left( \frac{1 - \xi}{2} \right)^k$ an interval of length
$\left( \frac{1 - \xi}{2} \right)^k\xi$, the length of each new interval is
(since each interval in $C_k$ is split into $2$ intervals in $C_{k + 1}$)
\[
\frac12 \left( \left( \frac{1 - \xi}{2} \right)^k - \left( \frac{1 - \xi}{2} \right)^k\xi \right)
 = \frac12 (1 - \xi) \left( \frac{1 - \xi}{2} \right)^k
 = \left( \frac{1 - \xi}{2} \right)^{k + 1}.
\]
Thus, since $\mu_*$ is finitely additive on (disjoint) intervals,
$\forall k \in \N$,
\[\mu_*(C_k)
 = \sum_{i = 1}^{2^k} \left( \frac{1 - \xi}{2} \right)^k
 = 2^k \left( \frac{1 - \xi}{2} \right)^k
 = (1 - \xi)^k.
\]
$\forall k \in \N$, since $\C_{\xi} \subseteq C_k$, by monotonicity and
non-negativity of $\mu_*$,
$0 \leq \mu_*(\C_{\xi}) \leq \mu_*(C_k) = (1 - \xi)^k$.
Since this holds $\forall k \in \N$ and
$\lim_{k \rightarrow \infty} (1 - \xi)^k = 0$, $\mu_*(C_{\xi}) = 0$. \qed

\end{enumerate}
\end{question}
\end{document}
