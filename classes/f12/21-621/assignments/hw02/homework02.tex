\documentclass[11pt]{article}
\usepackage{enumerate}
\usepackage{fullpage}
\usepackage{fancyhdr}
\usepackage{amsmath, amsfonts, amsthm, amssymb}
\usepackage{color}
\setlength{\parindent}{0pt}
\setlength{\parskip}{5pt plus 1pt}
\pagestyle{empty}

\def\indented#1{\list{}{}\item[]}
\let\indented=\endlist

\newcounter{questionCounter}
\newcounter{partCounter}[questionCounter]
\newenvironment{question}[2][\arabic{questionCounter}]{%
    \setcounter{partCounter}{0}%
    \vspace{.25in} \hrule \vspace{0.5em}%
        \noindent{\bf #2}%
    \vspace{0.8em} \hrule \vspace{.10in}%
    \addtocounter{questionCounter}{1}%
}{}
\renewenvironment{part}[1][\alph{partCounter}]{%
    \addtocounter{partCounter}{1}%
    \vspace{.10in}%
    \begin{indented}%
       {\bf (#1)} %
}{\end{indented}}

%%%%%%%%%%%%%%%%%%%%%%%HEADER%%%%%%%%%%%%%%%%%%%%%%%%%%%%%%
\newcommand{\myname}{Shashank Singh}
\newcommand{\myandrew}{sss1@andrew.cmu.edu}
\newcommand{\myclass}{21-621 Introduction to Lebesgue Integration}
\newcommand{\myhwnum}{2}
\newcommand{\duedate}{Wednesday, November 28, 2012}
\newcommand{\problemlist}{Chapter 1, Problems 5,7,11,13,16,17, provide the
details of Property 3 on page 29}
%%%%%%%%%%%%%%%%%%%%%%%%%%%%%%%%%%%%%%%%%%%%%%%%%%%%%%%%%%%

%%%%%%%%%%%%%%%%%%%%CONTENT MACROS%%%%%%%%%%%%%%%%%%%%%%%%%
\renewcommand{\qed}{\quad $\blacksquare$}
\newcommand{\mqed}{\quad \blacksquare}
\newcommand{\inv}{^{-1}}
\newcommand{\bx}{\mathbf{x}}
\newcommand{\by}{\mathbf{y}}
\newcommand{\bff}{\mathbf{f}}
\newcommand{\bzero}{\mathbf{0}}
\newcommand{\bxi}{\boldsymbol{\xi}}
\newcommand{\boldeta}{\boldsymbol{\eta}}
\newcommand{\sminus}{\backslash}
\newcommand{\N}{\mathbb{N}} % natural numbers
\newcommand{\Q}{\mathbb{Q}} % rational numbers
\newcommand{\R}{\mathbb{R}} % real numbers
\renewcommand{\O}{\mathcal{O}}
\newcommand{\pow}[1]{\mathcal{P} \left( #1 \right)} % power set of #1
%%%%%%%%%%%%%%%%%%%%%%%%%%%%%%%%%%%%%%%%%%%%%%%%%%%%%%%%%%%

\begin{document}
\thispagestyle{plain}

{\Large Homework \myhwnum} \\
\myclass \\
Name: \myname \\
Email: \myandrew \\
Due: \duedate \\
\problemlist \\

The following lemma is used in the solutions to problems 5 and 13:

{\bf Lemma 1:} If $C \subseteq \R^d$ is closed, $x \in C$ if and only
if $d(x,C) = 0$.

{\bf Proof of Lemma 1:} It is clear that, if $x \in C$, then $d(x,C) = 0$,
since $\|x - x\| = 0$. Suppose that $x \notin C$. Let $y \in C$, and let
$B = B(x, \|x - y\|)$ by the ball of radius $\|x - y\|$ centered at $x$. Let
$K := \overline{B \cap C}$, the closure of the intersection of $B$ and $C$.
Since $K$ is compact, the continuous function taking $y \in K$ to $\|x - y\|$
achieves a minimum on $K$, so that there must exist $y \in K$ with
$\|y - x\| = d(x,K)$. Thus, if $x \notin K$, $d(x,K) > 0$. From the definition
of $K$, it is clear that $d(x,K) = d(x,C)$, so that, if $x \notin C$,
$d(x,C) > 0$. \qed

{\bf Lemma 2:} $F \subseteq \R^d$ is a $G_{\delta}$ if and only if
$R^d \sminus F$ is an $F_{\sigma}$.

{\bf Proof of Lemma 2:} If $F = \bigcap_{i = 1}^{\infty} U_i$ with each $U_i$
open, then
\[ \R^d - F
 = \R^d - \bigcap_{i = 1}^{\infty} U_i 
 = \bigcup_{i = 1}^{\infty} \R^d - U_i,
\]
so that, since each $\R^d - U_i$ is closed, $\R^d - F$ is an $F_{\sigma}$.

If $\R^d - F = \bigcup_{i = 1}^{\infty} C_i$, with each $C_i$ closed, then
\[ F
 = \R^d - (\R^d - F)
 = \R^d - \bigcup_{i = 1}^{\infty} C_i 
 = \bigcap_{i = 1}^{\infty} \R^d - C_i,
\]
so that, since $\R^d - C_i$ is open, $F$ is a $G_{\delta}$. \qed

\begin{question}{Chapter 1, Problem 5}
\begin{enumerate}[(a)]
\item Suppose $E$ is compact. $\forall \in \N$, since $\O_n$ is open, $\O_n$
is measurable. Since $E$ is compact, it is bounded by some ball $B$ of radius
$r$ centered at the origin. Then, clearly, $\O_1$ is bounded (by the ball of
radius $r + 1$ centered at the origin), so that $\mu_*(\O_n) < \infty$.

Thus, by part (ii) of Corollary 3.3, it suffices to show that $\O_n \searrow E$
as $n \rightarrow \infty$. Clearly,
$\forall n \in \N$, $\O_n \supseteq \O_{n + 1}$. It is also apparent that
$E \subseteq \O_n, \forall n \in \N$, so that
$E \subseteq \bigcap_{i = 1}^{\infty} \O_n$. If $x \in \R^d$, then, by Lemma 1
above, $d(x,E) > 0$, so that, for some $n \in \N$, $x \notin \O_n$, and so
$E \supseteq \bigcap_{i = 1}^{\infty} \O_n$. \qed

\newpage
\item Let $C = \{(x,0) : x \in \R\} \subseteq \R^2$, so that $C$ is closed but
unbounded. $\forall \epsilon > 0$, we can cover $C$ with the countably
many almost disjoint rectangles of dimension $1 \times \frac{\epsilon}{2^i}$,
each centered at $(\pm i,0)$. Thus, by monotonicity of $\mu_*$,
\[\mu_*(C)
 \leq 2 \sum_{i = 1}^{\infty} \frac{\epsilon}{2^i}
 =    \epsilon
 <    \infty
.\]
On the other hand, $\forall n \in \N$, if
$\O_n = \{x \in \R^2 : d(x,C) < 1/n\}$, then $\O_n$ is the union of
countably many almost disjoint rectangles of dimension $1 \times 1/n$, each
centered at $(\pm i, 0)$, so that
\[
\mu_*(\O_n)
 = \sum_{i = 1}^{\infty} 1/n
 = \infty.
\]
Therefore, $\mu_*(\O_n)$ does not converge to $\mu_*(C)$ as
$n \rightarrow \infty$. \qed
\end{enumerate}
\end{question}

\begin{question}{Chapter 1, Problem 7}
Let $\mathcal{R}$ be the set of rectangles in $\R^d$.
Clearly, $\{R_1,R_2,\ldots\} \subseteq \mathcal{R}$ is a countable cover of
$E$ if and only if $\{\delta R_1,\delta R_2, \ldots\} \subseteq \mathcal{R}$
is a countable cover of $\delta E$. Thus, since, $\forall i \in \N$, it is
clear that $\mu_*(\delta R_i) = \mu_*(R_i) \prod_{i = 1}^d \delta_i$,
\begin{align*}
\mu_*(\delta E)
 & = \inf \left\{ \sum_{i = 1}^{\infty} \mu_*(\delta R_i) \right|
          \left. \delta E \subseteq \bigcup_{i = 1}^{\infty} \delta R_i,
            \delta R_i \in \mathcal{R} \right\} \\
 & = \inf \left\{ \left( \sum_{i = 1}^{\infty} \mu_*(\delta R_i) \right)
     \prod_{i = 1}^d \delta_i \right|
        \left. E \subseteq \bigcup_{i = 1}^{\infty} R_i,
            R_i \in \mathcal{R} \right\} \\
 & = \left( \prod_{i = 1}^d \delta_i \right)
   \inf \left\{ \sum_{i = 1}^{\infty} \mu_*(\delta R_i) \right|
        \left. E \subseteq \bigcup_{i = 1}^{\infty} R_i,
            R_i \in \mathcal{R} \right\}
   = \mu_*(E) \prod_{i = 1}^d \delta_i.
\end{align*}

Thus, it remains only to show that $\delta E$ is measurable, so that
\[ \mu(\delta E)
 = \mu_*(\delta E)
 = \mu_*(E) \prod_{i = 1}^d \delta_i
 = \mu(E) \prod_{i = 1}^d \delta_i.
\]
Let $\epsilon > 0$. Since $E$ is measurable, there is an open set
$U \supseteq E$ with
$\mu_*(U - E) < \frac{\epsilon}{\prod_{i = 1}^d \delta_i}$. Then,
\[\mu_*(\delta U - \delta E)
 = \mu_*(\delta(U - E))
 = \mu_*(U - E) \prod_{i = 1}^d \delta_i
 < \epsilon,
\]
so that, since $\delta U$ is clearly open, by Theorem 3.4, $\delta E$ is
measurable. \qed
\end{question}

\newpage
\begin{question}{Chapter 1, Problem 11}
Let $S$ be the set in question.
The proof that $\mu(S) = 0$ is identical to the proof given for Exercise 3
that Cantor sets of constant dissection have measure $0$, by first showing, in
a proof identical to that given for part (a) of Exercise 2, that $S$ can be
written in terms of removing from $[0,1]$ countably many intervals (first
$(0.4,0,5)$, then $(0.14,0.15),(0.24,0.25),\ldots$, and $(0.94,0.95)$, and so
on). It's really a bit too tedious to do in full again. \qed
\end{question}

\begin{question}{Chapter 1, Problem 13}
\begin{enumerate}[(a)]
\item Let $C \subseteq \R^d$ be closed, and, $\forall n \in \N$, let
$\O_n := \{x \in \R^d : d(x,C) < 1/n\}$. Clearly, $\forall n \in \N$,
$C \subseteq \O_n$, so that $C \subseteq \bigcap_{i = 1}^{\infty} \O_n$.
$C \subseteq $. It follows from Lemma 1 above that, if $x \notin C$, then
$d(x,C) > 0$, so that, for some $n \in \N$, $x \notin \O_n$. Therefore,
$\bigcap_{i = 1}^{\infty} \O_n \subseteq C$, so that $C$ is a $G_{\delta}$.
\qed

Let $U \subseteq \R^d$ be open, so that $C := \R^d - U$ is closed. Then, as
shown above, $C$ is a $G_{\delta}$, so that, by Lemma 2 above, $U$ is an
$F_{\sigma}$. \qed

\item Let $f$ be a bijection from $\N$ to $\Q$. Then,
$\Q = \bigcup_{i = 1}^{\infty} \{f(i)\}$, so that, since each $\{f(i)\}$ is
closed, $\Q$ is an $F_{\sigma}$. Thus, by Lemma 2 above, $\R - \Q$ is a
$G_{\delta}$. However, $\Q$ is not a $G_{\delta}$, since this would contradict
the Baire category theorem, as $\emptyset = \Q \cap (\R - \Q)$ would be the
countable intersection of open dense sets (as any set with a dense subset is
dense). \qed

\item By the same argument as in part (b) $(\Q \cap [1,2])$ is not a
$G_{\delta}$ and thus, by Lemma 2, $(\R - \Q) \cap [-2,-1]$ is not
an $F_{\sigma}$ (as, clearly, any translation of an $F_{\sigma}$ is, itself an
$F_{\sigma}$), $S := (\Q \cap [1,2]) \cup ((\R - \Q) \cap [-2,-1])$ is neither a
$G_{\delta}$ nor an $F_{\sigma}$.

Since $\Q$ is an $F_{\sigma}$ and $\R - \Q$ is a $G_{\delta}$, and
$[1,2]$ and $[-2,-1]$ are closed (and thus, all these sets are Borel),
since a $\sigma$-algebra is closed under finite unions and intersections,
$S$ is Borel. \qed
\end{enumerate}
\end{question}

\begin{question}{Chapter 1, Problem 16}
Note that $E = \bigcap_{i = 1}^{\infty} \bigcup_{j = i}^{\infty} E_j$, since
$x$ is in the latter if and only if $x \in E_k$ for infinitely many $k$.
\begin{enumerate}[(a)]
\item Since the set of Lebesgue measurable sets is a $\sigma$-algebra, and
thus closed under countable unions and intersections, it follows immediately
from the above characterization of $E$ that $E$ is Lebesgue measurable, since
each $E_k$ is measurable. \qed

\newpage
\item Since $\forall k \in \N$,
$\bigcap_{i = 1}^{\infty} \bigcup_{j = i}^{\infty} E_j
\subseteq \bigcup_{j = k}^{\infty} E_j$,
it follows from monotonicity and countable subadditivity of $\mu$, given the
above characterization of $E$, that
\[
\mu(E)
 =    \mu \left( \bigcap_{i = 1}^{\infty} \bigcup_{j = i}^{\infty} E_j \right)
 =    \mu \left( \bigcup_{j = k}^{\infty} E_j \right)
 \leq \sum_{j = k}^{\infty} \mu(E_j).
\]
Since $\sum_{i = 1}^{\infty} \mu(E_j) < \infty$,
\[
\mu(E)
 =    \lim_{k \rightarrow \infty} \mu(E)
 \leq \lim_{k \rightarrow \infty} \sum_{j = k}^{\infty} \mu(E_j)
 =    0. \mqed
\]

\end{enumerate}
\end{question}

\begin{question}{Chapter 1, Problem 17}
$\forall n \in \N$, since $f_n$ is finite almost everywhere,
$\exists c_n \in \R$ such that $f_n$ is bounded by $c_n/n$ except on a set of
measure less than $2^{-n}$, so that, for $E_n := \{x : |f_n(x)/c_n| > 1/n\}$,
$m(E_n) < 2^{-n}$. Then,
\[\sum_{i = 1}^{\infty} \mu(E_i)
 < \sum_{i = 1}^{\infty} \frac{1}{2^i}
 = 1
 < \infty
.\]
Since each $(c_n/n, \infty)$ and $f_n$ are measurable, each
$S_n = (|f_n|)\inv((c_n/n, \infty))$ is also measurable.
Thus, for $E := \limsup_{i \rightarrow \infty} (E_i)$, by the Borel-Cantelli
lemma, $\mu(E) = 0$.

Since the set of points where $\frac{f_n(x)}{c_n}$ does converge to $0$
is precisely $E$, $\frac{f_n(x)}{c_n} \rightarrow 0$ as $n \rightarrow \infty$
a.e. \qed
\end{question}

\begin{question}{Details of Proof of Property 3}
Since each $f_n$ is measurable, $\forall a \in \R$, if, $\forall n \in \N$,
$E_n = \{x \in \R : f_n(x) > a\}$, $E_n$ is measurable. Note that
$\sup_n f_n(x) > a$ if and only if $\exists n \in \N$ such that $f_n(x) > a$,
so that
$E := \{x \in \R : \sup_n f_n(x) > a\} = \bigcup_{i = 1}^{\infty} E_i$.
Thus, since the set of Lebesuge measurable sets is a $\sigma$-algebra, and
thus closed under countable unions, $E$ is measurable. Then, since this holds
$\forall a \in \R$, the function $x \mapsto \sup_n f(x)$ is measurable. The
proof for the pointwise infimum of countably many measurable functions is
analagous.
\qed

Since each $f_n$ is measurable, as shown above, $\forall k \in \N$,
the functions $x \mapsto \sup_n f_n(x)$ is measurable. Thus, since the
pointwise infimum of countable many measurable functions is measurable,
the function
\[x
 \mapsto \limsup_{n \rightarrow \infty} f_n(x)
 =       \inf_k \{\sup_{n \geq k} f_n(x) : k \in \N\}
\]
is measurable. The proof for the pointwise limit inferior of countable many
measurable functions is analagous. \qed
\end{question}
\end{document}
