\documentclass[11pt]{article}
\usepackage{enumerate}
\usepackage{fullpage}
\usepackage{fancyhdr}
\usepackage{amsmath, amsfonts, amsthm, amssymb}
\usepackage{color}
\setlength{\parindent}{0pt}
\setlength{\parskip}{5pt plus 1pt}
\pagestyle{empty}

\def\indented#1{\list{}{}\item[]}
\let\indented=\endlist

\newcounter{questionCounter}
\newcounter{partCounter}[questionCounter]
\newenvironment{question}[2][\arabic{questionCounter}]{%
    \setcounter{partCounter}{0}%
    \vspace{.25in} \hrule \vspace{0.5em}%
        \noindent{\bf #2}%
    \vspace{0.8em} \hrule \vspace{.10in}%
    \addtocounter{questionCounter}{1}%
}{}
\renewenvironment{part}[1][\alph{partCounter}]{%
    \addtocounter{partCounter}{1}%
    \vspace{.10in}%
    \begin{indented}%
       {\bf (#1)} %
}{\end{indented}}

%%%%%%%%%%%%%%%%%%%%%%%HEADER%%%%%%%%%%%%%%%%%%%%%%%%%%%%%%
\newcommand{\myname}{Shashank Singh}
\newcommand{\myandrew}{sss1@andrew.cmu.edu}
\newcommand{\myclass}{21-295 Putnam Seminar}
\newcommand{\myhwnum}{1}
\newcommand{\duedate}{Friday, September 7, 2012}
%%%%%%%%%%%%%%%%%%%%%%%%%%%%%%%%%%%%%%%%%%%%%%%%%%%%%%%%%%%

%%%%%%%%%%%%%%%%%%%%CONTENT MACROS%%%%%%%%%%%%%%%%%%%%%%%%%
\renewcommand{\qed}{\quad $\blacksquare$}
\newcommand{\mqed}{\quad \blacksquare}
\newcommand{\inv}{^{-1}}
\newcommand{\bx}{\mathbf{x}}
\newcommand{\by}{\mathbf{y}}
\newcommand{\bff}{\mathbf{f}}
\newcommand{\bzero}{\mathbf{0}}
\newcommand{\bxi}{\boldsymbol{\xi}}
\newcommand{\boldeta}{\boldsymbol{\eta}}
%%%%%%%%%%%%%%%%%%%%%%%%%%%%%%%%%%%%%%%%%%%%%%%%%%%%%%%%%%%

\begin{document}
\thispagestyle{plain}

{\Large Homework \myhwnum} \\
\myclass \\
Name: \myname \\
Email: \myandrew \\
Due: \duedate \\
\begin{question}{Putnam 1964/B6}
Suppose, for sake of contradiction, that there exist congruent sets $A$ and
$B$ with $A \cup B = D$ and $A \cap B = \emptyset$, and let
$f:A \rightarrow B$ be a geometric transformation between $A$ and $B$.
Since $O := (0,0) \in D$, $O$ is in exactly one of $A$ and $B$; without loss
of generality, we suppose $O \in A$. Let $Y = f(O) \in B$. Since
$O$ is center of the disk and $Y \neq O$ (as $Y \not \in A$),
there is a unique diameter $L$ of $D$ going through both $O$ and $Y$.
Furthermore, there is a unique diameter $L^{\prime}$ of $D$ that is
perpendicular to $L$. Let $P$ and $Q$ be the endpoints of $L^{\prime}$ (i.e.,
$P$ and $Q$ are the intersection points of $L^{\prime}$ and the boundary of
$D$).

Since $L^{\prime}$ is a diameter, $PQ = 2$. Note that any geometric
transformation is an isometry (i.e., it preserves the distance between any two
points). Since $PO = OQ = 1$, and $PY$ and $YQ$ are hypotenuses of
nondegenerate triangles having either $PO$ or $OQ$ as a leg, $PY, YQ > 1$.
Thus, every point $X$ on the half-circle with endpoints $P$ and $Q$ (including
$P$ and $Q$) has $XY > 1$, so that $X \not \in B$. $PQ = 2$, so that
$f(P),f(Q) \in B$ are the endpoints of some diameter of $D$ (as they have
distance $2$). But this is impossible if $B$ contains no points on the
aforementioned half-circle, providing the desired contradiction. \qed
\end{question}

\begin{question}{Putnam 1962/A6.}
Since either $1 \in X$ or $-1 \in X$ and $X$ is closed under multiplication,
$1 \cdot 1 = (-1) \cdot (-1) = 1 \in X$. Since $X$ is closed under addition
and any postive integer can be constructed by adding $1$ a finite number of
times, the set of positive integers $\mathbb{Z}^+ \subseteq X$.

Therefore, if, for some $a,b \in \mathbb{Z}^+$, $-\frac{a}{b} \in X$ (i.e., if
there were a negative rational number in $X$), then, since $X$ is closed under
multiplication $-a = b \cdot \left(-\frac{a}{b}\right) \in X$. However, this
contradicts the fact that $a \in X$, so no negative rational number is in $X$.
Therefore, every postive rational number is in $X$, so that, since
$X \subseteq \mathbb{Q}$, $X$ is the set of positive rationals. \qed
\end{question}
\end{document}
