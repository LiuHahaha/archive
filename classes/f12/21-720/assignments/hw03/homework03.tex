\documentclass[11pt]{article}
\usepackage{enumerate}
\usepackage{fullpage}
\usepackage{fancyhdr}
\usepackage{amsmath, amsfonts, amsthm, amssymb}
\usepackage{color}
\setlength{\parindent}{0pt}
\setlength{\parskip}{5pt plus 1pt}
\pagestyle{empty}

\def\indented#1{\list{}{}\item[]}
\let\indented=\endlist

\newcounter{questionCounter}
\newcounter{partCounter}[questionCounter]
\newenvironment{question}[2][\arabic{questionCounter}]{%
    \setcounter{partCounter}{0}%
    \vspace{.25in} \hrule \vspace{0.5em}%
        \noindent{\bf #2}%
    \vspace{0.8em} \hrule \vspace{.10in}%
    \addtocounter{questionCounter}{1}%
}{}
\renewenvironment{part}[1][\alph{partCounter}]{%
    \addtocounter{partCounter}{1}%
    \vspace{.10in}%
    \begin{indented}%
       {\bf (#1)} %
}{\end{indented}}

%%%%%%%%%%%%%%%%%%%%%%%HEADER%%%%%%%%%%%%%%%%%%%%%%%%%%%%%%
\newcommand{\myname}{Shashank Singh}
\newcommand{\myandrew}{sss1@andrew.cmu.edu}
\newcommand{\myclass}{21-720 Measure and Integration}
\newcommand{\myhwnum}{2}
\newcommand{\duedate}{Wednesday, September 26, 2012}
%%%%%%%%%%%%%%%%%%%%%%%%%%%%%%%%%%%%%%%%%%%%%%%%%%%%%%%%%%%

%%%%%%%%%%%%%%%%%%%%CONTENT MACROS%%%%%%%%%%%%%%%%%%%%%%%%%
\renewcommand{\qed}{\quad $\blacksquare$}
\newcommand{\mqed}{\quad \blacksquare}
\newcommand{\inv}{^{-1}}
\newcommand{\bx}{\mathbf{x}}
\newcommand{\bq}{\mathbf{q}}
\newcommand{\bv}{\mathbf{v}}
\newcommand{\by}{\mathbf{y}}
\newcommand{\bff}{\mathbf{f}}
\newcommand{\bzero}{\mathbf{0}}
\newcommand{\balpha}{\boldsymbol{\alpha}}
\newcommand{\boldeta}{\boldsymbol{\eta}}
\newcommand{\dist}{\operatorname{dist}}
\newcommand{\area}{\operatorname{area}}
\newcommand{\sminus}{\backslash}
\newcommand{\N}{\mathbb{N}} % natural numbers
\newcommand{\Z}{\mathbb{Z}} % natural numbers
\newcommand{\Q}{\mathbb{Q}} % rational numbers
\newcommand{\R}{\mathbb{R}} % real numbers
\newcommand{\pow}[1]{\mathcal{P}\left(#1\right)} % power set of #1
\newcommand{\LR}[1]{\mathcal{L}\left(\R^{#1}\right)} % Lebesgue measurable sets of \R^{#1}
\newcommand{\B}{\mathcal{B}} % Borel family of sets
%%%%%%%%%%%%%%%%%%%%%%%%%%%%%%%%%%%%%%%%%%%%%%%%%%%%%%%%%%%

\begin{document}
\thispagestyle{plain}

{\Large Homework \myhwnum} \\
\myclass \\
Name: \myname \\
Email: \myandrew \\
Due: \duedate \\
\begin{question}{Problem 1}
Let $\mathcal{V} := \{V \in \pow{X} | V \mbox{ open }, \mu(V) = 0\}$ and let
$U = \bigcup_{V \in \mathcal{V}} V$. Since a topology is closed under
arbitrary unions, $U$ is open. Furthermore, it is immediate from the
definition of $U$ that, if $V \subseteq X$ is open with $\mu(V) = 0$, then
$V \subseteq U$. Thus, it remains only to show that $\mu(U) = 0$.

Suppose $K \subseteq U$ is compact. Since $\mathcal{V}$ is an open cover of
$K$, by definition of compactness, there exists a finite subcover
$\{V_1,V_2,\ldots,V_k\} \subseteq \mathcal{V}$ of $K$. By definition of
$\mathcal{V}$, we have, for each $i$, $\mu(V_i) = 0$, so that, by monotonicity
and countable subadditivity,
\[\mu(K)
 \leq \mu\left( \bigcup_{i = 1}^k V_i \right)
 \leq \sum_{i = 1}^k \mu(V_i)
 = 0.\]
Thus, since $\mu$ is regular,
\[\mu(U) = \sup\{\mu(K) | K \subseteq U, K \mbox{ compact}\} = 0. \mqed\]
\end{question}

TODO
\begin{question}{Problem 2}
\end{question}

\begin{question}{Problem 3}
\begin{enumerate}[(a)]
\item Since $C$ is a $\pi$-system and, $\forall i \in \N$,
$\mathcal{C} \subseteq \Lambda_i
 := \{S \in \Sigma | \mu(S \cap C_i) = \nu(S \cap C_i)\}$, by a theorem proven
in class, to show that $\sigma(\mathcal{C}) \subseteq \Lambda_i$, it is
sufficient to show that $\Lambda_i$ is a $\lambda$-system.

Since $C_i \in \mathcal{C}$,
$\mu(X \cap C_i) = \mu(C_i) = \nu(C_i) = \nu(X \cap C_i)$, so that
$X \in \Lambda_i$.

Suppose $A,B \in \Lambda$ with $A \subseteq B$. Since $A \subseteq B$,
$B \cap A = A$, so that, since $A,B \in \Lambda$ (noting that, since
$\mu(C_i) < \infty$, the below measures are all finite),
\[\mu( (B \sminus A) \cap C_i)
 = \mu(B \cap C_i) - \mu(A \cap C_i)
 = \nu(B \cap C_i) - \nu(A \cap C_i)
 = \nu( (B \sminus A) \cap C_i),\]
and thus $B \sminus A \in \Lambda$.

Suppose that $A_1,A_2,\ldots \in \Lambda$ with $A_i \subseteq A_{i + 1}$.
Then, $\{A_i \cap C_i\}$ increasing sequence, so that, for
$A := \bigcup_{i = 1}^{\infty} A_i$, since each
$\mu(A_i \cap C_i) = \nu(A_i \cap C_i)$
\[\mu(A \cap C_i)
 = \mu\left( \bigcup_{i = 1}^{\infty}A_i \cap C_i \right)
 = \lim_{i \rightarrow \infty} \mu(A_i \cap C_i)
 = \lim_{i \rightarrow \infty} \nu(A_i \cap C_i)
 = \nu\left( \bigcup_{i = 1}^{\infty}A_i \cap C_i \right)
 = \nu(A \cap C_i),
\]
so that $A \in \Lambda$. Therefore, $\Lambda$ is a $\lambda$-system, so that,
as discussed above, $\forall A \in \sigma(\mathcal{C})$,
$\mu(A \cap C_i) = \nu(A \cap C_i)$.

We now show that, $\mu = \nu$ on $\sigma(\mathcal{C})$. Suppose
$A \in \sigma(\mathcal{C})$. $\forall i \in \N$, let
\[A_i
 := \left( A \cap \bigcap_{k = 1}^{i - 1} C_k^c \right) \cap C_i
 \in \sigma(\mathcal{C}),\] so that $A_i$'s partition $A$ and
$\mu(A_i) = \nu(A_i)$. Then, by countable additivity,
\[\mu(A)
 = \mu\left( \bigcup_{i = 1}^{\infty} A_i \right)
 = \sum_{i = 1}^{\infty} \mu(A_i)
 = \sum_{i = 1}^{\infty} \nu(A_i)
 = \nu\left( \bigcup_{i = 1}^{\infty} A_i \right)
 = \nu(A). \mqed
\]

\item Suppose $X = \{velociraptor,bunny\}$, $\mathcal{C} = \{\{bunny\},X\}$,
$\Sigma = \pow{X}$. Note that $\mathcal{C}$ is closed under intersections and
thus $\mathcal{C}$ is a $\pi$-system, and that
$\sigma(\mathcal{C}) = \pow{X}$. If $\mu,\nu$ are the measures defined by
$\mu(\{bunny\}) = \nu(\{bunny\}) = \infty$, $\mu(\{velociraptor\}) = 0$,
$\nu(\{velociraptor\}) = 1$, then, $\mu = \nu$ on $\mathcal{C}$, but $\mu$ is
not $\nu$ on $\sigma(\mathcal{C})$, and, furthermore, for
$X = C_1 = C_2 = \ldots$, $\bigcup_{i = 1}^{\infty} C_i = X$. Thus, the
condition $\mu(C_i) < \infty$ is necessary.
\end{enumerate}
\end{question}

\begin{question}{Problem 4}
Suppose, for sake of contradiction, that there existed some
$\kappa \in (0,1)$, $E \in \LR{1}$ such that, $\forall a < b \in (0,1)$,
\[\kappa (b - a) \leq \lambda(E \cap (a,b)) \leq (1 - \kappa)(b - a).\]
Let $\epsilon = \frac{\kappa}{2}$.

By regularity of the Lebesgue measure, $\forall n,i \in \N$ with
$0 \leq i < 2^n$, there exists an open set $U_{n,i}$ with
\[\lambda(U_{n,i})
 \leq \lambda\left(
        E \cap
          \left( \frac{i}{2^n}, \frac{i + 1}{2^n} \right)
      \right)
    + \frac{\epsilon}{2^{2n}}\]
and $E \subseteq U \subseteq \left( \frac{i}{2^n}, \frac{i + 1}{2^n} \right)$ (we
can ensure the second inclusion by replacing $U_{n,i}$ with
$U_{n,i} \cap \left( \frac{i}{2^n}, \frac{i + 1}{2^n} \right)$).

Let $U := \bigcup_{n = 1}^{\infty} \bigcup_{i = 1}^{2^n} U_{n,i}$. Then, for
any interval $I_{n,i} = \left( \frac{i}{2^n}, \frac{i + 1}{2^n} \right)$, 

\begin{align*}
\lambda(U \cap I)
 & = \lambda(U \cap E \cap I) + \lambda(U \cap E^c \cap I)
                                     & \mbox{(since $E \in \LR{d}$)}        \\
 & = \lambda(E \cap I)
   + \lambda\left(
       \bigcup_{m = 1}^{\infty}
         \bigcup_{
           \left( \frac{j}{2^m}, \frac{j + 1}{2^m} \right)
             \subseteq
           \left( \frac{i}{2^n}, \frac{i + 1}{2^n} \right)}
       U_{m,j} \sminus E
     \right)
                                     & \mbox{(since $E \subseteq U$)}       \\
 & = \lambda(E) + \sum_{m = 1}^{\infty}
         \sum_{
           \left( \frac{j}{2^m}, \frac{j + 1}{2^m} \right)
             \subseteq
           \left( \frac{i}{2^n}, \frac{i + 1}{2^n} \right)}
        \lambda(U_{m,j} \sminus E)
                                     & \mbox{(by countable subadditivity)}  \\
 & = \lambda(E) + \sum_{m = 1}^{\infty} \frac{\epsilon}{2^{m + n}} \\
 & < \frac{1 - \kappa + \epsilon}{2^n}
   = \frac{1 - \kappa/2}{2^n}
   < \lambda(I_{n,i}).
\end{align*}
Thus, since every interval $(a,b) \subseteq (0,1)$ contains some $I_{n,i}$,
$\lambda(U \cap (a,b)) < b - a$.

Since $\lambda(U) \geq \lambda(E) > 0$, $U$ is nonempty. Since $U$ is a union
of open sets, $U$ itself is an open set, so that it contains some interval
$(a,b)$. But then, $\lambda(U \cap (a,b)) = b - a$, which is a contradiction.
\qed
\end{question}

%TODO
\begin{question}{Problem 5}
\begin{enumerate}[(a)]
\item Define
\[\Sigma
 = \{A \in \pow{X_1 \times X_2} |
     \forall x_1 \in X_1, x_2 \in X_2,
       S_{x_1}(A) \in \Sigma_2, S_{x_2}(A) \in \Sigma_1\}.
\]
Then, to show the desired result, it is sufficient to show that $\Sigma$ is a
$\sigma$-algebra and that
$\{A_1 \times A_2 | A_i \in \Sigma_i\} \subseteq \Sigma$.

If $A_1 \in \Sigma_1, A_2 \in \Sigma_2$, then, by definition of the Cartesian
product $\forall x_1 \in X_1,x_2 \in X_2$, ${S_{x_1}(A_1 \times A_2) = A_2}$ and
$S_{x_2}(A_1 \times A_2) = A_1$, so that $A_1 \times A_2 \in \Sigma$, and thus
${\{A_1 \times A_2 | A_i \in \Sigma_i\} \subseteq \Sigma}$. Note that this
implies $\emptyset \in \Sigma$.

Suppose $A \in \Sigma$. Then, $\forall x_1 \in X_1,x_2 \in X_2$, since
$S_{x_1}(A^c) = S_{x_1}(A)^c \in \Sigma_2$ and
${S_{x_2}(A^c) = S_{x_2}(A)^c \in \Sigma_1}$, $A^c \in \Sigma$.

Suppose $A_1,A_2,\ldots \in \Sigma$. Then, $\forall x_1 \in X_1,x_2 \in X_2$,
since $S_{x_1}(A_i) \in \Sigma_2$ and $S_{x_1}(A_i) \in \Sigma_1$,
\[S_{x_1}\left( \bigcup_{i = 1}^{\infty} A_i \right)
 = \bigcup_{i = 1}^{\infty} S_{x_1}(A_i) \in \Sigma_2
    \mbox{ \quad and \quad }
  S_{x_2}\left( \bigcup_{i = 1}^{\infty} A_i \right)
 = \bigcup_{i = 1}^{\infty} S_{x_2}(A_i) \in \Sigma_1,
\]
so that $\bigcup_{i = 1}^{\infty} A_i \in \Sigma$. Since $\Sigma$ is nonempty
and closed under complements and countable unions, $\Sigma$ is a
$\sigma$-algebra. \qed

%TODO
\item

%TODO
\item Let
\[\mathcal{A}
      := \left\{\bigcup_{i = 1}^{\infty} A_i \times B_i |
                A_i \in \Sigma_1, B_1 \in \Sigma_2,
                (A_i \times B_i)\mbox{'s pairwise disjoint}\right\},\]
and define,
$\forall A = \bigcup_{i = 1}^{\infty} A_i \times B_i \in \mathcal{A}$
\[\nu \left( \bigcup_{i = 1}^{\infty} A_i \times B_i \right)
 := \sum_{i = 1}^{\infty} \mu_1(A_i)\mu_2(B_i).\]

We show that $\mathcal{A}$ is an algebra, that $\mu_0$ is a well-defined
pre-measure on $\mathcal{A}$, and that $\forall A_i \in \Sigma_i$,
$\nu(A_1 \times A_2) = \mu_1(A_1)\mu_2(A_)$. Then, noting that
$\sigma(\mathcal{A}) = \Sigma_1 \otimes \Sigma_2$, by the Caratheodory
Extension Theorem (the result of Problem 6 on Homework 2) ensures the
existence of a measure on $(X_1 \times X_2,\Sigma_1 \otimes \Sigma_2)$ with
the desired properties.

The empty union, $\emptyset \in \mathcal{A}$.
Since a countable union of countable unions of $(A_i \times B_i)$'s is itself
a countable union of $(A_i \times B_i)$'s, $\mathcal{A}$ is closed under
countable unions.
If $A_i \in \Sigma_1,B_i \in \Sigma_2$, then
$(A_i \times B_i)^c = (X_2 \times A_i^c) \cup (X_1 \times B_i^c)$. Thus, since
intersections distribute over Cartesian products and unions,
\begin{align*}
\left( \bigcup_{i = 1}^{\infty} A_i \times B_i \right)^c
 & =   \bigcap_{i = 1}^{\infty} (A_i \times B_i)^c
   =   \bigcap_{i = 1}^{\infty} (X_2 \times A_i^c) \cup (X_1 \times B_i^c) \\
 & =   \left( X_2 \times \bigcap_{i = 1}^{\infty} A_i^c \right)
         \cup
       \left( X_1 \times \bigcap_{i = 1}^{\infty} B_i^c \right)
   \in \mathcal{A},
\end{align*}
since $\Sigma_1$ and $\Sigma_2$ are $\sigma$-algebras and thus closed under
complements and countable intersections (note that we can ``disjointify'' the
above set by replacing each term with a set difference with all terms of lower
indices). Therefore, $\mathcal{A}$ is closed under complements and thus an
algebra.

To show that $\nu$ is well-defined, it suffices to observe that, for
\[\bigcup_{i = 1}^{\infty} A_i \times B_i
 = \bigcup_{i = 1}^{\infty} C_i \times D_i \in \mathcal{A}\]
(with $(A_i \times B_i)$'s pairwise disjoint and $(C_i \times D_i)$'s pairwise
disjoint),
\begin{align*}
\nu\left( \bigcup_{i = 1}^{\infty} A_i \times B_i \right)
 & = \sum_{i = 1}^{\infty} \mu_1(A_i)\mu_2(B_i) \\
 & = \sum_{i = 1}^{\infty}
     \sum_{j = 1}^{\infty} \mu_1(A_i \cap C_i)\mu_2(B_i \cap D_i)
                                & \mbox{(since the sets are disjoint sets)} \\
 & = \sum_{j = 1}^{\infty}
     \sum_{i = 1}^{\infty} \mu_1(A_i \cap C_i)\mu_2(B_i \cap D_i)
                                & \mbox{(since $\mu_1,\mu_2 \geq 0$)} \\
 & = \sum_{j = 1}^{\infty} \mu_1(C_j)\mu_2(D_i)
   = \nu\left( \bigcup_{i = 1}^{\infty} C_i \times D_i \right).
\end{align*}

It is immediate from the definition of $\nu$ that $\nu(\emptyset) = 0$ and
that $\nu$ is countably additive on $\mathcal{A}$ where necessary.

\end{enumerate}
\end{question}

\begin{question}{Problem 6}
Suppose $\mathbb{M} \in \pow{\pow{\pow{X}}}$ is a set of monotone classes,
and suppose that $\mathcal{M} = \bigcap_{M \in \mathbb{M}} M$. Suppose,
$\forall i \in \N$, $A_i,B_i \in \mathcal{M}$ with $A_i \subseteq A_{i + 1}$
and $B_i \supseteq B_{i + 1}$, so that $\forall M \in \mathbb{M}$,
$A_i,B_i \in M$. Then, by definition of a monotone class,
$\forall M \in \mathbb{M}$, $A := \bigcup_{i = 1}^{\infty} A_i \in M$ and
$B := \bigcap_{i = 1}^{\infty} B_i \in M$, so that $A,B \in \mathcal{M}$, and
thus $\mathcal{M}$ is a monotone class. Thus, we can safely define the
smallest monotone class $\mathcal{M}$ containing $\mathcal{A}$ as the
intersection of all monotone classes containing $\mathcal{A}$ (clearly, this
intersection contains $\mathcal{A}$).

Since any $\sigma$-algebra is closed under countable unions and intersections,
$\sigma(\mathcal{A})$ is a monotone class containing $\mathcal{A}$, so that,
since $\mathcal{M}$ is the \emph{smallest} monotone class containing
$\mathcal{A}$, $\mathcal{M} \subseteq \sigma(\mathcal{A})$.

Since any nonempty monotone class that is closed under complements is a
$\sigma$-algebra and $\emptyset \in \mathcal{A} \subseteq \mathcal{M}$, to
show that $\sigma(A) \subseteq \mathcal{M}$, it is sufficient to show that
$\mathcal{M}$ is closed under complements.

Let $\mathcal{M}_c := \{A \in \mathcal{M} | A^c \in \mathcal{M}\}$. To
show that $\mathcal{M}$ is closed under complements, it is sufficient to show
that $\mathcal{M}_c$ is a monotone class containing $\mathcal{A}$, since then
$\mathcal{M} \subseteq \mathcal{M}_c$.

Since $\mathcal{A} \subseteq \mathcal{M}$ is an algebra and thus closed under
complements, clearly $\mathcal{M}_c$ contains $\mathcal{A}$.
Suppose
$A_1,A_2,\ldots \in \mathcal{M}_c$ with $A_i \subseteq A_{i + 1}$. Then,
$A_1^c,A_2^c,\ldots \in \mathcal{M}$, and $A_i^c \supseteq A_{i + 1}^c$.
Thus, since $\mathcal{M}$ is a monotone class,
\[\left( \bigcup_{i = 1}^{\infty} A_i \right)^c
 = \bigcap_{i = 1}^{\infty}A_i^c \in \mathcal{M},\]
so that $\bigcup_{i = 1}^{\infty} A_i \in \mathcal{M}_c$.
Similarly, if $B_1,B_2,\ldots \in \mathcal{M}_c$ with
$B_i \supseteq B_{i + 1}$, then
$B_1^c,B_2^c,\ldots \in \mathcal{M}$, and $B_i^c \subseteq B_{i + 1}^c$, so
that, since $\mathcal{M}$ is a monotone class,
\[\left( \bigcap_{i = 1}^{\infty} B_i \right)^c
 = \bigcup_{i = 1}^{\infty}B_i^c \in \mathcal{M},\]
and thus $\bigcap_{i = 1}^{\infty} B_i \in \mathcal{M}_c$.
Therefore, $\mathcal{M}_c$ is a monotone class containing $\mathcal{A}$. \qed
\end{question}

\end{document}
