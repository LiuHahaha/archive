\documentclass[11pt]{article}
\usepackage{enumerate}
\usepackage{fullpage}
\usepackage{fancyhdr}
\usepackage{amsmath, amsfonts, amsthm, amssymb}
\usepackage{color}
\setlength{\parindent}{0pt}
\setlength{\parskip}{5pt plus 1pt}
\pagestyle{empty}

\def\indented#1{\list{}{}\item[]}
\let\indented=\endlist

\newcounter{questionCounter}
\newcounter{partCounter}[questionCounter]
\newenvironment{question}[2][\arabic{questionCounter}]{%
    \setcounter{partCounter}{0}%
    \vspace{.25in} \hrule \vspace{0.5em}%
        \noindent{\bf #2}%
    \vspace{0.8em} \hrule \vspace{.10in}%
    \addtocounter{questionCounter}{1}%
}{}
\renewenvironment{part}[1][\alph{partCounter}]{%
    \addtocounter{partCounter}{1}%
    \vspace{.10in}%
    \begin{indented}%
       {\bf (#1)} %
}{\end{indented}}

%%%%%%%%%%%%%%%%%%%%%%%HEADER%%%%%%%%%%%%%%%%%%%%%%%%%%%%%%
\newcommand{\myname}{Shashank Singh}
\newcommand{\myandrew}{sss1@andrew.cmu.edu}
\newcommand{\myclass}{21-720 Measure and Integration}
\newcommand{\myhwnum}{1}
\newcommand{\duedate}{Wednesday, September 12, 2012}
%%%%%%%%%%%%%%%%%%%%%%%%%%%%%%%%%%%%%%%%%%%%%%%%%%%%%%%%%%%

%%%%%%%%%%%%%%%%%%%%CONTENT MACROS%%%%%%%%%%%%%%%%%%%%%%%%%
\renewcommand{\qed}{\quad $\blacksquare$}
\newcommand{\mqed}{\quad \blacksquare}
\newcommand{\inv}{^{-1}}
\newcommand{\bv}{\mathbf{v}}
\newcommand{\bx}{\mathbf{x}}
\newcommand{\by}{\mathbf{y}}
\newcommand{\bff}{\mathbf{f}}
\newcommand{\bzero}{\mathbf{0}}
\newcommand{\bxi}{\boldsymbol{\xi}}
\newcommand{\boldeta}{\boldsymbol{\eta}}
\newcommand{\dist}{\operatorname{dist}}
\newcommand{\area}{\operatorname{area}}
\newcommand{\sminus}{\backslash}
%%%%%%%%%%%%%%%%%%%%%%%%%%%%%%%%%%%%%%%%%%%%%%%%%%%%%%%%%%%

\begin{document}
\thispagestyle{plain}

{\Large Homework \myhwnum} \\
\myclass \\
Name: \myname \\
Email: \myandrew \\
Due: \duedate \\
\begin{question}{Problem 1}
\begin{enumerate}[(a)]
\item Let $S \subseteq \mathbb{R}^d$ be a countable set, so that there exists
a bijection $f: S \rightarrow \mathbb{N}$. Let $\epsilon > 0$, and,
$\forall i \in \mathbb{N}$, let $R_i$ be the closed hypercube of sidelength
$\sqrt[n]{\epsilon2^{-i}}$ centered at $f\inv (i)$, so that $R_i$ has
$d$-dimensional volume $\ell(R_i) = \epsilon2^{-i}$.

Since,
$\forall x \in S$, $x \in R_{f(x)}$, and, since $f$ is a bijection,
\[S \subseteq \bigcup_{x \in S} R_{f(x)} = \bigcup_{i = 1}^{\infty} R_i,\]
and, clearly, each $R_i$ is a cell. Thus, by definition of the Lebesgue outer
measure,
\[m^*(S)
 \leq \sum_{i = 1}^{\infty} \ell(R_i)
 = \epsilon \sum_{i = 1}^{\infty} 2^{-i}
 = \epsilon.
\]
Since this holds for all $\epsilon > 0$, \fbox{$m^*(S) = 0$.}

\item $\forall i \in \mathbb{N}$, let $C_i$ be the
following recursively defined family of sets:
\begin{eqnarray*}
C_0 & = & [0,1] \\
C_i
 & = & \frac{C_{i - 1}}{3} \cup \left(\frac23 + \frac{C_{i - 1}}{3}\right)
\end{eqnarray*}
(where addition and multiplication of sets by real numbers denote translation
and dilation, respectively).
Let $C$ denote the Cantor set. It can be shown inductively
$\forall i \in \mathbb{N}\cup\{0\}$, that $C \subseteq C_i$ and that
\[C_i = \bigcup_{k = 1}^{2^i} I_k,\]
where each $I_k$ is a cell of length $\ell(I_i) = \left(\frac13\right)^i$.
Thus, by definition of the Lebesgue outer measure, $\forall i \in \mathbb{N}$.
\[m^*(C)
 \leq \sum_{i = 1}^{2^i} \left(\frac13\right)^i
 = \left(\frac23\right)^i.
\]
Since this is true for all $i$, \fbox{$m^*(C) = 0$.}

\item For notational convenience, let
$S = \{x \in [0,1] | x \not \in \mathbb{Q}\}$.
Since $\mathbb{Q}$ is countable, it follows from the result of part (a)
that $m^*(\mathbb{Q}) = 0$. Since $[0,1]$ is a cell, $m^* = 1 - 0 = 1$.
By sub-additivity of the Lebesgue outer measure,
\[1 = m^*([0,1]) \leq m^*([0,1] \cap \mathbb{Q}) + m^*(S) = m^*(S).\]
By monotonicity, $m^*(S) \leq m^*([0,1]) = 1$. Therefore, \fbox{$m^*(S) = 1$.}
\end{enumerate}
\end{question}

\begin{question}{Problem 2}
\begin{enumerate}[(a)]
\item Any subspace $V$ of $\mathbb{R}^d$ is the image of the subspace $W$
whose basis is the set of the first $n := \dim(V)$ canonical basis vectors of
$\mathbb{R}^d$, under some rotation. Thus, since all rotations are
orthonormal transformations, by the result of part (b) of problem 3, it
suffices to show that $\lambda(W) = 0$.

Let $\epsilon > 0$. Since $\mathbb{N}^n$ is countable, let
$\bff: \mathbb{N} \rightarrow \mathbb{N}^n$ be a bijection.

Then, $\forall i \in \mathbb{N}$, let
\begin{align*}
I_i = 
&        (f_1(i),f_1(i) + 2)
  \times (f_2(i),f_2(i) + 2)
  \times \cdots
  \times (f_n(i),f_n(i) + 2)                                              \\
& \times \left(-\frac{\epsilon}{2^{n + 1 + i}},
                                    \frac{\epsilon}{2^{n + 1 + i}}\right)
  \times \left(-\frac12,\frac12\right)
  \times \cdots
  \times \left(-\frac12,\frac12\right)
\subseteq \mathbb{R}^d,
\end{align*}
where $f_1(i),\ldots,f_n(i)$ are the components of $\bff(i)$.

Since each $I_i$ is a cell, $\lambda(I_i) = \ell(I_i) = \frac{\epsilon}{2^i}$.
Thus, by monotonicity and then by subadditivity,
\[\lambda(V)
 \leq \lambda\left(\bigcup_{i = 1}^{\infty} I_i \right)
 \leq                 \sum_{i = 1}^{\infty} \lambda(I_i)
 =    \epsilon        \sum_{i = 1}^{\infty} \frac{1}{2^i}
 = \epsilon.
\]
Since this holds for all $\epsilon > 0$, $\lambda(V) = 0$. \qed

\item
Since each segment of the boundary $\partial P$ is a subset of a line, which
is the translation of $1$-dimensional subspace of
$\mathbb{R}^2$, by monotonicity, translational invariance of $\lambda$, and
the result of part (a), each segment of $\partial P$ has Lebesgue measure $0$.
Since there finitely many boundary segments, by subadditivity,
$\lambda(\partial P) = 0$.

Since any triangle is the image of some triangle having at least $1$
side parallel to the $x$-axis, under a rotation, by the result of part (b) of
problem 3, to show the desired result in the case that $P$ is a triangle, we
can assume that $P$ has some side parallel to the $x$-axis. Let $I$ be the
smallest open cell containing $P$, so that $I$ has the same height and base
length as $P$. Let $v$ be the vertex of $P$ that is not on the side parallel
to the $x$-axis, and let $I_1$ and $I_2$ be the two cells into which $I$ is
split by the vertical line going through $v$. Then, $P_1 := P \cap I_1$ is a rotation
of $R_1 := (P^c \cap I_1)\sminus(\partial P)$, and $P_2 := P \cap I_2$ is a rotation
of $R_2 := (P^c \cap I_2)\sminus(\partial P)$. Therefore, by the result of
part(b) of problem 3,
$\lambda(P_1) = \lambda(R_1)$ and
$\lambda(P_2) = \lambda(R_2)$.
Since $P_1$ and $P_2$ are open and the intersection of their boundaries is closed,
\begin{align*}
\lambda(P_1) + \lambda(R_1) + \lambda(\partial P_1 \cap \partial R_1) = \lambda(I_1),\\
\lambda(P_2) + \lambda(R_2) + \lambda(\partial P_2 \cap \partial R_2) = \lambda(I_2).
\end{align*}
Then, since, as explained above,
$\lambda(\partial P_1 \cap \partial R_1) = \lambda(\partial P_2 \cap \partial R_2) = 0$.
\begin{align*}
\lambda(P_1) + \lambda(P_1) = \lambda(P_1) + \lambda(R_1) = \lambda(I_1),\\
\lambda(P_2) + \lambda(P_2) = \lambda(P_2) + \lambda(R_2) = \lambda(I_2).
\end{align*}
Thus, $\lambda(P_1) = \frac12\lambda(I_1)$ and $\lambda(P_2) = \frac12\lambda(I_2)$,
so that
\[\lambda(P) = \lambda(P_1) + \lambda(P_2) = \frac12(\lambda(I_1) + \lambda(I_2)) = \frac12\lambda(I).\]
Therefore, the Lebesgue measure of a triangle is its area.

Any polygon $P$ can be written as the union of finitely many triangles
$T_1,T_2,\ldots,T_k$, which are disjoint except perhaps at their boundaries.
Thus,
\begin{align*}
\lambda(P)
 & = \lambda\left(\bigcup_{i = 1}^k T_i\right)
   = \lambda\left(\bigcup_{i = 1}^k T_i^{\circ} \cup \partial T_i\right) & \\
 & = \sum_{i = 1}^k \lambda(T_i^{\circ})
   + \sum_{i = 1}^k \lambda(\partial T_i)                                &
                  \mbox{(since $T_i^{\circ}$ open, $\partial T_i$ closed)} \\
 & = \sum_{i = 1}^k \lambda(T_i^{\circ})
   = \sum_{i = 1}^k \area(T_i)
   = \area(P).
\end{align*}

\end{enumerate}
\end{question}

\begin{question}{Problem 3}
\begin{enumerate}[(a)]
\item Let $c = \mu(I)  \in [0,\infty)$, where $I = [0,1)^d$. Divide $I$ into $k^n$ half-open
hypercubes $C_1,\ldots,C_{k^n}$ of sidelength $1/k$. Note that, since each
cube is a translation of every other cube, each cube has the same $\mu$
measure, so that, since half-open cells are in $\mathcal{L}$,
\[k^n\mu(C_1)
 = \sum_{i = 1}^{k^n} \mu(C_1)
 = \sum_{i = 1}^{k^n} \mu(C_i)
 = \mu\left(\bigcup_{i = 1}^{k^n} C_i\right)
 = \mu(I) = c.
\]
Then, $\mu(C_1) = \frac{c}{k^n} = c\lambda(C_1)$.

By monotonicity, if $S = (a_1,b_1) \times \cdots \times (a_d,b_d)$ is a cell,
then there exists a covering family $\mathcal{C}$ of $S$ with cubes of
sidelength $1/k$ such that
\[\sum_{i = 1}^d (1/k)(b_m - a_m - 1/k)^{(d - 1)},
 \leq \lambda(\cup \mathcal{C}) - \lambda(S)
 \leq \sum_{i = 1}^d (1/k)(b_m - a_m)^{(d - 1)},
\]
where $m = \operatorname{argmax}_m (b_m - a_m)$. Since this holds for all
$k \in \mathbb{N}$ and $\mu$ agrees with $\lambda$ on each cube of sidelength
$1/k$, by monotonicity, $\mu(S) = c\lambda(S)$, so that we have shown that
$\mu = c\lambda$ on half-open cells.

Suppose now that $S$ is any bounded set in $\mathcal{L}$. For any cover of $S$ with countably
many disjoint cells $C_1,C_2,\ldots$, there exists a cover of $S$ with
countably many disjoint half-open cells $H_1,H_2,\ldots,$ such that
\[\sum_{i = 1}^{\infty} c\lambda(H_i)
\left(\leq \sum_{i = 1}^{\infty} c\lambda(C_i)\right) + \epsilon,\]
(we can cover each cell $C_i$ with a half-open cell of sidelength at most
$\frac{\sqrt[d]{\epsilon}}{2^i}$ larger than the sidelength of $C$).
Then, by countable additivity,
\[\mu\left(\bigcup_{i = 1}^{\infty} H_i\right)
= \sum_{i = 1}^{\infty} \mu(H_i)
\leq \left(\sum_{i = 1}^{\infty} c\lambda(H_i)\right) + \epsilon
= \lambda\left(\bigcup_{i = 1}^{\infty} C_i\right) + \epsilon.\]
Then, taking the infimum over all covers $C_1,C_2,\ldots$ on both sides,
\[\mu(S) \leq c\lambda(S) + \epsilon.\]
Since this is true for all $\epsilon > 0$, $\mu(S) \leq c \lambda(S)$.
However, by the same argument, if $I$ is a half-open cell containing $S$ (such
an $I$ must exist since $S$ is bounded), then $\mu(I\sminus S) \leq c\lambda(I\sminus S)$
so that $\mu(I) - \mu(S) = c\lambda(I) - \mu(S) \leq c\lambda(I) - c\lambda(S)$, and thus
$\mu(S) \geq c\lambda(S)$. Thus, $\mu = c\lambda$ on all bounded sets in $\mathcal{L}$.

Now, letting $S$ be any set in $\mathcal{L}$, $S$ can be written as the union
of disjoint bounded sets $S_i$ in $\mathcal{L}$ (since $\mathbb{R}^d$ can be
covered by countably many half-open, disjoint cubes $C_1,C_2,\ldots$ of unit
sidelength, we can just take $S_i = C_i \cap S$), by countable additivity,
\[\mu(S)
 = \sum_{i = 1}^{\infty} \mu(S_i)
 = \sum_{i = 1}^{\infty} c\lambda(S_i)
 = c\lambda(S).\mqed\]

\item Since any orthogonal linear transformation is a composition of rotations
and reflections, $\forall \delta > 0$,
$\forall B(x,r) \in \mathcal{E}_{\delta}$ (where $\mathcal{E}_{\delta}$ is as
defined in part (B) of problem 4),
$T(B(x,r)) = B(T(x),r) \in \mathcal{E}_{\delta}$, so that,
$\rho(T(B(x,r))) = \rho(B(x,r))$.
Therefore,
\begin{align*}
H_{\alpha}(A)
 & = \lim_{\delta \rightarrow 0+} \inf
        \left\{\sum_{i = 1}^{\infty} \rho(B_i) \right|
        \left.B_i \in \mathcal{E}_{\delta},
            A \subseteq \bigcup_{i = 1}^{\infty} B_i \right\}       \\
 & = \lim_{\delta \rightarrow 0+} \inf
        \left\{\sum_{i = 1}^{\infty} \rho(T(B_i)) \right|
        \left.T(B_i) \in \mathcal{E}_{\delta},
            T(A) \subseteq \bigcup_{i = 1}^{\infty} T(B_i) \right\} \\
 & = \lim_{\delta \rightarrow 0+} \inf
        \left\{\sum_{i = 1}^{\infty} \rho(B_i) \right|
        \left.B_i \in \mathcal{E}_{\delta},
            T(A) \subseteq \bigcup_{i = 1}^{\infty} B_i \right\}
   = H_{\alpha}(T(A)).
\end{align*}
Thus, by the result of part (c) of problem 4, $\exists c \in (0,\infty)$ such
that
\[\lambda(T(A))
 = \frac{1}{c}H_d(T(A))
 = \frac{1}{c}H_d(A)
 = \lambda(A). \mqed\]
\end{enumerate}
\end{question}


\begin{question}{Problem 4}
\begin{enumerate}[(a)]
\item Clearly, $\mu: [0,1] \rightarrow [0,\infty]$, and since
$\emptyset \in \mathcal{E}$ and $\rho(\emptyset) = 0$, $\mu^*(\emptyset) = 0$.

% Monotonicity
Suppose $E \subseteq F \subseteq X$. If $E_1,E_2,\ldots \in \mathcal{E}$ such
that $F \subseteq \bigcup_{i = 1}^{\infty} E_i$, then
$E \subseteq \bigcup_{i = 1}^{\infty} E_i$, so that
\[\left\{
    \sum_{i = 1}^{\infty} \rho(E_i) \right|\left. E_i \in \mathcal{E},
        F \subseteq \bigcup_{i = 1}^{\infty} E_i
  \right\}
 \subseteq
 \left\{
    \sum_{i = 1}^{\infty} \rho(E_i) \right|\left. E_i \in \mathcal{E},
        E \subseteq \bigcup_{i = 1}^{\infty} E_i
 \right\}.
\]
Then, taking the infimum on both sides gives $\mu^*(E) \leq \mu^*(F)$, so that
$\mu^*$ is monotonic.

% Countable Subadditivity
Suppose $A_1,A_2,\ldots \subseteq X$. If
$\sum_{i = 1}^{\infty} \mu^*(A_i) = \infty$, then the inequality
\[m^*\left(\bigcup_{i = 1}^{\infty} A_i\right)
 \leq \sum_{i = 1}^{\infty}m^*(A_i)\]
trivially holds. Thus, we suppose $\sum_{i = 1}^{\infty} \mu^*(A_i) < \infty$.
Let $\epsilon > 0$. Since $\mu^*$ is an infimum, $\forall i \in \mathbb{N}$,
there is a family $\mathcal{G}_i \subseteq \mathcal{E}$ such that
$A_i \subseteq \bigcup \mathcal{G}_i$ and
\[\sum_{E \in \mathcal{G}_i} \mu^*(E)
 \leq \mu^*(A_i) + \frac{\epsilon}{2^i}.\]
Taking the sum over all $i \in \mathbb{N}$ gives
\[\sum_{i = 1}^{\infty} \sum_{E \in \mathcal{G}_i} \mu^*(E)
 \leq \sum_{i = 1}^{\infty} \mu^*(A_i) + \frac{\epsilon}{2^i}
 = \epsilon + \sum_{i = 1}^{\infty} \mu^*(A_i).\]
Since
$\displaystyle \bigcup_{i = 1}^{\infty} A_i
 \subseteq \bigcup_{i = 1}^{\infty} \bigcup_{E \in \mathcal{G}_i} E$, by
monotonicity of $\mu^*$,
\[\mu^*\left(\bigcup_{i = 1}^{\infty} A_i\right)
  \leq \epsilon + \sum_{i = 1}^{\infty} \mu^*(A_i).\]
Since this holds for all $\epsilon > 0$,
\[\mu^*\left(\bigcup_{i = 1}^{\infty} A_i\right)
  \leq \sum_{i = 1}^{\infty} \mu^*(A_i),\]
so that $\mu^*$ is countably subadditive. Thus, $\mu^*$ is an outer measure,
as desired. \qed

\item
%H^*_{\alpha} is an outer measure
The result of part (a) implies that $H^*_{\alpha,\delta}$ is an outer
measure. Since, $\forall \delta > 0$,
$H^*_{\alpha,\delta}: \mathcal{P}(X) \rightarrow [0,\infty]$ and
$H^*_{\alpha,\delta}(\emptyset) = 0$, taking the limit as
$\delta \rightarrow 0^+$,
$H^*_{\alpha}: \mathcal{P}(X) \rightarrow [0,\infty]$ and
$H^*_{\alpha}(\emptyset) = 0$.

%Monotonicity
Since, $\forall \delta > 0$, for $E \subseteq F \in \mathcal{P}(X)$,
$H^*_{\alpha,\delta}(E) \leq H^*_{\alpha,\delta}(F)$, taking the limit as 
$\delta \rightarrow 0^+$,
$H^*_{\alpha}(E) \leq H^*_{\alpha}(F)$, so that $H^*_{\alpha}$ is monotonic.

% Countable Subadditivity
Since $H^*_{\alpha,\delta}$ is nondecreasing in $\delta$ (as it is an infimum
and $\mathcal{E}_{\delta}$ becomes smaller as $\delta$ decreases),
$\forall \delta > 0$,$H^*_{\alpha,\delta} \leq H^*_{\alpha}$.
Therefore, since, $\forall \delta > 0$,
$\forall E_1,E_2,\ldots \in \mathcal{P}(X)$,
\[H^*_{\alpha,\delta}\left(\bigcup_{i = 1}^{\infty} E_i\right)
  \leq \sum_{i = 1}^{\infty} H^*_{\alpha,\delta}(E_i)
  \leq \sum_{i = 1}^{\infty} H^*_{\alpha}(E_i),
\]
so that, taking the limit as $\delta \rightarrow 0^+$,
\[H^*_{\alpha}\left(\bigcup_{i = 1}^{\infty} E_i\right)
  \leq \sum_{i = 1}^{\infty} H^*_{\alpha}(E_i).
\]
Therefore, $H^*_{\alpha}$ is countably subadditive and thus an outer measure.
\qed

% H_{\alpha} is a measure on \mathcal{B}
Since $H^*_{\alpha}$ is an outer measure, by Caratheodory, it suffices to show
that, $\forall B \in \mathcal{B}$, $\forall A \subseteq X$,
\[H^*_{\alpha}(A) \geq H^*_{\alpha}(A \cap B) + H^*_{\alpha}(A \cap B^c),\]
If either $H^*_{\alpha}(A \cap B) = \infty$ or
$H^*_{\alpha}(A \cap B^c) = \infty$, then, by monotonicity,
$H^*_{\alpha}(A) = \infty$ and the desired result trivially holds. Thus, we
assume that each of these measures is finite.

{\bf Lemma:} If $E,F \in \mathcal{P}$ with $d := \dist(E,F) > 0$, then
\[H^*_{\alpha}(E \cup F) \geq H^*_{\alpha}(E \cup F).\]

{\bf Proof of Lemma:} For $\delta < \frac{d}{2}$, if
$B_1,B_2,\ldots \in \mathcal{E}_{\delta}$, then, each $B_i$ has
$B_i \cap E = \emptyset$ or $B_i \cap F = \emptyset$ (for, if
$x \in B_i \cap E$ and $y \in B_i \cap F$, then
$\dist(E,F) \leq d(x,y) < 2\delta = d$, which is a contradiction).

Thus, let $B_{j_1},B_{j_2},\ldots \in \mathcal{E}_{\delta}$ be the subsequence
of $B_i$ such that $B_i \cap E \neq \emptyset$, and let
$B_{k_1},B_{k_2},\ldots \in \mathcal{E}_{\delta}$ be the subsequence of $B_i$
such that $B_i \cap F \neq \emptyset$ (if either subsequence is finite, we can
add countably many empty sets to the sequence).
Since we assumed that $H^*_{\alpha,\delta}(E) \leq H^*_{\alpha}(E) < \infty$
and $H^*_{\alpha,\delta}(F) \leq H^*_{\alpha}(F) < \infty$, and, in the end,
we are concerned only with infima, so that we can ignore infinite sums,
\[\sum_{i = 1}^{\infty} \rho(B_i)
 = \sum_{i = 1}^{\infty} \rho(B_{j_i})
 + \sum_{i = 1}^{\infty} \rho(B_{k_i}).
\]
This implies that
\begin{align*}
   & \left\{\sum_{i = 1}^{\infty} \rho(B_i) \right|
            \left. E \cup F \subseteq \bigcup_{i = 1}^{\infty} B_i,
                        B_i \in \mathcal{E}_{\delta}\right\} \\
 = & \left\{\sum_{i = 1}^{\infty} \rho(B_{j_i}) + \rho(B_{k_i})\right|
            \left. E \subseteq \bigcup_{i = 1}^{\infty} B_{j_i},
                   F \subseteq \bigcup_{i = 1}^{\infty} B_{k_i},
                        B_{j_i},B_{k_i} \in \mathcal{E}_{\delta}\right\}.
\end{align*}

Rewriting the latter set as the element-wise sum of two sets, by definition of
$H^*_{\alpha,\delta}$ and the fact that the infimum is nonincreasing with
respect to inclusion,
\[H^*_{\alpha,\delta}(E \cup F)
 \geq H^*_{\alpha,\delta}(E)
 +    H^*_{\alpha,\delta}(F),\]
so that taking the limit as $\delta \rightarrow 0^+$ proves the lemma:
\[H^*_{\alpha}(E \cup F) \geq H^*_{\alpha}(E) + H^*_{\alpha}(F).\]

We now return to showing that, $\forall B \in \mathcal{B}$,
$\forall A \subseteq X$,
\[H^*_{\alpha}(A) \geq H^*_{\alpha}(A \cap B) + H^*_{\alpha}(A \cap B^c),\]
Since $\mathcal{B}$ is the $\sigma$-algebra generated by all open sets and is
closed under the set complement, it is sufficient to this for all closed sets
$B$. Thus, let $B \in \mathcal{B}$, and let $A \subseteq X$.
$\forall i \in \mathbb{N}$, define
\begin{align*}
B_i & = \{x \in X | \dist(\{x\},B) < 1/i\}, \\
C_i & = A \cap (B_i \cap B_{i + 1}^c)
\end{align*}

Using the above lemma and then monotonicity gives
\[H^*_{\alpha}(A \cap B_i^c) + H^*_{\alpha}(A \cap B)
 \leq H^*_{\alpha}((A \cap B_i^c) \cup (A \cap B))
 \leq H^*_{\alpha}(A \cup B). \mqed
\]
Thus, it remains only to show that
\[\lim_{i \rightarrow \infty} \left(H^*_{\alpha}(A \cap B_i^c)\right)
 = H^*_{\alpha}(A \cap B^c).\]

Since $B$ is closed, for all $x \in B^c$, $\dist(\{x\},B) > 0$, so that,
$\forall i \in \mathbb{N}$,
\[A \cap B^c = (A \cap B_i^c) \cup \bigcup_{j = i + 1}^{\infty} C_j.\]
Thus, by monotonicity and then subadditivity,
\[H^*_{\alpha}(A \cap B_i^c)
 \leq H^*_{\alpha}(A \cap B^c)
 \leq H^*_{\alpha}\left((A \cap B_i^c) \cup \bigcup_{j = i + 1}^{\infty} C_j\right)
 \leq H^*_{\alpha}(A \cap B_i^c) + \sum_{j = i + 1}^{\infty} H^*_{\alpha}(C_j).
\]

Since the above summation is finite and each $H^*_{\alpha}(F_j) \geq 0$,
$\lim_{i \rightarrow \infty}
 \left(\sum_{j = i + 1}^{\infty} H^*_{\alpha}(C_j)\right) = 0$, and
taking the limit as $i \rightarrow \infty$ in the above inequality:
\[
 \lim_{i \rightarrow \infty}\left(H^*_{\alpha}(A \cap B_i^c)\right)
 \leq H^*_{\alpha}(A \cap B^c)
 \leq \lim_{i \rightarrow \infty}\left(H^*_{\alpha}(A \cap B_i^c)\right) \mqed
\]

\item $\forall \bx \in \mathbb{R}^d$, $\delta > 0$,
$\mathcal{E}_{\delta} = \mathcal{E}_{\delta} + \bx$, since a translated ball
is a ball of the same radius, and,
$\forall B(\by,r) \in \mathcal{E}_{\delta}$,
$\rho(B(\by,r))) = \rho(B(\by + \bx,r)))$ Thus, since,
for all Hausdorff measurable sets $E \in \mathcal{P}(\mathbb{R})$
\[H_{\alpha}(E)
 = \inf\left\{\sum_{i = 1}^{\infty} \rho(B_i)\right|
        \left. B_i \in \mathcal{E}_{\delta},
            E \subseteq \bigcup_{i = 1}^{\infty} B_i
\right\},\]
$\forall \bx \in \mathbb{R}$,
$H_{\alpha}(E) = H_{\alpha}(E + \bx)$.

Thus, $H_d$ is translation invariant on $\mathcal{L}$, so that, by the result
of problem 3, part (a), $\exists c \geq 0$ such that $H_d = c\lambda$. It
remains only to show that $c \in (0,\infty)$. Let $B = B(\bzero,\sqrt[d]{2})$,
let $C_1 = [-1,1]^d$, and let $C_2 = [-2,2]^d$.

Since $C_1$ and $C_2$ are cells, $\lambda(C_1) = (1 - (-1))^d = 2^d$ and
$\lambda(C_2) = (2 - (-2))^d = 4^d$.

Thus, $H_{\alpha}(B),\lambda(C_1),\lambda(C_2) \in (0,\infty)$. By
monotonicity (noting $C_1 \subseteq B \subseteq C_2$),
\[c\lambda(C_1) = H_d(C_1) \leq H_d(B) \leq H_d(C_2) = c\lambda(C_2).\]
The first inequality implies that $c \neq \infty$, and the latter inequality
implies that $c \neq 0$. \qed

\item {\bf Lemma:} If, for some $\alpha,\beta \in [0,\infty)$,
$\alpha < \beta$ and $H_{\alpha}(S) < \infty$, then $H_{\beta}(S) = 0$.

{\bf Proof of Lemma:} Let $\delta > 0$.
Because $H_{\alpha,\delta}$ is an infimum, we can find a sequence of balls
$B(x_1,r_1),B(x_2,r_2),\ldots \in \mathcal{E}_{\delta}$, with
$S \subseteq \bigcup_{i = 1}^{\infty} B(x_i,r_i)$, such that
\[\sum_{i = 1}^{\infty} c_{\alpha} r_i^{\alpha}
 = \sum_{i = 1}^{\infty} \rho(B(x_i,r_i))
 \leq H_{s,\delta + 1}.\]
Furthermore, since $H_{\alpha,\delta}$ is an infimum and $\mathcal{E}_{\delta}$
becomes smaller as $\delta$ decreases, $H_{\alpha,\delta}$ increases as $\delta$
decreases. Thus, taking the limit as $\delta \rightarrow 0^+$,
\[\sum_{i = 1}^{\infty} c_{\alpha} r_i^{\alpha}
 \leq H_{\alpha,\delta}(S) + 1
 \leq H_{\alpha}(S) + 1.
\]

Therefore,
\begin{eqnarray*}
H_{\beta,\delta}(S)
 & \leq & \sum_{i = 1}^{\infty} c_{\beta}r_i^{\beta}
   \leq   \frac{c_{\beta}}{c_{\alpha}}
        \sum_{i = 1}^{\infty} c_{\alpha} r_i^{\alpha}r_i^{\beta - \alpha} \\
 & \leq & \frac{c_{\beta}}{c_{\alpha}} \delta^{\beta - \alpha}
        \sum_{i = 1}^{\infty} c_{\alpha}r_i^{\alpha}
                                    \quad \mbox{\quad(each $r_i \leq \delta$)}\\
 & \leq & \frac{c_{\beta}}{c_{\alpha}} \delta^{\beta - \alpha}(H_{\alpha}(S) + 1).
\end{eqnarray*}

Since $H_{\alpha}(S) < \infty$ and $\beta > \alpha$, taking the limit as
$\delta \rightarrow 0$ proves the lemma.

Let $d = \sup\{\alpha \in [0,\infty] | H^*_{\alpha}(S) = \infty\}$. Suppose
$\alpha \in (d,\infty)$. By choice of $d$, for
$\beta = \frac{\alpha - d}{2} > d$, $H_{\beta}(S) < \infty$, so that, by the
above lemma, $H_{\alpha}(S) = 0$. On the other hand, suppose $\alpha \in (0,d)$.
By the above lemma, if $H_{\alpha}(S) \neq \infty$, then,
$\forall \beta \in (\alpha,d]$, $H_{\beta}(S) = 0$, contradicting the choice
of $d$ as the supremum. Thus, $d$ has the desired properties. Note $d$ is
unique, as, if $d^{\prime} \neq d$ (without loss of generality,
$d^{\prime} > d$), also had the desired properties, then, for
$\alpha \in (d,d^{\prime}$, $\alpha = 0$ and $\alpha = \infty$, which is
impossible. \qed

\item {\bf Lemma:} $\forall A \subseteq \mathbb{R}$, $c \in \mathbb{R}$, if
$cA$ is the dilation of $A$ by $c$, then,
$H_{\alpha}(cA) = c^{\alpha}H_{\alpha}(A)$.

{\bf Proof of Lemma:} Note that, if $B_1,B_2,\ldots \in \mathcal{E}_{\delta}$
with $A \subseteq \bigcup_{i = 1}^{\infty} B_i$, then
$cB_1,cB_2,\ldots \in \mathcal{E}_{\delta}$ with
$cA \subseteq \bigcup_{i = 1}^{\infty} cB_i$. Also, for any ball $B(x,r)$,
$\rho(cB(x,r)) = \rho(B(cx,cr)) = c^{\alpha}\rho(B(x,r))$. Thus,
\[H_{\alpha}(cA)
 = \lim_{\delta \rightarrow 0+} \inf
        \left\{\sum_{i = 1}^{\infty} c^{\alpha}\rho(B_i) \right|
        \left.B_i \in \mathcal{E}_{\delta},
            cA \subseteq \bigcup_{i = 1}^{\infty} B_i \right\}
 = c^{\alpha}H_{\alpha}(A),
\]
proving the lemma.

The Cantor set $C$ has the property that
\[C = \frac13\left(C \cup (C + 2)\right),\]
where addition denotes translation and multiplication denotes dilation.

It was shown in the proof of part (c) that $H_d$ is translation invariant.
Note also that, since $C \subseteq [0,1]$, $\dist(C,C + 2) > 0$, and thus
that, by the Lemma shown in in part (b),
$H_{\alpha}(C \cup (C + 2)) = H_{\alpha}(C) + H_{\alpha}(C + 2)$.
Therefore, by the above lemma, $\forall \alpha \in [0,\infty]$,
\begin{align*}
H_{\alpha}(C)
 & = H_{\alpha}\left(\frac13\left(C \cup (C + 2)\right)\right) \\
 & = \left(\frac13\right)^{\alpha} H_{\alpha}(C \cup (C + 2))           & \mbox{by above lemma}            \\
 & = \left(\frac13\right)^{\alpha} H_{\alpha}(C) + H_{\alpha}(C + 2)    & \mbox{since $\dist(C,C + 2) > 0$} \\
 & = \left(\frac13\right)^{\alpha} 2H_{\alpha}(C).                      & \mbox{(by translation invariance)}
\end{align*}
Suppose, then, that there exists some $\alpha \in (0,\infty)$ such that
$H_{\alpha}(C) \in (0,\infty)$. Then, for that value of $\alpha$, we can
divide both sides of the above equation by $H_{\alpha}(C)$, so that
\[2 = 3^{\alpha}.\]
Then,
\[\alpha = \log_{3}(2) = \mbox{\fbox{$\displaystyle \frac{\ln(2)}{\ln(3)}$.}}\]

\end{enumerate}
\end{question}

\end{document}
