\documentclass[11pt]{article}
\usepackage{enumerate}
\usepackage{fullpage}
\usepackage{fancyhdr}
\usepackage{amsmath, amsfonts, amsthm, amssymb}
\usepackage{color}
\setlength{\parindent}{0pt}
\setlength{\parskip}{5pt plus 1pt}
\pagestyle{empty}

\def\indented#1{\list{}{}\item[]}
\let\indented=\endlist

\newcounter{questionCounter}
\newcounter{partCounter}[questionCounter]
\newenvironment{question}[2][\arabic{questionCounter}]{%
    \setcounter{partCounter}{0}%
    \vspace{.25in} \hrule \vspace{0.5em}%
        \noindent{\bf #2}%
    \vspace{0.8em} \hrule \vspace{.10in}%
    \addtocounter{questionCounter}{1}%
}{}
\renewenvironment{part}[1][\alph{partCounter}]{%
    \addtocounter{partCounter}{1}%
    \vspace{.10in}%
    \begin{indented}%
       {\bf (#1)} %
}{\end{indented}}

%%%%%%%%%%%%%%%%%%%%%%%HEADER%%%%%%%%%%%%%%%%%%%%%%%%%%%%%%
\newcommand{\myname}{Shashank Singh}
\newcommand{\myandrew}{sss1@andrew.cmu.edu}
\newcommand{\myclass}{21-720 Measure and Integration}
\newcommand{\myhwnum}{5}
\newcommand{\duedate}{Wednesday, October 10, 2012}
%%%%%%%%%%%%%%%%%%%%%%%%%%%%%%%%%%%%%%%%%%%%%%%%%%%%%%%%%%%

%%%%%%%%%%%%%%%%%%%%CONTENT MACROS%%%%%%%%%%%%%%%%%%%%%%%%%
\renewcommand{\qed}{\quad $\blacksquare$}
\newcommand{\mqed}{\quad \blacksquare}
\newcommand{\inv}{^{-1}}
\newcommand{\bx}{\mathbf{x}}
\newcommand{\bq}{\mathbf{q}}
\newcommand{\bv}{\mathbf{v}}
\newcommand{\by}{\mathbf{y}}
\newcommand{\bff}{\mathbf{f}}
\newcommand{\bzero}{\mathbf{0}}
\newcommand{\balpha}{\boldsymbol{\alpha}}
\newcommand{\boldeta}{\boldsymbol{\eta}}
\newcommand{\dist}{\operatorname{dist}}
\newcommand{\area}{\operatorname{area}}
\newcommand{\vol}{\operatorname{vol}}
\newcommand{\sminus}{\backslash}
\newcommand{\N}{\mathbb{N}} % natural numbers
\newcommand{\Z}{\mathbb{Z}} % natural numbers
\newcommand{\Q}{\mathbb{Q}} % rational numbers
\newcommand{\R}{\mathbb{R}} % real numbers
\newcommand{\pow}[1]{\mathcal{P}\left(#1\right)} % power set of #1
\newcommand{\LR}[1]{\mathcal{L}\left(\R^{#1}\right)} % Lebesgue measurable sets of \R^{#1}
\newcommand{\B}{\mathcal{B}} % Borel family of sets
%%%%%%%%%%%%%%%%%%%%%%%%%%%%%%%%%%%%%%%%%%%%%%%%%%%%%%%%%%%

\begin{document}
\thispagestyle{plain}

{\Large Homework \myhwnum} \\
\myclass \\
Name: \myname \\
Email: \myandrew \\
Due: \duedate \\

\begin{question}{Problem 1}
\begin{enumerate}[(a)]
\item Since any step function is also a simple function, it is immediate from
the definitions of the Riemann and Lebesgue integrals of step functions and
simple functions, respectively, that the Riemann and Lebesgue integrals agree
when $f$ is a step function.

Suppose that $s,t$ are step functions with $s \leq f \leq t$. Then,
\[\int_I s \, dx
 =    \int_I s \, d\lambda
 \leq \int_I f \, d\lambda
 =    \int_I t \, d\lambda
 =    \int_I t \, dx,
\]
where the first and last terms are Riemann integrals and the other terms are
Lebesgue integrals.
Since $f$ is Riemann integrable, taking the supremum over all lower and upper
step functions $s$ and $t$ gives
\[\int_I f \, dx
 \leq \int_I f \, d\lambda
 \leq \int_I f \, dx,
\]
so that the Lebesgue integral exists and is equal to the Riemann integral.
\qed

\item Let $f$ be defined by $f(x) = \frac{\sin(x)}{x}$,
$\forall x \in [1,\infty)$. Integration by parts gives that, for the Riemann
integral,
\begin{align*}
\int_1^{\infty} f(x) \, dx
 & =    \lim_{x \rightarrow \infty} \int_1^x \frac{(1 - \cos(t))^{\prime}}{t} \, dt \\
 & =    \lim_{x \rightarrow \infty} \left. \frac{1 - \cos(t)}{t} \right|_{t = 1}^x + \lim_{x \rightarrow \infty} \int_1^x \frac{1 - \cos(t)}{t^2} \, dt \\
 & =    0 + \lim_{x \rightarrow \infty} \int_1^x \frac{1 - \cos(t)}{t^2} \, dt \\
 & \leq \lim_{x \rightarrow \infty} \int_1^x \frac{2}{t^2} \, dt < \infty. \\
\end{align*}
Since $\int_1^x \frac{1 - \cos(t)}{t^2} \, dt$ is monotone and bounded, the limit
exists, so that $f$ is improper Riemann integrable on $[1,\infty)$.

On the other hand
\begin{align*}
\int_{[1,\infty)} f^+(x), \int_{[1,\infty)} f^-(x) \geq \sum_{i = 1}^{\infty} cH_n = \infty,
\end{align*}
where $c$ is some positive constant and $H_n$ is the $n^{th}$ harmonic number,
so that $f$ is not Lebesgue integrable.

\end{enumerate}
\end{question}

\begin{question}{Problem 2}
\begin{enumerate}[(a)]
\item We first show the result under the additional hypothesis that $f$ and
$g$ are simple functions. Let $\{f_1,\dots f_m\}$ be the range of $f$, let
$\{g_1,\dots,g_n\}$ be let the range of $g$, let
$F_i := f\inv(\{f_i\}), G_j := g\inv(\{g_j\})$, and let
$N := \{x \in X : f \neq g\}$, so that $\mu(N) = 0$.
Then,
\begin{align*}
\int_X f \, d\mu
 & = \sum_{i = 1}^m f_i \mu(F_i) \\
 & = \sum_{i = 1}^m f_i (\mu(F_i \cap N) + \mu(F_i \cap N^c))
                    & (\mbox{$\Sigma$ complete $\Rightarrow N \in \Sigma$}) \\
 & = \sum_{i = 1}^m f_i \mu(F_i \cap N^c)
                  & (\mbox{monotonicity $\Rightarrow \mu(F_i \cap N) = 0$}) \\
 & = \int_{X \sminus N} f \, d\mu
   = \int_{X \sminus N} g \, d\mu
   = \int_X g \, d\mu, & (\mbox{$f = g$ on $X \sminus N$})
\end{align*}
where the last equality holds due to symmetry.

We now relax the assumption that $f$ and $g$ are simple. Suppose
$s: X \rightarrow [-\infty,\infty]$ is simple with $s \leq f^+$. Then, for
$t: X \rightarrow [-\infty,\infty]$ defined by
\[t(x) :=   \left\{
                \begin{array}{lcr}
                    s(x) & : & f(x) = g(x) \\
                    0    & : & \mbox{otherwise}
                \end{array}
            \right.,\]
$s = t$ a.e. (since $f = g$ a.e.), so that
$\int_X s \, d\mu = \int_X t_s \, d\mu$ Thus, since $t \leq g^+$, taking
the supremum over all $s$ on both sides gives,
$\int_X f^+ \, d\mu \leq \int_X g^+ \, d\mu$.
A symmetric argument shows that
$\int_X f^+ \, d\mu \geq \int \int_X g^+ \, d\mu$
and then an identical argument shows that
$\int_X f^- \, d\mu = \int_X g^- \, d\mu$, so that, as desired,
\[\int_X f \, d\mu
 = \int_X f^+ \, d\mu - \int_X f^- \, d\mu
 = \int_X g^+ \, d\mu - \int_X g^- \, d\mu
 = \int_X g \, d\mu. \mqed\]

\item $\forall n \in \N$, define $g_n: X \rightarrow [-\infty,\infty]$ by
\[g_n(x) := \left\{
                \begin{array}{lcr}
                    f_n(x)  & : & \lim_{n \rightarrow \infty} f_n(x) = (x) \\
                    f(x)    & : & \mbox{otherwise}
                \end{array}
            \right.,\]
Then, $g_n = f_n$ almost everywhere, so that, by the result of part (a), and
then by Fatou's Lemma (since $(g_n) \rightarrow f$ pointwise on $E$),
\[\lim \inf_n \int_E f_n \, d\mu
 =    \lim \inf_n \int_E g_n d\mu \,
 \geq \int_E f \, d\mu. \mqed\]
\end{enumerate}
\end{question}

\newpage
\begin{question}{Problem 3}
If $\mu_{f\inv}$ is the pushforward measure of $\lambda$ by $f\inv$,
then $\mu_{f\inv}(I) = 0$ for all cells $I$. By the result of part (a) of
Problem 3 on Assignment 3, since the set of cells is a $\pi$-system,
$\mu_{f\inv} = 0$ on the $\sigma$-algebra generated by cells, which is the
Borel $\sigma$-algebra. Since $\LR{d}$ is the completion of the Borel
$\sigma$-algebra under the Lebesgue measure, $\mu_{f\inv}(E) = 0$ except
perhaps for sets of Lebesgue measure $0$. But then, $f = 0$ almost
everywhere. \qed
\end{question}

\begin{question}{Problem 4}
\begin{enumerate}[(a)]
\item Suppose $x \in [0,\infty)$. Then, by the dominated convergence
theorem, since $|e^{-st}f(t)| \leq |f(t)|$ for $s,t \in [0,\infty)$ and
$\int_0^{\infty} |f(t)| \, dt < \infty$,
\[\lim_{s \rightarrow x} F(s)
   = \lim_{s \rightarrow x} \int_0^{\infty} e^{-st}f(t) \, dt
   = \int_0^{\infty} \lim_{s \rightarrow x} e^{-st}f(t) \, dt \\
   = \int_0^{\infty} e^{-xt}f(t) \, dt
   = F(x).\]
Thus, $F$ is continuous. \qed

\end{enumerate}
\end{question}

\begin{question}{Problem 5}
\begin{enumerate}[(a)]
\item If $\det(T) = 0$, then $T(A)$ is a subspace of $\R^d$ with
$\dim(T(A)) < d$. Thus, by the result of part (a) of Problem 2 on Assignment
1, $\lambda(T(A)) = 0 = |\det(T)| \lambda(A)$.

Note that, since the image of a cell under a scaling transformation is a cell
whose volume is the product of the scaling factors in each dimension and the
volume of the original cell (the same as the determinant of such a
transformation), for any cell $I$, if $E$ is a scaling matrix,
\[\lambda(E(I)) = \vol(E(I)) = |\det(E)|\vol(I) = |\det(E)|\lambda(I).\]
It follows then, from the definition of the Lebesgue measure that the above
equality holds for any $I \in \LR{d}$.

Recall also that, by the result of part (b) of Problem 3 on Assignment 1,
if $E$ is an orthogonal matrix, then, since $\det(E) = \pm 1$,
\[\lambda(E(I)) = \lambda(I) = |\det(E)|\lambda(I).\]

We now consider the case that $T$ is any nonsingular linear transformation.
Then, $T$ has a unique polar decomposition, so that $T = UP$, where $U$ is
orthogonal and $P$ is symmetric. Since $P$ is symmetric, $P$ can be
diagonalized on an orthogonal basis, so that $P = Q\Lambda Q^T$, where $Q$,
and thus $Q^T$, are orthogonal, and $\Lambda$ is diagonal (and thus a scaling
transformation). Then, $\forall E \in \LR{d}$,
\begin{align*}
\lambda(T(E))
 & = \lambda(UQ\Lambda Q^T(E))                                              \\
 & = \lambda(\Lambda Q^T(E))          & \mbox{(since $UQ$ is orthogonal)}   \\
 & = |\det(\Lambda)|\lambda(Q^T(E))   & \mbox{(since $\Lambda$ is diagonal)} \\
 & = |\det(\Lambda)|\lambda(E)        
   = |\det(T)|\lambda(E),
\end{align*}
since the determinant is multiplicative. \qed

%TODO
\item We first show the result in the case that $f$ is a simple function.
Suppose $\{c_1,c_2,\ldots,c_k\}$ is the range of $f$, and let
$A_i := \{x \in E | f(x) = c_k\}$. Then, (noting that
$\det(T)$ is nonzero, since $T$ is invertible),
\begin{align*}
\int_{E} f \, d\mu
 & = \sum_{i = 1}^k c_i \lambda(A_i) & \mbox{(since $f$ is simple)} \\
 & = \sum_{i = 1}^k c_i |\det(T)| \lambda(T\inv(A_i)) & \mbox{(by the result of part (a))} \\
 & = \int_{T\inv(E)} (f \, \circ \, T)|\det(T)| \, d\lambda & \mbox{(since $(f \, \circ \, T) |\det(T)|$ is simple)}
\end{align*}
since $T\inv(A_i) = \left\{ x \in T\inv(E) \; | \; (f \, \circ \, T)(x) |\det(T)| = c_i|\det(T)| \right\}$.

Now suppose $f$ is any Lebesgue Integrable function. Since the result holds
for simple functions, and, for any simple function
$s : E \rightarrow \R$ with $s \leq f$,
$(s \, \circ \, T) |\det(T)| : T\inv(E) \rightarrow \R$ is simple, by
definition of the Lebesgue integral for general functions (in terms of simple
functions), as desired,
\[\int_E f \, d\lambda
 = \int_{T\inv(E)} \, (f \, \circ \, T) |\det(T)| \, d\lambda. \mqed \]
\end{enumerate}
\end{question}
\end{document}
