\documentclass[11pt]{article}
\usepackage{enumerate}
\usepackage{fullpage}
\usepackage{fancyhdr}
\usepackage{amsmath, amsfonts, amsthm, amssymb}
\usepackage{color}
\setlength{\parindent}{0pt}
\setlength{\parskip}{5pt plus 1pt}
\pagestyle{empty}

\def\indented#1{\list{}{}\item[]}
\let\indented=\endlist

\newcounter{questionCounter}
\newcounter{partCounter}[questionCounter]
\newenvironment{question}[2][\arabic{questionCounter}]{%
    \setcounter{partCounter}{0}%
    \vspace{.25in} \hrule \vspace{0.5em}%
        \noindent{\bf #2}%
    \vspace{0.8em} \hrule \vspace{.10in}%
    \addtocounter{questionCounter}{1}%
}{}
\renewenvironment{part}[1][\alph{partCounter}]{%
    \addtocounter{partCounter}{1}%
    \vspace{.10in}%
    \begin{indented}%
       {\bf (#1)} %
}{\end{indented}}

%%%%%%%%%%%%%%%%%%%%%%%HEADER%%%%%%%%%%%%%%%%%%%%%%%%%%%%%%
\newcommand{\myname}{Shashank Singh}
\newcommand{\myandrew}{sss1@andrew.cmu.edu}
\newcommand{\myclass}{21-720 Measure and Integration}
\newcommand{\myhwnum}{1}
\newcommand{\duedate}{Wednesday, September 19, 2012}
%%%%%%%%%%%%%%%%%%%%%%%%%%%%%%%%%%%%%%%%%%%%%%%%%%%%%%%%%%%

%%%%%%%%%%%%%%%%%%%%CONTENT MACROS%%%%%%%%%%%%%%%%%%%%%%%%%
\renewcommand{\qed}{\quad $\blacksquare$}
\newcommand{\mqed}{\quad \blacksquare}
\newcommand{\inv}{^{-1}}
\newcommand{\bx}{\mathbf{x}}
\newcommand{\bq}{\mathbf{q}}
\newcommand{\bv}{\mathbf{v}}
\newcommand{\by}{\mathbf{y}}
\newcommand{\bff}{\mathbf{f}}
\newcommand{\bzero}{\mathbf{0}}
\newcommand{\balpha}{\boldsymbol{\alpha}}
\newcommand{\boldeta}{\boldsymbol{\eta}}
\newcommand{\dist}{\operatorname{dist}}
\newcommand{\area}{\operatorname{area}}
\newcommand{\sminus}{\backslash}
\newcommand{\N}{\mathbb{N}} % natural numbers
\newcommand{\Z}{\mathbb{Z}} % natural numbers
\newcommand{\Q}{\mathbb{Q}} % rational numbers
\newcommand{\R}{\mathbb{R}} % real numbers
\newcommand{\pow}[1]{\mathcal{P}\left(#1\right)} % power set of #1
\newcommand{\LRd}{\mathcal{L}\left(\R^d\right)} % Lebesgue measurable sets of \R^d
\newcommand{\LR}[1]{\mathcal{L}\left(\R^{#1}\right)} % Lebesgue measurable sets of \R^{#1}
\newcommand{\B}{\mathcal{B}} % Borel family of sets
%%%%%%%%%%%%%%%%%%%%%%%%%%%%%%%%%%%%%%%%%%%%%%%%%%%%%%%%%%%

\begin{document}
\thispagestyle{plain}

{\Large Homework \myhwnum} \\
\myclass \\
Name: \myname \\
Email: \myandrew \\
Due: \duedate \\
\begin{question}{Problem 1}
\begin{enumerate}[(a)]
\item Since $\mu$ is a measure, $\mu(\emptyset) = 0$, so that, since
$\emptyset \in \Sigma$ (as $\Sigma$ is a $\sigma$-algebra) and
$\emptyset \subseteq \emptyset$, $\mu^*(\emptyset) = 0$.

Suppose $A,B \in \pow{X}$ with $A \subseteq E$. If $B \subseteq E$, then
$A \subseteq E$, so that
\[\left\{\mu(E) \right| \left. A \subseteq E, E \in \Sigma\right\}
\supseteq \left\{\mu(E) \right| \left. B \subseteq E, E \in \Sigma\right\}.\]
Thus
\[\mu^*(A)
 = \inf\left\{\mu(E) \right| \left. A \subseteq E, E \in \Sigma\right\}
\leq \inf\left\{\mu(E) \right| \left. B \subseteq E, E \in \Sigma\right\}
 = \mu^*(B),\]
so that $\mu^*$ is monotonic.

$\forall E_1,E_2,\ldots \in \Sigma$
with $A_i \subseteq E_i$, for $E := \bigcup_{i = 1}^{\infty} E_i$,
$E \in \Sigma$, $\bigcup_{i = 1}^{\infty} A_i \subseteq E$, and 
\[\mu(E) \leq \sum_{i = 1}^{\infty} \mu(E_i).\]
Then, taking the infimum over all sequences $E_1,E_2,\ldots \in \Sigma$, and
noting that the infimum on the left does not increase when we replace $E$
with any superset of $\bigcup_{i = 1}^{\infty} A_i$, gives
\[\mu^*\left( \bigcup_{i = 1}^{\infty} A_i \right)
 \leq \sum_{i = 1}^{\infty} \mu^*(A_i).\]
Therefore, $\mu^*$ is countably subadditive, so that $\mu^*$ is an outer
measure. \qed

\item $\forall E_1,E_2,\ldots \in\Sigma$
with $E_i \subseteq A_i$, for $E := \bigcup_{i = 1}^{\infty} E_i$,
$E \in \Sigma$, $E \subseteq \bigcup_{i = 1}^{\infty} A_i$, and, since
the $A_i$'s are pairwise disjoint and $\mu$ is countably additive,
\[\mu(E) \geq \sum_{i = 1}^{\infty} \mu(E_i).\]
Then, taking the supremum over all sequences $E_1,E_2,\ldots \in \Sigma$, and
noting that the supremum on the left does not decrease when we replace $E$
with any subset of $\bigcup_{i = 1}^{\infty} A_i$, gives
\[\mu^*\left( \bigcup_{i = 1}^{\infty} A_i \right)
 \geq \sum_{i = 1}^{\infty} \mu^*(A_i).\]


\item By definition of $\mu^*$ and $\mu_*$ as an infimum and a supremum,
respectively, $\forall n \in \N$, $\exists E_i, F_i \in \Sigma$ such that
$A \subseteq E_i$, $F_i \subseteq A^c$, $\mu^*(E) \leq \mu(A) + 1/n$ and
$\mu(F_i) \geq \mu_*(A^c)- 1/n$. Without loss of generality, $\{E_i\}$ is
decreasing (otherwise, we can replace $E_i$ with
$\bigcap_{k = 1}^i E_k$) and $\{F_i\}$ is increasing (otherwise, we can
replace $F_i$ with $\bigcup_{k = 1}^i F_k$).

Note that $E_i^c \subseteq A^c$ and $A \subseteq F_i^c$, so that, for 
$S_i := E_i \cap F_i^c$ (and thus $S_i^c = E_i^c \cup F_i$),
\[\mu^*(A) \leq \mu(S_i) \leq \mu(E_i) \leq \mu^*(A) + 1/n,\]
and
\[\mu_*(A^c) \geq \mu(S_i^c) \geq \mu(F_i) \geq \mu_*(A^c) - 1/n.\]
Since $\{S_i\}$ is decreasing and $\{S_i^c\}$ is increasing,
for $S := \bigcup_{i = 1}^{\infty} S_i$, taking the limit as
$n \rightarrow \infty$ in the above inequalities,
\[\mu^*(A) = \lim_{i \rightarrow \infty} \mu(S_i) = \mu\left( S \right)\]
and
\[\mu_*(A^c) = \lim_{i \rightarrow \infty} \mu(S_i) = \mu\left( S^c \right)\]

Since $\Sigma$ is closed under countable unions and intersections,
$S \in \Sigma$, so that
\[\mu(A) = \mu(S) + \mu(S^c) = \mu^*(A) + \mu_*(A^c). \mqed\]

\item Suppose $A \in \pow{X}$ with $\mu_*(A) = \mu^*(A)$. By definition of
$\mu_*$ and $\mu^*$ as a supremum and infimum, respectively,
$\forall n \in \N$, $\exists E_n, F_n \in \Sigma$ such that
$E_n \subseteq A \subseteq F_n$ and
$\mu(E_n) + \frac{1}{n} \geq A \geq \mu(F_n) - \frac{1}{n}$. Furthermore,
we can assume $\{E_i\}$ is an increasing sequence (otherwise, we replace $E_i$
with $\bigcup_{k = 1}^i E_i$) and $\{F_i\}$ is a decreasing sequence
(otherwise, we replace $F_i$ with $\bigcap_{k = 1}^i F_i$).

Let \[E = \bigcup_{i = 1}^{\infty} E_i,
\quad F = \bigcap_{i = 1}^{\infty} F_i.\]
Since $\{E_i\}$ is an increasing sequence and $\{F_i\}$ is a decreasing
sequence, $\mu(E) = \lim_{n \rightarrow \infty}(\mu(E_i)) = \mu_*(A)$ and
$\mu(F) = \lim_{n \rightarrow \infty}(\mu(F_i)) = \mu^*(A)$.
Since $\sigma$-algebras are closed under countable unions and countable
intersections, $E,F \in \Sigma$. Thus, since
$\mu(E) = \mu_*(A) = \mu^*(A) = \mu(F)$, $\mu(F \sminus E) = 0$, so that
$A \in \Sigma_{\mu}$.

Suppose, on the other hand, that $A \in \Sigma_{\mu}$. Then,
$\exists E,F \in \Sigma$ with $E \subseteq A \subseteq F$, such that
$\mu(F \sminus E) = 0$. Then, by definition of $\mu_*$ and $\mu^*$,
$\mu(E) \leq \mu_*(A),\mu^*(A) \leq \mu(F)$, so that $\mu_*(A) = \mu^*(A)$.
\qed
\end{enumerate}
\end{question}

\begin{question}{Problem 2}
\begin{enumerate}[(a)]
\item Suppose $A$ is bounded. By regularity of
$\lambda$, $\forall \epsilon > 0$, there exist an open set $U$ and a compact
set $K$ with $K \subseteq A \subseteq U$ such that
\[\lambda(K) + \frac{\epsilon}{2} > \lambda(A) > \lambda(U) - \frac{\epsilon}{2}.\]
Then, $\lambda(U \sminus K) = \lambda(U) - \lambda(K) < \epsilon$.

Now consider the general case, where $A$ may be unbounded.
Since $\Z^d$ is countable, there is a bijection $f: \N \rightarrow \Z^d$ and a
sequence of (disjoint, closed) cubes $C_1,C_2,\ldots$ with $C_i$ centered at
$f(i)$ and having sidelength $1 - \frac{\epsilon}{d2^{i + 2}}$, and thus
volume at least $1 - \frac{\epsilon}{2^{i + 2}}$. Since $A \cap C_i$ is
bounded (so that $\lambda(A) < \infty$, there exists a compact set $K_i$ with
$K_i \subseteq A \cap C_i$ and
\[\lambda(K_i)
 > \lambda(A \cap C_i) - \frac{\epsilon}{2^{i + 2}}
 \geq \lambda(A \cap Q_i) - \frac{\epsilon}{2^{i + 1}},\]
where $Q_i$ is the half-closed cube of sidelength $1$ centered at $f(i)$. Thus, by
countable additivity, since the $K_i$'s are disjoint (and closed, so
measurable)
\begin{align*}
\lambda\left( \bigcup_{i = 1}^{\infty} K_i \right)
 & = \sum_{i = 1}^{\infty} \lambda(K_i)
   > \sum_{i = 1}^{\infty} \lambda(A \cap Q_i) - \frac{\epsilon}{2^{i + 1}} \\
 & = \lambda\left( \bigcup_{i = 1}^{\infty} (A \cap Q_i) \right) - \epsilon/2
   = \lambda(A) - \epsilon/2.        & \mbox{(since $Q_i$'s partition $\R^d$)}
\end{align*}
Then, since a countable union of disjoint closed sets is closed,
$C := \bigcup_{i = 1}^{\infty} K_i$ is a closed subset of $A$. Just as for bounded
$A$, we can still take an open set $U$ such that
$\lambda(U) < \lambda(A) + \epsilon/2$, and thus
$\lambda(U \sminus K) < \epsilon$. \qed

\item By the result of part (a), $\forall i \in \N$,
$\exists C_i,U_i \subseteq \LRd$ with $C_i$ closed and $U_i$ open, such that
$C_i \subseteq A \subseteq U_i$ and $\lambda(U_i \sminus C_i) 1/i$. Let
$F = \bigcup_{i = 1}^{\infty} C_i$ and let $G = \bigcap_{i = 1}^{\infty} U_i$.
Since $\{U_i \sminus C_i\}$ is a decreasing sequence (if not, we can replace
$C_n$ with $\bigcup_{i = 1}^n C_i$ and replace $U_n$ with
$\bigcap_{i = 1}^n U_i$) and $\lambda(U_1 \sminus C_1) < 1 < \infty$
\[\lambda(G \sminus F)
 = \lim_{i \rightarrow \infty} (\lambda(U_i \sminus C_i)) = 0.\]
Since $\sigma$-algebras are closed under countable unions and intersections,
$F,G \in \B$, so that, by definition of the completion, $A \in \B_{\lambda}$,
and $\LRd \subseteq \B_{\lambda}$. Then, since $\LRd$ is complete,
$B_{\lambda} \subseteq \LRd$, so $B_{\lambda} = \LRd$. \qed

\end{enumerate}
\end{question}

\begin{question}{Problem 3}
Suppose $\lambda(A) = 0$, $E \subseteq A$. By monotonicity, any subset
of $A$ has Lebesgue measure $0$. Thus,
\begin{align*}
\lambda(E \cap F) + \lambda (E^c \cap F)
 & = \lambda(E^c \cap F)                        & \mbox{(by monotonicity)}              \\
 & = \lambda\left( E^c \cap F \cap A \right)
   + \lambda\left( E^c \cap F \cap A^c\right)   & \mbox{(since $A \in LRd$)}            \\
 & = \lambda\left( E^c \cap F \cap A^c\right)   & \mbox{(by monotonicity)}              \\
 & = \lambda\left( F \cap A^c \right)           & \mbox{(since $E^c \cap A^c = A^c$)}   \\
 & = \lambda\left( F \cap A \right)
   + \lambda\left( F \cap A^c \right)           & \mbox{(by monotonicity)}              \\
 & = \lambda\left( F \right).                   & \mbox{(since $A \in LRd$)}            \\
\end{align*}
Therefore, $E \in \LRd$.

Suppose, on the other hand, that $\lambda(A) > 0$. $\forall \balpha \in A$,
let $C_{\balpha} = \left( \alpha + \Q^d \right) \cap A$. By the Axiom of Choice, there
exists a choice function $f$ for the family
$\mathcal{C} = \{C_{\balpha} : \balpha \in A, C_{\balpha} \neq \emptyset\}
\subseteq \pow{A}$, so that we
can define $E = \{f(C) : C \in \mathcal{C}\} \subseteq A$. Note that
$\mathcal{C}$ is a partition of $A$.

{\bf Claim 1:} $\forall \bq_1,\bq_2 \in \Q^d$,
$(E + \bq_1) \cap (E + \bq_2) = \emptyset$.

{\bf Proof of Claim 1:} Suppose that $\bx_1 + \bq_1 = \bx_2 + \bq_2$ for some
$\bx_1,\bx_2 \in E$. Then, $\bx_1 - \bx_2 = \bq_2 - \bq_1$, so that
$(\bx_1 - \bx_2) \in \Q^d$, implying that $\bx_1,\bx_2 \in C_{\bx_1}$.
However, since distinct $C_{\balpha}$'s are disjoint, this contradicts the
construction of $E$.

{\bf Claim 2:} $\forall$ compact $K \subseteq E$, $\lambda(K) = 0$.

{\bf Proof of Claim 2:}
$\forall \bq \in \Q_1 := \{\bq \in \Q^d : \|\bq\| \leq 1\}$, let
$K_{\bq} =  K + \bq$. Since $\lambda$ is translation invariant,
$\forall \bq \in \Q_1$, $\lambda(K_{\bq}) = \lambda(K)$. Since $K$ and $\Q_1$
are bounded,  $\bigcup_{\bq \in \Q_1} K_{\bq}$ is bounded, so that
\[\sum_{\bq \in \Q_1} \lambda(K)
 = \sum_{\bq \in \Q_1} \lambda(K_{\bq})
 = \lambda\left( \bigcup_{\bq \in \Q_1} K_{\bq} \right)
 < \infty\]
(Claim 1 implies that the $K_{\bq}$'s are disjoint, allowing us to use
countable additivity in the last equality). Since $|\Q_1| = |\N|$, this
implies that $\lambda(K) = 0$, proving the claim.

If follows from Claim 2 and the regularity of $\lambda$ that $\lambda(E) = 0$.

{\bf Claim 3:} $E$ is not Lebesgue measurable.

{\bf Proof of Claim 3:} Suppose $E \in \LRd$. Note that there exists some
$\Q^{\prime} \subseteq \Q^d$ such that
$A = \bigcup_{\bq \in \Q^{\prime}} \bq + E$. Therefore, since $\lambda$ is
translation invariant,
\[\sum_{\bq \in \Q^{\prime}}\lambda(E)
 = \sum_{\bq \in \Q^{\prime}} \lambda(\bq + E)
 = \lambda(A) > 0\]
(where the last equality follows from the supposed measurability of $E$).
However, this contradicts the fact that $\lambda(E) = 0$, so that $E$ is not
Lebesgue measurable. \qed
\end{question}

\begin{question}{Problem 4}
\begin{enumerate}[(a)]
\item
Since $\B(\R^{m + n})$ is the $\sigma$-algebra generated by the family of open
sets in $\R^{m + n}$, it is sufficient to show that, for any open set
$U \in \pow{\R^{m + n}}$,
$U \in \sigma(\{A \times B | A \in \B(\R^m), B \in \B(R^n)\})$.
Since any open cell in $R^m$ is in $\B(R^m)$ and any open cell in $R^n$ is in
$\B(\R^n)$, and any open cell in $\R^{m + n}$ is the cross-product of some
open cell in $\B(R^m)$ and some open cell in $\B(\R^n)$, every open cell in
$\R^{m + n}$ is in $\{A \times B | A \in \B(\R^m), B \in \B(\R^n)\}$.
Therefore, since any open set is a countable union of open rectangles, any
open set $U \in \pow{\R^{m + n}}$ is in
$\sigma(\{A \times B | A \in \B(\R^m), B \in \B(R^n)\})$.

Since $\B(\R^{m + n})$ is a $\sigma$-algebra, to show that
$\B(\R^{m + n})
 \supseteq \sigma(\{A \times B | A \in \B(\R^m), B \in \B(R^n)\})$, it is
sufficient to show that
$\B(\R^{m + n}) \supseteq \{A \times B | A \in \B(\R^m), B \in \B(\R^n)\}$.

\item Since $\LR{m + n}$ is a $\sigma$-algebra, to show that
$\LR{m + n} \supseteq \sigma(\{A \times B | A \in \LR{m}, B \in \LR{n}\})$, it
is sufficient to show that 
$\LR{m + n} \supseteq \{A \times B | A \in \LR{m}, B \in \LR{n}\}$. Note that
$\LR{m} = \B(\R^m) \cup \mathcal{N}_m$, where
$\mathcal{N}_m = \{N \in \pow{\R^m} | \exists E \in \B(\R^m)\}$. If
$A \in \B(\R^m)$ and $B \in \B(\R^n)$, then, by the result of part (a),
$A \times B \in \B(\R^{m + n}) \subseteq \LR{m + n}$. Otherwise, either
$A \in \mathcal{N}_m$ or $B \in \mathcal{N}_n$; without loss of generality,
suppose $A \in \mathcal{N}_m$ (the case $B \in \mathcal{N}_n$ is identical).
Then, there is some $E \in \B(\R^m)$ such that $A \subseteq E$ and
$\lambda_m(E) = 0$. But then, $\lambda_{m + n}(E \times B) = 0$, so that,
since $A \times B \subseteq E \times B$, by the result of problem 3,
$A \times B \in \LR{m + n}$.

We now show that
$\LR{m + n} \neq \sigma(\{A \times B | A \in \LR{m}, B \in \LR{n}\})$. We
showed in class the existence of a non-measurable set $E_1 \in \pow{\R^m}$.
Let $E_2 = E_1 \times \{(0,0,\ldots,0)\} \in \pow{\R^{m + n}}$. By the result
of problem 3, $E_2 \in \LR{m + n}$, since $\lambda_{m + n}(E_2) = 0$. (This
follows from the result of part (a) of problem 2 on Assignment 1, since
$E_2 \subseteq V$, for some subspace $V$ of $\R^{m + n}$ with
$\dim(V) \leq m < m + n$). However, $E_1 \not \in \LR{m}$. \qed

\item Since, $\mathcal{L}(\R) = \B_{\lambda}(\R) \supseteq \B(\R)$, it follows
from the results of parts (a) and (b) that
\[\mathcal{L}(\R^2)
 \supsetneq \sigma\left( \{A \times B | A,B \in \mathcal{L}(\R) \right)
 \supseteq  \sigma\left( \{A \times B | A,B \in \B(\R) \right)
 = \B(\R^2). \mqed
\]
\end{enumerate}
\end{question}

\begin{question}{Problem 5}
This solution was inspired by Example 8 in Chapter 2 of Rudin's \emph{Real and
Complex Analysis}.

Since $\Q$ is countable, there is a bijection $f: \N \rightarrow \Q$.
$\forall n \in \N$, let
$R_i := (f(i) - \frac{1}{3^{i + 1}},f(i) + \frac{1}{3^{i + 1}})$,
\[S_i := R_i \sminus \left( \bigcup_{k = i + 1}^{\infty} R_k \right).\]
Then, $\forall i \in \N$, let
$T_i
 = (f(i) - \frac{1}{2 \cdot 3^{i + 1}},f(i) + \frac{1}{2 \cdot 3^{i + 1}})$
(noting that $\lambda(T_i) = \lambda(R_i)/2$), and let
\[V = \bigcup_{i = 1}^{\infty} S_i \cap T_i.\]

We claim that, for all $a < b$, $0 < \lambda(V \cap (a,b)) < b - a$. Note
that, since any interval $(a,b)$ has some $R_n \subseteq (a,b)$ such that
since each $R_i \in \B(\R) \subseteq \mathcal{L}{(R)}$,
$b - a = \lambda(R_i) + \lambda(R_i^c \cap (a,b))$, it is sufficient to show
that $0 < \lambda(R_i \cap V) < \lambda(R_i)$.

We first show that $0 < \lambda(R_i \cap V)$.
Since, for any $i \in \N$, there must be some $k > i$ such that
$f(k) \not \in R_i$, $\bigcup_{k = i + 1} R_k$ contains some interval with no
points in $S_i$, and thus, by countable additivity of $\lambda$,
\[\lambda(R_i) - \sum_{k = i + 1}^{\infty} \lambda(R_k) < \lambda(S_i).\]
Therefore, since each $R_i$ is an interval, so that $\lambda(R_i) = 3^i$,
\begin{equation}
\frac{\lambda(R_i)}{2}
 = \lambda(R_i)\left( 1 - \sum_{k = 1}^{\infty} \frac{1}{3^i} \right)
 = \lambda(R_i) - \sum_{k = i + 1}^{\infty} \frac{1}{3^{k}}\lambda(R_i)
 < \lambda(S_i).
\end{equation}
Therefore, since $S_i,T_i \in \B(\R)$,
\begin{align*}
\lambda(S_i \cup T_i)
 & \leq \lambda(R_i) = \frac12 \lambda(R_i) + \frac12 \lambda(R_i )
                  & \mbox{(by monotonicity ($S_i \cup T_i \subseteq R_i$))} \\
 & <    \lambda(S_i) + \lambda(T_i)
                   & \mbox{(by (1), since $\lambda(T_i) = \lambda(R_i)/2$)} \\
 & =    \lambda(S_i \sminus T_i) + \lambda(T_i \sminus S_i) + 2\lambda(S_i \cap T_i) \\
 & =    \lambda(S_i \cup T_i) + \lambda(S_i \cap T_i).
\end{align*}
Subtracting $\lambda(S_i \cup T_i)$ then gives, by monotonicity,
$0 < \lambda(S_i \cap T_i) \leq \lambda(R_i \cap V)$, as desired.

We now show that $\lambda(R_i \cap V) < \lambda(R_i)$. By countable additivity
of $\lambda$ and the definition of $V$,
\begin{align*}
\lambda(R_i \cap V)
 & = \lambda \left( \bigcup_{k = i}^{\infty} R_i \cap S_k \cap T_k \right)
                          & \mbox{($R_i \cap S_k = \emptyset$ for $k < i$)} \\
 & \leq \sum_{k = i}^{\infty} \lambda(R_i \cap S_k \cap T_k)
   \leq \sum_{k = i}^{\infty} \lambda(R_i \cap T_k)
                   & \mbox{countable additivity, monotonicity of $\lambda$} \\
 & <    \sum_{k = i}^{\infty} \lambda(T_k)
           & \mbox{monotonicity, $\exists T_k$ with $T_k \cap V_n = \emptyset,
                                                         \lambda(T_k) > 0$} \\
 & =    \sum_{k = i}^{\infty} \frac{\lambda(R_k)}{2}
   =    \sum_{k = i}^{\infty} \frac{\lambda(R_i)}{2^{k + 1}}
   =    \lambda(R_i). \mqed\\
\end{align*}

\end{question}

\begin{question}{Problem 6} Let $\Sigma = \sigma(A)$, and define
$\mu: \Sigma \rightarrow [0,\infty]$ by
\[\mu(A)
 = \inf \left\{\sum_{i = 1}^{\infty} \mu_0(E_i) \right.
    \left|
        E_i \in \mathcal{A}, A \subseteq \bigcup_{i = 1}^{\infty} E_i
    \right\}
.\]
By the result of part (a) of Problem 4 on Assignment 1, $\mu$ is an outer
measure on $X$.

We first show that $\mu = \mu_0$ on $\mathcal{A}$, so that it is in fact an
extension $\mu_0$ to $\Sigma$. Note that, since $\mathcal{A}$ is closed under
complements and finite unions, by De Morgan's laws, $\mathcal{A}$ is also
closed under finite intersections. Suppose that $A \in \mathcal{A}$. Clearly,
since $A \subseteq A$ and $\mu$ is an infimum, $\mu(A) \leq \mu_0(A)$. Suppose
$A_1,A_2,\ldots \in \mathcal{A}$ with $A \in \bigcup_{i = 1}^{\infty} A_i$.
$\forall i \in \N$, let
\[B_i = A \cap A_i \cap \left( \bigcup_{k = 1}^i A_k \right)^c,\] so that the
$B_i$'s are pairwise disjoint, $A = \bigcup_{i = 1}^{\infty} B_i$, and
$B_i \in \mathcal{A}$. Thus,
\[\mu_0(A) = \mu_0\left( \bigcup_{i = 1}^{\infty} B_i \right)
 = \sum_{i = 1}^{\infty} \mu_0(B_i),\]
so that, by non-negativity and finite additivity of $\mu_0$ on $\mathcal{A}$,
\[\mu_0(A)
 = \sum_{i = 1}^{\infty} \mu_0(B_i)
 \leq \sum_{i = 1}^{\infty} \mu_0(B_i) + \mu_0(A_i \cap B_i^c)
 \leq \sum_{i = 1}^{\infty} \mu_0(A_i).
\]
Taking the infimum on both sides gives $\mu_0(A) \leq \mu(A)$, and
thus $\mu = \mu_0$ on $\mathcal{A}$.

We now show that, $\forall E \in \Sigma$, $A \in \pow{X}$,
$\mu(A) = \mu(A \cap E) + \mu(A \cap E^c)$.

Let $\epsilon > 0$. By definition of $\mu$ as an infimum, there exists a
sequence $A_1,A_2,\ldots \in \mathcal{A}$ such that
\[A \subseteq \bigcup_{i = 1}^{\infty} A_i
 \quad \mbox{ and } \quad
\sum_{i = 1}^{\infty} \mu_0(A_i)
 \leq \mu(A) + \epsilon.
\]
Then,
\begin{align*}
\mu(A \cap E) + \mu(A \cap E^c)
 & \leq \sum_{i = 1}^{\infty} \mu_0(A_i \cap E)
   +    \sum_{i = 1}^{\infty} \mu_0(A_i \cap E^c)
                                       & \mbox{(since $\mu$ is an infimum)} \\
 & =    \sum_{i = 1}^{\infty} \mu_0( (A_i \cap E) \cup (A_i \cap E^c) )
                              & \mbox{(since $\mu_0$ is finitely additive)} \\
 & =    \sum_{i = 1}^{\infty} \mu_0(A_i)
   \leq \mu(A) + \epsilon.
\end{align*}
Then, countable subadditivity of $\mu$ gives that
$\mu(A) \leq \mu(A \cap E) + \mu(A \cap E^c) \leq \mu(A) + \epsilon$ for all
$\epsilon > 0$, so that $\mu(A) = \mu(A \cap E) + \mu(A \cap E^c)$.
Therefore, by the Caratheodory Criterion, $\mu$ is a measure on $\Sigma$, and,
as shown above, it extends $\mu_0$, as desired. \qed

\end{question}
\end{document}
