\documentclass{article} % For LaTeX2e
\usepackage{amsmath, amsfonts, amsthm, amssymb}
\usepackage{nips14submit_e,times}
\usepackage{hyperref}
\usepackage{url}

% For bibliography
\usepackage{natbib}
\renewcommand{\bibsection}{}

\title{11-745 Advanced Statistical Learning Seminar\\Final Paper}

\author{
Shashank Singh \\
Statistics \& Machine Learning Departments \\
Carnegie Mellon University \\
Pittsburgh, PA 15213 \\
\texttt{sss1@andrew.cmu.edu} \\
}

\newcommand{\fix}{\marginpar{FIX}}
\newcommand{\new}{\marginpar{NEW}}

%%%%%%%%%%%%%%%%%%%%CONTENT MACROS%%%%%%%%%%%%%%%%%%%%%%%%%
\renewcommand{\qed}{\quad \ensuremath{\blacksquare}}    % QED blacksquare
\newcommand{\inv}{^{-1}}                            % inverse operator
\newcommand{\sminus}{\backslash}                    % set minus
\newcommand{\N}{\mathbb{N}}                         % natural numbers
\newcommand{\R}{\mathbb{R}}                         % real numbers
\newcommand{\pow}{\mathcal{P}}                      % power set
\newcommand{\Se}{\mathcal{S}}                       % partition
\newcommand{\e}{\varepsilon}                        % \varepsilon
\newcommand{\X}{\mathcal{X}}                        % X domain
\newcommand{\Y}{\mathcal{Y}}                        % Y domain
\newcommand{\Z}{\mathcal{Z}}                        % Z domain
\newcommand{\A}{\mathcal{A}}                        % sub-domain
\newcommand{\E}{\mathbb{E}}                         % expected value
\newcommand{\V}{\mathbb{V}}                         % variance
\newcommand{\pr}{\mathbb{P}}                        % probability
\newcommand{\cpest}{\widehat{p}_h}                  % clipped estimated p_n
\newcommand{\cqest}{\widehat{q}_h}                  % clipped estimated q_n
\newcommand{\pest}{\widetilde{p}_h}                 % estimated p_n
\newcommand{\qest}{\widetilde{q}_h}                 % estimated q_n
\newcommand{\vx}{\vec{x}}                           % vector x
\newcommand{\vy}{\vec{y}}                           % vector y
\newcommand{\vz}{\vec{z}}                           % vector z
\newcommand{\vv}{\vec{v}}                           % vector v
\newcommand{\vu}{\vec{u}}                           % vector u
\newcommand{\vi}{{\vec{i}}}                         % multi-index vector i
\newcommand{\dist}{\operatorname{dist}}             % distance operator
\newcommand{\C}{\mathcal{C}}                        % cannot-link set
\newcommand{\M}{\mathcal{M}}                        % must-link set
\newcommand{\B}{\mathcal{B}}                        % border region of [0,1]^d
\newcommand{\acro}[1]{\textsc{\MakeLowercase{#1}}}
%%%%%%%%%%%%%%%%%%%%%%%%%%%%%%%%%%%%%%%%%%%%%%%%%%%%%%%%%%%

\nipsfinalcopy % Uncomment for camera-ready version

\begin{document}
\maketitle

\begin{abstract}
We review some advances over the past decade in four topics in large scale
machine learning: Unsupervised Clustering, Classification with Structured
Learning, Semi-Supervised Clustering/Metric Learning, and Non-negative Matrix
Factorization.
\end{abstract}

% \section{Introduction}
% \subsection*{Organization}
% 
\section{Large-Scale Unsupervised Clustering}
Papers:
    \cite{gopal14VMF,
          banerjee05VMF,
          blei03LDA}
\quad
Background:
          \cite{bishop06machinelearning}

These papers study large-scale unsupervised learning, primarily in the context
of organizing a large document corpus without supervision (topic modeling),
especially in a hierarchical manner that facilitates user exploration of large
datasets.

Early work (\cite{blei03LDA}) proposed a generative model (Latent Dirichlet
Allocation) describing the following generation procedure for each document:
\begin{enumerate}
\item A topic mixture $\theta$ is sampled from a $k$-dimensional Dirichlet
prior.
\item For each word $w_n$, a topic $z_n$ is drawn according to the weights of
$\theta$.
\item Each word $w_n$ is then sampled from a multinomial distribution
depending on $z_n$ according to a prior distribution $\beta_{z_n}$.
\end{enumerate}
To learn the prior parameters $\alpha$ and $\beta$, \cite{blei03LDA} uses a
variational inference procedure which maximizes a lower bound on the likelihood
(or, equivalently, minimizes the KL-divergence between the estimated posterior
and an approximation of the true posterior).

It has been fairly well-established that normalizing documents to lie on a unit
sphere via Term frequency-Inverse Document frequency (Tf-Idf) normalization
significantly improves the performance of many text-mining algorithms. The
two-parameter von Mises-Fisher (vMF) distribution plays the role of a Gaussian
distribution (a smooth, unimodal, symmetric distribution decaying exponentially
away from its mean) on the unit sphere, and hence serves as a natural basis for
generative clustering models of documents. \cite{gopal14VMF} studies three
increasingly expressive vMF clustering models: a basic Bayesian vMF mixture
model, a hierarchical augmentation of this, and a temporal vMF model for
time-varying cluster centers. All three models are learned via variational
inference. Two schemes are proposed for estimating the posterior distribution
of the vMF concentration parameters: an MCMC sampling approach and a bounding
approach, which uses an asymptotic normality approximation (empirically, the
MCMC approach performs better). Empirical results also show that the
variational inference approach consistently outperforms a similar EM approach
for likelihood maximization.

General conclusions/observations:
\begin{itemize}
\item Variational Bayesian inference provides a powerful alternative to EM in
many settings.
\item von Mises-Fisher distributions are useful for modeling data distributed
on the unit sphere, and hence for modelling documents in combination with
Tf-Idf normalization.
\end{itemize}

\section{Web-Scale Classification with Structured Learning}
Papers:
    \cite{gopal12hierarchicalclass,
          gopal13recursiveregularization}
\quad
Background:
          \cite{bishop06machinelearning}

These papers study classification in the context that some classes arr subsets
or supersets of other classes; i.e., the classes form a heirarchy. This is
natural for web-scale classification, where, for example, topic or image labels
mined from text lie at different levels of the hierarchy.

\cite{gopal12hierarchicalclass} models hierarchies by using a Gaussian
distribution centered at the parameters of a class as a prior distribution for
the parameters of the of child nodes of that class. This encourages nearby
classes in the hierarchy to have similar parameters, and also naturally solves
the general problem of hyperparameter selection inherent in Bayesian models
(only the parameters for the root node must be set manually). Within each level
of the hierarchy, \cite{gopal12hierarchicalclass} propose to use (multi-class)
logistic regression and call the resulting model Hierarchical Bayesian Logistic
Regression. Empirically, HBLR significantly outperforms both non-Bayesian
hierarchical models, as well as flat (non-hierarchical) methods. As with the
hierarchical clustering models above, the model can be learned efficiently via
variational inference, and can also be parallelized by alternating iterations
between even and odd layers of the hierarchy.

\cite{gopal13recursiveregularization} follows up on
\cite{gopal12hierarchicalclass} with two main modeling differences. Firstly,
rather than a Bayesian model with the hierarchical relationships used to define
the priors of the parameters, they use the hierarchical dependencies to define
regularization terms for the parameters in a manner that still encourages
classes nearby in the hierarchy to share similar model parameters. Secondly,
they generalize to the case where dependencies between class-labels can be an
arbitrary graph rather than specifically a hierarchy. Within each layer of the
hierarchy, \cite{gopal13recursiveregularization} propose to use either an SVM
(HR-SVM) or logistic regression (HR-LR). As opposed to previous hierarchical
classification models, incorporating dependencies into the regularization term
rather than constraints reduces the dependency between the parameters, which in
turn allows produces a highly parallelizable optimization problem. For general
graphs, the algorithm can be parallelized up to the chromatic number of the
graph, which reduces to alternatively optimizing odd and even layers, in the
case of hierarchial models. Empirically, HR-SVM consistently outperforms HR-LR,
as well as several established benchmarks on some of the largest datasets that
have been studied.

General conclusions/observations:
\begin{itemize}
\item Hierarchies are effective, efficient organizational structures for basic
supervised and unsupervised machine learning problems, and can make even simple
(e.g., linear) models quite powerful.
\item Again, variational Bayesian inference provides a powerful alternative to
EM.
\item Regularization/priors are very important in large-scale
clustering/classification tasks where many (or even the majority) of classes
have only a few instances.
\end{itemize}

\section{Semi-Supervised Clustering and Metric Learning}
Papers:
    \cite{xing02metriclearning,
          bilenko04metriclearning,
          gopal14supervisedclust}

These papers study the problem of learning an appropriate metric (for
$K$-means clustering or otherwise) using information about the similarity of
a small number of given data points (i.e., limited supervision). They all
operate on $n$ data points in $\R^d$, and take in supervision in the form of
two sets, a ``must-link'' set $\M$ of pairs of points that should lie in the
same cluster and a ``cannot-link'' set $\C$ of pairs of points that should lie
in different clusters.

Older work largely follows one of two approaches to using supervision to
improve clustering: Probabilistic Constrained Clustering (PCC) and Distance
Metric Learning (DML). The first two papers, \cite{xing02metriclearning} and
\cite{bilenko04metriclearning}, are of the latter (DML) approach, which
attempts to learn a metric under which the distances are reduced between pairs
in $\M$ and increased between pairs in $\C$. Specifically,
\cite{xing02metriclearning} learns the correlation matrix $A$ of a metric
of the form $d_A(x,y) = \sqrt{(x - y)^TA(x - y)}$ by solving the optimization
problem
\begin{align*}
\min_A                      & \sum_{(x,y) \in \M} d_A^2(x,y)        \\
\mbox{ such that } \quad    & \sum_{(x,y) \in \C} d_A^2(x,y) \geq 1 \\
\mbox{ and }       \quad    & A \succeq 0.
\end{align*}
This step is performed as a preprocessing step, and $K$-means is then run on
the data with the resulting metric. \cite{bilenko04metriclearning} modifies
this by combining the metric learning and clustering steps. Specifically, in
each iteration, for each pair $(x,y) \in \M$ that are placed in different
clusters, the objective includes a penalty increasing with $d_A(x,y)$, and,
similarly, for each pair $(x,y) \in \C$ that are placed in the same cluster,
the objective includes a penalty which decreases with $d_A(x,y)$. Their
algorithm then alternates between optimizing the objective over the clustering
assignments and the covariance $A$ of the metric.

The third paper, \cite{gopal14supervisedclust}, argues that DML approaches lead
to overfitting, because even well clustered data will be transformed so as to
exaggerate certain distances, and also that, because the metric learning step
is typically ignorant of the clustering objective
(\cite{bilenko04metriclearning} notwithstanding), it need not learn a metric
that is necessarily optimal for clustering. Their approach is to learn a
transformation of the data that maps points in $\M$ and $\C$ in naturally
well-separated clusters according to a specified clustering objective. Their
framework is general, but is worked out for the case of Gaussian or vMF models.

% Empirically, both \cite{xing02metriclearning} and
% \cite{bilenko04metriclearning} both test their methods on (several) relatively
% small datasets from the UCI repository, whereas \cite{gopal14supervisedclust}
% runs experiments on comparatively large datasets.
% 
General conclusions/observations:
\begin{itemize}
\item Metric learning is strongly dependent on the task at hand; e.g., it is
difficult to define a ``good'' metric without a goal, such as clustering, in
mind.
\item Even more specifically, in the case of semi-supervised clustering, it
helps to know the rough form of the clusters, or at least of the clustering
objective or model (e.g., Gaussians Mixture, von Mises-Fisher, etc.) being
used.
% \item Metric learning is only useful when it learns aspects of the data that
% cannot be learned by the learning algorithms into which it feeds. For example,
% $K$-means benefits from learning a metric $d_A$ where $A$ is diagonal, but does
% not benefit much more from learning a full matrix $A$.
\end{itemize}


\section{Large-Scale Matrix Factorization}
Papers:
    \cite{hoyer04NMFwithsparseness,
          jenatton10structuredsparsePCA,
          gemulla11matrixfactorization}

Given an data matrix $X \in \R^{n \times d}$, these papers seek factor
matrices $W \in \R^{n \times k}$ and $H \in \R^{k \times d}$ such that
$X \approx WH$. Typically, $k < d$, so that the factorization implicitly
performs dimension reduction, and it is also desirable for $W$ and $H$ to be
sparse.

One approach is non-negative matrix factorization (NMF), where $X$ has
non-negative entries, and we want $W$ and $H$ to also have non-negative
entries. In principle, this gives an additive parts-based decomposition of
components (e.g. decomposing portrait pictures into face and hair). Because it
gives a parts-based decomposition, NMF tends naturally to give sparse
solutions, but this sparsenss is intrinsic rather than being controlled by a
predefined parameter. Hence, defining sparseness as in terms of the $\ell_1$
and $\ell_2$ norms,
\[\text{sp}(x) = \frac{\sqrt{n} - \|x\|_1/\|x\|_2}{\sqrt{n} - 1},\]
\cite{hoyer04NMFwithsparseness} studies NMF with constraints on the sparseness
$\text{sp}(w_i)$ and $\text{sp}(h_i)$ of the columns of $W$ and the rows of
$H$. They use a multiplicative optimization procedure studied previously for
NMF and project in each iteration to satisfy the desired sparsity constraint.

A second approach is Structured Sparse Principal Component Analysis (SSPCA)
\cite{jenatton10structuredsparsePCA}. PCA can be thought of as learning an
orthogonal basis, or dictionary, which explains the data with low
reconstruction error. As in LASSO and group LASSO, this dictionary can be
sparsified by adding an appropriate regularization term (e.g., an $\ell_1$
norm), and structured sparsity can be enforced with a grouped regularization
term (e.g., an $\ell_1$ norm of $\ell_2$ norms of the appropriate groups).
\cite{jenatton10structuredsparsePCA} propose an optimization scheme which
cycles through optimizing over $W$, $H$, and the groups weights.
While NMF finds a sparse dictionary, its elements are not necessarity
structured. On the other hand, given a dataset of face images, SSPCA finds
elements such as eyes and mouths; that is, in this context, SSPCA identifies
local features.

Finally, \cite{gemulla11matrixfactorization} studies a general distributed
scheme, distributed stochastic gradient descent (DSGD), for matrix
factorization via stochastic gradient descent at very large scale. DSGD can be
adapted for NMF as well as for other matrix factorization problems. DSGD is
compatible with MapReduce, and provably converges under certain conditions.
Furthermore, experiments show DSGD significantly outperforming other
state-of-the-art algorithms for very large matrix factorization problems (e.g.,
Netflix dataset).

General conclusions/observations:
\begin{itemize}
\item Several basic problems for large datasets can be phrased in terms of
matrix factorizations (e.g., variants of PCA, NFM, etc.).
\item Matrix factorization algorithms can be effectively distributed and
performed very efficiently at a large scale.
\end{itemize}

\subsubsection*{References}
\setlength{\bibsep}{0.0pt}
{
%\small
\bibliographystyle{plain}
\bibliography{biblio}
}

\end{document}
