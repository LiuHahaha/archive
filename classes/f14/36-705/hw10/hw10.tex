\documentclass[11pt]{article}
\usepackage{enumerate}
\usepackage{fullpage}
\usepackage{fancyhdr}
\usepackage{amsmath, amsfonts, amsthm, amssymb}
\usepackage{color}
\setlength{\parindent}{0pt}
\setlength{\parskip}{5pt plus 1pt}
\pagestyle{empty}

%%%%%%%%%%%%%%%%%%%%%%%HEADER%%%%%%%%%%%%%%%%%%%%%%%%%%%%%%
\newcommand{\myname}{Shashank Singh\footnote{sss1@andrew.cmu.edu}}
\newcommand{\myclass}{36-705 Intermediate Statistics}
\newcommand{\myhwnum}{10}
\newcommand{\duedate}{Thursday, December 4, 2014}
%%%%%%%%%%%%%%%%%%%%%%%%%%%%%%%%%%%%%%%%%%%%%%%%%%%%%%%%%%%

%%%%%%%%%%%%%%%%%%%%CONTENT MACROS%%%%%%%%%%%%%%%%%%%%%%%%%
\renewcommand{\qed}{\quad \ensuremath{\blacksquare}}
\newcommand{\inv}{^{-1}}
\newcommand{\bv}{\mathbf{v}}
\newcommand{\bx}{\mathbf{x}}
\newcommand{\by}{\mathbf{y}}
\newcommand{\bff}{\mathbf{f}}
\newcommand{\bzero}{\mathbf{0}}
\newcommand{\bxi}{\boldsymbol{\xi}}
\newcommand{\boldeta}{\boldsymbol{\eta}}
\newcommand{\dist}{\operatorname{dist}}
\newcommand{\area}{\operatorname{area}}
\newcommand{\vspan}{\operatorname{span}}
\newcommand{\Gr}{\operatorname{Gr}} % graph of a function
\renewcommand{\sp}{\operatorname{span}} % span of a set
\newcommand{\sminus}{\backslash}
\newcommand{\E}{\mathbb{E}} % expected value
\newcommand{\F}{\mathcal{F}}
\newcommand{\pr}{\mathbb{P}} % probability
% \newcommand{\Var}{\operatorname{Var}} % variance
\newcommand{\Var}{\mathbb{V}} % variance
\newcommand{\Cov}{\operatorname{Cov}} % covariance
\newcommand{\N}{\mathbb{N}} % natural numbers
\newcommand{\Z}{\mathbb{Z}} % integers
\newcommand{\Q}{\mathbb{Q}} % rational numbers
\newcommand{\R}{\mathbb{R}} % real numbers
\newcommand{\A}{\mathcal{A}}
\newcommand{\B}{\mathcal{B}}
\newcommand{\C}{\mathcal{C}} % compact functions
\newcommand{\K}{\mathbb{K}} % underlying field of a linear space
\newcommand{\Ran}{\mathcal{R}} % range of a linear operator
\newcommand{\Nul}{\mathcal{N}} % null-space of a linear operator
\renewcommand{\L}{\mathcal{L}} % bounded linear functions
\newcommand{\pow}[1]{\mathcal{P}\left(#1\right)} % power set of #1
\newcommand{\e}{\varepsilon} % \varepsilon
\newcommand{\wto}{\rightharpoonup} % weak convergence
\newcommand{\wsto}{\stackrel{*}{\rightharpoonup}} % weak-* convergence
\renewcommand{\P}{\mathbb{P}}   % probability
\newcommand{\ol}{\overline}
%%%%%%%%%%%%%%%%%%%%%%%%%%%%%%%%%%%%%%%%%%%%%%%%%%%%%%%%%%%

\begin{document}
\thispagestyle{plain}

{\Large Homework \myhwnum} \\
Name: \myname \\
\myclass \\
Due: \duedate

\begin{enumerate}
\item Since $R(g) = \E[|Y - g(X)|] = \E[\E[|Y - g(X)|\;|\;X]]$, it suffices to
show that, for any $X$, $\E[|Y - g(X)|\;|\;X] \geq \E[|Y - m(X)|\;|\;X]$. If
$m(X) \geq g(X)$, then
\begin{align*}
 &  \E[|Y - g(X)|\;|\;X] - \E[|Y - m(x)|\;|\;X]    \\
 &  \geq \int_{-\infty}^{m(X)} |g(X) - m(X)| p(y|X) \, dy
    - \int_{m(X)}^{\infty} |g(X) - m(X)| p(y|X) \, dy
    = 0.
\end{align*}
where the equality is by definition of $m$. The case $g(X) \leq m(X)$ is
identical up to signs. \qed

\item For $i \in \{n + 1,\dots,2n\}$, let $Z_i := (Y_i - \hat g(X_i))^2$, and
let $Z_i' = Z_i | \mathcal{D}_1$. Then, for all $n \in \N$,
$\ol{Z'} = \ol Z | \mathcal{D}_1 = \hat R | \mathcal{D}_1$. Also, by the
triangle inequality, $|Z'| \leq (|Y| + |\hat g(X)|)^2 \leq (B + C)^2$, and so
all moments of $Z'$ exist. Since $R = \E[Z']$, by the Weak Law of Large
Numbers,
\[(\hat R - R) | \mathcal{D}_1 = \left( \ol{Z'} - \E[Z'] \right) \to 0\]
in probability. \qed

\item Let $\alpha := \E[Y | X = 1] - \E[Y | X = 0]$. Then,
\begin{align*}
\E[2X_iY_i]
 &  = \E[2X_iY_i | X_i = 0]\pr[X_i = 0] + \E[2X_iY_i | X_i = 1]\pr[X_i = 1] \\
 &  = 0 \cdot \frac12 + 2\E[Y_i | X_i = 1] \cdot \frac12
    = \E[Y_i | X_i = 1],
\end{align*}
and, similarly, $\E[2(1 - X_i)Y_i] = \E[Y_i | X = 0]$. Thus, by the Weak Law of
Large Numbers, (assuming the necessary moments of $Y$ are finite),
$\hat\alpha$ is a consistent estimator of $\alpha$.

Note that, given $X$, $Y = Y(X)$, and, since $X$ is randomly assigned,
$\E[Y(1) | X = 1] = \E[Y(1)]$ and $\E[Y(0) | X = 0] = \E[Y(0)]$. Hence,
\begin{align*}
\alpha
 &  = \E[Y | X = 1] - \E[Y | X = 0] \\
 &  = \E[Y(1) | X = 1] - \E[Y(0) | X = 0]
    = \E[Y(1)] - \E[Y(0)]
    = \theta. \qed
\end{align*}

\end{enumerate}
\end{document}
