\documentclass[11pt]{article}
\usepackage{enumerate}
\usepackage{fullpage}
\usepackage{fancyhdr}
\usepackage{amsmath, amsfonts, amsthm, amssymb}
\usepackage{color}
\setlength{\parindent}{0pt}
\setlength{\parskip}{5pt plus 1pt}
\pagestyle{empty}

\def\indented#1{\list{}{}\item[]}
\let\indented=\endlist

\newcounter{questionCounter}
\newcounter{partCounter}[questionCounter]
\newenvironment{question}[2][\arabic{questionCounter}]{%
    \setcounter{partCounter}{0}%
    \vspace{.25in} \hrule \vspace{0.5em}%
        \noindent{\bf #2}%
    \vspace{0.8em} \hrule \vspace{.10in}%
    \addtocounter{questionCounter}{1}%
}{}
\renewenvironment{part}[1][\alph{partCounter}]{%
    \addtocounter{partCounter}{1}%
    \vspace{.10in}%
    \begin{indented}%
       {\bf (#1)} %
}{\end{indented}}

%%%%%%%%%%%%%%%%%%%%%%%HEADER%%%%%%%%%%%%%%%%%%%%%%%%%%%%%%
\newcommand{\myname}{Shashank Singh}
\newcommand{\myandrew}{sss1@andrew.cmu.edu}
\newcommand{\myclass}{21-260 Differential Equations}
\newcommand{\myhwnum}{8}
\newcommand{\duedate}{Tuesday, July 31, 2012}
%%%%%%%%%%%%%%%%%%%%%%%%%%%%%%%%%%%%%%%%%%%%%%%%%%%%%%%%%%%

%%%%%%%%%%%%%%%%%%%%CONTENT MACROS%%%%%%%%%%%%%%%%%%%%%%%%%
\renewcommand{\qed}{\quad $\blacksquare$}
\newcommand{\mqed}{\quad \blacksquare}
\newcommand{\diffeq}{differential equation }
\newcommand{\inv}{^{-1}}
\newcommand{\ba}{\mathbf{a}}
\newcommand{\bb}{\mathbf{b}}
\newcommand{\bx}{\mathbf{x}}
\newcommand{\bzero}{\mathbf{0}}
\newcommand{\bxi}{\boldsymbol{\xi}}
\newcommand{\boldeta}{\boldsymbol{\eta}}
%%%%%%%%%%%%%%%%%%%%%%%%%%%%%%%%%%%%%%%%%%%%%%%%%%%%%%%%%%%

\begin{document}
\thispagestyle{plain}

{\Large Homework \myhwnum} \\
\myclass \\
Name: \myname \\
Email: \myandrew \\
Due: \duedate \\
\begin{question}{Section 6.1, Problem 16}
If $F$ is the Laplace transform of $f$, then, by definition,
\[F(s)
 = \int_0^{\infty} e^{-st} f(t) \, dt
 = \int_0^{\infty} e^{-st} t \sin(at) \, dt.
\]
Integration by Parts (differentiating the $te^{-st}$ term and integrating the
$\sin(at)$ term) gives
\begin{eqnarray*}
F(s)
 & = & -te^{-st}\frac{\cos(at)}{a}\big|_{t = 0}^{t = \infty}
   + \frac1a\int_0^{\infty}\left(e^{-st}
                                  - \frac{t}{s}e^{-st}\right)\cos(at) \, dt \\
 & = & \frac1a\int_0^{\infty}\left(e^{-st}
                                  - \frac{t}{s}e^{-st}\right)\cos(at) \, dt.
\end{eqnarray*}

Integration by Parts again (differentiating the $\frac{t}{s}e^{-st}$ term,
integrating the $\cos(at)$ term, and ignoring the $e^{-st}\cos(at)$ term by
linearity of the integral for the time being)
\begin{eqnarray*}
F(s)
 & = & \frac1a\int_0^{\infty}
                        - \frac{1}{sa}\int_0^{\infty}te^{-st}\cos(at) \, dt \\
 & = & \frac1a\int_0^{\infty}
                - \frac{1}{sa}\left( \frac{F(s)}{sa}
                                 - \int_0^{\infty}e^{-st}\sin(at) \, dt\right).
\end{eqnarray*}
Solving for $F(s)$ in the above equation gives
\[F(s)
 = \left(1 + \frac{1}{s^2a^2}\right)\inv
   \left(\frac1a g_1(s) + \frac{1}{sa^2}g_2(s)\right)
 = \frac{s^2a}{s^2a^2 + 1}g_1(s) + \frac{s}{s^2a^2 + 1}g_2(s)
\]
where $g_1(s) = \int_0^{\infty} e^{-st}\cos(at) \, dt$ and
$g_2 = \int_0^{\infty} e^{-st} \sin(at) \, dt$.

Table 6.2.1 shows that
\[\int_0^{\infty}e^{-st}\cos(at) \, dt
 = \mathcal{L}\{\cos(at)\} = \frac{s}{s^2 + a^2},\]
and that
\[\int_0^{\infty}e^{-st}\sin(at) \, dt
 = \mathcal{L}\{\sin(at)\} = \frac{a}{s^2 + a^2}.\]

Thus,
\[F(s)
 = \mbox{
    \fbox{
        $\displaystyle
            \frac{s^3a}{(s^2a^2 + 1)(s^2 + a^2)} + \frac{sa}{(s^2a^2 + 1)(s^2 + a^2)}
        $.
    }
   }
\]
\end{question}

\begin{question}{Section 6.1, Problem 26abc}
\begin{enumerate}[(a)]
\item Suppose $p > 0$. Then, Integration by Parts (differentiating $x^p$ and
integrating $e^{-x}$) gives that
\begin{eqnarray*}
\Gamma(p + 1)
 & = &  \int_0^{\infty} x^pe^{-x}                      \, dx \\
 & = & -x^pe^{-x}\big|_{x = 0}^{x = \infty}
   -    \int_0^{\infty} px^{p - 1}\left(-e^{-x}\right) \, dx \\
 & = & p\int_0^{\infty} x^{p - 1}e^{-x}                \, dx
   =   p\Gamma(p). \mqed
\end{eqnarray*}

\item
\[\Gamma(1)
 = \int_0^{\infty} e^{-x} \, dx
 = -e^{-x}\big|_{x = 0}^{x = \infty}
 = \lim_{x \rightarrow \infty}\left(-e^{-x}\right) - \left(-e^{-0}\right)
 = 0 - (-1)
 = 1. \mqed
\]

\item Since the factorial is defined by the recurrence relation
\[n! = n\cdot(n - 1)! \mbox{ for } n \in \mathbb{N}\backslash\{0\},\quad 0! = 1,\]
the desired result follows immediately by the Principle of Mathematical
Induction (the result of part (b) above is the base case, while the result of
part (a) above is the general case). \qed
\end{enumerate}
\end{question}

\begin{question}{Section 6.2, Problem 14}
As given in equation (16) is Section 2 of Chapter 6, if $Y$ is the Laplace
transform of the solution to the given initial value problem, then
\[Y(s)
 = \frac{(s - 4) + 1}{s^2 - 4s + 4}
 = \frac{(s - 2) - 1}{s^2 - 4s + 4}
 = \frac{1}{s - 2} - \frac{1}{(s - 2)^2}.
\]
Table 6.2.1 shows that $\frac{1}{s - 2}$ is the Laplace transform of the
function $e^{2t}$ of $t$, and that $\frac{1}{(s - 2)^2}$ is the Laplace
transform of the function $te^{2t}$ of $t$. Thus, by linearity of the inverse
Laplace transform, the solution to the given initial value problem is
\[y(t)
 = \mathcal{L}\inv      \{ Y(s)                      \}
 = \mathcal{L}\inv \left\{ \frac{1}{s - 2}     \right\}
 - \mathcal{L}\inv \left\{ \frac{1}{(s - 2)^2} \right\}
 = \mbox{\fbox{$e^{2t} - te^{2t}$.}}
\]
\end{question}

\newpage
\begin{question}{Section 6.2, Problem 20}
As given in equation (16) is Section 2 of Chapter 6, if $Y$ is the Laplace
transform of the solution to the given initial value problem, then (using Table
6.2.1) to compute $\mathcal{L}\{\cos(2t)\}$)
\[Y(s)
 = \frac{s}{s^2 + \omega^2}
 + \frac{s}{(s^2 + 4)(s^2 + \omega^2)}
 = \frac{s}{s^2 + \omega^2}
 + \frac{s}{(\omega^2 - 4)(s^2 + 4)}
 - \frac{s}{(\omega^2 - 4)(s^2 + \omega^2)}.
\]
Table 6.2.1 shows that $\frac{s}{s^2 + \omega^2}$ is the Laplace transform of
the function $\cos(\omega t)$, and that $\frac{s}{s^2 + 4}$ is the Laplace
transform of $\cos(2t)$. Thus, by linearity of the inverse Laplace transform
(noting that the $\frac{1}{\omega^2 - 4}$ terms are constants in $s$),
the solution to the given initial value problem is
\begin{eqnarray*}
y(t)
   =                         \mathcal{L}\inv      \{ Y(s)                           \}
 & = &                       \mathcal{L}\inv \left\{ \frac{s}{s^2 + \omega^2} \right\}
   +   \frac{1}{\omega^2 - 4}\mathcal{L}\inv \left\{ \frac{s}{s^2 + 4}        \right\}
   -   \frac{1}{\omega^2 - 4}\mathcal{L}\inv \left\{ \frac{s}{s^2 + \omega^2} \right\} \\
 & = & \mbox{
         \fbox{
           $\displaystyle
               \cos(\omega t)
             + \frac{\cos(2t)}{\omega^2 - 4}
             + \frac{\cos(\omega t)}{\omega^2 - 4}
           $.
         }
       }
\end{eqnarray*}
\end{question}

\begin{question}{Section 6.3, Problem 12}
On the interval $[0,\infty)$,
\begin{eqnarray*}
f(t)
 & = & t + (2 - t)u_2(t) + (7 - t - 2)u_5(t) - (7 - t)u_7(t) \\
 & = & \mbox{
         \fbox{
           $t + (2 - t)u_2(t) + (5 - t)    u_5(t) - (7 - t)u_7(t)$.
         }
       }
\end{eqnarray*}
\end{question}
\end{document}
