\documentclass[11pt]{article}
\usepackage{enumerate}
\usepackage{fullpage}
\usepackage{fancyhdr}
\usepackage{amsmath, amsfonts, amsthm, amssymb}
\setlength{\parindent}{0pt}
\setlength{\parskip}{5pt plus 1pt}
\pagestyle{empty}

\def\indented#1{\list{}{}\item[]}
\let\indented=\endlist

\newcounter{questionCounter}
\newcounter{partCounter}[questionCounter]
\newenvironment{question}[2][\arabic{questionCounter}]{%
    \setcounter{partCounter}{0}%
    \vspace{.25in} \hrule \vspace{0.5em}%
        \noindent{\bf #2}%
    \vspace{0.8em} \hrule \vspace{.10in}%
    \addtocounter{questionCounter}{1}%
}{}
\renewenvironment{part}[1][\alph{partCounter}]{%
    \addtocounter{partCounter}{1}%
    \vspace{.10in}%
    \begin{indented}%
       {\bf (#1)} %
}{\end{indented}}

%%%%%%%%%%%%%%%%%%%%%%%HEADER%%%%%%%%%%%%%%%%%%%%%%%%%%%%%%
\newcommand{\myname}{Shashank Singh}
\newcommand{\myandrew}{sss1@andrew.cmu.edu}
\newcommand{\myclass}{21-260 Differential Equations}
\newcommand{\myhwnum}{2}
\newcommand{\duedate}{Tuesday, July 10, 2012}
%%%%%%%%%%%%%%%%%%%%%%%%%%%%%%%%%%%%%%%%%%%%%%%%%%%%%%%%%%%

%%%%%%%%%%%%%%%%%%%%%%%%MACROS%%%%%%%%%%%%%%%%%%%%%%%%%%%%%
\renewcommand{\qed}{\quad $\blacksquare$}
\newcommand{\mqed}{\quad \blacksquare}
\newcommand{\diffeq}{differential equation }
\newcommand{\inv}{^{-1}}
%%%%%%%%%%%%%%%%%%%%%%%%%%%%%%%%%%%%%%%%%%%%%%%%%%%%%%%%%%%

\begin{document}
\thispagestyle{plain}

{\Large Homework \myhwnum} \\
\myclass \\
Name: \myname \\
Email: \myandrew \\
Due: \duedate \\

\begin{question}{Section 2.3, Problem 6}
\begin{enumerate}[(a)]
\item The gravitational potential energy of a particle with mass $m$ at height
$h$ is $mgh$, whereas the kinetic energy of such a particle after having
fallen from height $h$ is $\frac12 mv^2$, where $v$ is the speed of the
particle; equating the two by Conservation of Energy and solving for $v$
gives (by Torricelli's principle) the desired quantity:
\[v = \sqrt{\frac{2mgh}{m}} = \sqrt{2gh}.\mqed\]

\item The instantaneous rate of change of the volume $V$ of liquid in the tank
is the product of the cross-sectional area of the tank (as seen from above)
and the instantaneous rate of change in the height of the liquid in the tank:
$\frac{dV}{dt} = A(h) \frac{dh}{dt}$.

Assuming the density of the water in the tank is uniform, since the mass is
conserved, this is the same as the volumetric rate of outflow from the tank,
which is the product of the cross-sectional area of the outflow stream and the
outflow velocity $v$:
$\frac{dV}{dt} = -\alpha av$.

Thus, by the result of part (a), as desired,
\begin{equation}
A(h) \frac{dh}{dt} = -\alpha a\sqrt{2gh}. \mqed
\end{equation}

\item Since the tank is a right cylinder, for $0\,m < h < 3\,m$,
$A = \pi\,m^2$ in equation (1) above is constant with respect to $h$. Thus,
rewriting equation (1) as
\[Ah^{-1/2} \frac{dh}{dt} = -\alpha a\sqrt{2g}\]
shows that the equation is separable, so that, as shown in class,
\[2A\sqrt{h}
 = \int Ah^{-1/2} \; dh
 = \int -\alpha a\sqrt{2g} \; dt
 = (-\alpha a\sqrt{2g})t + C,
\]
for some $C \in \mathbb{R}$. Since the tank is initially full,
$h(0) = 3 \, m$, giving $C = 2\sqrt{3}A$. Thus, when $h = 0$,
\[t
 =         \frac{2\sqrt{3}A}{\alpha a\sqrt{2g}}
 \approx   \mbox{
               \fbox{
                   $130\,s$.
               }
           }
\]
\end{enumerate}
\end{question}

\begin{question}{Section 2.3, Problem 8}
\begin{enumerate}[(a)]
\item As derived in example 2 in section 2.3 of the textbook,
\begin{equation}
S(t) = \mbox{\fbox{$\displaystyle \frac{k}{r}\left(e^{rt} - 1)\right)$.}}
\end{equation}

\item Solving equation (2) above in the case $S = 10^6$, $r = 0.075$, $t = 40$
for $k$ gives
\[k
 = \frac{0.075\cdot10^6}{e^{0.075\cdot40} - 1}
 \approx \mbox{\fbox{3930}}
\]

\item Solving equation (2) above in the case $S = 10^6$, $k = 2000$, $t = 40$
for $r$ gives
\[r \approx \mbox{\fbox{0.098.}}\]

\end{enumerate}
\end{question}

\begin{question}{Section 2.4, Problem 30}
The \diffeq can be re-written in the form
\begin{equation}
y^{-3}y^{\prime} - \epsilon y^{-2} = -\sigma.
\end{equation}
For $v = y^{1 - n} = y^{-2}$, $v^{\prime} = -2y^{-3}y^{\prime}$ (where
differentiation is with respect to the independent variable). Thus,
substituting $v$ and $v^{\prime}$ appropriately into equation (1) gives:
\[-\frac12v^{\prime} - \epsilon v = -\sigma,\]
which is a linear first-order ODE, whose standard form is
\[v^{\prime} + 2\epsilon v = 2\sigma.\]
Let $\mu = e^{\int 2\epsilon \, dt} = e^{2\epsilon t}$. Then, multiplying by
$\mu$ gives
\[v^{\prime}\mu + 2\epsilon v\mu = 2\sigma\mu,\]
so that
\[v\mu
 = \int \left( v^{\prime}\mu + 2\epsilon v\mu \right)\,dt
 = \int 2\sigma\mu\,dt
 = \frac{2\sigma e^{2\epsilon t}}{2 \epsilon} + C,
\]
for some $C \in \mathbb{R}$. Thus,
\[v
 = \frac{2\sigma e^{2\epsilon t}}{2\epsilon\mu} + \frac{C}{\mu}
 = \frac{2\sigma}{2\epsilon} + Ce^{-2\epsilon t},
\]
so that
\[y
 = v^{-1/2}
 = \mbox{
    \fbox{
        $\displaystyle \left(
                                \frac{2\sigma}
                                     {2\epsilon}
                                + Ce^{-2\epsilon t}
                       \right)^{-1/2}$.
    }
   }
\]

\end{question}

\begin{question}{Section 2.5, Problem 10}
Figure 1 (see last page) shows sketch of graph of $f$. $f(y)$ has as roots
$y \in \{-1,0,1\}$. Furthermore, $\forall y \in (-\infty,-1), f(y) > 0$,
$\forall y \in (-1,0), f(y) < 0$, $\forall y \in (0,1), f(y) > 0$, and,
$\forall y \in (1,\infty), f(y) < 0$. \fbox{Thus, $y = -1$ and $y = 1$ are stable
equilibrium points, and $y = 0$ is an unstable equilibrium point.}
\end{question}

\begin{question}{Section 2.5, Problem 21}
\begin{enumerate}[(a)]
\item The quadratic formula gives that $\frac{dy}{dt} = 0$ if and only if
\[y = \mbox{\fbox{$\displaystyle \frac{r \pm \sqrt{r^2 - 4rh/K}}{2r/K}$.}}\]
If $h < rK/4$, then the discriminant $r^2 - 4rh/K$ above is strictly positive,
so that there exist $2$ distinct values of $y$ such that $\frac{dy}{dt} = 0$,
as determined by the above equation. \qed

\item Since $\frac{dy}{dt}$ is quadratic polynomial in terms of $y$ and it
first coefficient is negative, for $y < y_1$, $\frac{dy}{dt} < 0$, and, for
$y_1 < y < y_2$, $\frac{dy}{dt} > 0$, so that $y_1$ is an unstable equilibrium
point. Similarly, since, for $y_1 < y < y_2$, $\frac{dy}{dt} > 0$, and, for
$y > y_2$, $\frac{dy}{dt} < 0$, $y_2$ is an asymptotically stable equilibrium
point. \qed

\item Since, as shown in Figure 2 (see last page), $\frac{dy}{dt} > 0$ when
$y_1 < y < y_2$ and $\frac{dy}{dt} < 0$ when $y > y_2$, when $y_1 < y < y_2$,
$y$ increases over time, and, when $y > y_2$, $y$ decreases over time, in
either case approaching $y_2$. Similarly, since $\frac{dy}{dt} < 0$ when
$y < y_1$, when $y < y_1$ decreases over time, approaching $-\infty$. \qed

\item For $h > rK/4$, the discriminant found in part (a) above is negative, so
that the quadratic ($\frac{dy}{dt}$) has no zeros. Since the leading
coefficient of the quadratic is negative, this is possible only if it takes
only negative values, so that, $\forall y \in \mathbb{R}$,
$\frac{dy}{dt} < 0$, and thus $y$ always decreases over time. \qed

\item For $h = rK/4$, the discriminant fround in part (a) above is $0$, so
that $\frac{dy}{dt}$ has exactly $1$ root, $y = K/2$, and so $y = K/2$ is the
single equilibrium point. Since, for $y \neq K/2$, $\frac{dy}{dt} < 0$, this
equilibrium point is semistable (it resists increase in $y$, but not decrease
in $y$). \qed
\end{enumerate}
\end{question}
\end{document}
