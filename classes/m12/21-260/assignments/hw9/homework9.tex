\documentclass[11pt]{article}
\usepackage{enumerate}
\usepackage{fullpage}
\usepackage{fancyhdr}
\usepackage{amsmath, amsfonts, amsthm, amssymb}
\usepackage{color}
\setlength{\parindent}{0pt}
\setlength{\parskip}{5pt plus 1pt}
\pagestyle{empty}

\def\indented#1{\list{}{}\item[]}
\let\indented=\endlist

\newcounter{questionCounter}
\newcounter{partCounter}[questionCounter]
\newenvironment{question}[2][\arabic{questionCounter}]{%
    \setcounter{partCounter}{0}%
    \vspace{.25in} \hrule \vspace{0.5em}%
        \noindent{\bf #2}%
    \vspace{0.8em} \hrule \vspace{.10in}%
    \addtocounter{questionCounter}{1}%
}{}
\renewenvironment{part}[1][\alph{partCounter}]{%
    \addtocounter{partCounter}{1}%
    \vspace{.10in}%
    \begin{indented}%
       {\bf (#1)} %
}{\end{indented}}

%%%%%%%%%%%%%%%%%%%%%%%HEADER%%%%%%%%%%%%%%%%%%%%%%%%%%%%%%
\newcommand{\myname}{Shashank Singh}
\newcommand{\myandrew}{sss1@andrew.cmu.edu}
\newcommand{\myclass}{21-260 Differential Equations}
\newcommand{\myhwnum}{8}
\newcommand{\duedate}{Tuesday, July 31, 2012}
%%%%%%%%%%%%%%%%%%%%%%%%%%%%%%%%%%%%%%%%%%%%%%%%%%%%%%%%%%%

%%%%%%%%%%%%%%%%%%%%CONTENT MACROS%%%%%%%%%%%%%%%%%%%%%%%%%
\renewcommand{\qed}{\quad $\blacksquare$}
\newcommand{\mqed}{\quad \blacksquare}
\newcommand{\diffeq}{differential equation }
\newcommand{\inv}{^{-1}}
\newcommand{\ba}{\mathbf{a}}
\newcommand{\bb}{\mathbf{b}}
\newcommand{\bx}{\mathbf{x}}
\newcommand{\bzero}{\mathbf{0}}
\newcommand{\bxi}{\boldsymbol{\xi}}
\newcommand{\boldeta}{\boldsymbol{\eta}}
%%%%%%%%%%%%%%%%%%%%%%%%%%%%%%%%%%%%%%%%%%%%%%%%%%%%%%%%%%%

\begin{document}
\thispagestyle{plain}

{\Large Homework \myhwnum} \\
\myclass \\
Name: \myname \\
Email: \myandrew \\
Due: \duedate \\
\begin{question}{Section 6.4, Problem 16ab}
\begin{enumerate}[(a)]
\item Figure 1 below shows the graph of $g(t)$.
\vspace{1.5in}

\item
As shown in Section 2 of Chapter 6, for $Y(s) = \mathcal{L}\{u(t)\}$,
since
$\mathcal{L}\{ku_{3/2}(t) - ku_{5/2}(t)\}
= \frac{k}{s}\left(e^{-\frac32s} - e^{-\frac52s}\right)$,
\[Y(t) = k\left(\frac{e^{-\frac32s} - e^{-\frac52s}}{s\left(s^2 + \frac14s + 1\right)}\right)\]

Partial fraction decomposition shows that
\begin{eqnarray*}
Y(t)
 & = & k\left(\frac{1}{s} - \frac{s + 1/4}{s^2 + \frac14s + 1}\right)\left(e^{-\frac32s} - e^{-\frac52s}\right) \\
 & = & k\left(\frac{1}{s} - \frac{s + 1/8}{s^2 + \frac14s + 1} - \frac18\frac{1}{s^2 + \frac14s + 1}\right)\left(e^{-\frac32s} - e^{-\frac52s}\right) \\
 & = & k\left(\frac{1}{s} - \frac{s + 1/8}{(s + 1/8)^2+63/64} - \frac{64}{504}\frac{63/64}{(s + 1/8)^2+63/64}\right)\left(e^{-\frac32s} - e^{-\frac52s}\right)
\end{eqnarray*}
Thus, Table 6.2.1 shows that
\begin{eqnarray*}
y(t)
 & = & ku_{3/2}(t)\left(1 - e^{-\frac18t}\cos\left(\frac{63}{64}\left(t - \frac32\right)\right) - \frac{64}{504}e^{-\frac18t}\sin\left(\frac{63}{64}\left(t - \frac32\right)\right)\right) \\
 & - & ku_{5/2}(t)\left(1 - e^{-\frac18t}\cos\left(\frac{63}{64}\left(t - \frac52\right)\right) - \frac{64}{504}e^{-\frac18t}\sin\left(\frac{63}{64}\left(t - \frac52\right)\right)\right).
\end{eqnarray*}
\end{enumerate}
\end{question}

\newpage
\begin{question}{Section 6.5, Problem 6a}
\begin{enumerate}[(a)]
\item
As shown in Section 2 of Chapter 6, for $Y(s) = \mathcal{L}\{y(t)\}$,
since $\mathcal{L}\{\delta(t - 4\pi)\} = e^{-4\pi s}$,
\[Y(s)
 = \frac{s/2 + e^{-4\pi s}}{s^2 + 4}
 = \frac12\frac{s}{s^2 + 4} + \frac12e^{-4\pi s}\frac{2}{s^2 + 4}.
\]
Table 6.2.1 shows that
\[\frac{s}{s^2 + 4} = \mathcal{L}\{\cos(2t)\}
    \quad \mbox{ and that } \quad
  e^{-4\pi s}\frac{s}{s^2 + 4} = \mathcal{L}\{u_{4\pi}(t) \sin(2(t - 4\pi))\}
.\]
Thus, by linearity of the inverse Laplace transform,
\[y(t)
 = \mathcal{L}\inv\{Y(s)\}
 = \mbox{
    \fbox{
        $\displaystyle
            \frac12\left(
                \cos(2t)
              + u_{4\pi}(t)\sin(2(t - 4\pi))
            \right)
        $.
    }
   }
\]

\end{enumerate}
\end{question}

\begin{question}{Section 6.5, Problem 16abc}
\begin{enumerate}[(a)]
\item
As shown in Section 2 of Chapter 6, for $Y_k(s) = \mathcal{L}\{\phi(t,k)\}$,
since
$\mathcal{L}\{f_k(t)\}
 = \frac{1}{2ks}\left(e^{-(4 - k)s} - e^{-(4 + k)s}\right)$,
\[Y_k(s)
 = \frac{1}{2ks}
   \left(
    e^{-(4 - k)s} - e^{-(4 + k)s}
   \right)
   \left(
    \frac{1}{s^2 + 1}
    \right).
\]

Table 6.2.1 shows that $\frac{1}{s} = \mathcal{L}\{1\}$ and that
$\frac{s}{s^2 + 1} = \mathcal{L}\{\cos(t)\}$. Thus, by linearity of the
inverse Laplace transform,
\[\phi(t,k)
 = \mathcal{L}\inv\{Y(s)\}
 = \mbox{\fbox{$\displaystyle \frac{1}{2k}
   \left(
     u_{4 - k}(t)(1 - \cos(4 - k - t))
   - u_{4 + k}(t)(1 - \cos(4 + k - t))
   \right)$.}}
\]

\item
\[\lim_{k \rightarrow 0} \left(\phi(t,k)\right)
 = \mbox{\fbox{$u_4(t)\sin(t - 4)$.}}
\]

\item
As shown in Section 2 of Chapter 6, for $Y(s) = \mathcal{L}\{\phi_0(t)\}$,
since
$\mathcal{L}\{\delta(t - 4)\} = e^{-4s}$,
\[Y(s)
 = \frac{e^{-4s}}{s^2 + 1}.
\]
Table 6.2.1 shows that
\[\phi_0(t)
 = \mathcal{L}\inv\{Y(s)\}
 = \mbox{\fbox{$u_4(t)\sin(t - 4)$,}}
\]
so that $\phi_0(t) = \lim_{k \rightarrow 0}\left(\phi(t,k)\right)$.
\end{enumerate}
\end{question}

\begin{question}{Section 6.6, Problem 16}
As shown in Section 2 of Chapter 6, for $Y(s) = \mathcal{L}\{y(t)\}$,
\[Y(s)
 = \frac{s + \mathcal{L}\{1 - u_{\pi}(t)\}}{s^2 + s + \frac54}
 = \frac{s + \mathcal{L}\{1 - u_{\pi}(t)\}}{\left(s + \frac12\right)^2 + 1}.
\]
By linearity of the inverse Laplace transform, then,
\[y(t)
 = \mathcal{L}\inv\{Y(s)\}
 = \mathcal{L}\inv\left\{\frac{s + \frac12}{\left(s + \frac12\right)^2 + 1}\right\}
 - \frac12\mathcal{L}\inv\left\{\frac{1}{\left(s + \frac12\right)^2 + 1}\right\}
 + \mathcal{L}\inv\left\{\frac{\mathcal{L}\{1 - u_{\pi}(t)\}}{\left(s + \frac12\right)^2 + 1}\right\}.
\]
Table 6.2.1 shows that
$\frac{s + \frac12}{\left(s + \frac12\right)^2 + 1}
 = \mathcal{L}\{e^{-\frac12 t}\cos(t)\}$
and that
$\frac{1}{\left(s + \frac12\right)^2 + 1}
 = \mathcal{L}\{e^{-\frac12 t}\sin(t)\}$.
Thus, by Theorem 6.6.1,
\[y(t)
 = \mbox{\fbox{$\displaystyle e^{-\frac12 t}\cos(t)
 - \frac12e^{-\frac12 t}\sin(t)
 + (1 - u_{\pi}(t))*e^{-\frac12 t}\sin(t)$,}}
\]
where $*$ denotes convolution.
\end{question}
\end{document}
