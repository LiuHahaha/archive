\documentclass[11pt]{article}
\usepackage{enumerate}
\usepackage{fullpage}
\usepackage{fancyhdr}
\usepackage{amsmath, amsfonts, amsthm, amssymb}
\usepackage{color}
\setlength{\parindent}{0pt}
\setlength{\parskip}{5pt plus 1pt}
\pagestyle{empty}

\def\indented#1{\list{}{}\item[]}
\let\indented=\endlist

\newcounter{questionCounter}
\newcounter{partCounter}[questionCounter]
\newenvironment{question}[2][\arabic{questionCounter}]{%
    \setcounter{partCounter}{0}%
    \vspace{.25in} \hrule \vspace{0.5em}%
        \noindent{\bf #2}%
    \vspace{0.8em} \hrule \vspace{.10in}%
    \addtocounter{questionCounter}{1}%
}{}
\renewenvironment{part}[1][\alph{partCounter}]{%
    \addtocounter{partCounter}{1}%
    \vspace{.10in}%
    \begin{indented}%
       {\bf (#1)} %
}{\end{indented}}

%%%%%%%%%%%%%%%%%%%%%%%HEADER%%%%%%%%%%%%%%%%%%%%%%%%%%%%%%
\newcommand{\myname}{Shashank Singh}
\newcommand{\myandrew}{sss1@andrew.cmu.edu}
\newcommand{\myclass}{21-260 Differential Equations}
\newcommand{\myhwnum}{5}
\newcommand{\duedate}{Thursday, July 19, 2012}
%%%%%%%%%%%%%%%%%%%%%%%%%%%%%%%%%%%%%%%%%%%%%%%%%%%%%%%%%%%

%%%%%%%%%%%%%%%%%%%%CONTENT MACROS%%%%%%%%%%%%%%%%%%%%%%%%%
\renewcommand{\qed}{\quad $\blacksquare$}
\newcommand{\mqed}{\quad \blacksquare}
\newcommand{\diffeq}{differential equation }
\newcommand{\inv}{^{-1}}
\newcommand{\ba}{\mathbf{a}}
\newcommand{\bb}{\mathbf{b}}
\newcommand{\bx}{\mathbf{x}}
\newcommand{\bzero}{\mathbf{0}}
\newcommand{\bxi}{\boldsymbol{\xi}}
\newcommand{\boldeta}{\boldsymbol{\eta}}

%%%%%%%%%%%%%%%%%%%%%%%%%%%%%%%%%%%%%%%%%%%%%%%%%%%%%%%%%%%

\begin{document}
\thispagestyle{plain}

{\Large Homework \myhwnum} \\
\myclass \\
Name: \myname \\
Email: \myandrew \\
Due: \duedate \\

\begin{question}{Section 7.5, Problem 4}
\begin{enumerate}[(a)]
\item If $A$ is the matrix of constant coefficients of the given system, then,
if $\lambda$ is an eigenvalue of $A$, then,
\[0
 = (1 - \lambda)(-2 - \lambda) - 4
 = (\lambda + 3)(\lambda - 2),
\]
so that $\lambda \in \{-3,2\}$. Thus, the eigenvectors of $A$ associated with
eigenvalues $\lambda = -3$ and $\lambda = 2$ are, respectively,
\[\bxi^{(1)}
 = \begin{bmatrix}
        1  \\
        -4
   \end{bmatrix}, \quad
 \bxi^{(2)}
 = \begin{bmatrix}
        1 \\
        1
   \end{bmatrix}.
\]
Thus, solutions to the system of differential equations are of the form
\[\bx = \mbox{\fbox{$c_1\bxi^{(1)}e^{-3t} + c_2\bxi^{(2)}e^{2t}$.}}\]
Since the eigenvalues of $A$ have different signs, the solution will approach
multiples a basis vector of the general solution (in particular,
$\bxi^{(2)}$) as $t \rightarrow \infty$.

\item Figure 1 below shows a direction field as well as some sample
trajectories for $\bx$.
\vspace{2in}

\newpage
\end{enumerate}
\end{question}

\begin{question}{Section 7.6, Problem 6}
\begin{enumerate}[(a)]
\item
If $A$ is the matrix of constant coefficients of the given system of
differential equations, then, if $\lambda$ is an eigenvalue of $A$, then
$\lambda = \pm 3i$. Thus, the eigenvectors of $A$ associated with eigenvalues
$\lambda = 3i$ and $\lambda = -3i$, respectively, are
\[\bxi^{(1)}
 = \mbox{\fbox{$\begin{bmatrix}
        2     \\
        -1 + 3i
   \end{bmatrix}$,}} \quad
 \bxi^{(2)}
 = \mbox{\fbox{$\begin{bmatrix}
        2     \\
        -1 - 3i
   \end{bmatrix}$.}}
\]
For
$\ba = \begin{bmatrix} 2 \\ -1 \end{bmatrix}$,
$\bb = \begin{bmatrix} 0 \\ 3  \end{bmatrix}$, since $\bxi = \ba + i\bb$, for
\begin{eqnarray*}
\bx^{(1)}& = & e^{0\cdot t}(\ba\cos(3 t) - \bb\sin(3 t))
            = \ba\cos(3 t) - \bb\sin(3 t) \\
\bx^{(2)} & = & e^{0\cdot t}(\ba\sin(3 t)+ \bb\cos(3 t))
 = \ba\sin(3 t)+ \bb\cos(3 t),
\end{eqnarray*}
$\{\bx^{(1)},\bx^{(2)}\}$ is a fundamental set of real-valued solutions, so
that solutions to the given system of differential equations are of the form
\[\bx = \mbox{\fbox{$c_1\bx^{(1)} + c_2\bx^{(2)}$,}}\]
for some $c_1,c_2 \in \mathbb{R}$.

\item Since the eigenvalues of $A$ have no real part, motion of $\bx$ over
time is purely elliptical and there is no limit in $\bx$ as
$t \rightarrow \infty$. Figure 2 below shows a direction field as well as
some sample trajectories for $\bx$.
\vspace{2in}
\end{enumerate}
\end{question}

\newpage
\begin{question}{Section 7.8, Problem 4}
\begin{enumerate}[(a)]
\item Figure 3 below shows a direction field as well as some sample
trajectories for $\bx$.
\vspace{2in}

\item If $A$ is the matrix of constant coefficients of the given system of
differential equations, then, since (as shown in part (c)), the only
eigenvalue of $A$ is negative, all solutions to the system of differential
equations approach $\bzero$ as $t \rightarrow \infty$.

\item If $\lambda$ is an eigenvalue of $A$, then
\[0
 = (-3 - \lambda)(2 - \lambda) + 25/4
 = \left(\lambda + \frac12\right)^2,
\]
so that $\lambda = -\frac12$. The only eigenvector of $A$ is
\[\bxi
 = \begin{bmatrix}
        1     \\
        1
   \end{bmatrix}.
\]
Thus,
\[\bx^{(1)} = \bxi e^{-\frac12t}\]
is one solution to the system of differential equations. For $\boldeta$
satisfying $(A + \frac12I)\boldeta = \bxi$, another linearly independent
solution to the system of differential equations is
\[\bx^{(2)} = \bxi te^{-\frac12t} + \boldeta e^{-\frac12t}.\]

One such value for $\boldeta$ is $\begin{bmatrix} 2/5 \\ 0 \end{bmatrix}$.

Thus, solutions to the given system of differential equations are of the form
\[\bx
 = \mbox{
    \fbox{
        $\displaystyle
            c_1\begin{bmatrix} 1 \\ 1 \end{bmatrix}e^{-\frac12t}
            + c_2\left(
                \begin{bmatrix} 1 \\ 1 \end{bmatrix}te^{-\frac12t}
                + \begin{bmatrix} 2/5 \\ 0 \end{bmatrix} e^{-\frac12t}
            \right)
        $.
    }
   }
\]

\end{enumerate}
\end{question}

\newpage
\begin{question}{Section 9.1, Problem 8}
\begin{enumerate}[(a)]
\item If $A$ is the matrix of constant coefficients of the given system of
differential equations, then, if $\lambda$ is an eigenvalue of $A$, then
\[0
 = (-1 - \lambda)(-0.25 - \lambda)
\]
so that \fbox{$\lambda \in \{-1,-0.25\}$.} Thus, the eigenvectors of $A$
associated with eigenvalues $\lambda = -3$ and $\lambda = 2$, respectively,
are
\[\bxi^{(1)}
 = \mbox{\fbox{$\begin{bmatrix}
        1     \\
        0
   \end{bmatrix}$,}} \quad
 \bxi^{(2)}
 = \mbox{\fbox{$\begin{bmatrix}
        -0.75 \\
        1
   \end{bmatrix}$.}}
\]

\item Since both eigenvalues of $A$ are negative, the origin is an
\fbox{asymptotically stable} equilibrium point.

\item Figure 4 below shows several sample trajectories of $\bx$, as well as
some typical graphs of $x_1$ versus $t$.
\vspace{2in}
\end{enumerate}
\end{question}

\newpage
\begin{question}{Section 9.1, Problem 10}
\begin{enumerate}[(a)]
\item As found in the solution to part (a) of Problem 6 in chapter 7.6, if $A$
is the matrix of constant coefficients of the given system of differential
equations, then, if $\lambda$ is an eigenvalue of $A$, then
\fbox{$\lambda = \pm 3i$,} and the eigenvectors of $A$ associated with
eigenvalues $\lambda = 3i$ and $\lambda = -3i$, respectively, are
\[\bxi^{(1)}
 = \mbox{\fbox{$\begin{bmatrix}
        2     \\
        -1 + 3i
   \end{bmatrix}$,}} \quad
 \bxi^{(2)}
 = \mbox{\fbox{$\begin{bmatrix}
        2     \\
        -1 - 3i
   \end{bmatrix}$.}}
\]

\item As explained in the solution to part (a) of Problem 6 in chapter 7.6,
all solutions to the given system of differential equations are elliptical,
so that the origin is a \fbox{stable (but not asymptotically}
\fbox{stable)} equilibrium point.

\item Figure 5 below shows several sample trajectories of $\bx$, as well as
some typical graphs of $x_1$ versus $t$.
\vspace{2in}
\end{enumerate}
\end{question}
\end{document}
