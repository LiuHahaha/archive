\documentclass[11pt]{article}
\usepackage{enumerate}
\usepackage{fullpage}
\usepackage{fancyhdr}
\usepackage{amsmath, amsfonts, amsthm, amssymb}
\usepackage{color}
\setlength{\parindent}{0pt}
\setlength{\parskip}{5pt plus 1pt}
\pagestyle{empty}

\def\indented#1{\list{}{}\item[]}
\let\indented=\endlist

\newcounter{questionCounter}
\newcounter{partCounter}[questionCounter]
\newenvironment{question}[2][\arabic{questionCounter}]{%
    \setcounter{partCounter}{0}%
    \vspace{.25in} \hrule \vspace{0.5em}%
        \noindent{\bf #2}%
    \vspace{0.8em} \hrule \vspace{.10in}%
    \addtocounter{questionCounter}{1}%
}{}
\renewenvironment{part}[1][\alph{partCounter}]{%
    \addtocounter{partCounter}{1}%
    \vspace{.10in}%
    \begin{indented}%
       {\bf (#1)} %
}{\end{indented}}

%%%%%%%%%%%%%%%%%%%%%%%HEADER%%%%%%%%%%%%%%%%%%%%%%%%%%%%%%
\newcommand{\myname}{Shashank Singh}
\newcommand{\myandrew}{sss1@andrew.cmu.edu}
\newcommand{\myclass}{21-260 Differential Equations}
\newcommand{\myhwnum}{3}
\newcommand{\duedate}{Friday, August 10, 2012}
%%%%%%%%%%%%%%%%%%%%%%%%%%%%%%%%%%%%%%%%%%%%%%%%%%%%%%%%%%%

%%%%%%%%%%%%%%%%%%%%CONTENT MACROS%%%%%%%%%%%%%%%%%%%%%%%%%
\renewcommand{\qed}{\quad $\blacksquare$}
\newcommand{\mqed}{\quad \blacksquare}
\newcommand{\diffeq}{differential equation }
\newcommand{\inv}{^{-1}}
\newcommand{\ba}{\mathbf{a}}
\newcommand{\bb}{\mathbf{b}}
\newcommand{\bx}{\mathbf{x}}
\newcommand{\bzero}{\mathbf{0}}
\newcommand{\bxi}{\boldsymbol{\xi}}
\newcommand{\boldeta}{\boldsymbol{\eta}}
%%%%%%%%%%%%%%%%%%%%%%%%%%%%%%%%%%%%%%%%%%%%%%%%%%%%%%%%%%%

\begin{document}
\thispagestyle{plain}

{\Large Extra Credit Project \myhwnum} \\
\myclass \\
Name: \myname \\
Email: \myandrew \\
Due: \duedate \\
\begin{question}{Section 8.1, Problem 23}
\begin{enumerate}[(a)]
\item By definition of $E_n$ and $E_{n + 1}$, equation (20) gives
\[E_{n + 1}
 = E_n + h(f(t_n,\phi(t_n)) - f(t_n,y_n)) + \frac12\phi(\bar{t}_n)h^2.\]
Then, since $h > 0$, the triangle inequality then gives
\[|E_{n + 1}|
 \leq |E_n| + h|f(t_n,\phi(t_n)) - f(t_n,y_n)| + \frac12h^2|\phi(\bar{t}_n)|.\]

Since $f$ is Lipschitz in its second argument with Lipschitz constant $L$,
$|f(t_n,\phi(t_n)) - f(t_n,y_n)| \leq L|\phi(t_n) - y_n| = L|E_n|$, so that
\begin{equation}
|E_n| + h|f(t_n,\phi(t_n)) - f(t_n,y_n)| \leq  |E_n| + hL|E_n| = \alpha|E_n|.
\end{equation}

Since, $\frac12h^2 \geq 0$ and, by definition of $\beta$,
$|\phi^{\prime\prime}(\bar{t}_n)| \leq \beta$,
\begin{equation}
\frac12h^2|\phi(\bar{t}_n)| \leq \beta h^2.
\end{equation}
Adding equations (1) and (2) gives
\[|E_n| + h|f(t_n,\phi(t_n)) - f(t_n,y_n)| + \frac12h^2|\phi(\bar{t}_n)|
 \leq \alpha|E_n| + \beta h^2,
\]
so that, as desired,
\[|E_{n + 1}|
 \leq |E_n| + h|f(t_n,\phi(t_n)) - f(t_n,y_n)| + \frac12h^2|\phi(\bar{t}_n)|
 \leq \alpha|E_n| + \beta h^2. \mqed
\]

\item By definition of $\alpha$,
\[|E_n|
   \leq   \beta h^2\frac{\alpha^n - 1}  {\alpha - 1}
   =      \beta h^2\frac{(1 + hL)^n - 1}{(1 + hL) - 1}
   =      \beta h  \frac{(1 + hL)^n - 1}{L}. \mqed
\]

\item Since, $\forall x \in \mathbb{R}$, $1 + x \leq e^x$ (this can be shown
for $x \geq 0$ by noting that the first derivative of $e^x - 1$ is positive
and $e^0 - 1 = 0$, and can be shown for $x < 0$ by noting that $e^x$ is
everywhere positive), $1 + hL \leq e^{hl}$. Thus, $(1 + hL)^n \leq e^{nhl}$.
\qed
\end{enumerate}
\end{question}
\end{document}
