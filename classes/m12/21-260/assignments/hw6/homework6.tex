\documentclass[11pt]{article}
\usepackage{enumerate}
\usepackage{fullpage}
\usepackage{fancyhdr}
\usepackage{amsmath, amsfonts, amsthm, amssymb}
\usepackage{color}
\setlength{\parindent}{0pt}
\setlength{\parskip}{5pt plus 1pt}
\pagestyle{empty}

\def\indented#1{\list{}{}\item[]}
\let\indented=\endlist

\newcounter{questionCounter}
\newcounter{partCounter}[questionCounter]
\newenvironment{question}[2][\arabic{questionCounter}]{%
    \setcounter{partCounter}{0}%
    \vspace{.25in} \hrule \vspace{0.5em}%
        \noindent{\bf #2}%
    \vspace{0.8em} \hrule \vspace{.10in}%
    \addtocounter{questionCounter}{1}%
}{}
\renewenvironment{part}[1][\alph{partCounter}]{%
    \addtocounter{partCounter}{1}%
    \vspace{.10in}%
    \begin{indented}%
       {\bf (#1)} %
}{\end{indented}}

%%%%%%%%%%%%%%%%%%%%%%%HEADER%%%%%%%%%%%%%%%%%%%%%%%%%%%%%%
\newcommand{\myname}{Shashank Singh}
\newcommand{\myandrew}{sss1@andrew.cmu.edu}
\newcommand{\myclass}{21-260 Differential Equations}
\newcommand{\myhwnum}{6}
\newcommand{\duedate}{Wednesday, July 25, 2012}
%%%%%%%%%%%%%%%%%%%%%%%%%%%%%%%%%%%%%%%%%%%%%%%%%%%%%%%%%%%

%%%%%%%%%%%%%%%%%%%%CONTENT MACROS%%%%%%%%%%%%%%%%%%%%%%%%%
\renewcommand{\qed}{\quad $\blacksquare$}
\newcommand{\mqed}{\quad \blacksquare}
\newcommand{\diffeq}{differential equation }
\newcommand{\inv}{^{-1}}
\newcommand{\ba}{\mathbf{a}}
\newcommand{\bb}{\mathbf{b}}
\newcommand{\bx}{\mathbf{x}}
\newcommand{\bzero}{\mathbf{0}}
\newcommand{\bxi}{\boldsymbol{\xi}}
\newcommand{\boldeta}{\boldsymbol{\eta}}

%%%%%%%%%%%%%%%%%%%%%%%%%%%%%%%%%%%%%%%%%%%%%%%%%%%%%%%%%%%

\begin{document}
\thispagestyle{plain}

{\Large Homework \myhwnum} \\
\myclass \\
Name: \myname \\
Email: \myandrew \\
Due: \duedate \\

\begin{question}{Section 3.1, Problem 14}
The characteristic equation of the given \diffeq is $2r^2 + r - 4 = 0$, whose
solutions are $r_1 = \frac14\left(\sqrt{33} - 1\right)$ and
$r_2 = \frac14\left(-\sqrt{33} - 1\right)$. Thus, general solutions to the
given \diffeq are of the form
\[y = \mbox{\fbox{$c_1e^{r_1t} + c_2e^{r_2t}$,}}\]
where, for $t_0 = 0$,
\begin{eqnarray*}
c_1 & = & \frac{y^{\prime}(t_0) - y(t_0)r_2}{r_1 - r_2}e^{-r_1t_0}
      = \mbox{\fbox{$\displaystyle \frac{2}{\sqrt{33}}$}}\\
c_2 & = & \frac{y(t_0)r_1 - y^{\prime}(t_0)}{r_1 - r_2}e^{-r_2t_0}
      = \mbox{\fbox{$\displaystyle -\frac{2}{\sqrt{33}}$.}}
\end{eqnarray*}
\end{question}

\begin{question}{Section 3.2, Problem 18}
If the Wronskian of $f$ and $g$ is $t^2e^t$, then,
$f(t)g^{\prime}(t) - f^{\prime}(t)g(t) = t^2e^t$. Thus, since $f(t) = t$,
so that $f^{\prime} = 1$, \[tg^{\prime}(t) - g(t) = t^2e^t.\] Since this is a
linear first-order differential equation, we can solve it by writing it in
standard form and multiplying by an integration factor of
$\mu := e^{-\int 1/t \, dt} = t^{-1}$, giving
\[\mu(t)g(t) = \int e^t \, dt = e^t + C,\] for some $C \in \mathbb{R}$.
Then, \[g(t) = \frac{e^t + C}{\mu(t)} = \mbox{\fbox{$te^t + Ct$}}\]
(note that, without any initial value of $g$, we cannot further specify $C$).
\end{question}

\begin{question}{Section 3.3, Problem 12}
The characteristic equation of the given \diffeq is $4r^2 + 9 = 0$, whose
solutions are $r_1 = \frac32i$ and $r_2 = -\frac32i$.

Thus, the general solution of the given \diffeq is
\[y
 = \mbox{
    \fbox{
        $\displaystyle
            c_1\cos\left(\frac32t\right)
          + c_2\sin\left(\frac32t\right)
        $.
    }
   }\
\]
\end{question}

\begin{question}{Section 3.4, Problem 12}
The characteristic equation of the given \diffeq is $r^2 - 6r + 9 = 0$, whose
unique solution is $r = 3$. Thus, one solution to the \diffeq is
\[y_1(t) = e^{3t}.\]
Employing d'Alembert's method, to find another linearly independent solution
gives the solution
\[y_2(t) = te^{3t}.\]
Thus, the general solution of the given differential equation is
\[y
 = \mbox{
    \fbox{
        $c_1e^{3t} + c_2te^{3t}$
    ,}
   }
\]
for some $c_1,c_2 \in \mathbb{R}$. Since $y(0) = -1$, \fbox{$c_1 = -1$,} and,
since $y^{\prime}(0) = 2$, \fbox{$c_2 = 5$.}
\end{question}

\begin{question}{Section 3.4, Problem 28}
Suppose another solution to the given \diffeq is of the form
$y_2(x) = v(x)y_1(x)$. Substituting this into the given \diffeq and
simplifying gives
\[e^xv^{\prime\prime}(x) + \left(2e^x + \frac{x}{1 - x}e^x\right)v^{\prime}
 = 0,\]
so that cancelling the $e^x$ terms gives
\[v^{\prime\prime}(x) + \left(2 + \frac{x}{1 - x}\right)v^{\prime}
 = 0.\]
Letting $w = v^{\prime}$ gives the first order linear differential equation
\[w^{\prime} + \left(2 + \frac{x}{1 - x}\right)w = 0,\] which can be written
in the form
$\frac1w w^{\prime} = -\left(2 + \frac{x}{1 - x}\right),$
so that it is seperable. Then,
\[\ln|w|
 = \int \frac1w \, dw
 = \int -\left(2 + \frac{x}{1 - x}\right) \, dx
 = \log(1 - x) - x + C_1,
\]
for some $C_1 \in \mathbb{R}$. Thus, $w = (1 - x)e^{-x + C_1}$, so that
$v = \int w \, dx = xe^{-x + C_1} + C_2$, for some $C_2 \in \mathbb{R}$.
Therefore,
\[y_2 = \mbox{\fbox{$(xe^{-x + C_1} + C_2)e^x = xe^{C_1} + C_2e^{x}$.}}\]
\end{question}
\end{document}
