\documentclass[11pt]{article}
\usepackage{enumerate}
\usepackage{fullpage}
\usepackage{fancyhdr}
\usepackage{amsmath, amsfonts, amsthm, amssymb}
\setlength{\parindent}{0pt}
\setlength{\parskip}{5pt plus 1pt}
\pagestyle{empty}

\def\indented#1{\list{}{}\item[]}
\let\indented=\endlist

\newcounter{questionCounter}
\newcounter{partCounter}[questionCounter]
\newenvironment{question}[2][\arabic{questionCounter}]{%
    \setcounter{partCounter}{0}%
    \vspace{.25in} \hrule \vspace{0.5em}%
        \noindent{\bf #2}%
    \vspace{0.8em} \hrule \vspace{.10in}%
    \addtocounter{questionCounter}{1}%
}{}
\renewenvironment{part}[1][\alph{partCounter}]{%
    \addtocounter{partCounter}{1}%
    \vspace{.10in}%
    \begin{indented}%
       {\bf (#1)} %
}{\end{indented}}

%%%%%%%%%%%%%%%%%%%%%%%HEADER%%%%%%%%%%%%%%%%%%%%%%%%%%%%%%
\newcommand{\myname}{Shashank Singh}
\newcommand{\myandrew}{sss1@andrew.cmu.edu}
\newcommand{\myclass}{21-260 Differential Equations}
\newcommand{\myhwnum}{7b}
\newcommand{\duedate}{Tuesday, July 17, 2012}
%%%%%%%%%%%%%%%%%%%%%%%%%%%%%%%%%%%%%%%%%%%%%%%%%%%%%%%%%%%

%%%%%%%%%%%%%%%%%%%%CONTENT MACROS%%%%%%%%%%%%%%%%%%%%%%%%%
\renewcommand{\qed}{\quad $\blacksquare$}
\newcommand{\mqed}{\quad \blacksquare}
\newcommand{\diffeq}{differential equation }
\newcommand{\inv}{^{-1}}
\newcommand{\bx}{\mathbf{x}}
\newcommand{\bzero}{\mathbf{0}}
\newcommand{\bxi}{\boldsymbol{\xi}}
%%%%%%%%%%%%%%%%%%%%%%%%%%%%%%%%%%%%%%%%%%%%%%%%%%%%%%%%%%%

\begin{document}
\thispagestyle{plain}

{\Large Homework \myhwnum} \\
\myclass \\
Name: \myname \\
Email: \myandrew \\
Due: \duedate \\

\begin{question}{Section 7.3, Problem 8}
The vectors are linearly dependent:
\[-2\bx^{(3)} + 5\bx^{(2)} = \bx^{(1)}.\]
\end{question}

\begin{question}{Section 7.3, Problem 22}
If $A$ is the given matrix, then characteristic polynomial of $A$ (computed by
cofactor expansion about the first row) is
\[\left|A - \lambda I\right|
 = (1 - \lambda)((1 - \lambda)^2 + 4)
 = (1 - \lambda)(\lambda^2 - 2\lambda + 5).
\]
Aside from the obvious root $\lambda = 1$, the quadratic formula gives the two
roots
\[\lambda = \frac{2 \pm \sqrt{4 - 4\cdot5}}{2} = 1 \pm 2i,\]
so that the eigenvalues of $A$ are
\[\lambda \in \mbox{\fbox{$\displaystyle\left\{1,1 + 2i, 1 - 2i\right\}$.}}\]
Solving for the eigenvector $\bx_1$ associated with the eigenvalue
$\lambda = 1$ gives the system of equations (the first line does not constrain
$\bx_1$)
\begin{eqnarray*}
 x_1               & = & x_1 \\
2x_1 +  x_2 - 2x_3 & = & x_2 \\
3x_1 + 2x_2 +  x_3 & = & x_3,
\end{eqnarray*}
whose solutions are multiples the first eigenvector,
\[\bx_1
 = \mbox{\fbox{$\displaystyle \begin{bmatrix}
    \frac{2}{3} \\
    -1          \\
    \frac{2}{3} 
   \end{bmatrix}$.}}
\]
Solving for the eigenvectors $\bx_2$ and $\bx_3$ associated with the
eigenvalues $\lambda = 1 + 2i$ and $\lambda = 1 - 2_i$, respectively,
gives the systems of equations
\begin{eqnarray*}
 x_1              & = & (1 \pm 2i)x_1 \\
2x_1 + x_2 - 2x_3 & = & (1 \pm 2i)x_2 \\
3x_1 + 2x_2 + x_3 & = & (1 \pm 2i)x_3,
\end{eqnarray*}
which have the solutions which are multiples of
\[\bx_2
 = \mbox{\fbox{$\displaystyle \begin{bmatrix}
    0  \\
    1  \\
    -i
   \end{bmatrix}$,}} \quad
\bx_3
 = \mbox{\fbox{$\displaystyle \begin{bmatrix}
    0  \\
    1  \\
    i
   \end{bmatrix}$.}}
\]
\end{question}


\begin{question}{Section 7.3, Problem 31}
The cleanest proof is immediate from the fact that the determinant of a matrix
is the product of its eigenvalues, but this fact doesn't appear to be at our
disposal.

It suffices however to observe that, if a square matrix $A$ with $0$ as an
eigenvalue were invertible, then, for some associated non-zero eigenvector
$\bx$,
\[\bx = I\bx = (A^{-1}A)\bx = A^{-1}(A\bx) = A(0\cdot\bx) = A\bzero = \bzero,\]
which is impossible. Therefore, any matrix with $0$ as an eigenvalue is not
invertible, so that it has determinant $0$. \qed
\end{question}

\begin{question}{Section 7.4, Problem 6}
\begin{enumerate}[(a)]
\item By definition of the Wronskian,
\[W\left[\bx^{(1)},\bx^{(2)}\right]
 =  \left|
        \begin{bmatrix}
            t & t^2 \\
            1 & 2t
        \end{bmatrix}
    \right|
 = 2t^2 - t^2
 = \mbox{
    \fbox{
        $t^2$.
        }
    }
\]

\item Since the $2 \times 2$ matrix with columns $\bx^{(1)}$ and $\bx^{(2)}$
as columns has determinant $0$ if and only if its columns are linearly
dependent and its determinant is the Wronskian computed in part (a),
$\bx^{(1)}$ and $\bx^{(2)}$ are linearly independent if and only if
$t^2 \neq 0$. Thus, the solutions are linearly independent on
\fbox{$(-\infty,0)$} and on \fbox{$(0,\infty)$.}

\item Since $\bx^{(1)}$ and $\bx^{(2)}$ are linearly independent on the
intevals given in part (b), by Theorem 7.4.2, all solutions of the set of
homogeneous differential equations satisfied by $\bx^{(1)}$ and $\bx^{(2)}$
are of the form
\[\bx(t) = c_1\bx^{(1)}(t) + c_2\bx^{(2)}(t),\]
for some $c_1, c_2 \in \mathbb{R}$
(i.e., they are linear combinations of $\bx^{(1)}$ and $\bx^{(2)}$).

\item The system of equations is 
\[\bx^{\prime} = \bx_2\]
and
\end{enumerate}
\end{question}

\begin{question}{Section 7.5, Problem 12}
If $A$ is the matrix such that $\bx^{\prime} = A\bx$, then the eigenvalues of
$A$ are $\lambda \in \{-1,8\}$, where the eigenvalue $\lambda = -1$ has
algebraic multiplicity $2$. The eigenvectors associated with the eigenvalues
$(-1)$, $(-1)$, and $8$, respectively, are
\[\bxi_1
 = \begin{bmatrix}
        -0.4941 \\
        -0.4720 \\
         0.7301
   \end{bmatrix}, \quad
\bxi_2
 = \begin{bmatrix}
        -0.5580 \\
        -0.8161 \\
         0.1500
   \end{bmatrix}, \quad
\bxi_3
 = \begin{bmatrix}
        -0.6667 \\
        -0.3333 \\
         0.6667
   \end{bmatrix}.
\]

Thus, solutions to the homogeneous system of linear, first-order
differential equations are functions of the form
\[\bx(t)
 = \mbox{
    \fbox{
        $\displaystyle
            c_1\bxi_1e^{-t} + c_2\bxi_2e^{-t} + c_3\bxi_3e^{8t}
        $,
    }
   }
\]
with $c_1,c_2,c_3 \in \mathbb{R}$.
\end{question}
\end{document}
