\documentclass[11pt]{article}
\usepackage{enumerate}
\usepackage{fullpage}
\usepackage{fancyhdr}
\usepackage{amsmath, amsfonts, amsthm, amssymb}
\usepackage{color}
\setlength{\parindent}{0pt}
\setlength{\parskip}{5pt plus 1pt}
\pagestyle{empty}

\def\indented#1{\list{}{}\item[]}
\let\indented=\endlist

\newcounter{questionCounter}
\newcounter{partCounter}[questionCounter]
\newenvironment{question}[2][\arabic{questionCounter}]{%
    \setcounter{partCounter}{0}%
    \vspace{.25in} \hrule \vspace{0.5em}%
        \noindent{\bf #2}%
    \vspace{0.8em} \hrule \vspace{.10in}%
    \addtocounter{questionCounter}{1}%
}{}
\renewenvironment{part}[1][\alph{partCounter}]{%
    \addtocounter{partCounter}{1}%
    \vspace{.10in}%
    \begin{indented}%
       {\bf (#1)} %
}{\end{indented}}

%%%%%%%%%%%%%%%%%%%%%%%HEADER%%%%%%%%%%%%%%%%%%%%%%%%%%%%%%
\newcommand{\myname}{Shashank Singh}
\newcommand{\myandrew}{sss1@andrew.cmu.edu}
\newcommand{\myclass}{21-260 Differential Equations}
\newcommand{\myhwnum}{10}
\newcommand{\duedate}{Wednesday, August 8, 2012}
%%%%%%%%%%%%%%%%%%%%%%%%%%%%%%%%%%%%%%%%%%%%%%%%%%%%%%%%%%%

%%%%%%%%%%%%%%%%%%%%CONTENT MACROS%%%%%%%%%%%%%%%%%%%%%%%%%
\renewcommand{\qed}{\quad $\blacksquare$}
\newcommand{\mqed}{\quad \blacksquare}
\newcommand{\diffeq}{differential equation }
\newcommand{\inv}{^{-1}}
\newcommand{\ba}{\mathbf{a}}
\newcommand{\bb}{\mathbf{b}}
\newcommand{\bx}{\mathbf{x}}
\newcommand{\bzero}{\mathbf{0}}
\newcommand{\bxi}{\boldsymbol{\xi}}
\newcommand{\boldeta}{\boldsymbol{\eta}}
%%%%%%%%%%%%%%%%%%%%%%%%%%%%%%%%%%%%%%%%%%%%%%%%%%%%%%%%%%%

\begin{document}
\thispagestyle{plain}

{\Large Homework \myhwnum} \\
\myclass \\
Name: \myname \\
Email: \myandrew \\
Due: \duedate \\
\begin{question}{Section 10.1, Problem 6}
An obvious particular solution to the given nonhomogeneous \diffeq is
$y(t) = \frac12x$.

The characteristic equation of the homogeneous \diffeq associated with the
given nonhomogeneous equation is $r^2 + 2 = 0$, whose roots are
$r \in \{-\sqrt{2}i,\sqrt{2}i\}$. Thus, the solutions to the associated
homogeneous equation are of the form
\[y(t) = c_1\cos\left(\sqrt{2}x\right) + c_2\sin\left(\sqrt{2}x\right),\]
for some $c_1,c_2 \in \mathbb{R}$, and so, since the particular and general
solutions are independent, solutions to the given nonhomogeneous equation are
of the form
\[y(t)
 = \mbox{\fbox{$\displaystyle c_1\cos\left(\sqrt{2}x\right)
 + c_2\sin\left(\sqrt{2}x\right)
 + \frac12x$.}}
.\]
Since $y(0) = 0$, \fbox{$c_1 = 0$.} Then, since $y(\pi) = 0$,
$c_2
 =       \frac{-\pi}{\sin\left(\sqrt{2}\pi\right)}
 \approx \mbox{\fbox{$3.26$.}}$
\end{question}

\begin{question}{Section 10.1, Problem 19}
In the first case, suppose $\lambda > 0$. Then, the characteristic equation of
the given \diffeq is $r^2 - \mu^2 = 0$, whose roots are $r = \pm\mu$. Then,
solutions to the \diffeq are of the form
\[y(x) = c_1\cosh(\mu x) + c_2\sinh(\mu x),\quad c_1,c_2 \in \mathbb{R}.\]
Since $y(0) = 0$, $c_1 = 0$. Since $0 = y^{\prime}(L) = c_2\mu\cosh(\mu L)$,
and since $\mu \neq 0$ and $\cosh$ is everywhere strictly positive, $c_2$.
Thus, the given \diffeq has no positive eigenvalues.

In the second case, suppose $\lambda = 0$, so that $y^{\prime\prime} = 0$.
Then, solutions to the \diffeq are of the form
\[y(x) = c_1x + c_2,\quad c_1,c_2 \in \mathbb{R}.\]
Since $y(0) = 0$, $c_2 = 0$, and, since $0 = y^{\prime}(L) = c_1$, $c_1 = 0$.
Thus, the given \diffeq does not have zero as an eigenvalue.

In the last case, suppose $\lambda < 0$. Then, the characteristic equation of
the given \diffeq is $r^2 + \mu^2 = 0$, whose roots are $r = \pm\mu i$. Then,
solutions to the \diffeq are of the form
\[y(x) = c_1\cos(\mu x) + c_2\sin(\mu x),\quad c_1,c_2 \in \mathbb{R}.\]
Since $y(0) = 0, c_1 = 0$. Since $y^{\prime}(L) = 0$, so that
$c_2\mu\cos(\mu L) = 0$ and we are interested only in the case $c_2 \neq 0$
(and $\mu \neq 0$), $\cos(\mu L) = 0$, and thus $\mu L - \frac{\pi}{2}$ is an
integer multiple of $\pi$. Then, \[\mu = \frac{\pi/2 + n\pi}{L}\] and the
eigenvalues of the given \diffeq are, $\forall n \in \mathbb{N}$,
\[\lambda_n
 = \mbox{\fbox{$\displaystyle\left(\frac{\pi/2 + n\pi}{L}\right)^2$,}}\]
with associated
eigenfunctions
\[y_n(x)
 = \mbox{\fbox{$\displaystyle\sin\left(\frac{\pi/2 + n\pi}{L}x\right)$.}}\]
\end{question}

\begin{question}{Section 10.2, Problem 16}
\begin{enumerate}[(a)]
\item Three periods of the graph of $f$ are plotted below:
\vspace{1.5in}

\item Suppose $a_n$ and $b_{n + 1}$ ($\forall n \in \mathbb{N}$) are such that
\[f(x)
 = \frac{a_0}{2}
 + \sum_{n = 1}^{\infty} \left(a_n \cos(n\pi x)
                             + b_n \sin(n\pi x)\right).\]
$a_0 = \int_{-1}^1 f(x) \, dx = \mbox{\fbox{$1$.}}$
$\forall n \in \mathbb{N}\backslash\{0\}$, the Euler-Fourier formulas and
integration by parts give that
\begin{eqnarray*}
a_n
 & = &  \int_{-1}^1 f(x)   \cos(n\pi x) \, dx
   =   2\int_0^1    f(x)   \cos(n\pi x) \, dx \\
 & = & 2\int_0^1    (1 - x)\cos(n\pi x) \, dx \\
 & = & \mbox{\fbox{$\displaystyle \frac{2 - 2\cos(\pi n)}{\pi^2n^2}$.}}
\end{eqnarray*}
Since $f$ is even, $\forall n \in \mathbb{N}\backslash\{0\}$,
$b_n = \mbox{\fbox{$0$.}}$
\end{enumerate}
\end{question}

\begin{question}{Section 10.3, Problem 4}
\begin{enumerate}[(a)]
\item Suppose $a_n$ and $b_{n + 1}$ ($\forall n \in \mathbb{N}$) are such that
\[f(x)
 = \frac{a_0}{2}
 + \sum_{n = 1}^{\infty} \left(a_n \cos(n\pi x)
                             + b_n \sin(n\pi x)\right).\]
$a_0 = \int_{-1}^1 f(x) \, dx = \mbox{\fbox{$4/3$.}}$

$\forall n \in \mathbb{N}\backslash\{0\}$, the Euler-Fourier formulas and
integration by parts give that
\begin{eqnarray*}
a_n
 & = &  \int_{-1}^1 f(x)     \cos(n\pi x) \, dx
   =   2\int_0^1    f(x)     \cos(n\pi x) \, dx \\
 & = & 2\int_0^1    (1 - x^2)\cos(n\pi x) \, dx \\
 & = & \mbox{\fbox{$\displaystyle \frac{4\sin(\pi n) - 4\pi n\cos(\pi n)}{\pi^3n^3}$.}}
\end{eqnarray*}
Since $f$ is even, $\forall n \in \mathbb{N}\backslash\{0\}$,
$b_n = \mbox{\fbox{$0$.}}$

\item Three periods of the graph of $f$ are plotted below:
\vspace{1.5in}
\end{enumerate}
\end{question}

\newpage
\begin{question}{Section 10.4, Problem 16}
\begin{enumerate}[(a)]
\item The Euler-Fourier formula for a sine series and integration by parts
give that, for
\[f(x) = \sum_{n = 1}^{\infty} b_n\sin\left(\frac{n\pi x}{2}\right),\]
$\forall n \in \mathbb{N}\backslash\{0\}$,
\begin{eqnarray*}
b_n
 & = & \frac22\int_0^2 f(x)\sin\left( \frac{n\pi x}{2} \right) \\
 & = &        \int_0^1  x  \sin\left( \frac{n\pi x}{2} \right)
   +          \int_1^2     \sin\left( \frac{n\pi x}{2} \right) \\
 & = & \frac{4}{\pi^2n^2}  \sin\left( \frac{\pi n} {2} \right)
   -   \frac{2}{\pi n}     \cos\left( \frac{\pi n} {2} \right)
   +   \frac{2}{\pi n}     \cos\left( \frac{\pi n} {2} \right)
   -   \frac{2}{\pi n}     \cos\left(       \pi n      \right) \\
 & = & \mbox{\fbox{$\displaystyle
       \frac{4}{\pi^2n^2}  \sin\left( \frac{\pi n} {2} \right)
   -   \frac{2}{\pi n}     \cos\left(       \pi n      \right)
       $.}}
\end{eqnarray*}

\item Three periods of the graph of $f$ are plotted below:
\vspace{1.5in}
\end{enumerate}
\end{question}
\end{document}
