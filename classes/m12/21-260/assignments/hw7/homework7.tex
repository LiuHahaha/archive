\documentclass[11pt]{article}
\usepackage{enumerate}
\usepackage{fullpage}
\usepackage{fancyhdr}
\usepackage{amsmath, amsfonts, amsthm, amssymb}
\usepackage{color}
\setlength{\parindent}{0pt}
\setlength{\parskip}{5pt plus 1pt}
\pagestyle{empty}

\def\indented#1{\list{}{}\item[]}
\let\indented=\endlist

\newcounter{questionCounter}
\newcounter{partCounter}[questionCounter]
\newenvironment{question}[2][\arabic{questionCounter}]{%
    \setcounter{partCounter}{0}%
    \vspace{.25in} \hrule \vspace{0.5em}%
        \noindent{\bf #2}%
    \vspace{0.8em} \hrule \vspace{.10in}%
    \addtocounter{questionCounter}{1}%
}{}
\renewenvironment{part}[1][\alph{partCounter}]{%
    \addtocounter{partCounter}{1}%
    \vspace{.10in}%
    \begin{indented}%
       {\bf (#1)} %
}{\end{indented}}

%%%%%%%%%%%%%%%%%%%%%%%HEADER%%%%%%%%%%%%%%%%%%%%%%%%%%%%%%
\newcommand{\myname}{Shashank Singh}
\newcommand{\myandrew}{sss1@andrew.cmu.edu}
\newcommand{\myclass}{21-260 Differential Equations}
\newcommand{\myhwnum}{7}
\newcommand{\duedate}{Friday, July 27, 2012}
%%%%%%%%%%%%%%%%%%%%%%%%%%%%%%%%%%%%%%%%%%%%%%%%%%%%%%%%%%%

%%%%%%%%%%%%%%%%%%%%CONTENT MACROS%%%%%%%%%%%%%%%%%%%%%%%%%
\renewcommand{\qed}{\quad $\blacksquare$}
\newcommand{\mqed}{\quad \blacksquare}
\newcommand{\diffeq}{differential equation }
\newcommand{\inv}{^{-1}}
\newcommand{\ba}{\mathbf{a}}
\newcommand{\bb}{\mathbf{b}}
\newcommand{\bx}{\mathbf{x}}
\newcommand{\bzero}{\mathbf{0}}
\newcommand{\bxi}{\boldsymbol{\xi}}
\newcommand{\boldeta}{\boldsymbol{\eta}}
%%%%%%%%%%%%%%%%%%%%%%%%%%%%%%%%%%%%%%%%%%%%%%%%%%%%%%%%%%%

\begin{document}
\thispagestyle{plain}

{\Large Homework \myhwnum} \\
\myclass \\
Name: \myname \\
Email: \myandrew \\
Due: \duedate \\
\begin{question}{Section 3.5, Problem 12}
Taking inspiration from the given hint, we suppose that a solution to the
given \diffeq is of the form $Y(t) = Ae^{2t} + Be^{-2t}$, for
some $A,B \in \mathbb{R}$. Then,
\[Y^{\prime} = 2Ae^{2t} - 2Be^{-2t} \quad \mbox{ and } \quad
Y^{\prime\prime} = 4Ae^{2t} + 4Be^{-2t},\]
so that plugging $Y$ into the \diffeq gives
$(1 + 2 + 4)A = \frac12$ and $(1 - 2 + 4)B = \frac12$.
Thus,
\[Y(t) = \mbox{\fbox{$\displaystyle \frac{1}{14}e^{2t} + \frac16e^{-t}$.}}\]
The characteristic equation of the homogeneous equation corresponding to the
given \diffeq is $r^2 - r - 2 = 0$, whose roots are $r \in \{1,2\}$.
Thus, the solutions to the given \diffeq are of the form 
\[y
 = \mbox{\fbox{$c_1e^t + c_2e^{2t} + Y(t)$,}} \quad c_1,c_2 \in \mathbb{R}.\]
\end{question}

\begin{question}{Section 3.5, Problem 16}
Suppose that some particular solution of the given \diffeq is of the form
\[Y(t) = (A_1t + A_0)e^{2t}.\]
Then, $Y^{\prime} = A_1e^{2t} + 2Y(t)$ and
$Y^{\prime\prime} = 2A_1e^{2t} + 2Y^{\prime}(t)$.
Thus, plugging this solution into the given \diffeq and simplifying gives
$(2A_1 - 3A_0)e^{2t} - 3A_1te^{2t} = 3te^{2t}$,
so that $-3A_1 = 3$ and $2A_1 - 3A_0 = 0$, and thus $A_1 = -1$ and
$A_0 = -\frac23$.

The characteristic equation of the homogeneous equation associated with the
given \diffeq is $r^2 - 2r - 3 = 0$, whose roots are $r \in \{-1,3\}$.
Therefore, the general solution of the given equation is of the form
\[y
 = \mbox{\fbox{$c_1e^{-t} + c_2e^{3t} + Y(t)$,}} \quad c_1,c_2 \in \mathbb{R},\]
\[Y(t)
 = \mbox{\fbox{$\displaystyle -\left(t + \frac23\right)e^{2t}$.}}\]
Solving for $c_1$ and $c_2$ using the givens $y(0) = 1$ and
$y^{\prime}(0) = 0$ gives \fbox{$c_1 = \frac23$} and \fbox{$c_2 = 1$.}
\end{question}

\begin{question}{Section 3.6, Problem 16}
Since, for $y = y_1$,
\[(1 - t)y^{\prime\prime} + ty^{\prime} - y
 = (1 - t)e^t + te^t - e^t
 = (1 - t)e^t - (1 - t)e^t
 = 0,
\]
and,
for $y = y_2$,
\[(1 - t)y^{\prime\prime} + ty^{\prime} - y
 = (1 - t)\cdot0 + t\cdot1 - t
 = t - t
 = 0,
\]
$y_1$ and $y_2$ indeed satisfy the homogeneous \diffeq corresponding to the
given differential equation. Furthermore, since the Wronskian of $y_1$ and
$y_2$ is $(1 - t)e^t$, which is $0$ only when $t = 1$, $\{y_1,y_2\}$ is a
fundamental set of solutions of this corresponding homogeneous \diffeq on the
interval $t \in (0,1)$.

Thus, by Theorem 3.6.1, a particular solution of the given \diffeq is, for
$t_0 = 0$,
\begin{eqnarray*}
Y(t)
 & := & -e^t \int_{t_0}^t \frac{2s(s - 1)^2e^{-s}}   {(1 - s)e^s} \, ds
    +   t    \int_{t_0}^t \frac{2e^s(s - 1)^2e^{-s}} {(1 - s)e^s} \, ds \\
 &  = & -e^t \int_{t_0}^t 2s(1 - s)e^{-2s}                        \, ds
    +   t    \int_{t_0}^t 2(1 - s)e^{-s}                          \, ds \\
 &  = & -e^t \left( s^2e^{-2s} \right) \big|_{s = t_0}^{s = t}
    +   t    \left( 2se^{-s}   \right) \big|_{s = t_0}^{s = t}         \\
 &  = & -t^2e^{-t} + 2t^2e^{-t} = \mbox{\fbox{$t^2e^{-t}$.}}
\end{eqnarray*}
Thus, the general solution to the given \diffeq is of the form
\[y = \mbox{\fbox{$c_1e^t + c_2t + Y(t)$,}} \quad c_1,c_2 \in \mathbb{R}.\]
\end{question}

\begin{question}{Section 3.7, Problem 18}
As derived in the context of the mass-on-a-spring system, a system obeying the
\diffeq
\[ay^{\prime\prime} + by^{\prime} + cy = 0\]
(where $a,b,c \in \mathbb{R}$) is critically damped when $b = 2\sqrt{ac}$.
Thus, since the given series circuit obeys the differential equation
\[LQ^{\prime\prime} + RQ^{\prime} + \frac{1}{C}Q = 0,\]
the circuit is critically damped when
\[R
 = 2\sqrt{\frac{L}{C}}
 = 2\sqrt{\frac{0.2 \, H}{0.8 \times 10^{-6} \, F}}
 = \mbox{\fbox{$1000 \, \Omega$.}}
\]
\end{question}

\newpage
\begin{question}{Section 3.8, Problem 12}
This system obeys the \diffeq
\[mu^{\prime\prime} + \gamma^{\prime} + ku = 3\cos(3t)N - 2\sin(3t)N,\]
where $m = 2 \, kg$, $\gamma = 1 \, kg/s$, and $k = 3N/m$.
Suppose that a particular solution of this system is of the form
\[Y(t) = A\cos(3t) + B\sin(3t), \quad A,B \in \mathbb{R}.\]
Then, plugging $Y$ into the given \diffeq gives $-15A + 3B = 3$ and
$-3A - 15B = 2$, so that $A = -17/78$ and $B = -7/78$, and thus
\[Y(t)
 = \mbox{
    \fbox{
        $\displaystyle
              -\frac{17}{78}\cos(3t)
            + -\frac{7}{78}\sin(3t)
        $
    .}
   }
\]

The characteristic equation of the homogeneous \diffeq associated with the
given \diffeq is $2r^2 + r + 3 = 0$, whose roots are
$r = -\frac14(1\pm\sqrt{23}i)$,
Thus, the general solution the given \diffeq are of the form
\[y
 = 
   c_1e^{-\frac14t}\cos(\sqrt{23}t)
 + c_2e^{-\frac14t}\sin(\sqrt{23}t)
 + Y(t),
\]
for some $c_1,c_2 \in \mathbb{R}$.

Since the real parts of both roots of the above characteristic equation are
negative, both solutions to the homogeneous equation are transient, and the
steady state response of the system is identical to the particular solution
$Y$ of the given nonhomogeneous equation.
\end{question}
\end{document}
