\documentclass[11pt]{article}
\usepackage{enumerate}
\usepackage{fullpage}
\usepackage{fancyhdr}
\usepackage{amsmath, amsfonts, amsthm, amssymb}
\setlength{\parindent}{0pt}
\setlength{\parskip}{5pt plus 1pt}
\pagestyle{empty}

\def\indented#1{\list{}{}\item[]}
\let\indented=\endlist

\newcounter{questionCounter}
\newcounter{partCounter}[questionCounter]
\newenvironment{question}[2][\arabic{questionCounter}]{%
    \setcounter{partCounter}{0}%
    \vspace{.25in} \hrule \vspace{0.5em}%
        \noindent{\bf #2}%
    \vspace{0.8em} \hrule \vspace{.10in}%
    \addtocounter{questionCounter}{1}%
}{}
\renewenvironment{part}[1][\alph{partCounter}]{%
    \addtocounter{partCounter}{1}%
    \vspace{.10in}%
    \begin{indented}%
       {\bf (#1)} %
}{\end{indented}}

%%%%%%%%%%%%%%%%%%%%%%%%%%%%%%%%%%%%%%%%%%%%%%%%%%%%%%%%%%%
\newcommand{\myname}{Shashank Singh}
\newcommand{\myandrew}{sss1@andrew.cmu.edu}
\newcommand{\myclass}{21-238 Mathematical Studies Algebra II}
\newcommand{\myhwnum}{1}
\newcommand{\duedate}{Monday, February 6, 2012}
%%%%%%%%%%%%%%%%%%%%%%%%%%%%%%%%%%%%%%%%%%%%%%%%%%%%%%%%%%%

\begin{document}
\thispagestyle{plain}

{\Large Homework \myhwnum} \\
\myclass \\
Name: \myname \\
Email: \myandrew \\
Due: \duedate

\begin{question}{Exercise 1}
Since the determinant of $AB$ is $0$, rank$(AB) < 3$. Furthermore, no row of
$AB$ is a multiple of another, so that rank$(AB) > 2$, and thus
dim(Im($AB)) = 2$. Note that $(AB)^2 = 9AB$. Therefore, the
dim(Im$(A(BA)A)) \geq 2$. Since the dimension of the image of a composition of
linear functions is at most the minimum of the dimensions of the images of
those functions, dim(Im($BA)) \geq 2$. Since $BA$ is a $2 \times 2$ matrix of
rank $2$, $BA$ is invertible. Note that $(AB)^2 = 9AB$. Therefore,
$B(AB)^2A = 9BABA$, so that, right-multiplying by $(BABA)^{-1}$ gives
\[BA = 9I_2 = \left[\begin{array}{cc} 9 & 0 \\ 0 & 9 \end{array}\right].
\qquad \blacksquare\]
\end{question}

\begin{question}{Exercise 2}
\begin{enumerate}[i.]
\item Since the determinant is multiplicative and $AB = -BA = -IBA$,
$|AB| = |-I||B||A|$. Since $(-I)$ is a diagonal matrix, its determinant is the
product of the elements in its diagonal, so that $|-I| = (-1)^n$. Therefore,
since $|BA| = |A||B| = |AB|$, $n$ must be even. \qquad $\blacksquare$

\item Since the polynomial $x^2 - 1$ over $E$ has all of its roots in $E$ and
$A^2 - I_n$ is the minimal polynomial of $A$ over $E$, $A$ is diagonalizable.
Thus, for some invertible $P$ and some diagonal matrix $S$, $A = PSP^{-1}$, so
that $A = ABB = -BAB^{-1} = (BP)S(BP)^{-1}$.

Since the elements along the diagonals of both $S$ and $(-S)$ are the
eigenvalues of $A$, $A$ must have the same number of positive and negative
eigenvalues. Since the eigenvalues of $A^2$ are the squares of the eigenvalues
of $A$, and $A^2 = I_n$, the eigenvalues of $A$ are $1$ and $(-1)$, each with
multiplicity $m$.
\end{enumerate}
\end{question}
\newpage
\begin{question}{Exercise 5}
Suppose, for sake of contradiction that there existed a square matrix $A$ such
that \[\sin A = \left[\begin{array}{cc} 1 & 1996 \\ 0 & 1 \end{array}\right].\]

As with $\sin$, we can define $\cos A$ by the usual power series:
\[\cos A = \sum_{n = 0}^{\infty} \frac{(-1)^n}{(2n + 1)!} A^{2n + 1}.\]
Then,
\[\cos^2 A = I_2 - \sin^2 A = \left[\begin{array}{cc} 0 & -3992 \\ 0 & 0 \end{array}\right].\]
Let $a,b,c,d$ be the elements of $\cos A$, so that
\[\cos^2 A = \left[\begin{array}{cc} a & b \\ c & d \end{array}\right]^2 =
  \left[\begin{array}{cc} a^2 + bc & ab + bd \\ ca + cd & bc + d^2 \end{array}\right].\]

This gives the system of simultaneous equations:
\begin{eqnarray*}
a^2     & = & - bc   \\
ab + bd & = & - 3992 \\
ca      & = & - cd   \\
cb      & = & - d^2.
\end{eqnarray*}
If $c = 0$, then $a = d = 0$, so that $ab + bd = 0 \neq -3992$, which is a
contradiction. Similarly, if $c \neq 0$, then $a = - d$, so that
$ab + bd = 0 \neq -3992$. Therefore, there does not exist a matrix $A$ such
that
\[\sin A = \left[\begin{array}{cc} 1 & 1996 \\ 0 & 1 \end{array}\right].\qquad \blacksquare\]
\end{question}
\end{document}
