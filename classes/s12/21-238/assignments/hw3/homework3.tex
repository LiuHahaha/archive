\documentclass[11pt]{article}
\usepackage{enumerate}
\usepackage{fullpage}
\usepackage{fancyhdr}
\usepackage{amsmath, amsfonts, amsthm, amssymb}
\setlength{\parindent}{0pt}
\setlength{\parskip}{5pt plus 1pt}
\pagestyle{empty}

\def\indented#1{\list{}{}\item[]}
\let\indented=\endlist

\newcounter{questionCounter}
\newcounter{partCounter}[questionCounter]
\newenvironment{question}[2][\arabic{questionCounter}]{%
    \setcounter{partCounter}{0}%
    \vspace{.25in} \hrule \vspace{0.5em}%
        \noindent{\bf #2}%
    \vspace{0.8em} \hrule \vspace{.10in}%
    \addtocounter{questionCounter}{1}%
}{}
\renewenvironment{part}[1][\alph{partCounter}]{%
    \addtocounter{partCounter}{1}%
    \vspace{.10in}%
    \begin{indented}%
       {\bf (#1)} %
}{\end{indented}}

%%%%%%%%%%%%%%%%%%%%%%%%%%%%%%%%%%%%%%%%%%%%%%%%%%%%%%%%%%%
\newcommand{\myname}{Shashank Singh}
\newcommand{\myandrew}{sss1@andrew.cmu.edu}
\newcommand{\myclass}{21-238 Mathematical Studies Algebra II}
\newcommand{\myhwnum}{3}
\newcommand{\duedate}{Wednesday, February 29, 2012}
%%%%%%%%%%%%%%%%%%%%%%%%%%%%%%%%%%%%%%%%%%%%%%%%%%%%%%%%%%%

\begin{document}
\thispagestyle{plain}

{\Large Homework \myhwnum} \\
\myclass \\
Name: \myname \\
Email: \myandrew \\
Due: \duedate

\begin{question}{Exercise 11}
\begin{enumerate}[i)]
%TODO
\item Note that $\forall A \in L(\mathbb{R}^n,\mathbb{R}^n)$,
$\forall v \in \mathbb{C}^n, j \in \{1,2,\infty\}$,
\[\frac{\|Av\|_j}{\|v\|_j} =
\frac{\|Av\|_j}{\|v\|_j}\cdot\frac{\|v\|_j}{\|v\|_j}
= \frac{\|A\frac{v}{\|v\|_j}\|_j}{\|\frac{v}{\|v\|_j}\|_j},\] so that,
in computing $\||A\||_j$, we need only consider vectors $v \in B_j$, the unit
ball determined by the norm $\|\cdot\|_j$.

Computing $|||A|||_1$:

Let $A \in L(\mathbb{R}^n,\mathbb{R}^n)$, and let $v \in B_{\infty}$.
Then, letting $M = \max_
{i \in \{1,2,\ldots,n\}} \sum_{j = 1}^n |a_{j,i}|,$
\[\|Av\|_1 = \sum_{i = 1}^n \sum_{j = 1}^n |A_{j,i}v_i| = \sum_{i = 1}^n |v_i| \sum_{j = 1}^n |A_{i,j}|
 \leq \sum_{i = 1}^n |v_i| M = M \|v\|_1 = M.\]

Furthermore, if $v \in B_1$ is such that,
for $i = \arg \max_{i \in {1,2,\ldots,n}} \sum_{j = 1}^n A_{j,i}$, $v_i = 1$ and,
$\forall j \in \{1,2,\ldots,n\}\backslash\{i\}$, $v_j = 0$, then $AV = M$, so that
\[|||A|||_1
 = \displaystyle \max_{i \in \{1,2,\ldots,n\}} \left(\sum_{j = 1}^n |a_{j,i}|\right). \qquad \blacksquare\]

Computing $|||A|||_{\infty}$:

Let $A \in L(\mathbb{R}^n,\mathbb{R}^n)$, let $v \in B_{\infty}$,
and let $b = Av$, with components $b_1,b_2,\ldots,b_n$. Then,
$\forall i \in \{1,2,\ldots,n\}$, $b_i = A_i \cdot v$, where $A_i$ is the
$i^{th}$ row vector of $A$. $b_i$ is maximized over $B_{\infty}$ by the vector
$v$ with $j^{th}$ component $v_i = \frac{A_{i,j}}{|A_{i,j}|}$, so that
$\max_{v \in B_{\infty}} b_i = \sum_{j = 1}^n |A_{i,j}|$. Thus, we find the
induced matrix norm by maximizing this sum over $i \in \{1,2,\ldots,n\}$, so that
\[|||A|||_{\infty}
 = \max_{i \in \{1,2,\ldots,n\}} \left(\sum_{j = 1}^n |a_{i,j}|\right).\]

Suppose that, for some $A \in L(\mathbb{R}^n,\mathbb{R}^n)$, $\lambda$ is an
eigenvalue of $A$. Let $v$ be the eigenvector corresponding to $\lambda$, with
components $v_1,v_2,\ldots,v_n$. Then, $\forall i \in \{1,2,\ldots,n\}$, %TODO

%TODO
\item

%TODO
\item
\end{enumerate}
\end{question}

\begin{question}{Exercise 12} Let $V$ be a Euclidean space.
\begin{enumerate}[i)]
\item Let $n = \dim V$, and suppose $n \geq 2$.

Let $M_1, M_2 \in L(V,V)$ such that $M_1 = 0$ and

\[M_2 =
\begin{bmatrix}
   0   &    1   &    0   &    0   & \ldots &    0   \\
  -1   &    0   &    0   &    0   & \ldots &    0   \\
   0   &    0   &    0   &    0   & \ldots &    0   \\
   0   &    0   &    0   &    0   & \ldots &    0   \\
\vdots & \vdots & \vdots & \vdots & \ddots & \vdots \\
   0   &    0   &    0   &    0   & \ldots &    0   \\
\end{bmatrix}.\]
Then, $\forall v = (v_1,v_2,\ldots,v_n) \in V$,
$(M_1v,v) = 0 = v_1v_2 - v_1v_2 = (M_2v,v)$, but $M_1 \neq M_2$. Thus, $\geq$
is not antisymmetric $L(V,V)$ when $\dim V \geq 2$, so it is not a partial
order on $L(V,V)$.

Since $\forall M \in L_s(V,V)$, $\forall v \in V$ $(Mv,v) \geq (Mv,v)$, so
$\geq$ is reflexive. For $M_1,M_2,M_3 \in L(V,V)$, if, $\forall v \in V$,
$(M_1v,v) \geq (M_2v,v)$ and $(M_2v,v) \geq (M_3v,v)$, then, since $\geq$ is
transitive on $\mathbb{R}$, $(M_1v,v) \geq (Mv_3,v)$, so that $\geq$ is
transitive on $L_s(V,V)$. Let $M_1,M_2 \in L_s(V,V)$ such that $M_1 \geq M_2$
and $M_2 \geq M_1$. Then, $\forall v \in V$, since the inner product is linear
in its first argument, $((M_1 - M_2)v,v) = (M_1v,v) - (M_2v,v) = 0$. Since
$M_1$ and $M_2$ are symmetric, $(M_1 - M_2)$ is also symmetric, so that it has
$n$ eigenvalues, and is diagonalizable, so that $(M_1 - M_2) = SDS^{-1}$ for
some invertible $S$ and some diagonal matrix $D$. Furthermore, since,
$\forall v \in V, ((M_1 - M_2)v,v) = 0$, all eigenvalues of $(M_1 - M_2)$ are
$0$. Since the non-zero entries of $D$ are the eigenvalues of $(M_1 - M_2)$,
$D = 0$. Therefore, $M_1 - M_2 = 0$, so $M_1 = M_2$. Therefore, $\geq$ is
antisymmetric on $L_s(V,V)$, so that it is a partial order on $L_s(V,V)$.
\qquad $\blacksquare$

%TODO
\item

%TODO
\item
\end{enumerate}
\end{question}

\begin{question}{Exercise 13}
\begin{enumerate}[i)]
%TODO
\item Since $M$ is diagonalizable, it can be diagonalized on an orthogonal
basis, so that $M = SDS^{-1} = SDS^T$, where $D$ is diagonal and $S$ is
orthogonal. Let $v \in V$, and let $w = Sv$, so that $v = S^Tw$. Then,
$(Mv,v) = (SDS^Tv,v) = v^TSDS^Tv = w^TDw$, and, since the identity commutes
with all matrices, $(aIv,v) = v^TaIv = w^TSaIS^Tw = aw^TSS^Tw = aw^Tw$. Thus,
$aI \leq M$ if and only if $aI \leq D$. Similarly, $M \leq bI$ if and only if
$D \leq bI$.

%TODO
\item Let $n$ be the degree of $P$, and, $\forall i \in \{0,1,\ldots,n\}$,
Suppose $P_i(x) = x^i$, where $a_i \in \mathbb{R}$ is the coefficient of
$x^i$ in $P$. Then, for some $C \in L(V,V)$,
\begin{eqnarray*}
\lim_{\epsilon \rightarrow 0} \frac{P_i(A + \epsilon B) - P_i(A)}{\epsilon}
 & = & \lim_{\epsilon \rightarrow 0} \frac{A^i +
   \epsilon \sum_{j = 1}^i A^jBA^{i - j - 1} + \epsilon^2 C - A^i}{\epsilon} \\
 & = & \lim_{\epsilon \rightarrow 0}
   \sum_{j = 1}^i A^jBA^{i - j - 1} + \epsilon C \\
 & = & \sum_{j = 1}^i A^jBA^{i - j - 1}.
\end{eqnarray*}

Since the limit is additive and multiplicative where it exists, and
$P = \sum_{i = 0}^n a_i P_i$ for some $a_i \in \mathbb{R}$,
$\lim_{\epsilon \rightarrow 0} \frac{P(A + \epsilon B) + P(A)}{\epsilon}$
exists. \qquad $\blacksquare$

Since we consider an orthonormal basis eigenvectors of $A$, $A$ can be written
as diagonal matrix, so that $A_{i,i} = \lambda_i$, the $i^{th}$ eigenvalue of
$A$. $\forall i \in \{1,2,\ldots,n\}$ let $P_i$ be as above. Then, for
$i = 0$, clearly $DP_i(A) : B = 0$, so that, since
$P_i^{\prime}(\lambda_i) = 0$ and $P(\lambda_i) - P(\lambda_j) = 0$,
$DP_i(A) : B$ has the desired form.

Suppose that, for some $i \in \mathbb{N}$, $C = DP_i (A) : B$ has the desired form.

As computed previously,
\[DP_{i + 1} (A) : B
 = \sum_{j = 1}^iA^jBA^{i - j - 1}
 = A^n \sum_{j = 1}^iA^jBA^{i - j - 1}\]

Again, since the limit is additive, since $DP_i(A) : B$ has the desired form
$\forall i \in \{1,2,\ldots,n\}$, 

\end{enumerate}
\end{question}


\begin{question}{Exercise 14}
\begin{enumerate}[i)]
%TODO
\item

%TODO
\item
\end{enumerate}
\end{question}

\begin{question}{Exercise 15}
Trivially, $B_0 \in L_s(V,V)$. Suppose, as an induction hypothesis, that, for
some $n \in \mathbb{N}$, $B_n \in L_s(V,V)$. Since $B_n$ must commute with
itself, $B_n^2$ is symmetric, so that, since a sum of symmetric matrices and a
multiple of symmetric matrices must both be symmetric,
\[B_{n + 1} = \frac{A + B^2}{2}\]
must also be symmetric, since $A$ is symmetric. Therefore, by induction on
$n$, $\forall n \in \mathbb{N}$, $B_n \in L_s(V,V)$.
\end{question}
\end{document}
