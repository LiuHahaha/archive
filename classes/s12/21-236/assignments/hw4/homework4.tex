\documentclass[11pt]{article}
\usepackage{enumerate}
\usepackage{fullpage}
\usepackage{fancyhdr}
\usepackage{amsmath, amsfonts, amsthm, amssymb}
\setlength{\parindent}{0pt}
\setlength{\parskip}{5pt plus 1pt}
\pagestyle{empty}

\def\indented#1{\list{}{}\item[]}
\let\indented=\endlist

\newcounter{questionCounter}
\newcounter{partCounter}[questionCounter]
\newenvironment{question}[2][\arabic{questionCounter}]{%
    \setcounter{partCounter}{0}%
    \vspace{.25in} \hrule \vspace{0.5em}%
        \noindent{\bf #2}%
    \vspace{0.8em} \hrule \vspace{.10in}%
    \addtocounter{questionCounter}{1}%
}{}
\renewenvironment{part}[1][\alph{partCounter}]{%
    \addtocounter{partCounter}{1}%
    \vspace{.10in}%
    \begin{indented}%
       {\bf (#1)} %
}{\end{indented}}

%%%%%%%%%%%%%%%%%%%%%%%%%%%%%%%%%%%%%%%%%%%%%%%%%%%%%%%%%%%
\newcommand{\myname}{Shashank Singh}
\newcommand{\myandrew}{sss1@andrew.cmu.edu}
\newcommand{\myclass}{21-236 Mathematical Studies Analysis II}
\newcommand{\myhwnum}{4}
\newcommand{\duedate}{Wednesday, March 7, 2012}
%%%%%%%%%%%%%%%%%%%%%%%%%%%%%%%%%%%%%%%%%%%%%%%%%%%%%%%%%%%

\begin{document}
\thispagestyle{plain}

{\Large Homework \myhwnum} \\
\myclass \\
Name: \myname \\
Email: \myandrew \\
%TODO
\begin{question}{Problem 1} Let
$f: (a,b) \times \mathbb{R} \rightarrow \mathbb{R}$ be of class $C^1$, such
that, $\forall x \in (a,b), \exists \alpha_x > 0$ such that,
$\forall y \in \mathbb{R}$,
\[\left|\frac{\partial f}{\partial y}(x,y)\right| \geq \alpha_x.\]

Suppose there existed $\mathbf{x}_1,\mathbf{x}_2 \in (a,b) \times \mathbb{R}$
such that $f(\mathbf{x}_1) < 0 < f(\mathbf{x}_2)$. Let
$h: [0,1] \rightarrow \mathbb{R}$ such that, $\forall t \in [0,1]$,
$h(t) = \frac{\partial f}{\partial y}(x_1 + t(x_2 - x_1)$. Since $f$ is $C^1$,
$\frac{\partial f}{\partial y}$ is continuous, and therefore $h$ is
continuous. Thus, by the Intermediate Value Theorem, there exists
$t_0 \in [0,1]$ such that $h(t_0) = 0$. However, then,
$\exists (x,y) \in (a,b) \times \mathbb{R}$ such that
\[\left|\frac{\partial f}{\partial y}(x,y)\right| = 0,\]
contradicting the given constraint. Therefore, $\frac{\partial f}{\partial y}$
is either everywhere positive or everywhere negative.
Without loss of generality, take it to be everywhere postive (since taking
$(-f)$ does not change $\{(x,y) \in (a,b) \times \mathbb{R} | f(x,y) = 0\}$).
Let $x \in (a,b)$. Then, since
\[\frac{\partial f}{\partial f} \geq \alpha_x > 0,\] by the Mean Value
Theorem, for $y = \frac{-f(0)}{\alpha_x}$, if $f(0) <  0$, then
$f(x,y) \geq f(x,0) + \alpha_x(y - 0) = 0$, and, if $f(x,0) \geq 0$, then
$f(x,y) \leq f(x,0) + \alpha_x(y - 0) = 0$; in any case, $f(y)$ and $f(0)$ have
different sign, so, by the intermediate value theorem, there exists
$y_0 \in [f(0),y] \cup [y,f(0)]$, such that $f(x,y_0) = 0$. Since $f$ is strictly
increasing in $y$, $y_0$ is unique. Therefore, $\forall x \in (a,b)$, there
exists a unique function $g: (a,b) \rightarrow \mathbb{R}$ (defined by
$g := (x \mapsto y_0)$) such that, $\forall x \in (a,b)$, $f(x,g(x)) = 0$.
\qquad $\blacksquare$

Let $x \in (a,b)$. Then, since $f(x,g(x)) = 0$ and
$\frac{\partial f}{\partial y} (x,g(x)) > \alpha_x$, by the implicit function
theorem, within open balls $B_1 = B(x,r_1) \subseteq (a,b)$,
$B_2 = B(y,r_2) \subseteq \mathbb{R}$, such that, $\forall x \in B_1$, there
is a unique function $g_x: B_1 \rightarrow B_2$ mapping $x$ to $y$ such that
$f(x,y) = 0$, and, furthermore, since $f$ is of class $C^1$, $g_x$ is also of
class $C^1$. However, the restriction of $g$ to $B_1$ has this property, so
that, since $g_x$ is the unique function with this property,
$g |_{B_1} = g_x$, and consequently $g$ is of class $C^1$ in a ball around
$x$. Since this is true for all $x \in (a,b)$, $g$ is of class $C^1$ on its
entire domain. \qquad $\blacksquare$
\end{question}

%TODO
\begin{question}{Problem 2}
Let $g: [0,1] \rightarrow Y$ such that, $\forall t \in [0,1]$,
$g(t) = f(x + t(y - x))$. Let $y_0 = f(y) - f(x) = g(1) - g(0) \in Y$. By the
Corollary of the Hahn-Banach Theorem proven in the notes, there exists a
linear function $L: Y \rightarrow \mathbb{R}$ such that, for
$y_0 = g(1) - g(0)$, $L(y_0) = \|y_0\|$, and,
$\forall y \in Y$, $L(y) \leq \|y\|$.

By the Mean Value Theorem,
$L(g(1)) - L(g(0)) \leq \sup(\frac{d (L \circ g)}{dt} (1 - 0)$.

Because $L$ is linear, so that, $\forall t_1, t_2 \in [0,1]$
$\frac{L(g(t_1)) - L(g(t_2))}{t_1 - t_2} = L\left(\frac{g(t_1) - g(t_2)}{t_1 - t_2}\right)$.
Thus, $\forall t \in [0,1]$, $(L \circ g)^{\prime}(t)  = L(g^{\prime}(t))$.

By the Chain Rule, $\forall w \in S$,
$\frac{d g}{d t}(w)
 = \frac{\partial f}{\partial v}(w) \|y - x\|_X$.

\begin{eqnarray*}
\|f(y) - f(x)\|_Y
 & =    & \|g(1) - g(0)\|_Y \\
 & =    & \|L(g(1) - g(0))\|_Y \\
 & =    & \|(L \circ g)(1) - (L \circ g)(0)\|_Y \\
 & \leq & \sup_{t \in [0,1]} \left\|\left(L \circ g)^{\prime}(t)\right)\right\|_Y|1 - 0| \\
 & =    & \sup_{t \in [0,1]} \left\| L (g^{\prime}(t)) \right\|_Y \\
 & =    & \sup_{w \in S}\left\| L \left(\frac{\partial f}{\partial v}(w)\right)\| y - x \|_X \right\|_Y \\
 & \leq & \sup_{w \in S}\left\| \frac{\partial f}{\partial v}(w) \right\|_Y \| y - x \|_X \\
\end{eqnarray*}
\end{question}

\begin{question}{Problem 3} Let
\begin{enumerate}[(a)]
\item Let $(x,y) \in \{(x,y) : f(x) < y\}$. By the result of part (a) of
Problem 1 on Assignment 2, there exists at least one point
$(t,s) \in E$ such that dist$((x,y), E) = \|(x,y) - (t,s)\|$. Suppose there
exist distinct $x_1,x_2 \in [a,b]$ such that 
dist$((x,y), E) = \|(x,y) - (x_1,f(x_1))\|$ and
dist$((x,y), E) = \|(x,y) - (x_2,f(x_2))\|$.
Without loss of generality, $x_1 < x_2$.
Let $B = B((x,y),g(x,y))$. By definition of the distance function, there
cannot be $t \in [a,b]$ such that $\|(x,y) - (t,f(t))\| < g(x,y)$, so that, at
$(x_1,f(x_1))$ and $(x_2,f(x_2))$, $B$ is tangent to the graph of $f$ (insofar
as $B \cap \{(x,f(x)) : x \in [a,b]\} = \emptyset$ and
$(x_1,f(x_1)), (x_2,f(x_2)) \in \partial B \cap \{(x,f(x)) : x \in [a,b]\}$).
Thus, the curvature of $f$ at some point must be at least that of the boundary
of $B$, so that, since the curvature of a circle of radius $\delta$ is
$\frac{1}{\delta}$, $f^{\prime\prime} \geq \frac{1}{\delta}$.

On the other hand, since $f$ is of class $C^2$, $f^{\prime\prime}$ is
continuous, so that, by the Weierstrass Theorem, $f^{\prime\prime}$ has an
upper bound $M$ on $[a,b]$. Therefore, letting $\delta < \frac{1}{M}$, ensures
that $(t,s)$ is unique, $\forall (x,y) \in U_{\delta}$. \qquad $\blacksquare$

\item Let $(x,y) \in U_\delta$, for $\delta$ as in part (a), and let $(t,s)$
be the unique point shown to exist in part (a). Since the tangent vector of
$f$ at $(t,s)$ is $(t,f^{\prime}(t))$, the normal vector of $f$ at $(t,s)$ is
$(-f^{\prime},t)$. Since $(x,y) - (t,s)$ is in the direction of the normal
vector of $f$ at $(t,s)$, we have
\[(x,y) = (t,s) + \|(x,y) - (t,s)\|\frac{(t,-\frac{1}{f^{\prime}(t)})}{\|(t,-\frac{1}{f^{\prime}(t)})\|}.\]
\qquad $\blacksquare$

\item Let $L: \mathbb{R}^2 \rightarrow \mathbb{R}$ such that,
$\forall (x,y) \in \mathbb{R}^2$, $L(x,y) = x^2 + y^2$. Then,
Letting $h: \mathbb{R} \rightarrow \mathbb{R}$ such that,
$\forall (x,y) \in \mathbb{R}^2$,
\[h(x,y) = 
L((t,s) + \|(x,y) - (t,s)\|\frac{(t,-\frac{1}{f^{\prime}(t)})}{\|(t,-\frac{1}{f^{\prime}(t)})\|} - (x,y)),\]
by the result of part (b) above, fixing $(L \circ h)(x,y) = 0$, since the
derivative of $L \circ h$ is non-zero, by the Implicit Function Theorem, in
some ball around $(t,s)$, there exists a unique function $g_x$ such that
$h(x,g_x(x)) = 0$, and, furthermore, $g_x$ is of class $C^1$. However,
$(L \circ h)(x,y) = 0$ if and only if $g(x,y) = \|(x,y) - (t,s)\|$, so that
$g$ also has this property. Therefore, since $g_x$ is unique $g = g_x$, so
that $g$ is of class $C^1$. \qquad $\blacksquare$
\end{enumerate}
\end{question}

\begin{question}{Problem 4}
\begin{enumerate}[(a)]
\item Let $f : \mathbb{R}^2 \rightarrow \mathbb{R}$ such that,
$\forall (x,y) \in \mathbb{R}^2$,
\[f(x,y) = \alpha \log(1 + xy) + \alpha^2xy - 2 \sin x + y - 1.\]
Evaluating $f$ at $(0,1)$ gives $f(0,1) = 0$.
Evaluating
\[\frac{\partial f}{\partial y} (x,y) = \frac{\alpha x}{1 + xy} + \alpha^2 x + 1\]
at $(0,1)$ gives 
\[\frac{\partial f}{\partial y} (0,1) = 1 \neq 0.\] Therefore, by the Implicit
Function Theorem, fixing $f(x,y) = 0$ determines, in a ball $B_x = B(0,r_1)$,
a unique function $g: B_x \rightarrow B_y = B(1,r_2)$ such that,
$\forall x \in B_x$, $f(x,g(x)) = 0$. \qquad $\blacksquare$
\end{enumerate}
\end{question}
\end{document}
