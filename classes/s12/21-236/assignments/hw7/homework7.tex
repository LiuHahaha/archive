\documentclass[11pt]{article}
\usepackage{enumerate}
\usepackage{fullpage}
\usepackage{fancyhdr}
\usepackage{amsmath, amsfonts, amsthm, amssymb}
\setlength{\parindent}{0pt}
\setlength{\parskip}{5pt plus 1pt}
\pagestyle{empty}

\def\indented#1{\list{}{}\item[]}
\let\indented=\endlist

\newcounter{questionCounter}
\newcounter{partCounter}[questionCounter]
\newenvironment{question}[2][\arabic{questionCounter}]{%
    \setcounter{partCounter}{0}%
    \vspace{.25in} \hrule \vspace{0.5em}%
        \noindent{\bf #2}%
    \vspace{0.8em} \hrule \vspace{.10in}%
    \addtocounter{questionCounter}{1}%
}{}
\renewenvironment{part}[1][\alph{partCounter}]{%
    \addtocounter{partCounter}{1}%
    \vspace{.10in}%
    \begin{indented}%
       {\bf (#1)} %
}{\end{indented}}

%%%%%%%%%%%%%%%%%%%%%%%HEADER%%%%%%%%%%%%%%%%%%%%%%%%%%%%%%
\newcommand{\myname}{Shashank Singh}
\newcommand{\myandrew}{sss1@andrew.cmu.edu}
\newcommand{\myclass}{21-236 Mathematical Studies Analysis II}
\newcommand{\myhwnum}{7}
\newcommand{\duedate}{Monday, April 23, 2012}
%%%%%%%%%%%%%%%%%%%%%%%%%%%%%%%%%%%%%%%%%%%%%%%%%%%%%%%%%%%

%%%%%%%%%%%%%%%%%%%%%%%%MACROS%%%%%%%%%%%%%%%%%%%%%%%%%%%%%
\renewcommand{\qed}{\quad $\blacksquare$}
\newcommand{\mqed}{\quad \blacksquare}
\newcommand{\bvarphi}{\boldsymbol{\varphi}}
\newcommand{\bpsi}{\boldsymbol{\psi}}
\newcommand{\Var}{\operatorname{Var}}
\newcommand{\meas}{\operatorname{meas}}
\newcommand{\diam}{\operatorname{diam}}
\newcommand{\bx}{\mathbf{x}}
\newcommand{\by}{\mathbf{y}}
\newcommand{\bff}{\mathbf{f}}
\newcommand{\bfg}{\mathbf{g}}
\newcommand{\bfh}{\mathbf{h}}
\newcommand{\pjm}{Peano-Jordan measurable }
%%%%%%%%%%%%%%%%%%%%%%%%%%%%%%%%%%%%%%%%%%%%%%%%%%%%%%%%%%%

\begin{document}
\thispagestyle{plain}

{\Large Homework \myhwnum} \\
\myclass \\
Name: \myname \\
Email: \myandrew \\

\begin{question}{Problem 1}
\begin{enumerate}[(a)]
\item Let $\Gamma$ be the range of $\gamma_1$, and let
$\bvarphi: [a,b] \rightarrow \mathbb{R}^N$ be a parametrization of $\gamma_1$.
Since $U$ is open, $\forall \mathbf{x} \in U$, $\exists r > 0$ such that
$B(\mathbf{x},r) \subseteq U$, so that there is a family
$\{U_{\alpha}\}_{\alpha}$ of open balls around every point in $\Gamma$ with
\[\bigcup \{U_{\alpha}\}_{\alpha} \subseteq U.\] Clearly,
$\{U_{\alpha}\}_{\alpha}$ covers $\Gamma$. Since $\gamma_1$ is continuous and
$[a,b]$ is compact, $\Gamma$ is compact. Thus, there
exists some $\delta > 0$ (Lebesgue's number) such that,
$\forall E \subseteq \Gamma$ with $\diam E \leq \delta$,
$E \subseteq U_{\alpha}$ for some $\alpha$.
Note that, since $\gamma_1$ is continuous and $[a,b]$ is compact, $\bvarphi$
is uniformly continuous, so that, $\exists \delta_2 > 0$ such that,
$\forall x,y \in [a,b]$, $|x - y| < \delta_2$ implies
$\|\bvarphi(x) - \bvarphi(y)\| < \delta$. Define
$k = \lceil\frac{b - a}{\delta_2/2}\rceil$, and,
$\forall i \in \{0,1,\ldots,k\}$, let $t_i = a + i\frac{b - a}{k}$.
Then, $\forall i \in \{0,1,\ldots,k\}$, $t_i - t_{i - 1} < \delta_2$, so that,
by choice of $\delta_2$, $\diam \bvarphi([t_{i - 1},t_i]) < \delta$, and
therefore, by choice of $\delta$,
$\bvarphi([t_{i - 1},t_i]) \subseteq U_{\alpha}$ for some $\alpha$.
Define $\bfh: [a,b] \times [0,1]$ such that,
$\forall (s,t) \in [a,b] \times [0,1]$, if $s \in [t_{i - 1},t_i]$,
\[\bfh(s,t) = \bvarphi(s) + t(L_i(s) - \bvarphi(s)),\]
where
\[L_i(s) := \bvarphi(t_{i - 1}) + \frac{s - t_{i - 1}}{t_i - t_{i - 1}}(\bvarphi(t_{i}) - \bvarphi(t_{i - 1}))\]
is the point that is the same fraction along the line segment from
$\bvarphi(t_{i -1})$ to $\bvarphi(t_i)$ as is $s$ on the line segment from
$t_{i - 1}$ to $t_i$.

Since each $\bvarphi([t_{i - 1},t_i])$ is contained in some ball $U_{\alpha}$,
which must be convex, $\bfh([a,b] \times [0,1]) \subseteq U$.
$\forall s \in [a,b]$, $\bfh(s,0) = \bvarphi(s)$, and, $\forall t \in [0,1]$,
$\bfh(a,t) = \bvarphi(a) = \bvarphi(b) = h(b,t)$. Finally, each
$\bfh([t_i,t_{i - 1}],1)$ is linear, so that $h([a,b],1)$ is a closed polygonal
path. Therefore, $h$ is a homotopy from $\gamma_1$ to a closed polygonal path.
\qed

\item Let $bfh_1 = \bfh$ as defined in part (a), and let
$\bpsi: [a,b] \rightarrow U$ such that, $\forall s \in [a,b]$,
$\bpsi(s) = \bfh_1(s,1)$, so that $\bpsi$ parametrizes a polygonal path to
which $\gamma_1$ is homotopic. Since $U$ is open,
$\forall i \in \{1,2,\ldots,k\}$, $\exists r_i > 0$ such that
$B_i := B(\bpsi(t_i), r_i) \subseteq U$. Note that, since $\bpsi$ is
continuous, $\bpsi^{-1}(B_i)$ is an open interval $(a_i,b_i)$, and consider
$\bfh_2:[a,b] \times [0,1]$ such that,
$\forall t \in [0,1]$,
$\forall s \in (a_i,b_i)$,
\[\bfh_2(s,t)
 = L_i(s) + t\frac{s - a_i}{b_i - a_i}(L_{i + 1}(s) - L_i(s)),\]
where $L_i$ is as defined in part (a), and, for all other $s \in [a,b]$,
$\bfh_2(s,t) = L_i(s)$.

Let $\bfh:[a,b] \times [0,1]$ so that, $\forall (s,t) \in [a,b] \times [0,1]$,
\[\bfh(s,t)
  = \left\{
      \begin{array}{cc}
        \bfh_1(s,2t) & t \in [0,\frac12] \\
        \bfh_2(s,2t - 1) & t \in [\frac12,1]
      \end{array}
    \right.
\]
Then, $\forall s \in [a,b]$, $\bfh(s,0) = \bvarphi(s)$ (where $\bvarphi$ is
the parametrization of $\gamma_1$ from part (a)) and $\bfh(s,1)$ parametrizes
a closed $C^1$ curve, and $\forall t \in [0,1]$, $\bfh(a,t) = \bfh(b,t)$.
Since each $B_i \subseteq U$ is convex and
$\bfh_1([a,b] \times [0,1]) \subseteq U$,
$\bfh([a,b] \times [0,1]) \subseteq U$.
Therefore, $\bfh$ is a homotopy in $U$ from $\gamma_1$ to a closed $C^1$
curve. \qed
\end{enumerate}
\end{question}

\begin{question}{Problem 2}
\begin{enumerate}[(a)]
\item
{\bf Lemma:} If $F = \cup_{i = 1}^{\infty} F_i$ is a union of disjoint \pjm
sets and $F$ is Peano-Jordan measurable, then
\[\meas F \leq \sum_{i = 1}^{\infty} \meas F_i.\]

{\bf Proof:} Suppose, for sake of contradiction, that
$\sum_{i = 1}^{\infty} \meas F_n < \meas F$. Note that, since
$\cup_{i = 2}^n F_i \subseteq F\backslash F_1$,
$\sum_{i = 1}^n \meas F_i \leq \meas (F\backslash F_1)$, so that
\[\sum_{i = 1}^{\infty} \meas F_i \leq \meas (F\backslash F_1).\]

Thus,
\begin{eqnarray*}
\meas F_1
 & =    & \sum_{i = 1}^{\infty} \meas F_i - \sum_{i = 2}^{\infty} \meas F_i \\
 & <    & \meas F - \sum_{i = 2}^{\infty} \meas F_i \\
 & \leq & \meas F - \meas(F\backslash F_1) \\
 & =    & \meas (F\backslash(F\backslash F_1))
   =      \meas F_1,
\end{eqnarray*}
which is a contradiction, proving the Lemma.

Note that, since $\{\meas E_n\}$ is increasing (as
$E_i \subseteq E_{i + 1}$) and bounded above by $\meas E$ (as
$E_i \subseteq E$), $\lim_{n \rightarrow \infty} \meas E_n$ exists and
$\lim_{n \rightarrow \infty} E_n \leq \meas E$.

$\forall i \in \mathbb{N}^*$, let $F_i := E_i \backslash E_{i - 1}$. For each
$i \in \mathbb{N}^*$, since $\{E_n\}$ is exhausting,
$E_i = \cup_{j = 1}^i F_i$, so that $\meas E_i = \sum_{j = 1}^i \meas F_i$.
Since $E = \cup_{i = 1}^{\infty} F_i$, by the Lemma, since the $F_i$'s are
\pjm and disjoint,
\[\meas E
 \leq \sum_{i = 1}^{\infty} \meas F_i
 = \lim_{n \rightarrow \infty} \sum_{i = 1}^n \meas F_i
 = \lim_{n \rightarrow \infty} \meas E_i,\]
and, therefore, $\lim_{n \rightarrow \infty} \meas E_n = \meas E$. \qed

\item
Since $f$ is Riemann integrable, it is bounded above by some constant $M$ and
below by some constant $m$.
Since $E$ is \pjm, $E$ is bounded, so that it is contained
in some rectangle $R$. Therefore, by the result of part (a) above,
\begin{eqnarray*}
0 & =    & M\lim_{n \rightarrow \infty} \meas(E \backslash E_n) \\
  & =    & M\lim_{n \rightarrow \infty} \int_R \chi_{E\backslash E_n}(\bx) \; d\bx \\
  & =    & M\lim_{n \rightarrow \infty} \int_R (\chi_E - \chi_{E_n})(\bx) \; d\bx \\
  & \geq & \lim_{n \rightarrow \infty} \int_R f(\bx)(\chi_E - \chi_{E_n})(\bx) \; d\bx \\
  & =    & \lim_{n \rightarrow \infty} \left(\int_R f(\bx)\chi_E(\bx) \; d\bx
    -                                        \int_R f(\bx)\chi_{E_n}(\bx) \; d\bx\right). \\
\end{eqnarray*}
Subtracting $\int_R f(\bx)\chi_E(\bx) \; d\bx$ shows that
\[\int_R f(\bx)\chi_E(\bx) \; d\bx
 \leq \int_R f(\bx)\chi_{E_n}(\bx) \; d\bx
.\]
A similar proof with $m$ shows that
\[\int_R f(\bx)\chi_E(\bx) \; d\bx
  \geq \lim_{n \rightarrow \infty}  \int_R f(\bx)\chi_{E_n}(\bx) \; d\bx
.\]
Thus, $\lim_{n \rightarrow \infty}  \int_R f(\bx)\chi_{E_n}(\bx) \; d\bx$
exists and
\[\int_R f(\bx)\chi_E(\bx) \; d\bx = 
\lim_{n \rightarrow \infty}  \int_R f(\bx)\chi_{E_n}(\bx) \; d\bx
,\]
as desired. \qed
\end{enumerate}
\end{question}

\begin{question}{Problem 3}
For $k \in \{1,2,\ldots,N\}$, let $i_k$ be the smallest number such that
$(\nabla \bfg_k)_{i_k} \neq 0$ and $i_k \neq i_1,i_2,\ldots,i_{k -1}$ (such
$i_k$ must exist, since $\det J_{\bfg}(\bx) \neq 0$).

Let $\bfh = \bff_1 \circ \bff_2 \circ \cdots \circ \bff_N$, where each
$\bff_j$ is a flip switching $j$ with $i_j$.

Then, $\forall k \in \{1,2,\ldots,N\}$, let
\[\bfh_i(\bx)
 = (x_1,x_2,\ldots,x_{i - 1},
      \bfg_i((\bfh_{k - 1}\circ\bfh_{k - 2}\circ\ldots\circ\bfh_1)^{-1}(\bx)),
          x_{i + 1},\ldots,x_N)\]
(noting that, by the inverse function theorem and the fact that the
composition of invertible functions is invertible,
$\bfh_{k - 1}\circ\bfh_{k - 2}\circ\ldots\circ\bfh_1$ is invertible in some
ball $U_k$).
Then, \[\bfg = \bfh_N \circ \bfh_{N - 1} \circ \ldots \circ \bfh_1 \circ \bfh,\]
and, $\det J_{\bfh} = 1$, and, since $\det J_{\bfg} \neq 0$,
$\det_{\bfh_i} \neq 0$. \qed
\end{question}

\begin{question}{Problem 4}
\begin{enumerate}[(a)]
\item $E$ appears as follows:
\vspace{3cm}

We show that $E$ is \pjm by computing the integral of its characteristic
function (using a Change of Variables into polar coordinates):
\begin{eqnarray*}
\int_E 1 \; d\bx
 & = & \int_{\pi/4}^{\pi/2} \int_{\frac{\alpha}{\sin \theta}}^1 r \; dr \, d\theta \\
 & = & \int_{\pi/4}^{\pi/2} \frac12 - \frac{\alpha^2}{2\sin^2 \theta} \; d\theta \\
 & = & \frac12(\pi/2) + \frac{\alpha^2\cot(\pi/2)}{2} \\
   -   \frac12(\pi/4) + \frac{\alpha^2\cot(\pi/4)}{2}
 & = & \mbox{\fbox{$\displaystyle \frac{\pi + 4\alpha^2}{8}$.}}
\end{eqnarray*}

\item We show $f$ is integrable over $E$ by computing its integral (using a
Change of Variables into polar coordinates):
\vspace{3cm}

\item
\vspace{3cm}

\item
\end{enumerate}
\end{question}
\end{document}
