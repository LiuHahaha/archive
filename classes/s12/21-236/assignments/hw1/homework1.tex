\documentclass[11pt]{article}
\usepackage{enumerate}
\usepackage{fullpage}
\usepackage{fancyhdr}
\usepackage{amsmath, amsfonts, amsthm, amssymb}
\setlength{\parindent}{0pt}
\setlength{\parskip}{5pt plus 1pt}
\pagestyle{empty}

\def\indented#1{\list{}{}\item[]}
\let\indented=\endlist

\newcounter{questionCounter}
\newcounter{partCounter}[questionCounter]
\newenvironment{question}[2][\arabic{questionCounter}]{%
    \setcounter{partCounter}{0}%
    \vspace{.25in} \hrule \vspace{0.5em}%
        \noindent{\bf #2}%
    \vspace{0.8em} \hrule \vspace{.10in}%
    \addtocounter{questionCounter}{1}%
}{}
\renewenvironment{part}[1][\alph{partCounter}]{%
    \addtocounter{partCounter}{1}%
    \vspace{.10in}%
    \begin{indented}%
       {\bf (#1)} %
}{\end{indented}}

%%%%%%%%%%%%%%%%%%%%%%%%%%%%%%%%%%%%%%%%%%%%%%%%%%%%%%%%%%%
\newcommand{\myname}{Shashank Singh}
\newcommand{\myandrew}{sss1@andrew.cmu.edu}
\newcommand{\myclass}{21-236 Mathematical Studies Analysis II}
\newcommand{\myhwnum}{1}
\newcommand{\duedate}{Monday, January 30, 2012}
%%%%%%%%%%%%%%%%%%%%%%%%%%%%%%%%%%%%%%%%%%%%%%%%%%%%%%%%%%%

\begin{document}
\thispagestyle{plain}

{\Large Homework \myhwnum} \\
\myclass \\
Name: \myname \\
Email: \myandrew \\
Due: \duedate
\begin{question}{Problem 1}
\begin{enumerate}[(a)]
\item Let $U \subseteq \mathbb{R}^N$ be open and convex, and let
$f: U \rightarrow \mathbb{R}$ be differentiable in $U$.

Suppose $f$ is Lipschitz continuous in $U$, with Lipschitz constant $L \geq 0$.
Let $x \in U$, and let
\[i = \arg \max_{k = 1, \ldots, n} \left( \frac{\partial{f}}{\partial e_i} \right)\]
(so that $\frac{\partial{f}}{\partial e_i} (\mathbf{x})$ is the partial
derivative of $f$ at $\mathbf{x}$ of greatest magnitude).
Then,
\begin{eqnarray*}
||\nabla{f(\mathbf{x})}||
 & = & \sqrt{\sum_{k = 1}^n \left(\frac{\partial{f}}{\partial e_k} (\mathbf{x})\right)^2} \\
 & \leq & \sqrt{\sum_{k = 1}^n \left(\frac{\partial{f}}{\partial e_i} (\mathbf{x})\right)^2} \\
 & = & \sqrt{N \left(\frac{\partial{f}}{\partial e_i} (\mathbf{x})\right)^2} \\
\end{eqnarray*}
Letting $\mathbf{y} = \mathbf{x} - t\mathbf{e}_i$, since $f$ is Lipschitz
continuous with Lipschitz constant $L$,
\[L \geq \frac{|f(\mathbf{x}) - f(\mathbf{y})|}{||\mathbf{x} - \mathbf{y}||}
 = \frac{|f(\mathbf{x} + t\mathbf{e}_i) - f(\mathbf{x})|}{||t\mathbf{e}_i||},\]
so that
\[L \geq \lim_{t \rightarrow 0} \frac{|f(\mathbf{x} + t\mathbf{e}_i) - f(\mathbf{x})|}{||t\mathbf{e}_i||}
 = \frac{\partial{f}}{\partial e_i} (\mathbf{x}).\]
Therefore, for
\[M = \sqrt{N L^2} = L \sqrt{N} > 0,\] we have $||\nabla{f(\mathbf{x})}|| \leq M$. \qquad $\blacksquare$

Suppose, on the other hand, that, for some $M > 0$, $\forall x \in U$,
$\nabla f(\mathbf{x}) \leq M$. 
Since $f$ is differentiable in $U$, by the Cauchy-Schwarz inequality, $\forall \mathbf{x} \in U$,
\[\frac{\partial f}{\partial \mathbf{v}} (\mathbf{x})
 = \nabla f (\mathbf{x}) \cdot \mathbf{v}
 \leq ||\nabla f (\mathbf{x})|| \cdot ||\mathbf{v}||
 = ||\nabla f (\mathbf{x})|| \leq M. \]

Let $\mathbf{x}, \mathbf{y} \in U$. Let
$\mathbf{v} = \frac{\mathbf{y} - \mathbf{x}}{||\mathbf{y} - \mathbf{x}||}$.
Since $f$ is differentiable on $U$, by the Mean Value Theorem, for some
$\Theta \in (0,1)$ (noting that, since $U$ is convex, so that
$(\Theta \mathbf{x} + (1 - \Theta) \mathbf{y}) \in U$),
\[\frac{|f(\mathbf{x}) - f(\mathbf{y})|}{||\mathbf{x} - \mathbf{y}||}
 = \frac{\partial f}{\partial \mathbf{v}} (\Theta \mathbf{x} + (1 - \Theta) \mathbf{y})
 \leq M.\]
Thus, $f$ is Lipschitz continuous on $U$. \qquad $\blacksquare$

\item $f$ has continuous partial derivatives on $U$ and is thus differentiable
on $U$. The partial derivatives of $f$ are bounded above and below by $2$ and
$-2$, respectively. Thus, for $M = 2\sqrt{2}$, $||\nabla f (x,y)|| \leq M$,
$\forall (x,y) \in U$. For $x = 1$, for $y > 0$, $f(x,y) > 1$, whereas, for
$y < 0$, $f(x,y) < -1$, so that, $\forall L > 0$,
$\exists (x_1,y_1),(x_2,y_2) \in U$ such that
\[|f(x_1,y_1) - f(x_2,y_2)| > L||fx_1,y_1) - fx_2,y_2).||\] Therefore, $f$ is
not Lipschitz continuous on $U$. \qquad $\blacksquare$
\end{enumerate}
\end{question}

\begin{question}{Problem 2}
Let $f : B(\mathbf{x}_0, r) \rightarrow \mathbb{R}$ be Lipschitz continuous,
with Lipschitz constant $L$.
\begin{enumerate}[(a)]
\item This follows immediately from part (b).

\item Since $\partial B(\mathbf{0}$ is a closed and bounded set in $R^N$, $E$ is compact in $R^N$.
Thus, since $\forall k \in \mathbb{N}$, $C := \{B(\mathbf{v}, \frac{1}{2k}) |
\mathbf{v} \in E\}$ is an open cover of $E$, there exists a finite open
subcover $S \subseteq C$ of $E$. Since $E$ is dense, $\forall B \in S$,
$\exists \mathbf{v} \in E \cap B$. Thus, $\forall \mathbf{x} \in B(\mathbf{x}_0,r)$, for some $\mathbf{v} \in E$,
\[\frac{f(\mathbf{x} - \mathbf{x}_0 - \nabla f(\mathbf{x}) (\mathbf{x} - \mathbf{x}_0)}{||\mathbf{x} - \mathbf{x}_0||}
= \frac{f(\mathbf{x} - \mathbf{x}_0 - \frac{\partial f}{\partial \mathbf{v}} (mathbf{x} - \mathbf{x}_0)}{||\mathbf{x} - \mathbf{x}_0||}
 \rightarrow 0\]
as $\mathbf{x} \rightarrow \mathbf{x}_0$. Thus, since $\mathbf{x}_0$ is an interior point of $B(\mathbf{x}_0, r)$,
$f$ is differentiable at $\mathbf{x}_0$. \qquad $\blacksquare$
\end{enumerate}
\end{question}

\begin{question}{Problem 3} Let
$f: \mathbb{R}^2 \backslash \{(x,y) \mathbb{R}^2 | xy = 0, x^2 + y^2 \neq 0\}
    \rightarrow \mathbb{R}$ such
that, $\forall (x,y) \in \mathbb{R}$, \[f(x,y) = \frac{|xy|}{xy} (x^2 + y^2).\]
Then,
\[
\lim_{(x,y) \rightarrow \mathbf{0}}
    \frac{f(x,y) - f(\mathbf{0}) - 0}
        {||(x,y) - \mathbf{0}||}
 =
\lim_{(x,y) \rightarrow \mathbf{0}}
    \frac{|xy|(x^2 + y^2)}{xy \sqrt{x^2 + y^2}}
 = 0,\]
since $\frac{|xy|}{xy}$ is bounded and $\sqrt{x^2 + y^2} \rightarrow 0$ as
$(x,y) \rightarrow \mathbf{0}$,
so that $f$ is differentiable and thus continuous at $\mathbf{0}$.
However, since $f$ is undefined on the $x$- and $y$-axes (except at $\mathbf{0}$),
$f$ has no partial derivatives at $\mathbf{0}$. \qquad $\blacksquare$

\end{question}

\begin{question}{Problem 4}
\begin{enumerate}[(a)]
\item $f$ has continuous partial derivatives and is thus differentiable and
continuous everywhere except $\mathbf{0}$. At $\mathbf{0}$, $f$ is continuous
if and only if $m + n \geq 2$, and differentiable if and only if
$m + n \geq 3$.

\item $f$ has continuous partial derivatives and is thus differentiable and
continuous everywhere except $\mathbf{0}$. At $\mathbf{0}$, $f$ is continuous
if and only if $m \geq 2$, $n \geq 4$, or $m = 1$ and $n = 3$, and $f$ is
differentiable if and only if $m \geq 3$, $n \geq 5$, $m = 1$ and $n = 4$, or
$m = n = 2$.

\item $f$ has continuous partial derivatives and is thus differentiable and
continuous wherever $x^2 \neq y^2$. $f$ is discontinuous and thus not
differentiable wherever $x^2 = y^2$. Wherever $x^2 = y^2$, the directional
derivatives of $f$ exist only in those directions pointing towards and away
from the origin.

\item Let $g, h: \mathbb{R}^2 \rightarrow \mathbb{R}$ such that, $\forall (x,y) \in \mathbb{R}^2$,
$g(x,y) = x^2 \sin\left(\frac{1}{x}\right)$ and $h(x,y) = y^2 \sin\left(\frac{1}{y}\right)$.
Then, $g$ and $h$ are everywhere differentiable, so that, since $f = g + h$, $f$ is everywhere
differentiable. Therefore, $f$ is everywhere continuous, and all directional derivatives of $f$
exist at all points.
\end{enumerate}
\end{question}
\end{document}
