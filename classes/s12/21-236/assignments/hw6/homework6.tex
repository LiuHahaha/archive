\documentclass[11pt]{article}
\usepackage{enumerate}
\usepackage{fullpage}
\usepackage{fancyhdr}
\usepackage{amsmath, amsfonts, amsthm, amssymb}
\setlength{\parindent}{0pt}
\setlength{\parskip}{5pt plus 1pt}
\pagestyle{empty}

\def\indented#1{\list{}{}\item[]}
\let\indented=\endlist

\newcounter{questionCounter}
\newcounter{partCounter}[questionCounter]
\newenvironment{question}[2][\arabic{questionCounter}]{%
    \setcounter{partCounter}{0}%
    \vspace{.25in} \hrule \vspace{0.5em}%
        \noindent{\bf #2}%
    \vspace{0.8em} \hrule \vspace{.10in}%
    \addtocounter{questionCounter}{1}%
}{}
\renewenvironment{part}[1][\alph{partCounter}]{%
    \addtocounter{partCounter}{1}%
    \vspace{.10in}%
    \begin{indented}%
       {\bf (#1)} %
}{\end{indented}}

%%%%%%%%%%%%%%%%%%%%%%%%%%%%%%%%%%%%%%%%%%%%%%%%%%%%%%%%%%%
\newcommand{\myname}{Shashank Singh}
\newcommand{\myandrew}{sss1@andrew.cmu.edu}
\newcommand{\myclass}{21-236 Mathematical Studies Analysis II}
\newcommand{\myhwnum}{6}
\newcommand{\duedate}{Wednesday, April 11, 2012}
%%%%%%%%%%%%%%%%%%%%%%%%%%%%%%%%%%%%%%%%%%%%%%%%%%%%%%%%%%%

\begin{document}
\thispagestyle{plain}

{\Large Homework \myhwnum} \\
\myclass \\
Name: \myname \\
Email: \myandrew \\
\begin{question}{Problem 1}
Let $\boldsymbol{\varphi}: [a,b] \rightarrow \mathbb{R}^2$ parametrize
$\gamma$, and let
$\mathbf{h}_1,\mathbf{h}_2: [a,b] \times [0,1] \rightarrow \mathbb{R}^2$ such
that, $\forall (s,t) \in [a,b] \times [0,1]$, if $(\rho,\theta)$ is the polar
representation of $\boldsymbol{\varphi}(s)$ (i.e.,
$\boldsymbol{\varphi}(s) = (\rho \cos \theta, \rho \sin \theta)$, $\theta \in [0,2\pi]$),
\[\mathbf{h_1}(s,t) = (\rho + t(r - \rho), \theta)\]
and
\[\mathbf{h_2}(s,t) =
\left(1,
\theta + t\left((2 \pi \omega)\frac{(s - a)}{(b - a)} - \theta\right)\right)
,\]
where $\omega = \operatorname{ind}_{\gamma}(x_0,y_0)$.
Then, let $\mathbf{h}:[a,b] \times [0,1]$ such that,
$\forall (s,t) \in [a,b] \times [0,1]$,
\[\mathbf{h}(s,t) =
  \left\{
    \begin{array}{cc}
      (x_0,y_0) + \mathbf{h}_1(s,2t), & t \in \left[0,\frac12\right] \\
      (x_0,y_0) + \mathbf{h}_2(s,2t - 1), & t \in \left[\frac12,1\right]
    \end{array}
  \right.
.\]
Let $\gamma_2$ be the continuous closed curve parametrized by
$\boldsymbol{\phi}: [a,b] \rightarrow \mathbb{R}^2$ such that,
$\forall s \in [a,b]$,
$\boldsymbol{\phi}(s) = (x_0,y_0) + (r \cos(\omega s), r \sin(\omega s))$.
Then, $\forall s \in [a,b]$, $\mathbf{h}(s,0) = \boldsymbol{\varphi}(s)$ and
$\mathbf{h}(s,1) = \boldsymbol{\phi}(s)$, and, because $\sin 0 = \sin(2\pi)$
and $\cos 0 = \cos(2\pi)$ $\forall t \in [0,1]$,
$\mathbf{h}(a,t) = \mathbf{h}(b,t)$.
$\forall (s,t) \in [a,b] \times [0,\frac12]$, $\mathbf{h}(s,t)$ is on the line segment
from $\boldsymbol{\varphi}(s) \in U$ to $(x_0,y_0)$, and,

$\forall (s,t) \in [a,b] \times [\frac12,1]$ $\mathbf{h}(s,t) \in \partial B((x_0,y_0),r)$

so that, since $U$ is starshaped with respect to $(x_0,y_0)$ and by choice of
$r$, $\forall (s,t) \in [a,b] \times [0,1]$, $h(s,t) \in U$.

Therefore,
$\mathbf{h}$ is a homotopy from $\gamma$ to $\gamma_2$ so that, since
$\mathbf{g}$ is irrotational and of class $C^1$, by Theorem 143,
\[\int_{\gamma} \mathbf{g} = \int_{\gamma_2} \mathbf{g}.\]
However, $\gamma_2$ is the union of $\omega$ identical curves parametrized by
$\psi$, so that
\[\int_{\gamma} \mathbf{g}
 = \int_{\gamma_2} \mathbf{g}
 = \sum_{i = 1}^{\omega} \int_{\gamma_1} \mathbf{g}
 = \operatorname{ind}_{\gamma}\int_{\gamma_1} \mathbf{g}
\quad \blacksquare.\]


\end{question}

\begin{question}{Problem 2}
Let $\boldsymbol{\varphi}: [a,b] \subseteq \mathbb{R} \rightarrow V$
be a parametrization of $\gamma$ (with components
$\varphi_1,\varphi_2,\varphi_3$), let $\gamma_2$ be the curve parametrized by
$\boldsymbol{\phi}: [a,b] \rightarrow V$ such that,
$\forall s \in [a,b]$,
\[\boldsymbol{\phi}(s) = (\varphi_1(s),\varphi_2(s),0),\] and let
$\mathbf{h}:[a,b] \times [0,1] \rightarrow V$ such that,
$\forall (s,t) \in [a,b] \times [0,1]$,
\[\mathbf{h}(s,t) = (\varphi_1(s),\varphi_2(s),\varphi_3(s) - t\varphi_3(s)).\]
Note that, $\forall s \in [a,b]$, $\mathbf{h}(s,0) = \boldsymbol{\varphi}(s)$ and
$\mathbf{h}(s,1) = \boldsymbol{\phi}(s)$, and, $\forall t \in [0,1]$,
$\mathbf{h}(a,t) = \mathbf{h}(b,t)$. Since $\mathbf{h}$ is linear in $t$ and
$\boldsymbol{\varphi}$ is continuous (as $\gamma$ is continuous), $\mathbf{h}$
is continuous. Since $h_1 = \varphi_1$ and $h_2 = \varphi_2$, and
$\boldsymbol{\varphi}$ has range in $V$, $h$ has range in $V$. Therefore,
$\mathbf{h}$ is a homotopy from $\gamma$ to $\gamma_2$, and $\gamma$ and
$\gamma_2$ are homotopic in $V$. By Theorem 143, then, since $\mathbf{g}$ is
an irrotational field of class $C^1$,
\[\int_{\gamma} \mathbf{g} = \int_{\gamma_2} \mathbf{g}.\]

Note that $\gamma_2$ in $\mathbb{R}^3$ is the same curve as $\Pi_{\gamma}$ is
$\mathbb{R}^2$, so that
\[\int_{\gamma_2} \mathbf{g} = \int_{\Pi_{\gamma}} \mathbf{g}.\] By the result
of Problem 1 above, since $U := \mathbb{R}^2 \backslash \{(0,0)\}$ is open and
starshaped with respect to $(0,0)$, and $\partial B((0,0),1) \subseteq U$,
\[\int_{\gamma} \mathbf{g}
 = \int_{\gamma_2} \mathbf{g}
 = \int_{\Pi_{\gamma}} \mathbf{g}
 = \operatorname{ind}_{\Pi_{\gamma}}(0,0) \int_{\gamma_1} \mathbf{g},
\]
as $\gamma_1$ in $\mathbb{R}^3$ is the same curve as unit circle centered at
the origin in $\mathbb{R}^2$. \quad $\blacksquare$
\end{question}

\begin{question}{Problem 3}
\begin{enumerate}[(a)]
\item Let $h:U \rightarrow \mathbb{R}^3$ such that, $\forall (x,y,z) \in U$,
\[h(x,y,z) = \frac12\ln(x^2 + y^2).\]
Then, by the Quotient Rule,
\[\frac{\partial g_1}{\partial y}(x,y,z)
 = \frac{-xz(2y)}{(x^2 + y^2)^2}
 = \frac{\partial g_2}{\partial x}(x,y,z).\]
Furthermore,
\[\frac{\partial g_1}{\partial z}(x,y,z)
 = \frac{x}{x^2 + y^2}
 = \frac12\frac{2x}{x^2 + y^2}
 = \frac{\partial h}{\partial x}(x,y,z)
 = \frac{\partial g_3}{\partial x}(x,y,z),\]
and
\[\frac{\partial g_2}{\partial z}(x,y,z)
 = \frac{y}{x^2 + y^2}
 = \frac12\frac{2y}{x^2 + y^2}
 = \frac{\partial h}{\partial y}(x,y,z)
 = \frac{\partial g_3}{\partial y}(x,y,z).\]
Thus, $\mathbf{g}$ is irrotational. \quad $\blacksquare$

\item $\forall (x,y,z) \in \mathbb{R}^3$, if $z = 0$, then
$\mathbf{g}(x,y,z) = \mathbf{0}$. Thus, $\forall t \in [0,2\pi]$,
$\mathbf{g}(\boldsymbol{\varphi}(t)) = \mathbf{0}$, so that, by definition of the
curve integral,
\[\int_{\gamma} \mathbf{g}
 = \int_0^{2\pi} \sum_{i = 1}^3 0\cdot(\varphi_i^{\prime}(t)) \; dt
 = 0. \quad \blacksquare\]

\item Let $\gamma$ be a piecewise $C^1$ closed, oriented curve with range
$\Gamma \subseteq U$. By the result of Problem 2 above, since $\mathbf{g}$ is
irrotational and $C^1$,
\[\int_{\gamma} \mathbf{g}
 = \operatorname{ind}_{\Pi_{\gamma}}((0,0)) \int_{\gamma_1} \mathbf{g},\]
where $\gamma_1$ is the closed curve parametrized by
$\boldsymbol{\varphi}(t) = (\cos t, \sin t, 0), t \in [0,2 \pi]$.
Then, by the result of part (b) above,
\[\int_{\gamma} \mathbf{g}
 = \operatorname{ind}_{\Pi_{\gamma}}((0,0))\cdot 0 = 0.\]
By Theorem 130, then, $\mathbf{g}$ is conservative. \quad $\blacksquare$

\item Let $f: U \rightarrow \mathbb{R}^3$ such that, $\forall (x,y,z) \in U$,
\[f(x,y,z) = \frac{z}{2}\ln(x^2 + y^2) - \frac{\ln 2}{2}.\]
Then,
\begin{eqnarray*}
\frac{\partial f}{\partial x}(x,y,z) & = & \frac{xz}{x^2 + y^2} = g_1(x,y,z) \\
\frac{\partial f}{\partial y}(x,y,z) & = & \frac{yz}{x^2 + y^2} = g_2(x,y,z) \\
\frac{\partial f}{\partial z}(x,y,z) & = & \frac12 \log(x^2 + y^2) = g_3(x,y,z) \\
\end{eqnarray*}
(where $g_3 = h$ as defined in part (a). Furthermore,
$f(1,1,1) = \frac{1}{2} \ln(1^2 + 1^2) - \frac{\ln 2}{2} = 0$,
so that $f$ has the desired properties. \quad $\blacksquare$

\end{enumerate}
\end{question}

\begin{question}{Problem 4}
\begin{enumerate}[(a)]
\item {\bf Case 1:} ($N = M$) Let $L$ be the Lipschitz constant of
$\mathbf{g}$, and $R$ be a rectangle with $E \subseteq R$ (such a rectangle
exists because $E$ is Peano-Jordan Measurable). Since
$\operatorname{meas} E = 0$, $\forall \delta = \frac{\epsilon}{L^M}$, there exists a
partition $P$ of $R$ such that $U(\chi_E,P) < \delta$. Let $\epsilon > 0$, and
let $Q$ be a refinement of $P$ such that each rectangle $R$ in $Q$ has
diagonals of length no more than $\epsilon$, with rectangles
$Q_1,Q_2,\ldots,Q_l$. Since $Q$ is a refinement of $P$,
$U(\chi_E,Q) < \delta$. Since $\mathbf{g}$
is Lipschitz, $\forall \mathbf{x},\mathbf{y} \in R_i$,
$\|\mathbf{g}(\mathbf{x}) - \mathbf{g}(\mathbf{x})\|
                                              < L\|\mathbf{x} - \mathbf{y}\|,$
so that each $\mathbf{g}(Q_i)$ can be covered by a rectangle $C_i$ of diagonal
at most $L\epsilon$; let $S$ be a cover of $\mathbf{g}(E)$ of such rectangles.
Then, $U(\chi_{\mathbf{g}(E)},S) < L^MU(\chi_{\mathbf{g}(E)} < \epsilon$,
so that $\int_S \chi_{\mathbf{g}(E)} = 0$ and thus
$\operatorname{meas} \mathbf{g}(E) = 0$, concluding this case.

{\bf Case 2:} ($N < M$) Let \[E^{\prime}
         := \{(\mathbf{x},\mathbf{0}) \in \mathbb{R}^M : \mathbf{x} \in E\}.\]
If $\chi_E$ is integrable over $R$ over some rectangle $R$ with
$E \subseteq R$, then, $\forall k \in \mathbb{N}$, $\chi_{E^{\prime}}$ is
integrable over the rectangle
$R^{\prime} := R \times [0,0]^{M - N} \subseteq \mathbb{R}^M$.
In particular, since $\chi_{E^{\prime}}$ is bounded and
$\operatorname{meas}_{M} R^{\prime} = 0$, \[\int_{R^{\prime}} \chi_{E^{\prime}} = 0.\]

Let $\mathbf{f}: E^{\prime} \rightarrow \mathbb{R}^M$ such that,
$\forall \mathbf{y} = (\mathbf{x},\mathbf{0}) \in E^{\prime}$
($\mathbf{x} \in E$), $\mathbf{f}(\mathbf{y}) = \mathbf{g}(\mathbf{x})$. Then,
$\mathbf{f}$ is Lipschitz, and $E^{\prime}$ is Peano-Jordan Measurable with
measure zero, so that, by the result of Case 1 above, $\mathbf{f}(E^{\prime})$
is Peano-Jordan Measurable with measure zero. Therefore, since
$\mathbf{f}(E^{\prime}) = \mathbf{g}(E)$, $\mathbf{g}(E)$ is
Peano-Jordan Measurable with measure zero. \quad $\blacksquare$

\item Since $\mathbf{g}$ is of class $C^1$, its derivative is continuous.
Thus, since $\overline{E}$ is compact, (it is clearly closed, and it is
bounded because it is Peano-Jordan Measurable) and any continuous function on
a compact domain is bounded, the derivative of $\mathbf{g}$ is bounded on
$\overline{E}$. Therefore, $\mathbf{g}$ is Lipschitz over $\overline{E}$. By
the result of part (a), then, $\mathbf{g}(E)$ is Peano-Jordan measurable with
measure zero. \quad $\blacksquare$

\item
\begin{enumerate}[i.]
\item By the Inverse Function Theorem, $\forall \mathbf{x} \in E^{\circ}$,
$\exists r > 0$ such that $B(\mathbf{x},r)) \subseteq E^{\circ}$ and
$\mathbf{g}(B(\mathbf{x},r))$ is open. Thus, if $\mathbf{x} \in E^{\circ}$,
$\exists r_{\mathbf{x}} > 0$ such that $B(\mathbf{g}(\mathbf{x},r_\mathbf{x}))
 \subseteq \left(\mathbf{g}(E)\right)^{\circ}$. Therefore,
$\mathbf{g}(\mathbf{x}) \in \left(\mathbf{g}(E)\right)^{\circ}$, and so
$\mathbf{g}(E^{\circ}) \subseteq \left(\mathbf{g}(E)\right)^{\circ}$.
Since $\mathbf{g}$ is continuous,
$\overline{\mathbf{g}(E)} \subseteq \mathbf{g}(\overline{E})$.
Thus,
\[\partial\mathbf{g}(E)
 = \overline{\mathbf{g}(E)}\backslash (\mathbf{g}(E))^{\circ}
 \subseteq \overline{\mathbf{g}(E)}\backslash \mathbf{g}(E^{\circ})
 \subseteq \mathbf{g}(\overline{E})\backslash \mathbf{g}(E^{\circ})
 \subseteq \mathbf{g}(\partial E).\]

\item Since $E$ is Peano-Jordan measurable, by Theorem 177, $\partial E$ is
Peano-Jordan measurable with measure zero. Thus, by the result of part (b)
above, $\mathbf{g}(\partial E)$ is Peano-Jordan measurable with measure zero.
By the result of part i. above and Theorem 167,
\[\operatorname{meas} \partial \mathbf{g}(E)
  \leq \operatorname{meas} \mathbf{g}(\partial E) = 0,\]
so that $\operatorname{meas} \partial \mathbf{g}(E) = 0$. Therefore, by
Theorem 177, $\mathbf{g}(E)$ is Peano-Jordan measurable. \quad $\blacksquare$
\end{enumerate}
\end{enumerate}
\end{question}
\end{document}
