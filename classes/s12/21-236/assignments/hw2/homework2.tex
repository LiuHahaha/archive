\documentclass[11pt]{article}
\usepackage{enumerate}
\usepackage{fullpage}
\usepackage{fancyhdr}
\usepackage{amsmath, amsfonts, amsthm, amssymb}
\setlength{\parindent}{0pt}
\setlength{\parskip}{5pt plus 1pt}
\pagestyle{empty}

\def\indented#1{\list{}{}\item[]}
\let\indented=\endlist

\newcounter{questionCounter}
\newcounter{partCounter}[questionCounter]
\newenvironment{question}[2][\arabic{questionCounter}]{%
    \setcounter{partCounter}{0}%
    \vspace{.25in} \hrule \vspace{0.5em}%
        \noindent{\bf #2}%
    \vspace{0.8em} \hrule \vspace{.10in}%
    \addtocounter{questionCounter}{1}%
}{}
\renewenvironment{part}[1][\alph{partCounter}]{%
    \addtocounter{partCounter}{1}%
    \vspace{.10in}%
    \begin{indented}%
       {\bf (#1)} %
}{\end{indented}}

%%%%%%%%%%%%%%%%%%%%%%%%%%%%%%%%%%%%%%%%%%%%%%%%%%%%%%%%%%%
\newcommand{\myname}{Shashank Singh}
\newcommand{\myandrew}{sss1@andrew.cmu.edu}
\newcommand{\myclass}{21-236 Mathematical Studies Analysis II}
\newcommand{\myhwnum}{2}
\newcommand{\duedate}{Monday, February 6, 2012}
%%%%%%%%%%%%%%%%%%%%%%%%%%%%%%%%%%%%%%%%%%%%%%%%%%%%%%%%%%%

\begin{document}
\thispagestyle{plain}

{\Large Homework \myhwnum} \\
\myclass \\
Name: \myname \\
Email: \myandrew \\
Due: \duedate

\begin{question}{Problem 1}
\begin{enumerate}[(a)]
\item Let $\mathbf{x} \in \mathbf{R}^N$. Let $d = f(\mathbf{X}) + 1$, and let
$D = \overline{B(\mathbf{x}, d)} \cap C$. Since $D$ is the intersection of
two closed sets, $D$ is closed, and, since $\overline{B(\mathbf{x}, d)}$ is
bounded, $D$ is bounded, so that $D$ is compact. Let
$g: D \rightarrow \mathbb{R}$ such that, $\forall \mathbf{y} \in D$,
$g(\mathbf{y}) = \|\mathbf{x} - \mathbf{y}\|$. Since $g$ is a continuous
function on a compact set, by the Weierstrass Theorem,
$g$ attains a minimum on $D$; in particular,
$\exists \mathbf{y}_{\mathbf{x}} \in D$ such that
$g(\mathbf{y}_{\mathbf{x}}) = \min g(D)$. By definition of infimum, since
$g(\mathbf{y}_{\mathbf{x}})$ is a lower bound of
$\{\|\mathbf{x} - \mathbf{y}\| : \mathbf{y} \in C\}$. Since
$\mathbf{y}_{\mathbf{x}} \in C$,
$f(\mathbf{x}) = \|\mathbf{x} - \mathbf{y}_{\mathbf{x}}\|$.
\qquad $\blacksquare$

\item Let $\mathbf{y} \in C$. Since $\mathbf{z}$ is in the line segment with endpoints
$\mathbf{x}$ and $\mathbf{y}_{\mathbf{x}}$,
\[\|\mathbf{y}_{\mathbf{x}} - \mathbf{z}\|
 = \|\mathbf{y}_{\mathbf{x}} - \mathbf{x}\| - \|\mathbf{x} - \mathbf{z}\|.\]
By choice of $\mathbf{y}_{\mathbf{x}}$ as
$\inf\{\|\mathbf{x} - \mathbf{y} : \mathbf{y} \in C\|\}$,
\[\|\mathbf{y}_{\mathbf{x}} - \mathbf{x}\| \leq \|\mathbf{y} - \mathbf{x}\|.\]
Thus, by the Reverse Triangle Inequality,
\[\|\mathbf{y}_{\mathbf{x}} - \mathbf{z}\|
 \leq \|\mathbf{y} - \mathbf{x}\| - \|\mathbf{x} - \mathbf{z}\|
 \leq \|\mathbf{y} - \mathbf{z}\|.\] Therefore, since
$\|\mathbf{y}_{\mathbf{x}} - \mathbf{z}\|$ is a lower bound of
$\{\|\mathbf{z} - \mathbf{y}\| : \mathbf{y} \in C\}$ and
$\mathbf{y}_{\mathbf{x}} \in C$, by definition of infimum,
$f(\mathbf{z}) = \|\mathbf{y}_{\mathbf{x}}\|$.
\qquad $\blacksquare$

\item Let $\mathbf{x}_1, \mathbf{x}_2 \in \mathbb{R}^N$. As shown in part (a),
$\exists \mathbf{y}_1, \mathbf{y}_2 \in C$ such that
$f(\mathbf{x}_1) = \|\mathbf{x}_1 - \mathbf{y}_1\|,
f(\mathbf{x}_2) = \|\mathbf{x}_2 - \mathbf{y}_2\|$. By the Reverse Triangle
Inequality, and the choice of $\mathbf{y}_1$,
\[\|\mathbf{x}_1 - \mathbf{x}_2\| \geq
\|\mathbf{x}_1 - \mathbf{y}_2\| - \|\mathbf{y}_2 - \mathbf{x}_2\|
 \geq \|\mathbf{x}_1 - \mathbf{y}_1\| - \|\mathbf{y}_2 - \mathbf{x}_2\|
 = f(\mathbf{x}_1) - f(\mathbf{x}_2).\]
An identical proof shows that $\|\mathbf{x}_1 - \mathbf{x}_2\| \geq
f(\mathbf{x}_2) - f(\mathbf{x}_1)$, so that
$\|\mathbf{x}_1 - \mathbf{x}_2\| \geq |f(\mathbf{x}_1) - f(\mathbf{x}_2)|$,
so that $f$ is Lipschitz continuous with Lipschitz constant at most $1$.
If $f$ is differentiable at some $\mathbf{x} \in \mathbb{R}^N$, then
\[||\nabla f(\mathbf{x})|| \leq \lim_{\mathbf{y} \rightarrow \mathbf{x}}
\frac{|f(\mathbf{x}) - f(\mathbf{y})|}{\|\mathbf{x} - \mathbf{y}\|} \leq 1.
\qquad \blacksquare\]

\item Suppose $f$ is differentiable in $\mathbb{R}^N \backslash C$.
Since, $\forall$ directions $\mathbf{v} \in \mathbb{R}^N$,
$\forall \mathbf{x} \in \mathbb{R}^N$,
$\left|\frac{\partial f}{\partial \mathbf{v}} (\mathbf{x})\right|
 = \|\nabla f (\mathbf{x})\| \|\mathbf{v}\|$, it follows from the equality case
of the Cauchy-Schwarz Inequality that the gradient of $f$ is in the direction
in which the directional derivative of $f$ is maximised.
For $\mathbf{v} = \mathbf{y}_{\mathbf{x}} - \mathbf{x}$, by the Reverse
Triangle Inequality
\[\frac{\partial f}{\partial \mathbf{v}} (\mathbf{x})
 = \lim_{t \rightarrow 0} \left(\frac{f(\mathbf{x} + t\mathbf{v}) - f(\mathbf{v})}{t}\right)
 \leq \lim_{t \rightarrow 0} \left(\frac{\|t\mathbf{v}\|}{t}\right) = \|\mathbf{v}\| = 1.\]
Since, as shown in part (c), $\| \nabla f (\mathbf{x}) \| \leq 1$,
$\|\nabla f(\mathbf{x})\| = 1$. \qquad $\blacksquare$

\item Let $\mathbf{x} \in \mathbb{R}^N \backslash C$, and let
$\mathbf{y},\mathbf{z} \in C$ such that
$f(\mathbf{x}) = \|\mathbf{x} - \mathbf{y}\| = \|\mathbf{x} - \mathbf{z}\|$.
Suppose, for sake of contradiction, that $f$ were differentiable at
$\mathbf{x}$. Then, the gradient of $f$ is well-defined, and, furthermore, as
shown in the solution to part (d) above, \[\nabla f(\mathbf{x})
 = \frac{\mathbf{y} - \mathbf{x}}{\|\mathbf{y} - \mathbf{x}\|}
 = \frac{\mathbf{z} - \mathbf{x}}{\|\mathbf{z} - \mathbf{x}\|}.\]
Then,
$(\mathbf{y} - \mathbf{x})$ and $(\mathbf{z} - \mathbf{x})$ must be
parallel, so that there must exists a line segment containing $\mathbf{y}$,
$\mathbf{z}$, and $\mathbf{x}$. Since
 $\|\mathbf{y} - \mathbf{x}\| = \|\mathbf{z} - \mathbf{x}\|$,
$\mathbf{y} = \mathbf{z}$, contradicting the given that $\mathbf{y}$ and
$\mathbf{z}$ are distinct.

\item Let $N = 1$, let $C = \{-1, 1\} \subseteq \mathbb{R}^N$. Then, by the
result of part (e), $f$ is not differentiable at $0$. \qquad $\blacksquare$

\end{enumerate}
\end{question}

\begin{question}{Problem 2}
Let $E \subseteq \mathbb{R}^N$. Then, $E$ has property (a) if and only if $E$
has property (b).

Since, if $U$ and $V$ are disjoint sets, then $E \cap U \cap V = \emptyset$,
if $E$ has property (a), then $E$ has property (b).

Suppose, on the other hand, that $E$ has propety (b), so that there exist two
open sets  $U$ and $V$ such that
\[E \subseteq U \cup V, \; E \cap U \neq \emptyset \neq E \cap V, \mbox{ and }
E \cap U \cap V = \emptyset.\]
Let $U^{\prime} = U \cap E$, and let $V^{\prime} = V \cap E$. Since
$E \subseteq U \cup V$, $E \subseteq U^{\prime} \cup V^{\prime}$.
Let $f: \mathbb{R}^N \rightarrow [0,\infty)$ be the distance function from
$V^{\prime}$, as defined in problem 1. Since $U$ is open, $\forall
\mathbf{x} \in U^{\prime}$, $\exists r_{\mathbf{x}} > 0$ such that
$B(\mathbf{x},r_{\mathbf{x}}) \subseteq U$. Suppose, for sake of
contradiction, that, for some $\mathbf{x}_0 \in U^{\prime}$, $f(x) = 0$. Then,
$\exists \mathbf{y} \in V_{\prime}$ such that $\|\mathbf{y} - \mathbf{x}\|
 < r_{\mathbf{x}}$. However, this would imply that
$\mathbf{y} \in U \cap V_{\prime} = U \cap E \cap V$, contradicting the given.
Thus, $\forall \mathbf{x} \in U_{\prime}$, $f(\mathbf{x}) > 0$. A similar
proof by contradiction shows that, letting
$g: \mathbb{R}^N \rightarrow [0,\infty)$ be the distance function from
$U^{\prime}$, $\forall \mathbf{y} \in V_{\prime}$, $g(\mathbf{y}) > 0$. Let
$U^* = \cup_{\mathbf{x} \in U^{\prime}} B(\mathbf{x}, f(\mathbf{x}/4)$, and let
$V^* = \cup_{\mathbf{y} \in V^{\prime}} B(\mathbf{y}, g(\mathbf{y}/4)$. By
construction, $U^* \cap V^* = \emptyset$. Since a union of open sets is open,
$U^*$ and $V^*$ are open. Since $U^{\prime} \subseteq U^*$ and
$V^{\prime} \subseteq V^*$, $E \subseteq U^* \cup V^*$. Thus, $E$ has property
(a). \qquad $\blacksquare$
\end{question}

\begin{question}{Problem 3}
\begin{enumerate}[(a)]
\item Let $\epsilon > 0$, and let $\epsilon_2 = \frac{\epsilon}{b - a}$. Since
$g$ is continuous, $\exists \delta > 0$ such that,
$\forall f,h \in C^1([a,b])$ with
$\|(x,f(x),f^{\prime}) - (x,h(x),h^{\prime}(x))\| < \delta$,
$|g(x,f(x),f^{\prime}(x)) - g(x,f(x),f^{\prime}(x)| < \epsilon_2$.

Thus, $\forall f, h \in C^1([a,b])$ with
$\|(x,f(x),f^{\prime}) - (x,h(x),h^{\prime}(x))\| < \delta$, by the Triangle
Inequality,
\begin{eqnarray*}
|G(f) - G(h)|
 & = & \left|\int_a^b g(x,f(x),f^{\prime}) - g(x,h(x),h^{\prime}(x)\right| \; dx \\
 & \leq & \int_a^b \left|g(x,f(x),f^{\prime}) - g(x,h(x),h^{\prime}(x)\right| \; dx \\
 & < & \int \epsilon_2 dx = \epsilon_2(b - a) = \epsilon.
\end{eqnarray*}
Thus, $G$ is continuous. \qquad $\blacksquare$

\item

\item Suppose $f_0$ is such that \[\min_{f \in X} G(f) = G(f_0).\]
Let $v \in C^1([a,b])$ with $v(a) = v(b) = 0$. Consider the function
$h : \mathbb{R} \rightarrow \mathbb{R}$ such that, $\forall t \in \mathbb{R}$,
$h(t) = G(f + tv)$. Clearly, $h$ must have a local minimum at $t = 0$, since
$G$ has a local minimum at $f$. By Theorem 282 (in the notes for Real Analysis
I), $\frac{\partial G}{\partial v} (f) = \frac{dh}{dt} (0) = 0$.
\qquad $\blacksquare$

\item Let $h \in C([a,b])$ such that, $\forall v \in C^1([a,b])$ with
$v(a) = v(b) = 0$, \[\int_a^b h(x) v(x) \; dx = 0.\] Suppose, for sake of
contradiction, that $h \neq 0$. Without loss of generality, for some
$x_0 \in [a,b]$, $h(x_0) > 0$ (otherwise, we take $(-h)$). Since $h$ is
continuous, for some $\delta > 0$, letting
$c = \max \{x_0 - \delta, a\}, d = \min \{x_0 + \delta, b\}$.
$\forall x \in [c,d]$, $|f(x) - f(x_0)| < f(x_0)$, so that $f(x) > 0$.
Let $w = d - c$.
Define $v: [a,b] \rightarrow \mathbb{R}$ piecewise as follows:

\[
   f(x) = \left\{
     \begin{array}{lcr}
       0 & : & x \in [a,c] \cup [d,b] \\
       (w(x - c))^2  & : & x \in (c,c + \frac{w}{4}] \\
       \frac{1}{2} - (w(x - (c + 1)))^2  &
                         : & x \in (c + \frac{w}{4},c + \frac{3w}{4}] \\
       (w(x - (c + 2)))^2  & : & x \in (c + \frac{3w}{4}, d)
       
     \end{array}
   \right.
\]
Then, $v \in C^2([a,b])$, $\forall x \in (c,d), v(x) > 0$, and,
$\forall x \in [a,c] \cup [d,b]$, $v(x) = 0$. Thus,
$\forall x \in (c,d), h(x)v(x) > 0$, and,
$\forall x \in [a,c] \cup [d,b]$, $h(x)v(x) = 0$. Then, however,
\[\int_a^b h(x)v(x) \; dx > 0,\] which is a contradiction.
Therefore, $h = 0$. \qquad $\blacksquare$

\end{enumerate}
\end{question}

\begin{question}{Problem 4}
\begin{enumerate}[(a)]
\item $\min_{f \in X} G(f) =$ \fbox{$L^2 (b - a)$.} In particular, $G$ is minimized by
$f_0 := \left(x \mapsto L \frac{x - a}{b - a}\right)$. For suppose $f \in X$.
Then, letting $F = f - f_0$,
\[G(f) = G(F + f_0) = \int_a^b \left((F + f_0)^{\prime}\right)^2
 = \int_a^b \left(F^{\prime}\right)^2
 + 2F^{\prime}f_0^{\prime}
 + \left(f_0^{\prime}\right)^2.\]

By the Fundamental Theorem of Calculus, since $f_0^{\prime}(x)$ is a constant
with respect to $x$ and $f(a) = f_0(a) = 0$ and $f(b) = f_0(b) = L$, so that
$F(a) = F(b) = 0$,
\[\int_a^b 2F^{\prime}(x)f_0^{\prime}(x) \; dx
 = 2f_0^{\prime}(x)\int_a^b F^{\prime}(x) \; dx
 = 2f_0^{\prime}(x) \left(F(b) - F(a)\right)
 = 2f_0^{\prime}(x) (0) = 0.\]
Thus, since $f_0^{\prime}(x)$ is a constant with respect to $x$,
\[G(f) = \int_a^b \left(F^{\prime}\right)^2 + C,\] for some constant, so that
$G$ is minimized when $F = 0$, and thus $f = f_0$. \qquad $\blacksquare$

\item $\forall f \in C^1([a,b])$ and $\forall$ directions $v \in C^1([a,b])$ with $v(0) = v(1) = 0$,
integrating by parts gives \[G(f) = \int_0^1 \left(f^{\prime}(x)\right)^2 - x^2f^{\prime}(x) \; dx.\]
By the result of Exercise 3, part (c), if $f$ minimizes $G$, then
\[
\int_0^1 \left(2f^{\prime}(x)- x^2\right) v^{\prime}(x) \; dx
 = \int_0^1\frac{\partial}{\partial f^{\prime}} \left(\left(f^{\prime}(x)\right)^2 - x^2f^{\prime}(x)\right) \; dx
 = \frac{\partial G}{\partial v} (f)
 = 0.\]
Since $v(0) = v(1) = 0$, integrating by parts again shows that
\[\int_0^1 \left(f^{\prime \prime} - 2x\right) v(x) \; dx = 0,\] so that, by
the result of Exercise 3, part (d), $f^{\prime \prime} (x) = 2x$.

Integrating with respect to $x$ shows that, for $f \in C^1([a,b])$ such that,
$\forall x \in [a,b]$, $f(x) =$ \fbox{$\frac{1}{3}\left(x^3 - x\right)$,} $f$ is the unique
function satisfying the constraints $f^{\prime \prime} (x) = 2x$ and
$f(0) = f(1) = 0$. Thus, $G$ is minimized at $f$. \qquad $\blacksquare$
\end{enumerate}
\end{question}
\end{document}
