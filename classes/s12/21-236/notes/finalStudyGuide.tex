\documentclass[11pt]{article}
\usepackage{enumerate}
\usepackage{fullpage}
\usepackage{fancyhdr}
\usepackage{amsmath, amsfonts, amsthm, amssymb}
\setlength{\parindent}{0pt}
\setlength{\parskip}{5pt plus 1pt}
\pagestyle{empty}

\def\indented#1{\list{}{}\item[]}
\let\indented=\endlist

\newcounter{questionCounter}
\newcounter{partCounter}[questionCounter]
\newenvironment{question}[2][\arabic{questionCounter}]{%
    \setcounter{partCounter}{0}%
    \vspace{.25in} \hrule \vspace{0.5em}%
        \noindent{\bf #2}%
    \vspace{0.8em} \hrule \vspace{.10in}%
    \addtocounter{questionCounter}{1}%
}{}
\renewenvironment{part}[1][\alph{partCounter}]{%
    \addtocounter{partCounter}{1}%
    \vspace{.10in}%
    \begin{indented}%
       {\bf (#1)} %
}{\end{indented}}

%%%%%%%%%%%%%%%%%%%%%%%HEADER%%%%%%%%%%%%%%%%%%%%%%%%%%%%%%
\newcommand{\myname}{Shashank Singh}
\newcommand{\myandrew}{sss1@andrew.cmu.edu}
\newcommand{\myclass}{21-236 Mathematical Studies Analysis II}
\newcommand{\duedate}{Tuesday, May 8, 2012}
%%%%%%%%%%%%%%%%%%%%%%%%%%%%%%%%%%%%%%%%%%%%%%%%%%%%%%%%%%%

%%%%%%%%%%%%%%%%%%%%%%%%MACROS%%%%%%%%%%%%%%%%%%%%%%%%%%%%%
\renewcommand{\qed}{\quad $\blacksquare$}
\newcommand{\mqed}{\quad \blacksquare}
\newcommand{\bvarphi}{\boldsymbol{\varphi}}
\newcommand{\bpsi}{\boldsymbol{\psi}}
\newcommand{\balpha}{\boldsymbol{\alpha}}
\newcommand{\Var}{\operatorname{Var}}
\newcommand{\meas}{\operatorname{meas}}
\newcommand{\diam}{\operatorname{diam}}
\newcommand{\dist}{\operatorname{dist}}
\newcommand{\rank}{\operatorname{rank}}
\renewcommand{\div}{\operatorname{div}}
\newcommand{\curl}{\operatorname{curl}}
\newcommand{\bzero}{\mathbf{0}}
\newcommand{\ba}{\mathbf{a}}
\newcommand{\bb}{\mathbf{b}}
\newcommand{\be}{\mathbf{e}}
\newcommand{\bv}{\mathbf{v}}
\newcommand{\bw}{\mathbf{w}}
\newcommand{\bx}{\mathbf{x}}
\newcommand{\by}{\mathbf{y}}
\newcommand{\bz}{\mathbf{z}}
\newcommand{\bff}{\mathbf{f}}
\newcommand{\bfg}{\mathbf{g}}
\newcommand{\bfh}{\mathbf{h}}
\newcommand{\pjm}{Peano-Jordan measurable }
\newcommand{\Rint}{Riemann integrable }
%%%%%%%%%%%%%%%%%%%%%%%%%%%%%%%%%%%%%%%%%%%%%%%%%%%%%%%%%%%

\begin{document}
\thispagestyle{plain}

{\Large Final Study Guide} \\
\myclass \\
Name: \myname \\
Email: \myandrew \\
Exam Date: \duedate \\

\hrule
{\Large \bf Integration}
\vspace{1mm}
\hrule
\begin{enumerate}
\item \textbf{Theorem 160 (Repeated Integration)}
Let $S \subseteq \mathbb{R}^N$, $T \subseteq \mathbb{R}^M$ be rectangles, let
$f: S \times T \rightarrow \mathbb{R}$ be Riemann integrable, and suppose
that, $\forall \bx \in S$, $\by \in T \mapsto f(\bx,\by)$ is Riemann
Integrable. Then, the function $\bx \in S \mapsto \int_T f(\bx,\by) \; d\by$
is \Rint and
\[\int_{S \times T} f(\bx,\by) \; d(\bx,\by)
 = \int_S\left(\int_Tf(\bx,\by)\;d\by\right)\;d\bx.\]

{\bf Proof:} Construct a partition $\mathcal{P} = \{R_1,\ldots,R_k\}$ (by
refinement of partitions $\mathcal{P}$ and $\mathcal{Q}$) of $R := S \times T$
such that $R_i = S_i \times T_j$, where $\mathcal{P}_N = \{S_1,\ldots,S_m\}$
and $\mathcal{P}_M = \{T_i,\ldots,T_l\}$ are partitions of $S$ and $T$,
respectively. Using the fact that
\[\meas_{N + M} R_k
 = \meas_{N + M} (S_i \times T_j)
 = \meas_N S_i \meas_M T_j,\]
\[L(f,\mathcal{P})
 \leq \underline{\int_S} \left(\int_T f(\bx,\by) \; d\by\right) \; d\bx
 \leq \overline{\int_S} \left(\int_T f(\bx,\by) \; d\by\right) \; d\bx
 \leq U(f,\mathcal{Q}).
\]
Taking the supremum over all partitions $\mathcal{P}$ of $R$ gives
\[\underline{\int_{S \times T}} f(\bx,\by) \; d(\bx,\by)
 \leq \underline{\int_S} \left(\int_T f(\bx,\by) \; d\by\right) \; d\bx.
 \leq \overline{\int_S} \left(\int_T f(\bx,\by) \; d\by\right) \; d\bx
 \leq U(f,\mathcal{Q}).
\]
Taking the infimum over all partitions $\mathcal{Q}$ of $R$ gives
\[\underline{\int_{S \times T}} f(\bx,\by) \; d(\bx,\by)
 \leq \underline{\int_S} \left(\int_T f(\bx,\by) \; d\by\right) \; d\bx.
 \leq \overline{\int_S} \left(\int_T f(\bx,\by) \; d\by\right) \; d\bx
 \leq \overline{\int_{S \times T}} f(\bx,\by) \; d(\bx,\by).
\]
Therefore, since $f$ is Riemann integrable,
\[\int_{S \times T} f(\bx,\by) \; d(\bx,\by)
 = \underline{\int_S}\left(\int_T f(\bx,\by)\; d\by\right) \; d\bx
 = \overline{\int_S}\left(\int_T f(\bx,\by)\; d\by\right) \; d\bx. \mqed
\]
\end{enumerate}

\hrule
{\Large \bf Peano-Jordan Measure}
\vspace{1mm}
\hrule
\begin{enumerate}
\item \textbf{Theorem 177 (Peano-Jordan Measurability of a Set and its
                                                                    Boundary)}
A bounded set $E \subset \mathbb{R}^N$ is \pjm if and only if $\partial E$ is
\pjm and $\meas \partial E = 0$.

{\bf Proof:} Note that, if $P$ is a pluri-rectangle, then
$\meas \partial P = 0$, so that
$\meas P = \meas \overline{P} = \meas P^{\circ}$.

{\bf Step 1:} Suppose $E$ is \pjm and let $R$ be a rectangle containing $E$.
Then, $\forall \epsilon > 0$, there exist pluri-rectangles $P_1$ and $P_2$
with $P_1 \subseteq E \subseteq P_2$ such
\[0 \leq \meas P_2 - \meas P_1 \leq \epsilon.\] Then,
\[\meas (\overline{P_2}\backslash P_2^{\circ})
 = \meas \overline{P_2} - \meas P_1^{\circ}
 = \meas P_2 - \meas P_1
 \leq \epsilon.\]
Since $\overline{P_2}\backslash P_1^{\circ}$ is also a pluri-rectangle and
$P_1^{\circ} E^{\circ} \subseteq \overline{E} \subseteq \overline{P_2}$,
\[\partial E
 = \overline{E}\backslash E^{\circ}
 \subseteq \overline{P_2}\backslash P^{\circ}.\]
Thus,
\begin{eqnarray*}
0 & \leq & \sup\{\meas P: P\mbox{ pluri-rectangle, }\partial E\supseteq P\} \\
  & \leq & \inf\{\meas P: P\mbox{ pluri-rectangle, }\partial E\subseteq P\} \\
  & \leq & \epsilon,
\end{eqnarray*}
so that, letting $\epsilon \rightarrow 0^+$,
\begin{eqnarray*}
0 & = & \sup\{\meas P : P\mbox{ pluri-rectangle, }\partial E\supseteq P\} \\
  & = & \inf\{\meas P : P\mbox{ pluri-rectangle, }\partial E\subseteq P\}
    = 0
\end{eqnarray*}
Therefore, $\partial E$ is \pjm with $\meas E = 0$. \qed

{\bf Step 2:} Suppose $\partial E \subset \mathbb{R}^N$ is \pjm with
$\meas \partial E = 0$. Since $E$ is bounded, there exists a rectangle $R$
containing $\overline{E}$. Since $\meas \partial E = 0$, for $\epsilon > 0$,
there exists a pluri-rectangle $P$ containing $\partial E$ with
$\meas P \leq \epsilon$. Then, $R \backslash P$ is also a pluri-rectangle, so
that we can decompose it into a disjoint union of rectangles $R_1,\ldots,R_k$.
Let \[P_1 = \cup \{R_i : R_i \subseteq E\},\] and let $P_2 = P \cup P_1$, so
that $E \subseteq P_2$. Thus,
\[0 \leq \meas P_2 - \meas P_1 \leq \epsilon,\] so that $E$ is Peano-Jordan
measurable. \qed

\item \textbf{Theorem 178 (Riemann Integrability and Peano-Jordan
                                                               Measurability)}
Let $R \subseteq \mathbb{R}^N$ be a rectangle and let
$f: R \rightarrow [0,\infty)$ be a bounded function. Then, $f$ is \Rint
over $R$ if and only if the set
\[S_f := \{(\bx,y) \in R \times [0,\infty) : 0 \leq y \leq f(\bx)\}.\]
is \pjm in $\mathbb{R}^{N + 1}$, in which case
\[\meas_{N + 1} S_f = \int_R f(\bx) \; d\bx.\]

{\bf Proof:} Assume $f$ is \Rint over $R$. Then, For $\epsilon > 0$, there
exists a partition $\mathcal{P} = \{R_1,\ldots,R_k\}$ such that
\[0 \leq U(f,\mathcal{P}) - L(f,\mathcal{P}) \leq \epsilon.\]
Define
\[T_i := R_i \times \left[0,\inf_{R_i} f\right],
U_i := R_i \times \left[0,\sup_{R_i} f\right],\]
and
\[P_1 := \cup_{i = 1}^k T_i, \quad P_2 := \cup_{i = 1}^k U_i.\]
Then, $P_1$ and $P_2$ are pluri-rectangles, $P_1 \subseteq S_f \subseteq P_2$,
and
\begin{eqnarray*}
\meas_{N + 1} P_2 - \meas P_1
 & = & \sum_{i = 1}^k \meas_{N + 1} U_i - \sum_{i = 1}^k T_{N + 1} \\
 & = & \sum_{i = 1}^k\left(\sup_{R_i} f - 0\right)\meas_NR_i
   -   \sum_{i = 1}^k\left(\inf_{R_i} f - 0\right)\meas_NR_i \\
 & = & U(f,\mathcal{P}) - L(f,\mathcal{P}) \leq \epsilon.
\end{eqnarray*}
Thus, $S_f$ is \pjm with
\[\meas_{N + 1} S_f = \int_R f(\bx) \; d\bx.\]
Suppose, conversely, that $S_f$ is \pjm and let $R \times [0,M]$ be a
rectangle containing $S_f$.

\item \textbf{Corollary 182 (Riemann Integration over the Region Between Two
                                                                   Functions)}
Let $R \subset \mathbb{R}^N$ be a rectangle, let
$\alpha,\beta: R \rightarrow \mathbb{R}$ be \Rint with
$\alpha \leq \beta$, and let $f: E \rightarrow \mathbb{R}$ be a bounded,
continuous function. Then, $f$ is \Rint over $E$, and
\[\int_E f(\bx,y) \; d(\bx,y)
 = \int_R \left(\int_{\alpha(\bx)}^{\beta(\bx)} f(\bx,y) \; dy\right) \; d\bx
.\]

{\bf Proof:} Consider a rectangle $R \times [a,b]$ containing $E$ and let
\[g(\bx,y) :=
    \left\{
        \begin{array}{cc}
            f(\bx,y) & (\bx,y) \in E \\
            0        & (\bx,y) \in (R \times [a,b]) \backslash E
        \end{array}
    \right
.\]
Let $c \leq \alpha(\bx)$ in $R$. Then, for
\[T_{\beta}
 := \{(\bx,y) \in R \times [c,\infty) : c \leq y \leq \beta(\bx)\},\]
\[T_{\alpha}
 := \{(\bx,y( \in R \times [c,\infty) : c \leq y \leq \alpha(\bx),\}\]
if $E := T_{\beta} \backslash T_{\alpha}$,
$\partial E \subseteq \partial T_{\beta} \cup \partial T_{\alpha}$, so that,
since $\meas_{N + 1} T_{\beta} = \meas_{N + 1} T_{\alpha} = 0$,
$\meas_{N + 1} \partial E = 0$. Therefore, since $g$ is continuous in $E$, the
set of discontinuity points of $g$ (which must be in $\partial E$) has measure
zero, so that $g$ is \Rint in $R \times [a,b]$.

$\forall \bx \in R$, the function $y \in [a,b] \mapsto g(\bx,y)$ is \Rint
since it is continuous except at most at the points $y = \alpha(\bx)$ and
$y = \beta(\bx)$. Then, by Theorem 160 (Repeated Integration), the function
$\bx \in R \mapsto \int_{[a,b]} g(\bx,y) \; dy$ is \Rint and
\[\int_{R \times [a,b]}g(\bx,y) \; d(\bx,y)
 = \int_R \left(\int_{[a,b]} g(\bx,y)\; dy\right)\; d\bx
 = \int_R \left(\int_{\alpha(\bx)}^{\beta(\bx)} f(\bx,y)\; dy\right)\; d\bx.\]

\end{enumerate}
\hrule
{\Large \bf Improper Integrals}
\vspace{1mm}
\hrule
\begin{enumerate}
\item \textbf{Theorem 191 (Improper Riemann Integrability of Non-negative
                                                                   Functions)}
Let $E \subseteq \mathbb{R}^N$, and let $f: E \rightarrow [0,\infty)$. If
there exists \emph{one} exhausting sequence $\{E_n\}$ such that $f$ is \Rint
over each $E_n$ and
\[\lim_{n \rightarrow \infty} \int_{E_n} f(\bx) \; d\bx.\]

{\bf Proof:}

\item \textbf{Theorem 193 (Comparison for Improper Riemann Integrals)}
Let $E \subseteq \mathbb{R}^N$ and let $f,g: E \rightarrow \mathbb{R}$, and
assume $f$ is \Rint over every subset $F \subseteq E$ where $g$ is integrable.
Then,
\begin{enumerate}
\item If $|f| \leq g$ and $g$ is \Rint in the improper sense over $E$ with
$\int_E g(\bx) \; d\bx < \infty$, then $f$ and $|f|$ are \Rint in the improper
sense over $E$, and
\[\left|\int_E f(\bx) \; d\bx\right|
 \leq \int_E |f(\bx)| \; d\bx
 \leq \int_E g(\bx) \; d\bx.
\]

\item If $f \geq g \geq 0$ and $g$ is \Rint in the improper sense over $E$
with $\int_E g(\bx) \; d\bx = \infty$, then $f$ is \Rint in the improper sense
over $E$ and \[\int_E f(\bx) \; d\bx = \infty.\]
\end{enumerate}

\item \textbf{Theorem 197 (Improper Riemann Integrals of $f$ and $|f|$)}
Let $E \subseteq \mathbb{R}^N$ and $f: E \rightarrow \mathbb{R}$ be \Rint in
the improper sense Riemann integral. Then $|f|$ is \Rint in the improper sense
over $E$ and, furthermore,
\[\int_E |f| \; d\bx < \infty.\]
\end{enumerate}

\hrule
{\Large \bf Change of Variables}
\vspace{1mm}
\hrule
\begin{enumerate}
\item \textbf{Theorem 204 (Change of Variables)}
Let $U \subseteq \mathbb{R}^N$ be an open set, and let
$\bfg: U \rightarrow \mathbb{R}^N$ be injective and of class $C^1$, such that
$\det J_{\bfg} (\bx) \neq 0$ in $U$. Let $E \subset \mathbb{R}^N$ be \pjm with
$\overline{E} \subseteq U$ and let $f: \bfg(E) \rightarrow \mathbb{R}$ be
Riemann integrable. Then, the function
$\bx \in E \mapsto f(\bfg(\bx))|\det J_{\bfg}(\bx)|$ is \Rint and,
furthermore,
\[\int_{\bfg(E)} f(\by) \; d\by
 = \int_E f(\bfg(\bx))|\det J_{\bfg}(\bx)| \; d\bx
.\]

\item \textbf{Lemma 207 (One-Dimensional Change of Variables)}
Let $g: [a,b] \rightarrow \mathbb{R}$ be an injective, differentiable function
continuous derivative $g^{\prime}$ such that $g^{\prime}(x) \neq 0$ in
$[a,b]$, and let $f: g([a,b]) \rightarrow \mathbb{R}$ be bounded. Then,
\[\overline{\int_{g([a,b])}} f(y) \; dy
 = \overline{\int_a^b} f(g(x))g^{\prime}(x)| \; dx
.\]

\end{enumerate}

\hrule
{\Large \bf Spherical Coordinates in $\mathbb{R}^N$}
\vspace{1mm}
\hrule
\begin{enumerate}
\item \textbf{Lemma 209 (Measure of the $n$-Dimensional Unit Ball)}
Let $N \geq 1$. Then, $\forall \bx_0 \in \mathbb{R}^N$,
\[\meas B_N(\bx_0, r) = \frac{\pi^{N/2}}{\Gamma(1 + N/2)}r^N.\]

\item \textbf{Corollary 211 (Convenient Criteria for finite Riemann
                                                                   Integrals)}
Let $E \subseteq \mathbb{R}^N$, let $f: E \rightarrow \mathbb{R}$ be
continuous, and let $\bx_0 \in \mathbb{R}^N \backslash E$. Then, if there
exist $a,C > 0$ such that, in $E$
\[|f(\bx)| \leq \frac{C}{\|\bx - \bx_0\|^a},\] then
\begin{enumerate}[(i)]
\item If $E$ is \pjm and $a < N$, then $f$ is \Rint in the improper sense over
$E$ with finite improper Riemann integral.

\item If $E \subseteq \mathbb{R}^N \backslash B(\bx_0,r)$ admits an
exhausting sequence and $a > N$, then $f$ is \Rint in the improper sense over
$E$ with finite improper Riemann integral.
\end{enumerate}

{\bf Proof:} Since $f$ is continuous and
\[|f(\bx)| \leq g(\bx) := \frac{C}{\|\bx - \bx_0\|^a},\] $f$ is bounded
wherever $g$ is bounded, and therefore $f$ is \Rint on any $F \subseteq E$
such that $g$ is \Rint over $F$. The desired consequences then follow from
Theorem 193 (Comparisson for Improper Riemann Integrals). \qed

\item \textbf{Corollary 212 (Convenient Criteria for infinite Riemann
                                                                   Integrals)}
Let $E \subseteq \mathbb{R}^N$, let $f: E \rightarrow \mathbb{R}$ be
continuous, and let $\bx_0 \in \mathbb{R}^N \backslash E$. Then, if there
exist $a,C > 0$ such that, in $E$
\[|f(\bx)| \geq \frac{C}{\|\bx - \bx_0\|^a},\] then
\begin{enumerate}[(i)]
\item If $E$ is \pjm with $(B(\bx_0,r)\backslash\{\bx_0\}) \subseteq E$,
$a < N$, and $f$ is bounded on each $E_n := E \backslash B(\bx_0,\frac1n)$,
then $f$ is \Rint in the improper sense over $E$ with infinite improper
Riemann integral.

\item If $E$ admits an exhausting sequence and $\{E_n\}$ such that $f$ is
bounded on each $E_n$, $(\mathbb{R}^N\backslash B(\bx_0,r)) \subseteq E$, and
$a > N$, then $f$ is \Rint in the improper sense over $E$ with infinite
improper Riemann integral.
\end{enumerate}

{\bf Proof:} Since $f$ is continuous and $f$ is bounded on each $E_n$, $f$ is
\Rint over each $E_n$. The rest of the proof is identical to the proof of (ii)
in Theorem 193 (Comparisson for Improper Riemann Integrals). \qed
\end{enumerate}

\hrule
{\Large \bf Differential Surfaces}
\vspace{1mm}
\hrule
\begin{enumerate}
\item \textbf{Definition 214 (Definition of a Manifold)}
For $1 \leq k < N$, a nonempty set $M \subseteq \mathbb{R}^N$ is a
$k$-dimensional manifold of class $C^m$ if and only if, $\forall \bx_0 \in M$,
there exists an open set $U$ with $\bx_0 \in U$ and a function
$\bvarphi: W \subseteq \mathbb{R}^k \rightarrow \mathbb{R}^N$ of class $C^m$
such that $\bvarphi: W \rightarrow M \cap U$ is a homeomorphism and
$\rank J_{\bvarphi}(\by) = k$ in $W$.

\item \textbf{Proposition 219 (Equivalent Definition of a Manifold)}
Suppose $1 \leq k < N$, $M \subseteq \mathbb{R}^N$ is nonempty, and
$m \in \mathbb{N}$. Then, the following are equivalent:
\begin{enumerate}[(i)]
\item $M$ is a $k$-dimensional manifold of class $C^m$.

\item $\forall \bx_0 \in M$, there exists an open set
$U \subseteq \mathbb{R}^N$ with $\bx_0 \in U$ and
$\bfg: U \rightarrow \mathbb{R}^{N - k}$ of class $C^m$ such that
\[M \cap U = \{\bx \in U : \bfg(\bx) = \bzero\},\] and
$\rank J_{\bfg}(\bx) = N - k$ in $M \cap U$.
\end{enumerate}

{\bf Proof:} That (ii) $\Rightarrow$ (i) is an exercise. We show that (i)
implies (ii). Given $\bx_0 \in M$, let $U$,$W$, and $\bvarphi$ be as in the
definition of the manifold. Let $\by_0 \in W$ be such that
$\bvarphi(\by_0) = \bx_0$. Since $\rank J_{\bvarphi}(\by_0) = k$, there is a
$k \times k$ submatrix $A$ of $J_{\bvarphi}(\by_0)$ with $\det A \neq 0$;
without loss of generality,
\[\det
    \begin{bmatrix}
        \frac{\partial \varphi_1}{\partial y_1}(\by_0)
            & \cdots
                & \frac{\partial \varphi_1}{\partial y_k}(\by_0) \\
        \vdots & \ddots & \vdots \\
        \frac{\partial \varphi_k}{\partial y_1}(\by_0)
            & \cdots
                & \frac{\partial \varphi_k}{\partial y_k}(\by_0) \\
    \end{bmatrix}
\neq 0.\]
Consider $\Pi: \mathbb{R}^N \rightarrow \mathbb{R}^k$ defined by
$\Pi(\bx) := (x_1,\ldots,x_k)$, and let $\bff: W \rightarrow \mathbb{R}^k$ be
defined by
\[\bff(\by)
 := \Pi(\bvarphi(\by))
 = (\varphi_1(\by),\ldots,\varphi(\by)),\]
so that $\det J_{\bff} (\by) \neq 0$. Then, by the Inverse Function Theorem,
$\exists r > 0$ such that $B_k(\by_0,r) \subseteq W$, $\bff(B_k(\by_0,r))$ is
open, $\bff: B_k(\by_0,r) \rightarrow \bff(B_k(\by_0,r))$ is invertible, with
inverse $\bff^{-1}$ of class $C^m$. Let $\bz = (x_1,\ldots,x_k)$; then, we can
write $\by$ as a function of $\bz$ ($\by = \bff^{-1}(\bz)$). Let
$\bw := (x_{k + 1},\ldots,x_N)$, so that $\bx = (\bz,\bw)$. Since $\bvarphi$
is a homeomorphism, $\bvarphi(B_k(\by,r))$ is relatively open in $M$, so that
it can be written as some \[\bvarphi(B_k(\by,r)) = M \cap U_1\] for some open
set $U_1 \in \mathbb{R}^N$. Then,
\[M \cap U_1
 = \{\bvarphi(\by) : \by \in B_k(\by_0,r)\}
 = \{(\bz,\varphi_{k + 1}(\bff^{-1}(\bz)),
                             \ldots,\varphi_{n}(\bff^{-1}(\bz)): \bz \in U_1\}
,\]
so that $M \cap U_1$ is the graph of the function
\[\bz \in U_1
 \mapsto \varphi_{k + 1}(\bff^{-1}(\bz)),\ldots,\varphi_{n}(\bff^{-1}(\bz)).\]
Let $\bfg: U_1 \rightarrow \mathbb{R}^{N - k}$ be the class $C^m$ function
defined by
\[g_i(\bx) := x_{k + i} - \varphi_{k + i}\bff^{-1}(x_1,\ldots,x_k).\]
Then, $M \cap U_1 = \{\bx \in U_1 : \bfg(\bx) = \bzero\}$. Moreover, since,
for $i,j \geq k + 1$,
\[\frac{\partial g_i}{\partial x_j}(\bx)
 = \frac{\partial}{\partial x_j} (x_i - \varphi_i(\bff^{-1}(x_1,\ldots,x_k)))
 = \delta_{i,j} - 0
.\]
$J_{bfg}(\bx)$ contains $I_{N - k}$ as a submatrix, so that
$\rank J_{\bfg}(\bx) = N - k$. \qed
\end{enumerate}

\hrule
{\Large \bf Surface Integrals}
\vspace{1mm}
\hrule
\begin{enumerate}
\item \textbf{Definition of the Surface Integral}
Given a $k$-dimensional manifold $M$ of class $C^m$, $m \in \mathbb{N}$, if
$\bvarphi: V \rightarrow M$ is a local chart for $M$ with
$E \subseteq \bvarphi(V)$ such that $\bvarphi^{-1}(E)$ is \pjm and
$f: E \rightarrow \mathbb{R}$ is bounded, then we define the \emph{surface
integral of $f$ over $E$} as
\[\int_E \; d\mathcal{H}^k
 := \int_{\bvarphi(E)} f(\bvarphi(\by))
    \sqrt{
        \sum_{\alpha \in \Lambda_{N,k}}
            \left[
                \det\frac
                    {\partial(\varphi_{\alpha_1},\ldots,\varphi_{\alpha_k})}
                    {\partial(y_1,\ldots,y_k)}(\by)
            \right]^2
    } \; d\by,\]
provided the latter exists.
\end{enumerate}

\hrule
{\Large \bf Divergence Theorem}
\vspace{1mm}
\hrule
\begin{enumerate}
\item \textbf{Definition 227 (Regular Set)}
An open and bounded set $U \subset \mathbb{R}^N$ is \emph{regular} if and only
if there exists a function $\bfg \in C^1(\mathbb{R}^N)$ with
$\nabla g(\bx) = 0$ in $\partial U$, such that
\[U = \{\bx \in \mathbb{R}^N : g(\bx) = 0\},\]
\[\partial U = \{\bx \in \mathbb{R}^N : \nabla g(\bx) = 0\}.\]

\item \textbf{Theorem 230 (Divergence Theorem)}
Let $U \subset \mathbb{R}^N$ be regular, and let
$\bff: \overline{U} \rightarrow \mathbb{R}^N$ be bounded and continuous in
$\overline{U}$ such that all partial derivatives of $\bff$ exist and are
bounded and continuous in $U$.
\[\int_U \div \bff(\bx) \; d\bx
 = \int_{\partial U} \bff(\bx) \cdot \nu(\bx) \; d\mathcal{H}^{N - 1}(\bx).\]

{\bf Proof: Step 1:} First, prove the Theorem in the case that $U$ is the
rectangle $R = (a_1,b_1) \times \cdots (a_N,b_N)$. Let
$R^{\prime} = (a_2,b_2) \times \cdots (a_N,b_N)$, and let
$\bx^{\prime} = (x_2,\ldots,x_N)$. Then, by Theorem 160 (Repeated
Integration) and the Fundamental Theorem of Calculus,
\begin{eqnarray*}
\int_R \frac{\partial f_1}{\partial x_1}(\bx) \; d\bx
 & = & \int_{R^{\prime}}
            \left(
                \int_{a_1}^{b_1} \frac
                    {\partial f_1}
                    {\partial x_1}
                        (x_1,\bx^{\prime})
                \; dx_1
            \right)
        \; d\bx^{\prime} \\
 & = & \int_{R^{\prime}}
            \left(
                f_1(b_1,\bx^{\prime}) - f_1(a_1,\bx^{\prime})
            \right)
        \; d\bx^{\prime} \\
 & = & \int_{R^{\prime}}
                \bff(b_1,\bx^{\prime})\cdot\be_1
        \; d\bx^{\prime}
   +   \int_{R^{\prime}}
                \bff(a_1,\bx^{\prime})\cdot(-\be_1)
        \; d\bx^{\prime} \\
 & = & \int_{\{b_1\} \times R^{\prime}}
                \bff(\bx)\cdot\nu
        \; d\mathcal{H}^{N - 1}(\bx)
   +   \int_{\{a_1\} \times R^{\prime}}
                \bff(\bx)\cdot\nu
        \; d\mathcal{H}^{N - 1}(\bx)
\end{eqnarray*}
Applying this argument to each component and summing the resulting identities
give the desired result.

{\bf Step 2:} Next, prove the result in the case that $U$ is of the form
\[U = \{(x_1,\bx^{\prime}) \in \mathbb{R} \times \mathbb{R}^{N - 1}
    : h(\bx^{\prime}) < x_1 < b_1, \bx^{\prime} \in R^{\prime}\},\]
by using the Change of Variables
\[y_1 := x_1 - h(\bx^{\prime}), \quad \by^{\prime} := \bx^{\prime}.\]
A similar argument works in the case that $x_1$ is replaced by some other
$x_i$, and in the case that $b_i$ is switched approriately by $a_i$.

{\bf Step 3:} Now, prove that, for any $\bx \in \partial U$, there exists
$r_{\bx} > 0$ such that $U \cap Q(\bx,r_{\bx})$ is of a form addressed in Step
2.

{\bf Step 4:} Finally, we ``stitch'' together the results of Step 1, 2, and 3
to prove the result for general $U$. Since $\overline{U}$ is closed and
bounded, it is compact, so that it has a finite open cover
$\{Q(\bx,r_{\bx})\} = Q_1,\ldots,Q_k$. Note that, for each $\bx$, either
$Q(\bx,r_{\bx}) \subseteq U$ or $Q(\bx,r_{\bx}) \cap U$ is of the form covered
by Step 2 (by the result of part 3). Thus, we construct $\varphi_k$, the
partition of unity subordinated to the family of open sets $\{Q_i\}$. Note
that, since in each $Q_i$, $\varphi_i$ is constant,
\[\sum_{i = 1}^k \frac{\partial \varphi}{\partial x_i}(\bx)
 = \frac{\partial}{\partial x_i} (1) = 0\] in $U$.

By the results of Steps 1 and 2, since $\varphi_i\bff$ is nonzero only inside
$Q_i$,
\[\int_U \div (\varphi_k\bff) \; d\bx
 = \int_{\partial U} \varphi\bff \cdot \nu \; d\mathcal{H}^{N - 1}.\]
Since $\sum_{i = 1}^k \varphi_i = 1$ in $\overline{U}$, taking the sum over
all $i$ gives
\begin{eqnarray*}
\int_U \div \bff \; d\bx
 & = & \int_U \div \left(\sum_{i = 1}^k \varphi_i \bff \right) \; d\bx
   =   \sum_{i = 1}^k \int_U \div (\varphi_i \bff) \; d\bx \\
 & = & \sum_{i = 1}^k\int_{\partial U}\varphi_i \bff \cdot \nu
                                                        \;d\mathcal{H}^{N - 1}
   =   \int_{\partial U}\sum_{i = 1}^k\varphi_i \bff \cdot \nu
                                                        \;d\mathcal{H}^{N - 1}
   =   \int_{\partial U}\bff \cdot \nu \;d\mathcal{H}^{N - 1}. \mqed
\end{eqnarray*}

\item \textbf{Corollary 234 (Integration by Parts)}
Let $U \subset \mathbb{R}^N$ be regular, and let
$f,g: \overline{U} \rightarrow \mathbb{R}$ be bounded and continuous, such
that all partial derivates of $f$ and $g$ exist and are bounded and continuous
in $U$. Then, $\forall i \in \{1,2,\ldots,N\}$,
\[\int_U f(\bx) \frac{\partial g}{\partial x_i}(\bx) \; d\bx
 = -\int_U g(\bx)\frac{\partial f}{\partial x_i}(\bx) \; d\bx
 + \int_{\partial U} f(\bx)g(\bx)\nu_i(\bx) \; d\mathcal{H}^{N - 1}(\bx).\]

{\bf Proof:} Apply the Divergence Theorem to the function
$\bff: \overline{U} \rightarrow \mathbb{R}^N$ defined by
\[f_j(\bx) = \left\{
    \begin{array}{cc}
        f(\bx)g(\bx) & j = i,    \\
        0            & j \neq i. \\
    \end{array}
\right.\]
Then, since
\[\div \bff
 = \sum_{j = 1}^N \frac{\partial f_i}{\partial x_j}
 = \frac{\partial(fg)}{\partial x_j}
 = f\frac{\partial g}{\partial x_j} + g\frac{\partial f}{\partial x_j},\]
\[\int_U \left(f\frac{\partial g}{\partial x_j}
 + g\frac{\partial f}{\partial x_j}\right) \; d\bx
 = \int_U \div \bff \; d\bx
 = \int_{\partial U} \bff \cdot \nu \; d\mathcal{H}^{N - 1}
 = \int_{\partial U} fg\nu_i \; d\mathcal{H}^{N - 1}. \mqed\]

\item \textbf{Proposition 237 (Mollifiers)}
Given open sets $A,D \subset \mathbb{R}^N$ with $D$ bounded and
$\dist(A,D) \geq 3d > 0$ for some $d > 0$, there exists a function
$f \in C^1(\mathbb{R}^N)$ such that $0 \leq f \leq 1$, $f(\bx) = 0$ in $A$,
$f(\bx) = 1$ in $D$, and $\|\nabla f\| \leq \frac{C}{d}$, where $C > 0$
depends only on $N$.

{\bf Proof:} Construct a nonnegative function $g \in C^1(\mathbb{R}^N)$ with
$g(\bx) = 0$ in $\mathbb{R}^N\backslash B(\bzero,1)$, $g(\bx) > 0$ in
$B(\bzero,\frac12)$, and \[\int_{B(\bzero,1)} g(\bx) \; d\bx = 1.\]

\item \textbf{Theorem 242 (Generalization of the Divergence Theorem)}
The Divergence Theorem continues to hold if $U \subset \mathbb{R}^N$ is open
and bounded, and its boundary consists of sets $E_1, E_2$, where
$\meas_{N - 1} E_1 = 0$ and, $\forall \bx_0 \in E_2$, for $B := B(\bx_0,r)$,
there exists $g \in C^1(B)$ with $\nabla g(\bx) = 0$ in $B \cap \partial U$,
and
\[B \cap U = \{\bx \in B : g(\bx) < 0\},\]
\[B \backslash \overline{U} = \{\bx \in B : g(\bx) > 0\},\]
\[B \cap \partial U = \{\bx \in B : g(\bx) = 0\}.\]

{\bf Proof:} 

\item \textbf{Lemma 243 (Lemma for Generalization of the Divergence Theorem)}
Let $K \subset \mathbb{R}^N$ be a compact set with $\meas_{N - 1} K = 0$.
Then, $\forall r > 0$, if
\[A_r := \{\bx \in \mathbb{R}^N : \dist(\bx,K) < r\},\]
\[\lim_{r \rightarrow 0^+} \frac{\meas_o (A_r)}{r} = 0.\]

{\bf Proof:} Since $\meas_{N - 1} K = 0$, $\forall \epsilon > 0$,
$\exists r_{\epsilon} > 0$ such that, for $0 < r \leq r_{\epsilon}$, there
exists a finite family $\{Q(\bx_{n,r},r)\}$ of open cubes covering $K$ and
\[\sum_n \left(\sqrt{N}r\right)^{N - 1}
 = \sum_n (\diam Q(\bx_{n,r},r))^{N - 1}
 \leq \frac{N^{(N - 1)/2}}{2^N}\epsilon.\]
By choice of $A_r$, if $\bx \in A_r$, then $\exists \by \in K$ such that
$\|\bx - \by\| < r$. Since $\{Q(\bx_{n,r},r)\}$ covers $K$, $\by$ is in some
$Q(\bx_{n,r},r)$, so that $\bx \in Q(\bx_{n,r},2r)$, and therefore
$\{Q(\bx_{n,r},2r)\}$ covers $A_r$. Then,
\[\frac{1}{r}\meas_o(A_r)
 \leq \frac{1}{r}\sum_n\meas Q(\bx_{n,r},2r)
 = \frac{1}{r} \sum_n (2r)^N
 = \frac{2^N}{N^{(N - 1)/2}}\sum_n \left(\sqrt{N}r\right)^{N - 1}
 \leq \epsilon,\]
and therefore, $\lim_{r \rightarrow 0} \frac{1}{r} \meas_o (A_r) = 0$. \qed
\end{enumerate}

\hrule
{\Large \bf Stokes' Theorem}
\vspace{1mm}
\hrule
\begin{enumerate}
\item \textbf{Theorem 244 (Stokes' Theorem)}
Let $U \subseteq \mathbb{R}^3$ be an open set, and let
$\bff: U \rightarrow \mathbb{R}^3$ be of class $C^1$. Let
$M \subseteq \mathbb{R}^2$ be a $2$-dimensional manifold of class $C^2$,
with boundary $\Gamma$, of positive orientation. Then,
\[\int_M \curl \bff \cdot \nu \; d\mathcal{H}^2 = \int_{\Gamma} \bff.\]

{\bf Proof:} Don't need to know this one (just know how to apply the theorem).

\end{enumerate}
\end{document}
