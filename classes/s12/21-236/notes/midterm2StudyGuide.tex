\documentclass[11pt]{article}
\usepackage{enumerate}
\usepackage{fullpage}
\usepackage{fancyhdr}
\usepackage{amsmath, amsfonts, amsthm, amssymb}
\setlength{\parindent}{0pt}
\setlength{\parskip}{5pt plus 1pt}
\pagestyle{empty}

\def\indented#1{\list{}{}\item[]}
\let\indented=\endlist

\newcounter{questionCounter}
\newcounter{partCounter}[questionCounter]
\newenvironment{question}[2][\arabic{questionCounter}]{%
    \setcounter{partCounter}{0}%
    \vspace{.25in} \hrule \vspace{0.5em}%
        \noindent{\bf #2}%
    \vspace{0.8em} \hrule \vspace{.10in}%
    \addtocounter{questionCounter}{1}%
}{}
\renewenvironment{part}[1][\alph{partCounter}]{%
    \addtocounter{partCounter}{1}%
    \vspace{.10in}%
    \begin{indented}%
       {\bf (#1)} %
}{\end{indented}}

%%%%%%%%%%%%%%%%%%%%%%%HEADER%%%%%%%%%%%%%%%%%%%%%%%%%%%%%%
\newcommand{\myname}{Shashank Singh}
\newcommand{\myandrew}{sss1@andrew.cmu.edu}
\newcommand{\myclass}{21-236 Mathematical Studies Analysis II}
\newcommand{\duedate}{Friday, April 13, 2012}
%%%%%%%%%%%%%%%%%%%%%%%%%%%%%%%%%%%%%%%%%%%%%%%%%%%%%%%%%%%

%%%%%%%%%%%%%%%%%%%%%%%%MACROS%%%%%%%%%%%%%%%%%%%%%%%%%%%%%
\newcommand{\bvarphi}{\boldsymbol{\varphi}}
\newcommand{\bpsi}{\boldsymbol{\psi}}
\newcommand{\Var}{\operatorname{Var}}
%%%%%%%%%%%%%%%%%%%%%%%%%%%%%%%%%%%%%%%%%%%%%%%%%%%%%%%%%%%

\begin{document}
\thispagestyle{plain}

{\Large Midterm 2 Study Guide} \\
\myclass \\
Name: \myname \\
Email: \myandrew \\
Exam Date: \duedate \\

\hrule
{\Large \bf Lagrange Multipliers}
\vspace{1mm}
\hrule
\begin{enumerate}
\item \textbf{Theorem 85 (Lagrange Multipliers)}: Let
$U \subseteq \mathbb{R}^N$ be an open set, let $f: U \rightarrow \mathbb{R}$,
and $\mathbf{g}: U \rightarrow \mathbb{R}^M$ with $M < N$ be $C^1$, and
\[F := \{\mathbf{x} \in U : \mathbf{g}(\mathbf{x}) = \mathbf{0}\}.\]
Let $\mathbf{x}_0 \in F$, and assume $f$ attains a local extremum at
$\mathbf{x}_0$. Then, if $G_{\mathbf{g}}(\mathbf{x}_0)$ has maximum rank $M$,
there exist $\lambda_1,\lambda_2,\ldots,\lambda_M \in \mathbb{R}$ such that
\[\nabla f(\mathbf{x})
 = \lambda_1 \nabla g_1(\mathbf{x}_0) + \lambda_2 \nabla g_2(\mathbf{x}_0)
 + \ldots + \lambda_M \nabla g_M(\mathbf{x}_0).
\]
\begin{itemize}
\item Proof is $\approx 2$ pages, so probably too long.
\item Know how to use them to find extrema.
\end{itemize}
\end{enumerate}

\hrule
{\Large \bf Curves}
\vspace{1mm}
\hrule
\begin{enumerate}
\item \textbf{Theorem 94 (Peano Curve)} There exists a continuous function
$\bvarphi: [0,1] \rightarrow \mathbb{R}^2$ such that
$\bvarphi([0,1]) = [0,1]^2$.

% SEEMS VERY UNLIKELY
%{\bf Proof:} For $n \in \mathbb{N}$, divide $[0,1]$ into $4^n$ equal closed
%intervals $I_{k,n}$ and divide $[0,1]^2$ into $4^n$ equal closed squares
%$Q_{k,n}$.

\item \textbf{Theorem 101 (Length of a Smooth Curve)} Let $\gamma$ be a $C^1$
curve with parametrization $\bvarphi: I \rightarrow \mathbb{R}^N$. Recall
that
\[\Var_I \bvarphi
 := \sup \left\{
           \sum_{i = 1}^n \|\bvarphi(t_i) - \bvarphi(t_{i - 1})\}
         \right\}
,\]
and that $L(\gamma) := \Var_I \bvarphi$. Then,
\[L(\gamma) = \int_a^b \|\bvarphi^{\prime}(t)\|\; dt.\]
\begin{itemize}
\item Proof requires Lemmas 103 and 104.
\item Likely just need to know how to calculate length.
\end{itemize}

\item \textbf{Lemma 103 (Triangle Inequality for Integrals)} If
$\mathbf{f}: [c,d] \rightarrow \mathbb{R}^N$ is Riemann Integrable, then
$\|\mathbf{f}\|: [c,d] \rightarrow \mathbb{R}$ is Riemann Integrable and
\[\left\|\int_c^d \mathbf{f}(t) \; dt\right\|
                                      \leq \int_c^d \|\mathbf{f}(t)\| \; dt.\]

\item \textbf{Lemma 104 (Another Integral Inequality)} If
$\mathbf{f}: [c,d] \rightarrow \mathbb{R}^N$ is Riemann Integrable, then, for
$t_0 \in [c,d]$, 
\[\left\|\int_c^d \mathbf{f}(t) \; dt\right\|
  \geq \int_c^d \|\mathbf{f}(t)\| \; dt - 2\int_c^d\|\mathbf{f}(t)
  - \mathbf{f}(t_0)\| \; dt
.\]

\item \textbf{Proposition 116 (Absolute Continuity implies Finite Variation)}
If $\bvarphi: [a,b] \rightarrow \mathbb{R}^N$ is absolutely continuous, then
$\bvarphi$ has finite variation.

\begin{itemize}
\item Proof relies on previous exercise.
\end{itemize}

\item \textbf{Theorem 119 (Regular curves can be Arc-Length Parametrized)}
If $\gamma$ is regular (i.e., it is piecewise $C^1$ and admits a parametric
representation $\bvarphi: [a,b] \rightarrow \mathbb{R}^N$ with nonzero left
and right derivatives on $[c,d]$), then $\gamma$ can be parametrized by
arclength.

{\bf Proof:} Define a \emph{length} function
$s: [a,b] \rightarrow [0,L(\gamma)]$. Show that $s$ is invertible. Then, since
the inverse of a continuous, one-to-one function on a compact set is
continuous, $s^{-1}$ is continuous. Therefore, since, by the FTC,
$s^{\prime} > 0$ a.e., so that $s$ is piecewise $C^1$, $s^{-1}$ is $C^1$ with
the usual derivative, so that
$\bpsi(t) := \bvarphi(s^{-1}(t)), t \in [0,L(\gamma)]$ is equivalent to
$\bvarphi$, and, furthermore,
\[\bpsi^{\prime}(t)
 = \frac{\bvarphi^{\prime}(s^{-1}(t))}
        {\|\bvarphi^{\prime}(s^{-1}(t))\|}
,\] so that $\|\bpsi^{\prime}(t)\| = 1$. \quad $\blacksquare$

\end{enumerate}

\hrule
{\Large \bf Curve Integrals}
\vspace{1mm}
\hrule
\begin{enumerate}
\item \textbf{Theorem 129 (Fundamental Theorem of Calculus for Curves)} Let
$U \subseteq \mathbb{R}^N$ be open, let $f \in C^1(U)$, let
$\mathbf{x},\mathbf{y} \in U$, and let $\gamma$ be piecewise $C^1$ in $U$ with
parametric representation $\bvarphi: [a,b] \rightarrow \mathbb{R}^N$ such that
$\bvarphi(a) = \mathbf{y}$ and $\bvarphi(b) = \mathbf{x}$. Then,
\[\int_{\gamma}\nabla f = f(\mathbf{x}) - f(\mathbf{y}).\]

{\bf Proof:} Define $p(t) = f(\bvarphi(t))$. By the Chain Rule, $p$ is
piecewise $C^1$ with (a.e)
\[p^{\prime}(t)
 = \sum_{i = 1}^N
             \frac{\partial f}{\partial x_i}(\bvarphi(t))\varphi_i^{\prime}(t)
.\]
Thus, by the FTC,
\[\int_{\gamma}\nabla f
 = \int_a^b \sum_{i = 1}^N
        \frac{\partial f}{partial x_i}(\bvarphi(t))\varphi_i^{\prime}(t) \; dt
 = \int_a^b p^{\prime}(t) \; dt = p(b) - p(a) = f(\mathbf{y}) - f(\mathbf{x})
. \quad \blacksquare\]

\item \textbf{Theorem 130 (Conservative Field Equivalents)} Let
$U \subseteq \mathbb{R}^N$ be an open, connected set, and let
$\mathbf{g}: U \rightarrow \mathbb{R}^N$ be continuous. Then, the following
are equivalent:
\begin{enumerate}[(i)]
\item $\mathbf{g}$ is conservative.

\item For every piecewise $C^1$ closed oriented curve $\gamma$ in $U$,
\[\int_{\gamma} \mathbf{g} = 0.\]

\item $\forall \mathbf{x},\mathbf{y} \in U$ and for piecewise $C^1$ oriented
curves $\gamma_1$, $\gamma_2$ in $U$ with the same endpoints,
\[\int_{\gamma_1} \mathbf{g} = \int_{\gamma_2} \mathbf{g}.\]
\end{enumerate}

{\bf Proof:} That (i) $\Rightarrow$ (ii) follows immediately from the FTC for
Curves. That (ii) $\Leftrightarrow$ (iii) is trivial. Thus, we prove
(ii) $\Rightarrow$ (i).

\item \textbf{Theorem 132 (Conservative Fields are Irrotational)} Let
$U \subseteq \mathbb{R}^N$ be open, and let
$\mathbf{g}: U \rightarrow \mathbb{R}^N$ be conservative and $C^1$. Then,
$\mathbf{g}$ is irrotational.

{\bf Proof:\} Since $\mathbf{g}$ is conservative, $\mathbf{g} = \nabla f$ for
some $f: U \rightarrow \mathbb{R}$. Since $\mathbf{g}$ is $C^1$, $f$ is $C^2$.
Therefore, by the Schwartz Theorem, $\forall i, j \in \{1,2,\ldots,N\}$,
\[\frac{\partial g_i}{\partial x_j}
 = \frac{\partial^2 f}{\partial x_j x_i}
 = \frac{\partial^2 f}{\partial x_i x_j}
 = \frac{\partial g_j}{\partial x_i},
\]
so that $\mathbf{g}$ is irrotational. \quad $\blacksquare$

\item \textbf{Theorem 135 (Poincar\'e's Lemma I (Starshaped Domains))}

\item \textbf{Theorem 143 (Homotopic Curves Integrals in Irrotational Fields)}

\item \textbf{Theorem 144 (Poincar\'e's Lemma II (Simply Connected Domains))}

\item \textbf{Lemma 145 (Lebesgue Number)}

\item \textbf{Theorem 150 (Simple Connectivity Equivalents)}

\item \textbf{Theorem 154 (Integral Definitions of the Winding Number)}

\item \textbf{Theorem 155 (Winding Number Constant on Connected Components)}

\end{enumerate}
\end{document}
