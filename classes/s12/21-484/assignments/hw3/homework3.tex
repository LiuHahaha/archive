\documentclass[11pt]{article}
\usepackage{enumerate}
\usepackage{fullpage}
\usepackage{fancyhdr}
\usepackage{tikz}
\usepackage{amsmath, amsfonts, amsthm, amssymb}
\setlength{\parindent}{0pt}
\setlength{\parskip}{5pt plus 1pt}
\pagestyle{empty}

\def\indented#1{\list{}{}\item[]}
\let\indented=\endlist

\newcounter{questionCounter}
\newcounter{partCounter}[questionCounter]
\newenvironment{question}[2][\arabic{questionCounter}]{%
    \setcounter{partCounter}{0}%
    \vspace{.25in} \hrule \vspace{0.5em}%
        \noindent{\bf #2}%
    \vspace{0.8em} \hrule \vspace{.10in}%
    \addtocounter{questionCounter}{1}%
}{}
\renewenvironment{part}[1][\alph{partCounter}]{%
    \addtocounter{partCounter}{1}%
    \vspace{.10in}%
    \begin{indented}%
       {\bf (#1)} %
}{\end{indented}}

%%%%%%%%%%%%%%%%%%%%%%%%%%%%%%%%%%%%%%%%%%%%%%%%%%%%%%%%%%%
\newcommand{\myname}{Shashank Singh}
\newcommand{\myandrew}{sss1@andrew.cmu.edu}
\newcommand{\myclass}{21-484A Graph Theory}
\newcommand{\myhwnum}{3}
\newcommand{\duedate}{Wednesday, March 28, 2012}
%%%%%%%%%%%%%%%%%%%%%%%%%%%%%%%%%%%%%%%%%%%%%%%%%%%%%%%%%%%

\begin{document}
\thispagestyle{plain}

{\Large Homework \myhwnum} \\
\myclass \\
Name: \myname \\
Email: \myandrew \\
Due: \duedate
\begin{question}{Problem 2}
Note that
$V(\overline{G - \{v\}}) = V(G) \backslash \{v\} = V(\overline{G} - \{v\})$.
Suppose $e \in E(\overline{G - \{v\}})$. Then, $e$ is not incident to $v$ in
$G$ (since $v \not \in V(G - \{v\})$), and $e \not \in E(G)$, so that
$e \in E(\overline{G} - \{v\})$. Therefore,
$E(\overline{G - \{v\}}) \subseteq E(\overline{G} - \{v\})$.

Since, by the result of a previous homework problem, either
$\overline{G - \{v\}}$ or $G - \{v\}$ must be connected, and, by definition of
cut-vertex, $G - \{v\}$ is disconnected, $\overline{G - \{v\}}$ must be
connected. Then, since adding edges to a connected graph cannot disconnect it,
$\overline{G} - \{v\}$ is connected, so that $v$ is not a cut-vertex or
$\overline{G}$. \quad $\blacksquare$
\end{question}

\begin{question}{Problem 3}
Let $G$ be a connected graph with at least $2$ vertices.

Suppose $G$ is non-separable, and let $e_1 = \{t,u\},e_2 = \{u,v\}$ be two
adjacent edges in $G$. Since $G$ is non-separable, $G - \{u\}$, the graph
produced by removing $u$ and its incident edges from $G$, is connected. Thus,
there exists a path from $t$ to $v$ in $G - \{u\}$, so that there is a path
from $t$ to $v$ in $G$ that does not use $u$. Adding $u,t$ to this path gives
a cycle containing $e_1$ and $e_2$. \quad $\blacksquare$

Suppose, on the other hand, that any two adjacent edges in $G$ lie on a common
cycle. Let $t,u,v \in V(G)$, with $t \neq u \neq v$. Since $G$ is connected,
there exists a path from $t$ to $v$ in $G$. If that path does not contain $u$,
then that path is also in $G - \{u\}$, so $t$ and $v$ are connected in
$G - \{u\}$. Otherwise, let $u_-$ and $u_+$ be the vertices before and after
$u$, respectively, in the path from $t$ to $v$. Since $\{u_-,u\}$ and
$\{u,u_+\}$ are adjacent edges in $G$, they lie in a common cycle, so that
$u_-,u,$ and $u_+$ lie in a common cycle, and there is a path $P$ from $u_-$
to $u_+$ that does not include $u$. By choice of $u_-$ and $u_+$ there exist
paths $P_1$ and $P_2$ from $t$ to $u_-$ and from $u_+$ to $v$, respectively,
neither of which includes $u$. Therefore, if $P^{\prime}$ is $P$ without its
first and last vertices, $P_1P^{\prime}P_2$, the walk $W$ created by
concatenating $P_1,P^{\prime},$ and $P_2$ (in that order), is a walk from $t$
to $v$ that does not contain $u$. Therefore, $W$ is also in $G - \{u\}$, and,
since any walk from $t$ to $v$ contains a path from $t$ to $v$, there is a
path from $t$ to $v$ in $G - \{u\}$, so that $t$ and $v$ are connected in
$G - \{u\}$. Therefore, since $t$ and $v$ are arbitrary, $G - \{u\}$ is
connected, so that, since $u$ is arbitrary, $G$ is non-separable.
\quad $\blacksquare$
\end{question}

\begin{question}{Problem 7}
Let $G$ be a nontrivial, connected graph.

Suppose $G$ is Eulerian, let $m = |E(G)|$, and let
$C = e_0,e_1,\ldots,e_{n - 1}$ be an Eulerian circuit in $G$. Let
$v \in V(G)$. For every edge $e_k$ incident to $v$, exactly one of
$e_{k - 1 \pmod n}$ or $e_{k + 1 \pmod n}$ is also incident to $v$.
Furthermore, since $C$ is an Eulerian circuit, each edge incident to $v$
appears exactly once in $C$. Therefore, each edge in $G$ incident to $v$ can
be paired with exactly one other edge incident to $v$, so that there are an
even number of edges incident to $v$, so that $v$ has even degree. Thus,
every vertex in $G$ has even degree. \quad $\blacksquare$

Suppose, on the other hand, that every vertex of $G$ has even degree. Since
$G$ is connected and is not a tree (because it is non-trivial and cannot have
any leaves), it must contain a cycle.
We proceed by induction on the number of edges in $G$. The number of edges in
$G$ must be at least $n = |V(G)|$, the number of edges in a cycle on $n$
vertices. Clearly, there is an Eulerian circuit in a graph that is just a
cycle. Suppose, as an inductive hypothesis, that all $G$ with $k \geq 1$
cycles, $G$ is Eulerian.
Suppose $G$ contains at least $2$ cycles. Let $H$ be a graph produced
by removing all edges in some cycle of $G$. Each connected component of $H$
must have at least one vertex that was in that cycle (since $G$ was
connected), and each connected component has strictly fewer cycles than $G$.
Thus, by the inductive hypothesis, each connected component of $H$ must be
Eulerian, so that we can construct an Eulerian circuit in $G$ by connecting
the Eulerian circuits in the connected components of $H$ by their vertices
on the cycle removed from $G$. Thus, by induction, all $G$ with the above
properties are Eulerian. \quad $\blacksquare$
\end{question}

\begin{question}{Problem 8}
Suppose for sake of contradiction, that the Petersen graph $G$ were
Hamiltonian. Let $U$ be the set of vertices forming the pentagonal exterior of
$G$, and let $V$ be the set of vertices forming star-shaped interior of $G$.
Let $C$ be a Hamiltonian cycle in $G$, and suppose we traverse $C$ starting
from a vertex in $U$. Since we begin and end in $U$, we must traverse an even
number of edges spanning between $U$ and $V$. Since there are $5$ such edges
and we must visit all the vertices in $V$, we must traverse either $2$ or $4$
such edges. Thus, we break into two cases:

{\bf Case 1:} We traverse exactly $2$ edges spanning between $U$ and $V$.
Let $u_1 \in U$ be an endpoint of some such edge, and let $v_1 \in V$ be its
other endpoint. Then, in traversing $C$, we walk from $u_1$ to $v_1$, traverse
the star until we have visited all $5$ vertices, and end up on a vertex in $V$
adjacent to $v_1$. Then, we exit the star, so that we are at a vertex
$u_2 \in U$ that is not adjacent to $u_1$. Then, however, since we cannot
enter the star, if we walk clockwise, we cannot visit any vertices
counterclockwise from $u_2$ and clockwise from $u_1$ (of which at least one
must exist because $u_1$ and $u_2$ cannot be adjacent), and similarly, if we
walk counterclockwise, we fail to vist vertices clockwise from $u_2$ and
counterclockwise from $u_1$. Therefore, no such Hamiltonian cycle can exist,
concluding this case.

{\bf Case 2:} We traverse exactly $4$ edges spanning between $U$ and $V$.
Let $u_1 \in U$ be an endpoint of some such edge, and let $v_1 \in V$ be its
other endpoint. Since, in traversing $C$, we enter $V$ twice and cannot use
edges more than once, we must either vist $3$ vertices in $V$ the first time
we enter $V$, and $2$ vertices the second time, or we must visit $2$ vertices
the first time we enter $V$ and $3$ vertices the second time. However, without
loss of generality, we have the first case, because, in the second case, we
can traverse $C$ backwards to give the first case. Thus, we enter $V$ from
$u_1$, traverse $3$ vertices in $V$, and then move to a vertex in $U$ adjacent
to $u_1$. Then, we must traverse either $1$ or $3$ other vertices in $U$
before re-entering $V$, but, in the first case, it is not possible to visit
all vertices in $U$ and return to $u_1$, and in the second case, it is not
possible to visit all vertices in $V$ and return to $u_1$, so that there is
no such Hamiltonian cycle.

Therefore, the Petersen graph is not Hamiltonian. \quad $\blacksquare$
\end{question}

\end{document}
