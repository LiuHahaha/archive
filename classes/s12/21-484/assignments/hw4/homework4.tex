\documentclass[11pt]{article}
\usepackage{enumerate}
\usepackage{fullpage}
\usepackage{fancyhdr}
\usepackage{tikz}
\usepackage{amsmath, amsfonts, amsthm, amssymb}
\setlength{\parindent}{0pt}
\setlength{\parskip}{5pt plus 1pt}
\pagestyle{empty}

\def\indented#1{\list{}{}\item[]}
\let\indented=\endlist

\newcounter{questionCounter}
\newcounter{partCounter}[questionCounter]
\newenvironment{question}[2][\arabic{questionCounter}]{%
    \setcounter{partCounter}{0}%
    \vspace{.25in} \hrule \vspace{0.5em}%
        \noindent{\bf #2}%
    \vspace{0.8em} \hrule \vspace{.10in}%
    \addtocounter{questionCounter}{1}%
}{}
\renewenvironment{part}[1][\alph{partCounter}]{%
    \addtocounter{partCounter}{1}%
    \vspace{.10in}%
    \begin{indented}%
       {\bf (#1)} %
}{\end{indented}}

%%%%%%%%%%%%%%%%%%%%%%%%%%%%%%%%%%%%%%%%%%%%%%%%%%%%%%%%%%%
\newcommand{\myname}{Shashank Singh}
\newcommand{\myandrew}{sss1@andrew.cmu.edu}
\newcommand{\myclass}{21-484A Graph Theory}
\newcommand{\myhwnum}{4}
\newcommand{\duedate}{Wednesday, March 28, 2012}
%%%%%%%%%%%%%%%%%%%%%%%%%%%%%%%%%%%%%%%%%%%%%%%%%%%%%%%%%%%

%%%%%%%%%%%%%%%%%%%%%%%%MACROS%%%%%%%%%%%%%%%%%%%%%%%%%%%%%
\renewcommand{\qed}{\quad $\blacksquare$}
\newcommand{\mqed}{\quad \blacksquare}
\newcommand{\diam}{\operatorname{diam}}
%%%%%%%%%%%%%%%%%%%%%%%%%%%%%%%%%%%%%%%%%%%%%%%%%%%%%%%%%%%

\begin{document}
\thispagestyle{plain}

{\Large Homework \myhwnum} \\
\myclass \\
Name: \myname \\
Email: \myandrew \\
Due: \duedate
\begin{question}{Problem 1}
Let $C$ be the set of columns of the board, and let $R$ be the set of rows of
the board. Construct an undirected graph $G$ (with vertex set $C \cup R$)
such that there is an edge between two vertices $u$ and $v$ if and only if
there is a rook on a square shared by $u$ and $v$.

Since no pair of rows and no pair of columns can share a square, $G$ is
bipartite. Since each row and each column contains $k$ rooks, $G$ is
$k$-regular. Thus, since all regular, bipartite graphs have perfect matchings,
$G$ has a perfect matching $M$. Let $S$ be the set of rooks corresponding to
edges in $M$. Since each row is paired with exactly one column and each column
is paired with exactly one row, no two rooks in $S$ are in the same row or
column. Since $|C \cup R| = 2n$, $|S| = n$. Therefore, $S$ is a set of $n$
rooks with the desired properties. \qed
\end{question}

\begin{question}{Problem 3}
Suppose that, for some $k \in \mathbb{N}$, $\|G\| < {k \choose 2}$, and
consider a proper $k$ coloring of $G$. Since each edge connects at most $2$
vertices (so that $k \choose 2$ edges are required to have an edge between
vertices of each possible pair of colors), there exist two colors $b$ and $r$
such that there is no edge between any vertex colored $b$ and any vertex
colored $r$. Thus, all the vertices of color $b$ can be recolored with color
$r$ to give a proper $k - 1$ coloring, so that $\chi(G) < k$. Therefore, if
$\|G\| < {k \choose 2}$, then $\chi(G) < k$, so that the contrapositive, that,
if $\chi(G) \geq k$, then $\|G\| \geq {k \choose 2}$, also holds. \qed
\end{question}

\begin{question}{Problem 4}
Note that we discuss only connected graphs here, since disconnected graphs can
be colored one component at a time.
Let $G$ be a graph which neither an odd cycle nor a complete graph. Suppose
$G$ is not $\Delta(G)$-regular, so that there exists a vertex $v$ of degree
$d < \Delta(G)$. Consider a breadth-first search tree rooted at $v$ (since $G$
is connected, this tree includes every vertex in $G$). Color the tree from the
bottom up by repeatedly coloring the lowest uncovered vertex in the tree,
until $v$ has been colored. This can be done with $\Delta(G)$ colors, since
each vertex is colored before its parent (so that it has at most
$\Delta(G) - 1$ colored neighbors (it children)), with the exception of $v$,
which, by choice, has degree at most $\Delta(G) - 1$, so that it can be
colored with one of the $\Delta(G)$ colors. Therefore, any graph that is not
$\Delta(G)$-regular is $\Delta(G)$-colorable, so that Brooks' theorem holds
for all graphs that are not $\Delta(G)$-regular, or are $\Delta(G)$-regular
and $2$-connected, leaving only those graphs which are $k$-regular but not
$2$-connected.
\newpage
Name: \myname \\
Email: \myandrew \\

Let $K$ be a graph which is neither an odd cycle nor a complete graph. By the
previous result, we can color each biconnected component of $K$ with
$\Delta(K)$ colors. Let $\sigma$ be a cyclic permutation of the the
$\Delta(G)$ colors. Let $A$ and $B$ be two biconnected components thus
colored. As distinct biconnected components, $A$ and $B$ share at most $1$
vertex $v$ or are connected by at most one edge $e$. In the first case, we can
construct a proper coloring of $A \cup B$ (the subgraph of $K$ induced by
those vertices in either $A$ or $B$) by using colorings of $A$ and $B$, and,
in the case that $v$ is colored differently in the two colorings, repeatedly
permuting the colors of the vertices in $A$ by $\sigma$ until $v$ is the same
color in both the coloring of $A$ and the coloring of $B$. In the second case,
we can construct a proper coloring of $A \cup B$ by using the colorings of $A$
and $B$, and permuting the colors of $A$ once if the vertices to which $e$ is
incident are the same color. Furthermore, we can connect an arbitrary number
of biconnected components of $K$ in this manner, so that we can construct a
proper $\Delta(K)$-coloring of $K$ in this manner. Thus, Brooks' Theorem holds
for graphs which are not $2$-connected.

Therefore, Brooks' theorem holds for all graphs. \qed
\end{question}

\begin{question}{Problem 6}
Let $K$ be the graph whose vertices are the edges of $G$, with edges between
vertices in $K$ if and only if those vertices are adjacent edges in $G$. Since
$G$ contains $4k + 1$ vertices, no independent set of edges in $G$ contains
more than $2k$ edges (since $2$ vertices are required for each independent
edge). Thus, since a set of edges in $G$ is independent if and only the
corresponding set of vertices in $K$ is independent, is $\alpha(K)$ is the
maximum size of an independent of vertices in $K$, $\alpha(K) \leq 2k$. The
number of edges in $H$ is half the sum of the degres of the vertices in $H$,
$\frac12\frac{2k}{4k + 1} = 4k^2 + k$. Thus, the number of edges in $G$ is
$4k^2 + k - (k - 1) = 4k^2 + 1$, so that the number of vertices in $K$ is
$4k^2 + 1$.
By Theorem 10.5, then,
\[\chi(K)
 \geq \frac{|V(K)|}{\alpha(K)}
 = \frac{4k^2 + 1}{2k}
 > \frac{4k^2}{2k} = 2k
.\]
Thus, since a proper vertex coloring of $K$ corresponds exactly to a proper
edge coloring of $G$, $\chi_1(G) > 2k$. Thus, since Vizing's Theorem implies
that $\chi_1(G) = \Delta(G)$ or $\chi_1(G) = \Delta(G) + 1$ and
$\Delta(G) = 2k$, $\chi_1(G) = 2k + 1 = \Delta(G) + 1$. \qed
\end{question}

\begin{question}{Problem 7}
Suppose, for sake of contradiction that $G$ is planar. Let $H$ be a plane
drawing of $G$. Suppose $v$ and $u$ are vertices in $H$ that are also in $K_5$
or $K_{3,3}$ (as appropriate). Replace the vertices and edges in the path from
$u$ to $v$ (i.e., the path that was added in the subdivision) with a single
edge that traverses that path. Doing this for each pair of vertices $u$ and
$v$ that were in $K_5$ or $K_{3,3}$ (as appropriate) gives a plane drawing of
either $K_5$ or $K_{3,3}$, respectively. However, since both $K_5$ and
$K_{3,3}$ are not planar, this is a contradiction. \qed
\end{question}
\end{document}
