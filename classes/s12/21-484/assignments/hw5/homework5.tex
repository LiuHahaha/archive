\documentclass[11pt]{article}
\usepackage{enumerate}
\usepackage{fullpage}
\usepackage{fancyhdr}
\usepackage{tikz}
\usepackage{amsmath, amsfonts, amsthm, amssymb}
\setlength{\parindent}{0pt}
\setlength{\parskip}{5pt plus 1pt}
\pagestyle{empty}

\def\indented#1{\list{}{}\item[]}
\let\indented=\endlist

\newcounter{questionCounter}
\newcounter{partCounter}[questionCounter]
\newenvironment{question}[2][\arabic{questionCounter}]{%
    \setcounter{partCounter}{0}%
    \vspace{.25in} \hrule \vspace{0.5em}%
        \noindent{\bf #2}%
    \vspace{0.8em} \hrule \vspace{.10in}%
    \addtocounter{questionCounter}{1}%
}{}
\renewenvironment{part}[1][\alph{partCounter}]{%
    \addtocounter{partCounter}{1}%
    \vspace{.10in}%
    \begin{indented}%
       {\bf (#1)} %
}{\end{indented}}

%%%%%%%%%%%%%%%%%%%%%%%%%%%%%%%%%%%%%%%%%%%%%%%%%%%%%%%%%%%
\newcommand{\myname}{Shashank Singh}
\newcommand{\myandrew}{sss1@andrew.cmu.edu}
\newcommand{\myclass}{21-484A Graph Theory}
\newcommand{\myhwnum}{5}
\newcommand{\duedate}{Friday, May 4, 2012}
%%%%%%%%%%%%%%%%%%%%%%%%%%%%%%%%%%%%%%%%%%%%%%%%%%%%%%%%%%%

%%%%%%%%%%%%%%%%%%%%%%%%MACROS%%%%%%%%%%%%%%%%%%%%%%%%%%%%%
\renewcommand{\qed}{\quad $\blacksquare$}
\newcommand{\mqed}{\quad \blacksquare}
\newcommand{\diam}{\operatorname{diam}}
%%%%%%%%%%%%%%%%%%%%%%%%%%%%%%%%%%%%%%%%%%%%%%%%%%%%%%%%%%%

\begin{document}
\thispagestyle{plain}

{\Large Homework \myhwnum} \\
\myclass \\
Name: \myname \\
Email: \myandrew \\
Due: \duedate
\begin{question}{Problem 1}
Let $G$ be a graph whose vertices are all possible states of the system of
lights and the gnome (including the position of the gnome), with an edge from
vertex $u$ to vertex $v$ if and only if it is possible for the system to go
from state $u$ to state $v$ in a single step. Since there are a finite number
of states, some set of states must repeat after a finite number of steps, so
that the current state will eventually traverse a cycle $C$ in $G$ indefinitely.
Note that, from any state, we can determine the previous step (the gnome must
have been at the previous light, and, depending on whether the previous light
is on or off, the current light must have been in the same or opposite state,
and no other lights were changed). Therefore, the in-degree of each vertex is
$1$, so that, if a vertex $v$ is in a cycle, then the previous state (the
vertex with an edge to $v$) is also in that cycle. Consequently, once the
current state has entered $C$, it will eventually return to any vertex it
visited previously, including the start vertex (that in which the gnome is at
$1$ and all the lights are on). \qed
\end{question}

\begin{question}{Problem 4}
{\bf Lemma:} If $G$ is outerplanar, cyclic, and biconnected, the $G$ contains
a vertex of degree $2$.

{\bf Proof:}
Since $G$ is planar, let $G^*$ be the dual of $G$, and let
$G^{\prime}$ be the graph produced by removing from $G^*$ the vertex
representing the unbounded region of $G$ (noting that, since $G$ is cyclic,
$G^{\prime}$ is nonempty). Suppose, for sake of contradiction, that
$G^{\prime}$ contained a cycle. Then, there is a vertex $v$ in $G$
corresponding to the face surrounded by that cycle. But then, $v$ must be
surrounded by bounded regions in $G$, contradicting the fact that $G$ is
outerplanar. Therefore, $G^{\prime}$ contains no cycle, so that its connected
components are trees and it has a leaf $l$ (or a vertex $l$ of degree $0$).
The face $f$ in $G$ corresponding
to $l$ is adjacent to at most $1$ bounded region $r$. Then, the vertex $u$ on
the boundary of $f$ that is not on the boundary of $r$ has no neighbors not on
the boundary of $f$ (if it did, then removing $u$ would disconnect those
vertices them, contradicting the fact that $G$ is biconnected). Thus,
$\deg(u) = 2$, proving the lemma. \qed

We proceed to prove that all outerplanar graphs $G$ are $3$-colorable by
strong induction on the number of vertices in $G$. If $G$ has fewer than $4$
vertices, $G$ is trivially $3$-colorable. Suppose, as an inductive hypothesis,
that, for some $n \in \mathbb{N}$, $\forall k \in \mathbb{N}$ with $k \leq n$,
all outerplanar graphs on $k$ vertices are $3$-colorable. Let $G$ be an
outerplanar graph on $n + 1$ vertices. If $G$ is acyclic, then its
connected components are trees, so that they are $3$-colorable. If
$G$ is not biconnected, then each biconnected component of $G$ is an
outerplanar graph on at most $n$ vertices, so that, by the inductive
hypothesis, each biconnected component of $G$ is $3$-colorable; the colorings
of each biconnected component of $G$ can then be merged (as in the general
proof of Brooks' Theorem) to construct a proper $3$-coloring of $G$. If $G$ is
outerplanar, cyclic, and biconnected, then, by the above lemma, $G$ has a
vertex $v$ of degree $2$. Since removing a vertex preserves outerplanarity, by
the inductive hypothesis, the graph $K = G - \{v\}$ can be $3$-colored.
Therefore, $G$ can be $3$-colored by using the $3$-coloring from $K$, and then
coloring $v$ differently from its neighbors. \qed
\end{question}
\end{document}
