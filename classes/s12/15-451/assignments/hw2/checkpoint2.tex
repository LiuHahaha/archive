\documentclass[11pt]{article}
\usepackage{enumerate}
\usepackage{fullpage}
\usepackage{fancyhdr}
\usepackage{amsmath, amsfonts, amsthm, amssymb}
\setlength{\parindent}{0pt}
\setlength{\parskip}{5pt plus 1pt}
\pagestyle{empty}

\def\indented#1{\list{}{}\item[]}
\let\indented=\endlist

\newcounter{questionCounter}
\newcounter{partCounter}[questionCounter]
\newenvironment{question}[2][\arabic{questionCounter}]{%
    \setcounter{partCounter}{0}%
    \vspace{.25in} \hrule \vspace{0.5em}%
        \noindent{\bf #2}%
    \vspace{0.8em} \hrule \vspace{.10in}%
    \addtocounter{questionCounter}{1}%
}{}
\renewenvironment{part}[1][\alph{partCounter}]{%
    \addtocounter{partCounter}{1}%
    \vspace{.10in}%
    \begin{indented}%
       {\bf (#1)} %
}{\end{indented}}

%%%%%%%%%%%%%%%%% Identifying Information %%%%%%%%%%%%%%%%%
%% This is here, so that you can make your homework look %%
%% pretty when you compile it.                           %%
%%     DO NOT PUT YOUR NAME ANYWHERE ELSE!!!!            %%
%%%%%%%%%%%%%%%%%%%%%%%%%%%%%%%%%%%%%%%%%%%%%%%%%%%%%%%%%%%
\newcommand{\myname}{Shashank Singh}
\newcommand{\myandrew}{sss1@andrew.cmu.edu}
\newcommand{\myhwname}{Checkpoint 2}
\newcommand{\myrecitation}{1}
%%%%%%%%%%%%%%%%%%%%%%%%%%%%%%%%%%%%%%%%%%%%%%%%%%%%%%%%%%%

\begin{document}
\thispagestyle{plain}

\begin{center}
{\Large \myhwname} \\
\myname \\
\myandrew \\
\myrecitation \\
\today
\end{center}
\begin{question}{Finding Well Spaced Triples of Ones}
\begin{enumerate}[(a)] 
\item The following pseudocode gives an algorithm which returns the triple
\texttt{(j,i,k)} with the desired properties if such a triple exists, and
returns \texttt{none} otherwise.
    \begin{verbatim}
    for i = 0,1,..,(n - 1)
        for j = 0,1,..,(i - 1)
            let k = 2*i - j
            if(k < n)
                if(a_j * a_i * a_k == 1)
                    return (j,i,k)
    return none
    \end{verbatim}

\item For even $t \in \{0,1,\ldots, 2n - 2\}$, $P_t$ is odd if and only if
$a_{t/2} = 1$, and $P_t > 1$ if and only if, for some pair $(j,k)$ with
$|k - t/2| = |t/2 - j| \geq 1$, $a_j = a_k = 1$. Thus, $P_t$ is odd and
greater than $1$ if and only if $(a_j,a_{t/2},a_k)$ is a triple of ones with
the desired properties, In particular, in this case, $\frac{P_t - 1}{2}$ gives
the number of triples of ones centered around $a_{t/2}$. For odd $t$, $P_t$
tells us nothing relevant to the problem at hand. \\

\item Compute the convolution $P$ of $S$ with itself (this takes $O(n \log n)$
time using the discrete Fourier transform as discussed in lecture). Initialize
a counter $C$ at zero. For $t \in \{0,2,4,\ldots,2n - 2\}$, if $P_t \% 2 == 1$
and $P_t > 1$, then increment $C$ by $\frac{t - 1}{2}$. Then, return $C$. \\

\item For $i \in \{0,2,4,\ldots,2n - 2\}$, if $P_i > 1$ and $P_i$ is odd,
iterate over $j \in \{i/2,\ldots,1,0\}$. If $a_j, a_{i/2}$, and $a_{i - j}$
are all $1$, then return the triple $(j,i/2,i - j)$. If the outer loop
terminates without returning a triple, then no such triple exists, so return
none. Note that, since $P_i > 1$ and then inner loop is will return before it
goes through all of its iterations, so that the algorithm's runtime is $O(n)$.
\end{enumerate}
\end{question}
\begin{question}{Matrix Transposition In-Place}
\begin{enumerate}[(a)]
\item Divide input matrix $T$ of size $n \times n$ into four roughly square
blocks $T_1,T_2,T_3,T_4$, where $T_1$ is the upper left quadrant of $T$, $T_2$
is the upper right quadrant, $T_3$ is the lower left quadrant of $T$, and
$T_4$ is the lower right quadrant of $T$. Let $C_1, C_2, C_3, C_4$ be the
corresponding quadrants of the $T^T,$ the transpose of $T$. Then,
$C_1 = T_1^T$, $C_2 = T_3^T$, $C_3 = T_2^T$, and $C_4 = T_4^T$, so that $T^T$
can be computed recursively by computing the transposes of four matrices of
size $\frac{n}{2} \times \frac{n}{2}$. Both dividing the input matrix into
four submatrices and combining the four transposes of submatrices into the
output matrix can be done in constant time. Thus, the runtime of this
algorithm is given by the recurrence \[T(1) = 1, T(n) = 4T(n/2) + k,\]
for some positive constant $k$. By the Master Method, $T(n) \in \Theta(n^2)$,
as desired. \\


\item Rather than modifying the structure of the matrix, whenever prompted for
an entry in the matrix, say $a_{i,j}$, the entry in the $j^{th}$ column of the
$i^{th}$ row of the matrix, return the corresponding entry in the transpose of
the matrix, $a_{j,i}$. Thus, no work is done initially, and only constant time
work is done when a value in the matrix is read.
\end{enumerate}
\end{question}
\end{document}
