\documentclass[11pt]{article}
\usepackage{enumerate}
\usepackage{fullpage}
\usepackage{fancyhdr}
\usepackage{amsmath, amsfonts, amsthm, amssymb}
\setlength{\parindent}{0pt}
\setlength{\parskip}{5pt plus 1pt}
\pagestyle{empty}

\def\indented#1{\list{}{}\item[]}
\let\indented=\endlist

\newcounter{questionCounter}
\newcounter{partCounter}[questionCounter]
\newenvironment{question}[2][\arabic{questionCounter}]{%
    \setcounter{partCounter}{0}%
    \vspace{.25in} \hrule \vspace{0.5em}%
        \noindent{\bf #2}%
    \vspace{0.8em} \hrule \vspace{.10in}%
    \addtocounter{questionCounter}{1}%
}{}
\renewenvironment{part}[1][\alph{partCounter}]{%
    \addtocounter{partCounter}{1}%
    \vspace{.10in}%
    \begin{indented}%
       {\bf (#1)} %
}{\end{indented}}

%%%%%%%%%%%%%%%%%%%%%%%HEADER%%%%%%%%%%%%%%%%%%%%%%%%%%%%%%
\newcommand{\mynamek}{Kenny Lu}
\newcommand{\myandrewk}{krlu@andrew.cmu.edu}
\newcommand{\mynamem}{Mike Wehar}
\newcommand{\myandrewm}{mwehar@andrew.cmu.edu}
\newcommand{\mynames}{Shashank Singh}
\newcommand{\myandrews}{sss1@andrew.cmu.edu}
\newcommand{\myhwname}{Assignment 7}
\newcommand{\myhwnum}{10}
\newcommand{\duedate}{Friday, May 4, 2012}
%%%%%%%%%%%%%%%%%%%%%%%%%%%%%%%%%%%%%%%%%%%%%%%%%%%%%%%%%%%

%%%%%%%%%%%%%%%%%%%%%%%%MACROS%%%%%%%%%%%%%%%%%%%%%%%%%%%%%
\renewcommand{\qed}{\quad $\blacksquare$}
%%%%%%%%%%%%%%%%%%%%%%%%%%%%%%%%%%%%%%%%%%%%%%%%%%%%%%%%%%%

\begin{document}
\thispagestyle{plain}

\begin{center}
{\Large 15-451 \myhwname} \\
\begin{tabular}{ccc}
\mynamek & \mynamem & \mynames \\
\myandrewk & \myandrewm & \myandrews
\end{tabular}

\today
\end{center}
\begin{question}{Parallel 2-Dimensional Linear Programming}
\begin{enumerate}[(a)]
\item Given an array of n elements, Algorithm will proceed as follows: 

\begin{enumerate}
\item Spliting the array:

\begin{enumerate}[(i)]
\item Split array down the middle, dividing into subarrays of roughly equal
size
\item Recursively split each subarray with step 1, using separate processors
for each subarray. Stop when the subarray is a singleton. Final result should
be a delimited list of singletons

\end{enumerate}
\item Finding max/min element in result by comparisons:

\begin{enumerate}[(i)]
\item Merge adjacent singletons computed, and output the maximum of the two as
a new singleton. 
\item For each pair of singletons, use separate processors for each pair of
elements to perform step3, once again, we recursively compute the maximum of
two elements and stop when array of comparables is a singleton.
\end{enumerate}
\end{enumerate}

Amount of time is $O(\log n)$ since as each level of the recursion, we are
doing constant work, and the depth of the recursion tree is $O(\log n)$.\\
Amoung of work done is O n)  since then number of split and the number of
comparisons are both at most n.  

For solving 1-D LP problems, we want to maximize
$Cx, C\in\mathbb{R}, x\in \mathbb{R}^n$, subject to the constraints:\\
$a_ix \le b_i$ for each $ i\in\left\{1,.....n\right\}$ and $x \ge 0$. \\
The constraints imply $x \le \displaystyle\frac{b_i}{a_i}$, in which case 
we want to find $i \in\left\{1,.....n\right\}$ such that
$\displaystyle\frac{b_i}{a_i}$, we can do this using the above algorithm,
in O(logn) time and O(n) work. All that remains is to consider the following
cases: \\
if( $ \displaystyle\frac{b_i}{a_i}$) the result is in the infeasible, hence no
solution. \\
else if($c > 0$) $x =  \displaystyle\frac{b_i}{a_i}$  \\  
else if($c< 0$) $x = 0$\\ \qed

\item Consider a point P in our bounding box. For point P to be an optimal
point for our LP solution, it must be on the intersection of two half planes.

For the k-th half plane to have resulted in an unsatisfied constraint, the
k-th cut must intersect point P, because that would mean point P was
originally on the corner of a different region defined by the intersection of
the previous k-1 half-planes.

Following directly from the LP construction of P's position on the corner of
the feasible region, the probability for any single cut to intersect P is
$1/k$. Thus for a point P on the intersection of 2 half planes, it is $2/k$,
since only 1 needs to have given an unsatisfied constraint. \\

We then take the sum $\displaystyle\sum\limits_{k=0}^n \frac{2}{k} = 2$($k$th
harmonic number) = O(2*logn) $\in$ O(logn).\\ \qed

\item Given a list of $n$ half-planes and a point P (as in part (b)). Suppose
we have evaluated the intersection of the first $k$ half planes (and their
associated cuts). \\
We take the remaining $n-k$ half planes, and we cut the subarray in half,
with the two resultant subarrays have start and end indices
$\displaystyle( n-k,  \lfloor n - \frac{n-k}{2}) \rfloor$ and
$\displaystyle(\lfloor n - \frac{n-k}{2}, n) \rfloor$. We then use separate
processors for each recursive split on those subarrays. We stop splititng
until we have singleton arrays (similar to the splitting in part (a)). \\
We then evaluate the singletons in parallel using the processors we've called
during the recursive step. For each pair of singletons, we pass up to the next
step the lowest indexed (supposing we have list ranking or an indexed array)
cut that results an unsatisfied constraint, and recursively compare them each
above step until there is only 1 singleton S left. If P does not violate its
constaint, then there are no half-planes whose constraints are unsatisfied.
Otherwise, if P does violate constraint S, then index of S be the lowest of
all unsatisfied cuts in subarray $(n-k+1,n)$, making the half-plane of S the
first for which P violates its contraints. 

\item After building our list of n half-planes, we must fine each one that
results. Letting P be a point within our bounding box, we observe that P will
be outside the feasible region O(log n) times given n cuts across the bounding
box, by part (b). \\
Thus for each cut that does this, we search our list using the algorithm in
part(d), and then, use the algorithm in part(a) to solve the resulting 1-D LP
problem.

In total, we end up with a runtime of
$O(\log n \cdot (\log n + \log n)) = O(\log^2 n)$ in parallel, and work of
$O(logn \cdot (n+n)) = O(n \log n)$ operations. \\
\end{enumerate}
\end{question}

\newpage
\begin{question}{Maximum Independent Set in a Tree}
\begin{enumerate}[(a)]
\item Let $S_1$ be an independent set in $T$, and suppose $v$ is a leaf in $T$
with $v \not \in S_1$. Let $S_2 = (S_1 \cup \{v\})\backslash(N(v))$, where
$N(v)$ denotes the set of neighbors of $v$. Note that, since $v$ is a leaf,
$|N(v)| = 1$, so that, since $v \not \in S_1$, $|S_2| = |S_1|$. Since $S_2$
contains no neighbors of $v$ and $S_2\backslash\{v\} \subseteq S_1$
$S_2$ is independent. Assuming $T$ has more than two vertices, and is
connected, the neighbor of $v$ cannot itself be a leaf. Thus, for any maximum
independent set of $T$ not containing all the leaves of $T$, there is a
maximum independent of $T$ containing strictly more leaves of $T$, so that
there exists a maximum independent set containing all leaves of $T$. \qed

\item Beginning with the lowest level of the tree, iteratively perform the
following:
\begin{enumerate}
\item Add any leaves in the level to the independent set.

\item For each vertex $v$ that is not a leaf in the current level, if $v$ has
no child in $I$, add $v$ to $I$.

\item Repeat steps $1$ and $2$ on the level of the parents of the vertices in
the current level unless the current level is the root (in which case,
terminate).
\end{enumerate}
We claim, $I$ will be a maximum independent set. That $I$ is independent is
clear. To see that $I$ is maximum, consider a maximum independent set $S$
containing all the leaves of $T$ (such an $S$ much exist by the result of part
(a)). If $S \neq I$, then consider a lowest (farthest from the root) vertex
$v$ such that $v \in S$ but $v \not \in I$. Since $v \not \in I$, one of its
children $u$ must be in $I$. By choice of $v$, however, $u \in I$,
contradicting the fact that $S$ is independent. Therefore, no such $v$ can
exist so that $S \subseteq I$, and $|S| \leq |I|$. \qed

By implementing the algorithm as a post-order traversal, where constant time
(the time to check whether any child of the current vertex is in $v$) is spent
at each vertex, the algorithm runs in linear time, as desired. \qed

\item We give a constant-time linear-work reduction to a similar problem $P$
concerning arithmetic expression trees presented in lecture. Label each leaf
in the tree with a $1$, and every other vertex as
follows: if the vertex has one child, assign it the expression $(1 - c_1)$,
where $c_1$ is the value of that child; if the vertex has two children, assign
it the expression $(1 - c_1)(1 - c_2)$, where $c_1$ and $c_2$ are the the
values of its children. Evaluate the expression tree using tree contraction as
in lecture. Then, include a vertex in the maximum independent set $I$ if and
only if the value at that vertex is a $1$.

Since $(1 - c_1)(1 - c_2)$ (or $(1 - c_1))$ in the case of one child
node) is $1$ if and only if $c_1$ and $c_2$ are $0$, a vertex is in $I$
if and only if neither of its children is. Thus, the algorithm returns the
same tree as that returned by the algorithm
given in part (b) above, so that the same proof of correctness applies.
Since the reduction is constant-time linear-work, and the solution to $P$ runs
in $O(\log n)$ time with $O(n)$ work, the algorithm runs in
$O(1 + \log n) = O(\log n)$ with $O(n + n) = O(n)$ work, as desired. \qed

\end{enumerate}
\end{question}

\newpage
\begin{question}{NP-Completeness}
\begin{enumerate}[(a)]
\item Given a $3$-$CNF$ fomula $F$ = $C_{1} \wedge ... \wedge C_{n}$ and an
assignment $A$, if $A$ is a $\neq$-assignment for $F$ then $\neg A$ is a
$\neq$-assignment for $F$.

\paragraph{}
Proof.  If $A$ is a $\neq$-assignment for $F$, then for each $i$, $A$ assigns true to at least one literal in $C_{i}$ and false to at least one literal in $C_{i}$.  Therefore, for each $i$, $\neg A$ assignes true to at least one literal in $C_{i}$ and false to at least one literal in $C_{i}$.  Hence, $\neg A$ is a $\neq$-assignment for $F$.
\\\\
\item Claim: $\neq$-$SAT$ is $NP$-$Complete$.
\paragraph{}
Proof.  $\neq$-$SAT$ is in $NP$ since given an assignment $A$, we can check if $A$ is a $\neq$-assignment in linear time.

\paragraph{}
Now, it remains to show that $\neq$-$SAT$ is $NP$-$Hard$.  We will do this by reducing $3$-$SAT$ to $\neq$-$SAT$.  Let a $3$-$CNF$ formula $F$ = $C_{1} \wedge ... \wedge C_{n}$ be given.  Define $F^{\prime}$ = $D_{1} \wedge ... \wedge D_{n}$ such that $D_{i}$ $=$ $(y_{1} \vee y_{2} \vee z_{i})$ $\wedge$ $(\bar{z_{i}} \vee y_{3} \vee b)$ where $C_{i}$ $=$ $(y_{1} \vee y_{2} \vee y_{3})$ for each $i$.

\paragraph{}
Subclaim: $F$ has a satisfying assignment iff $F^{\prime}$ has a $\neq$-assignment.

\paragraph{}
$\rightarrow$:  Suppose that $F$ has a satisfying assignment $A$.  We will now construct a $\neq$-assignment $A^{\prime}$ for $F^{\prime}$.  (1) $A^{\prime}$ will assign the same values to variables in $F$ as $A$.  (2) $A^{\prime}$ assigns false to b.  (3) For each $i$, we will split into cases to define $z_{i}$.  Let $y_{1}$, $y_{2}$, and $y_{3}$ so that $D_{i}$ $=$ $(y_{1} \vee y_{2} \vee z_{i})$ $\wedge$ $(\bar{z_{i}} \vee y_{3} \vee b)$.  We assign true to $z_{i}$ if $A$ assigns true to $y_{3}$ and false to $y_{1}$ and $y_{2}$.  Otherwise, we assign false to $z_{i}$.  Therefore, for each clause in $F^{\prime}$, $A^{\prime}$ assigns true to at least one literal and false to at least one literal.  Hence, $A^{\prime}$ is a $\neq$-assignment for $F^{\prime}$.

\paragraph{}
$\leftarrow$:  Suppose that $F^{\prime}$ has a $\neq$-assignment $A^{\prime}$.  By part (a), $\neg A^{\prime}$ is a $\neq$-assignment for $F^{\prime}$.  Hence, either $A^{\prime}$ or $\neg A^{\prime}$ is a $\neq$-assignment that assigns false to b.  WLOG $A^{\prime}$ assigns false to b.  Define an assignment $A$ for $F$ such that $A$ agrees with $A^{\prime}$ on variables of $F$.  Now, I will show that $A$ is a satisfying assignment for $F$.  To do this, we just need to show that $C_{i}$ is satisfied for each $i$.  Let $y_{1}$, $y_{2}$, and $y_{3}$ so that $D_{i}$ $=$ $(y_{1} \vee y_{2} \vee z_{i})$ $\wedge$ $(\bar{z_{i}} \vee y_{3} \vee b)$ and $C_{i}$ $=$ $(y_{1} \vee y_{2} \vee y_{3})$.  Since $A^{\prime}$ is a $\neq$-assignment for $F^{\prime}$, it assigns true to $D_{i}$.  And, since b is assigned false, either $y_{1}$, $y_{2}$, or $y_{3}$ is assigned true.  Therefore, $A$ assigns true to $C_{i}$.  It follows that $A$ is a satisfying assignment for $F$.
\\\\
\item Claim: $MAXCUT$ is $NP$-$Complete$.

\paragraph{}
$MAXCUT$ is in $NP$ since given a cut of a graph $G$ and given a number $k$, we can check if $k$ or more vertices are in the cut in quadratic time.

\paragraph{}
Now, it remains to show that $MAXCUT$ is $NP$-$Hard$.  We will do this by reducing $\neq$-$SAT$ to $MAXCUT$.  Let a $3$-$CNF$ formula $F$ be given.  Construct the graph $G$ that is described in the problem hint.

\paragraph{}
Subclaim: $F$ has a $\neq$-assignment iff $(G,r)$ $\in$ $MAXCUT$ where $r$ $:=$ $2k + 9k^{2}n$, k is the number of clauses, and n is the number of variables.

\paragraph{}
$\rightarrow$: Suppose that $F$ has a $\neq$-assignment $A$.  Consider the cut associated with $A$, where each literal that is assigned true is in one set and each literal that is assigned false is in the other.  In total, there are $3k + 9k^{2}n$ edges where $9k^{2}n$ edges are adjacent to variables and their negations, and $3k$ edges are associated with clauses.  The cut associated with $A$ has the $9k^{2}n$ edges since each variable and its negation are separated.  The cut has 2k additional edges because there is one literal from each clause triange in each set.  (Each clause triangle adds 2 edges.)  Therefore, the cut has r edges and hence $(G,r)$ $\in$ $MAXCUT$.

\paragraph{}
$\leftarrow$: Suppose that $(G,r)$ $\in$ $MAXCUT$.  Choose a cut $(S,T)$ with at least r edges.  This cut must include the $9k^{2}n$ edges that are adjacent to variables and their negations.  Otherwise, it would have at most $9k^{2}(n-1) + 3k$ edges which is less than r.  Therefore, this cut separates variables and their negations and hence is associated with an assignment $A$.  Since the cut can have at most 2 edges per clause triangle and the cut has $2k$ more edges, there must be exactly two edges from each clause triange in the cut.  Therefore, for each clause, $A$ assigns true to at least one literal and false to at least one literal.  It follows that $A$ is a $\neq$-assignment for $F$.
\end{enumerate}
\end{question}
\end{document}
