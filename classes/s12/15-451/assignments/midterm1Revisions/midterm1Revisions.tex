\documentclass[11pt]{article}
\usepackage{enumerate}
\usepackage{fullpage}
\usepackage{fancyhdr}
\usepackage{amsmath, amsfonts, amsthm, amssymb}
\setlength{\parindent}{0pt}
\setlength{\parskip}{5pt plus 1pt}
\pagestyle{empty}

\def\indented#1{\list{}{}\item[]}
\let\indented=\endlist

\newcounter{questionCounter}
\newcounter{partCounter}[questionCounter]
\newenvironment{question}[2][\arabic{questionCounter}]{%
    \setcounter{partCounter}{0}%
    \vspace{.25in} \hrule \vspace{0.5em}%
        \noindent{\bf #2}%
    \vspace{0.8em} \hrule \vspace{.10in}%
    \addtocounter{questionCounter}{1}%
}{}
\renewenvironment{part}[1][\alph{partCounter}]{%
    \addtocounter{partCounter}{1}%
    \vspace{.10in}%
     \begin{indented}%
       {\bf (#1)} %
}{\end{indented}}

%%%%%%%%%%%%%%%%% Identifying Information %%%%%%%%%%%%%%%%%
%% This is here, so that you can make your homework look %%
%% pretty when you compile it.                           %%
%%     DO NOT PUT YOUR NAME ANYWHERE ELSE!!!!            %%
%%%%%%%%%%%%%%%%%%%%%%%%%%%%%%%%%%%%%%%%%%%%%%%%%%%%%%%%%%%
\newcommand{\myname}{Shashank Singh}
\newcommand{\myandrew}{sss1@andrew.cmu.edu}
\newcommand{\myhwname}{Midterm 1 Revisions}
\newcommand{\myrecitation}{1}
%%%%%%%%%%%%%%%%%%%%%%%%%%%%%%%%%%%%%%%%%%%%%%%%%%%%%%%%%%%

\begin{document}
\thispagestyle{plain}

{\Large \myhwname} \\
\myname \\
\myandrew \\
\myrecitation \\
\today
\begin{question}{Problem 1}
\begin{enumerate}[(a)]
\item $T(n) \in \Theta(n^2)$

\item $T(n) \in \Theta(n)$

\item The best possible algorithm will run in $\Theta(n)$ time.

\item Let $P_X,P_Y$ be polynomials with
\[P_X = \sum_{i = 0}^{2M} a_i x^i,
P_Y = \sum_{i = 0}^{2M} b_i x^i,\]
where $\forall i \in \{0,1,\ldots,2M\}$, $a_i = 1$ if $i - M \in X$ and
$a_i = 0$ otherwise, and $b_i = 1$ if $i - M \in Y$ and $b_i = 0$ otherwise.
Compute $P = P_XP_Y$ in $O(M \log M)$ time using a Fast Fourier Transform.
Then, $\forall i \in \{0,1,\ldots,4M^2\}$, let $c_i$ be the coefficient of
$x^i$ in $P$, so that $c_i = 1$ if $i - 2M \in X + Y$, and $c_i = 0$
otherwise. Thus, the elements of $X + Y$ can be read in linear time from the
coefficients of $P$, so that $X + Y$ is computed in $O(M \log M)$ time.

\item See written test.
\end{enumerate}
\end{question}

\begin{question}{Problem 2}
\begin{enumerate}[(a)]
\item See written test.

\item Let $\Phi(n)$ be the sum of the token values in all of the arrays,
i.e. $\Phi = \sum_{i = 0}^{\log_2 n - 1} i * (\log n - i)$.

\item The amortized cost of inserting the new unit size array is a constant
actual time, plus $\log n$ time for the change in the potential function,
giving $O(\log n)$ amortized time for inserting the new unit size array.

\item The amortized cost of merging two arrays of size $2^i$ is $2^i$ time for
actually merging the arrays, plus
$2^{i + 1} (\log n - (i + 1)) - 2 * 2^i (\log n - i) = 2^{i + 1}$ time for the
change in potential, giving $O(2^i)$ amortized time for merging two arrays of
size $2^i$.

\item Since in $n$ insertions, we add about $n * \log n$ stored tokens, and
the amortized cost of an insert is, in the worst case $2^{\log_2 n} = n$, the
average cost is at most $\frac{n \log n}{n} \in O(\log n)$.
\end{enumerate}
\end{question}
\end{document}
