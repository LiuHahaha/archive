\documentclass[11pt]{article}
\usepackage{enumerate}
\usepackage{fullpage}
\usepackage{fancyhdr}
\usepackage{amsmath, amsfonts, amsthm, amssymb}
\setlength{\parindent}{0pt}
\setlength{\parskip}{5pt plus 1pt}
\pagestyle{empty}

\def\indented#1{\list{}{}\item[]}
\let\indented=\endlist

\newcounter{questionCounter}
\newcounter{partCounter}[questionCounter]
\newenvironment{question}[2][\arabic{questionCounter}]{%
    \setcounter{partCounter}{0}%
    \vspace{.25in} \hrule \vspace{0.5em}%
        \noindent{\bf #2}%
    \vspace{0.8em} \hrule \vspace{.10in}%
    \addtocounter{questionCounter}{1}%
}{}
\renewenvironment{part}[1][\alph{partCounter}]{%
    \addtocounter{partCounter}{1}%
    \vspace{.10in}%
    \begin{indented}%
       {\bf (#1)} %
}{\end{indented}}

%%%%%%%%%%%%%%%%% Identifying Information %%%%%%%%%%%%%%%%%
%% This is here, so that you can make your homework look %%
%% pretty when you compile it.                           %%
%%     DO NOT PUT YOUR NAME ANYWHERE ELSE!!!!            %%
%%%%%%%%%%%%%%%%%%%%%%%%%%%%%%%%%%%%%%%%%%%%%%%%%%%%%%%%%%%
\newcommand{\myname}{Shashank Singh}
\newcommand{\myandrew}{sss1@andrew.cmu.edu}
\newcommand{\myhwname}{Homework 1}
\newcommand{\myrecitation}{B}
%%%%%%%%%%%%%%%%%%%%%%%%%%%%%%%%%%%%%%%%%%%%%%%%%%%%%%%%%%%

\begin{document}
\thispagestyle{plain}

\begin{center}
{\Large \myhwname} \\
\myname \\
\myandrew \\
\myrecitation \\
 \today
\end{center}
\begin{question}{Review}
\begin{enumerate}[(a)]
\item Note that, $\forall$ non-zero $x \in (-1,1)$,
\begin{eqnarray*}
\sum_{k = 0}^{\infty} (k + 1) x^k
 & = & \sum_{k = 0}^{\infty} \frac{d}{dx} \left( x^{k + 1} \right) \\
 & = & \frac{d}{dx} \sum_{k = 0}^{\infty} \left( x^{k + 1} \right) \\
 & = & \frac{d}{dx} \left( \left( \sum_{k = 0}^{\infty} x^{k} \right)
                                                                - 1 \right) \\
 & = & \frac{d}{dx} \left( \frac{1}{1 - x} - 1 \right) \\
 & = & (1 - x)^{-2}
\end{eqnarray*}

Letting $x = \frac{1}{2}$ in the above identity gives
\[\sum_{k = 0}^{\infty} \frac{k + 1}{2^k} = \left( 1 - \frac{1}{2} \right)
^{-2} = \mbox{\fbox{$4$.}}\]
 
\item $\forall n \in \mathbb{N}$,
\[\sum_{k = 0}^n 2^{k + 1} = \sum_{k = 1}^{n + 1} 2^{k}
 = \left( \sum_{k = 0}^{n + 1} 2^k \right) - 1 = \mbox{\fbox{$2^{n + 2} - 2$.}}\]
\end{enumerate}
\end{question}
\begin{question}{Asymptotic Notations}
\begin{enumerate}[(a)]
\item
    \begin{enumerate}[i.]
    \item $2n^3 + 25n^2 + \log n \in \Theta(n^3)$.
    \item $\log_4(n^3) \in \Theta(\log n)$.
    \item $4^{\log_8 n} \in \Theta(n^{2/3})$.
    \item $\log_{2n} \left( n^{3n} \right) \in \Theta(n)$.
    \end{enumerate}
\item
    \begin{enumerate}[i.]
    \item For $f(n) = n^2$, $g(n) = 4n^2 + 3n \log n$, $f \in \Theta(g)$.
    \item For $f(n) = n^{15}$, $g(n) = 3^n$, $f \in o(g)$.
    \item For $f(n) = \log \left( \sqrt{n} \right)$,
          $g(n) = \log \left( n^{12} \right)$, $f \in \Theta(g)$.
    \item For $f(n) = 2^{\log_3 n}$, $g(n) = 3^{\log_5 n}$, $f \in o(g)$.
    \end{enumerate}
\end{enumerate}
\end{question}
\begin{question}{Solving Recurrence Equations}
\begin{enumerate}[(a)]
\item For $a = b = 3$, $\log_b a = 1 > 0$. Thus, by the master method (in the
case where the leaves outweigh the root), $T(n) \in \Theta(n^{\log_b a}) =
 \Theta(n)$.
\item For $a = 3$, $b = 2$, $\log_b a > 1.58 > 0$. Thus, by the master
method (in the case where the leaves outweigh the root), $T(n) \in
 \Theta(n^{\log_b a}) = \Theta(n^{\log_2 3})$.
\item The recurrence tree appears as follows:

\begin{verbatim}
                              n                         = n
                            /   \
                          /       \
                        n/2   +   n/3                   = 5n/6
                        / \       / \
                       /   \     /   \
                     n/4 + n/6+n/6 + n/9                = 25n/36
                      .       .       .                    .
                      .       .       .                    .
                      .       .       .                    .
\end{verbatim}
Thus, for $i \in \{0, 1, \ldots, \log_3 n\}$, the sum of the terms in the
$i^{th}$ level of the tree is $(5/6)^i n$.
For $i \in \{(\log_3 n) + 1, \ldots, \log_2 n\}$, the sum of the
terms in the $i^{th}$ level of the tree is less than $(5/6)^i n$ (as the tree
continues to grow on the left but not on the right). Thus, for large $n$,
since $\lim_{n \rightarrow \infty} \log_2 n = \infty$,
\[T(n) \leq \sum_{i = 0}^{\log_2 n} \left( \frac{5}{6} \right)^i n =
n \left( \frac{1 - \left( \frac{5}{6}\right)^{\log_3 n}}{1 - \frac{5}{6}} \right)
 \approx n \left(\frac{1 - 0}{1 - \frac{5}{6}} \right) = 6n \in O(n).\]
Therefore, $T(n) \in O(n)$.

Clearly, since the root of the tree alone is $n$, $T(n) \geq n \in \Omega(n)$,
so that $T(n) \in \Omega(n)$.

It follows, then, that \fbox{$T(n) \in \Theta(n).$}
\end{enumerate}
\end{question}
\begin{question}{Strassen's Algorithm}
\begin{enumerate}[(a)]
\item $\forall n \in \mathbb{N}\backslash\{0\}$, let $T(n)$
denote the number of elementary additions and subtractions required to
multiply two $n \times n$ matrices using Strassen's Algorithm.
Each call requires $7$ recursive calls to problems of size $\frac{n}{2}$, and
$18$ additions of matrices with $\frac{n^2}{4}$ elements, giving the recurrence
\[T(1) = 1, T(n) = 7T\left(\frac{n}{2}\right) + 18\left(\frac{n^2}{4}\right).\]
Thus, for $i \in \{1, 2, \ldots, \log_2(n)\}$, the $i^{th}$ level of recursion
consists of $7^i$ calls to Strassen's algorithm, each doing
$\frac{9}{2} \left( \frac{n^2}{4^i} \right)$ non-recursive work. Therefore,
\[T(n) = \sum_{i = 0}^{\log_2 n} \frac{9}{2} n^2 \left(\frac{7}{4}\right)^i
 = \frac{9}{2} n^2 \left( \frac{\left( \frac{7}{4}\right)^{(\log_2 n) + 1} - 1}{\frac{7}{4} - 1} \right)
 = 6n^2 \left( \frac{7}{4}\left( \frac{7}{4}\right)^{(\log_2 (7/4))} - 1\right)\]
\[ = \mbox{\fbox{$\displaystyle \frac{21}{2} n^{2.81} - 6n^2$.}}\]
\item Let $k$ be the time taken in computing a single elementwise
multiplication. The time taken by Strassen's Algorithm is given by the
recurrence
\[S(1) = 1; S(n) = 8S\left(\frac{n}{2}\right) + \frac{9}{2}k,\]
the solution to which is $S(n) = 6k \left(n^{2.81} - n^2 \right)$.
The time taken by the naive algorithm is given by the recurrence
\[T(1) = 1; T(n) = 7T\left(\frac{n}{2}\right) + k,\]
the solution to which is $T(n) = k \left(n^3 - n^2 \right)$. Finding the
solutions to $S(n) = T(n)$ ($0$ and $1$) shows that Strassen's Algorithm has
fewer multiplications and thus better runtime for all non-trivial matrices.
\end{enumerate}
\end{question}
\begin{question}{Karatsuba's Algorithm}
$\forall n \in \mathbb{N}\backslash\{0\}$, let $T(n)$ denote the number of addition and
subtraction operations required to multiply two $n$-bit numbers using
Karatsuba's Algorithm.
Each call requires $3$ recursive calls to problems of size $\frac{n}{2}$,
$6$ additions of $\frac{n}{2}$-bit numbers, and $2$ additions of $n$-bit numbers, giving the recurrence
\[T(1) = 1,\; T(n) = 3T \left( \frac{n}{2} \right) + 4n.\]
Thus, for $i \in \{1, 2, \ldots, \log_2(n)\}$, the $i^{th}$ level of recursion
consists of $3^i$ calls to Karatsuba's algorithm, each doing
$4 \left( \frac{n}{2^i} \right)$ non-recursive work. Therefore, \[T(n)
 = \sum_{i = 1}^{\log_2 n} 4n \left( \frac{3}{2} \right)^i
 = 4n \left( \frac{ \left( \frac{3}{2} \right)^{\log_2 (n) + 1} - 1}{\frac{3}{2} - 1} \right)
 = \mbox{\fbox{$\displaystyle 8n \left( \left( \frac{3}{2} \right)^{\log_2 (n) + 1} - 1 \right)$.}} \]
\end{question}
\end{document}
