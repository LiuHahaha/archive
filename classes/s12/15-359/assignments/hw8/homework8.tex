\documentclass[11pt]{article}
\usepackage{enumerate}
\usepackage{fullpage}
\usepackage{fancyhdr}
\usepackage{amsmath, amsfonts, amsthm, amssymb}
\usepackage{tikz}
\setlength{\parindent}{0pt}
\setlength{\parskip}{5pt plus 1pt}
\pagestyle{empty}

\newcommand*\mycirc[1]{%
  \begin{tikzpicture}
    \node[draw,circle,inner sep=1pt] {#1};
  \end{tikzpicture}}


\def\indented#1{\list{}{}\item[]}
\let\indented=\endlist

\newcounter{questionCounter}
\newcounter{partCounter}[questionCounter]
\newenvironment{question}[2][\arabic{questionCounter}]{%
    \setcounter{partCounter}{0}%
    \vspace{.25in} \hrule \vspace{0.5em}%
        \noindent{\bf #2}%
    \vspace{0.8em} \hrule \vspace{.10in}%
    \addtocounter{questionCounter}{1}%
}{}
\renewenvironment{part}[1][\alph{partCounter}]{%
    \addtocounter{partCounter}{1}%
    \vspace{.10in}%
    \begin{indented}%
       {\bf (#1)} %
}{\end{indented}}

%%%%%%%%%%%%%%%%%%%%%%%%%%%%%%%%%%%%%%%%%%%%%%%%%%%%%%%%%%%
\newcommand{\myname}{Shashank Singh}
\newcommand{\myandrew}{sss1@andrew.cmu.edu}
\newcommand{\myclass}{15-359 Probability and Computing}
\newcommand{\myhwnum}{7}
\newcommand{\mysection}{B}
\newcommand{\duedate}{Friday, March 30, 2012}
%%%%%%%%%%%%%%%%%%%%%%%%%%%%%%%%%%%%%%%%%%%%%%%%%%%%%%%%%%%

\begin{document}
\thispagestyle{plain}

{\Large Assignment \myhwnum} \\
\myclass \\
Name: \myname \\
Email: \myandrew \\
Section: \mysection \\
Due: \duedate

\begin{question}{Problem 1: Upper tail (10 pts.)}
Clearly, examining the Taylor series expansion of $\ln(1 + \delta)$,
$\forall \delta \in (0,1]$,
\[\delta^2/3 + \delta
 \leq (1 + \delta)(\delta - \delta^2/2 + \delta^3/3 - \delta^4/4 \ldots )
 = (1 + \delta)\ln(1 + \delta).\]
Thus, since $x \mapsto e^x$ is an increasing function,
$\frac{e^{\delta}}{(1 + \delta)^{1 + \delta}} < e^{-\delta^2/3}$
Therefore,
\[\left(\frac{e^{\delta}}{(1 + \delta)^{1 + \delta}}\right)^{\mu}
 < = e^{- \mu\delta^2/3},\]
so that the strong Chernoff bound implies the simplified bound
\[P(X \geq (1 + \delta)\mu) < e^{- \mu \delta^2/3} \quad \blacksquare.\]
\end{question}

\begin{question}{Problem 2: The $\frac12$-th moment (10 pts.)}
Suppose $X \sim$ Exp$(1)$. Then, substituting $u = \sqrt{x}$ gives
\[E\left[\sqrt{X}\right]
 = \int_0^{\infty} \sqrt{X} f_X(x) \; dx
 = \int_0^{\infty} \sqrt{X} e^{-x} \; dx
 = \int_0^{\infty} 2u^2 e^{-u^2}   \; du
.\]
Integration by parts then gives
\[E\left[\sqrt{X}\right]
 = -0*e^{-0} - \lim_{x \rightarrow \infty}\left(-xe^{-x}\right)
                                             - \int_0^{\infty} -e^{-u^2} \; dx
 = \int_0^{\infty} e^{-u^2} \; dx
 = \frac12 \int_{-\infty}^{\infty} e^{-u^2} \; dx
,\]
(since $e^{-{x^2}}$ is even in $x$). This integral is well-known to be
$\sqrt{\pi}$, so that
\[E\left[\sqrt{X}\right] = \mbox{\fbox{$\displaystyle \frac{\sqrt{\pi}}{2}$.}}\]
\end{question}

\newpage
\begin{question}{Problem 3: Sorting our loose ends (20 pts.)}
\begin{enumerate}[A.]
\item $\sum_{i = 1}^t X_i$ is the number of times, in the first $t$ steps,
that $a$ ended up in the smaller of the two pieces of the array after
partitioning around the pivot. Each time an array is partitioned into two
pieces, the size of the smaller piece is at most half the size of the initial
array. Thus, if $a$ ended up in the smaller of the two pieces of the array
partitioned $X = \sum_{i = 1}^t X_i$ times, then it is in a part that is of
size at most $\frac{n}{2^X}$, so that, when $X$ exceeds $\log_2 n$, $a$ is in
a part of size $1$, so that it must be in its correct position in the sorted
array. \quad $\blacksquare$

\item Since each $X_i$ is $1$ with probability $\frac12$ and $0$ otherwise,
$X \sim$ Binomial$(\frac12,t)$, so that $\mu := E[X] = \frac12 C \log_2 n$.
For $\delta = \frac{C - 2}{2}$, then, by the simplified Chernoff bound for the
Lower Tail,
\[P(X < \log_2 n) = P(X < (1 - \delta)\mu) < e^{-\mu \delta^2/2}. \quad \blacksquare\]

\item If $A$ is the event that the array is not sorted, and $A_i$ is the event
that $a_i$ is not sorted, then, after $t = C \log_2 n$ iterations of
quicksort, by the Union Bound,
\[P(A) \leq \sum_{i = 1}^t P(A_i) \leq te^{-\mu \delta^2/2}.\]
For $C \leq 4.6$, $c(c - 2)^2/16 \geq 2$, so that
$p(A_i) \leq \frac{1}{n^2}$. \quad $\blacksquare$

\end{enumerate}
\end{question}

\begin{question}{Problem 5: Predictions are guesses anyway (8 pts.)}
Suppose $f(h) \in \{0,1\}$. Then, $F(h)$ is distributed according to
Binomial$(P(f(h) = 1),n)$, so that, since $F(\Lambda) = E[F(h)]$ we can apply
Hoeffding's Inequality to give that, $\forall \epsilon > 0$,
\[P(F(h) - F(\Lambda) > \epsilon) < e^{-2\epsilon^2n}. \quad \blacksquare\]
\end{question}

\newpage
\begin{question}{Problem 6: PB wrap (15 pts.)}
Let $s = F_m(h)$ and $d = F(h)$.
\begin{eqnarray*}
E_s(Z)
 & = & E_s\left(E_{h \sim \Pi}\left(\exp\left(m\left(s\ln\left(\frac{s}{d}\right) + (1 - s)\ln\left(\frac{1 - s}{1 - d}\right)\right)\right)\right)\right) \\
 & = & E_s\left(E_{h \sim \Pi}\left(\left(\frac{s}{d}\right)^{m\cdot s} \left(\frac{1 - s}{1 - d}\right)^{m(1 - s)}\right)\right).
\end{eqnarray*}
$s$ can take $m + 1$ values: $\{0/m,1/m,\ldots,m/m\}$ and has a binomial
distribution with parameter(s) $d$ and $m$. Thus,
\begin{eqnarray*}
E_s\left(\left(\frac{s}{d}\right)^{m\cdot s} \left(\frac{1 - s}{1 - d}\right)^{m(1 - s)}\right)
 & = & \sum_{k = 0}^m {m \choose k} d^k(1 - d)^{m - k}\left(\frac{k/m}{d}\right)^k\left(\frac{1 - k/m}{1 - d}\right)^{m - k} \\
 & = & \sum_{k = 0}^m {m \choose k} \left(\frac{k}{m}\right)^k \left(1 - \frac{k}{m}\right)^{m - k}.
\end{eqnarray*}
Since each term of this summation is a probability (e.g., of getting $k$ heads
out of $m$ flips of a biased coin with $k/m$ chance of getting heads), each of
the $m + 1$ terms is at most $1$. Thus, $E_s(Z) \leq m + 1$.
\quad $\blacksquare$
\end{question}
\end{document}
