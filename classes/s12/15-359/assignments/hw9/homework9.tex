\documentclass[11pt]{article}
\usepackage{enumerate}
\usepackage{fullpage}
\usepackage{fancyhdr}
\usepackage{amsmath, amsfonts, amsthm, amssymb}
\usepackage{tikz}
\setlength{\parindent}{0pt}
\setlength{\parskip}{5pt plus 1pt}
\pagestyle{empty}

\def\indented#1{\list{}{}\item[]}
\let\indented=\endlist

\newcounter{questionCounter}
\newcounter{partCounter}[questionCounter]
\newenvironment{question}[2][\arabic{questionCounter}]{%
    \setcounter{partCounter}{0}%
    \vspace{.25in} \hrule \vspace{0.5em}%
        \noindent{\bf #2}%
    \vspace{0.8em} \hrule \vspace{.10in}%
    \addtocounter{questionCounter}{1}%
}{}
\renewenvironment{part}[1][\alph{partCounter}]{%
    \addtocounter{partCounter}{1}%
    \vspace{.10in}%
    \begin{indented}%
       {\bf (#1)} %
}{\end{indented}}

%%%%%%%%%%%%%%%%%%%%%%%HEADER%%%%%%%%%%%%%%%%%%%%%%%%%%%%%%
\newcommand{\myname}{Shashank Singh}
\newcommand{\myandrew}{sss1@andrew.cmu.edu}
\newcommand{\myclass}{15-359 Probability and Computing}
\newcommand{\myhwnum}{9}
\newcommand{\mysection}{B}
\newcommand{\duedate}{Friday, April 13, 2012}
%%%%%%%%%%%%%%%%%%%%%%%%%%%%%%%%%%%%%%%%%%%%%%%%%%%%%%%%%%%
%%%%%%%%%%%%%%%%%%%%%%%%MACROS%%%%%%%%%%%%%%%%%%%%%%%%%%%%%
\newcommand{\Geo}{\operatorname{Geo}}


\begin{document}
\thispagestyle{plain}

{\Large Assignment \myhwnum} \\
\myclass \\
Name: \myname \\
Email: \myandrew \\
Section: \mysection \\
Due: \duedate

\begin{question}{Problem 4.1}
Sample two random variables $X,Y \sim \operatorname{Uniform}(0,1)$.
Note that $c := \max_{x \in [0,1]} f(x) = f\left(\frac12\right) = 15/8$
and that, $\forall x \in [0,1]$, $f(x) \geq 0$. Thus, $\forall x \in [0,1]$,
$f(x)/c \in [0,1]$, so that we can use the accept/reject method
(accepting if $f(X) > Y$ and rejecting otherwise) to generate the desired
probability distribution. \quad $\blacksquare$
\end{question}

\begin{question}{Problem 4.3}
Consider the following Java program \texttt{Simulation}:
\begin{verbatim}
public static void main(String[] args)
{
    double mean = 0;
    for(int i = 0; i < 200; i++)
        mean += trial(Double.parseDouble(args[0]));
    System.out.println(mean/200);
}
private static double trial(Double lambda) //one trial of 2000 runs
{
    //qLoad is the total work in the queue immediately after adding the
    //ith element
    double qLoad = 0;
    for(int i = 0; i < 2001; i++)
        qLoad = expDist(1) + Math.max(qLoad - expDist(lambda), 0);
    return qLoad;
}
private static double expDist(double lambda)
{
    double u = new Random().nextDouble();
    return -((Math.log(1 - u))/lambda);
}
\end{verbatim}
\texttt{Simulation 0.5} outputs $\approx 1.6$.

\texttt{Simulation 0.7} outputs $\approx 2.6$.

\texttt{Simulation 0.9} outputs $\approx 6.5$.
\end{question}

\begin{question}{Problem 8.1}
The stationary equations are:
\[
  \begin{bmatrix}
    \pi_C & \pi_M & \pi_U \\
    \pi_C & \pi_M & \pi_U \\
    \pi_C & \pi_M & \pi_U \\
  \end{bmatrix} = 
  \begin{bmatrix}
    \pi_C & \pi_M & \pi_U \\
    \pi_C & \pi_M & \pi_U \\
    \pi_C & \pi_M & \pi_U \\
  \end{bmatrix}
  \begin{bmatrix}
    0.7 & 0.2 & 0.1 \\
    0.8 & 0.1 & 0.1 \\
    0.9 & 0.1 & 0   \\
  \end{bmatrix}
\]
and $\pi_C + \pi_M + \pi_U = 1$.
The solution to this system of equations is
\[\vec{\pi} = (\pi_C,\pi_M,\pi_U)
 = \mbox{\fbox{$\displaystyle \left(\frac{89}{121},\frac{21}{121},\frac{1}{11}\right)$.}}\]
\end{question}

\begin{question}{Problem 8.2}
Let $P$ be an $m \times m$ transition matrix, and let $n \in \mathbb{N}$.
$\forall i,j \in \{1,2,\ldots,m\}$, let
$p_{i,j}$ be the entry in the $i^{th}$ row of the $j^{th}$ column of $P^n$.
Let $i \in \{1,2,\ldots,m\}$, and, $\forall j \in \{1,2,\ldots,n\}$, let $E_j$
be the event that we end at state $j$ after $n$ transitions given that we are
initially at state $i$, so that $E_j = p_{i,j}$.
Thus, \[\sum_{j = 1}^m p_{i,j} = \sum_{j = 1}^m E_j = 1,\]
since $E_1,E_2,\ldots,E_n$ partition the probability space of end states
given that we start at state $i$. Therefore, $\forall n \in \mathbb{N}$, the
sum of the elements in each row of $P^n$ is $1$. \quad $\blacksquare$
\end{question}

\begin{question}{Problem 8.3}
Suppose $A$ be an $m \times m$ transition matrix with the given properties,
and let
\[P := \lim_{n \rightarrow \infty} \left(A^n\right)
 = \begin{bmatrix}
     \vec{\pi} \\
     \vec{\pi} \\
     \vdots    \\
     \vec{\pi}
 \end{bmatrix}
\]
be its limiting distribution.
Since, $\forall k \in \mathbb{N}$,
$\left(A^k\right)^T = \left(A^T\right)^k$ (where $^T$ denotes the transpose
operator),
\[P^T = \lim_{n \rightarrow \infty} \left(\left(A^T\right)^n\right)
                              = (\vec{\pi}^T,\vec{\pi}^T,\ldots,\vec{\pi}^T)
,\]
so $P^T$ is the limiting distribution for the Markov chain with transition
matrix $A^T$. Thus, since the rows of any limiting distribution are identical,
\[\pi_1 = \pi_2 = \ldots = \pi_m. \quad \blacksquare\]
\end{question}

\begin{question}{Problem 8.4}
Consider a Markov chain with one state for each $i \in \mathbb{N}$, where the
state representing $i$ transitions to the state representing $0$ with
probability $b$, the state representing $i + 1$ with probability $p$, and the
state representing $i$ (itself) with probability $s$. Let $\vec{pi}$ be the
limiting distribution of for this Markov chain. Then, $\pi_0 = s\pi_0 + b$,
and, $\forall i \geq 1$, $p_i = p\pi_{i - 1} + s\pi_i$,
yielding the recurrence
\[\pi_0 = \frac{b}{1 - s},
 \pi_i = \frac{p\pi_{i - 1}}{1 - s}, \forall i \geq 1.\]
The solution to this recurrence is
\[\pi_i = \mbox{\fbox{$\displaystyle b\frac{p^i}{(1 - s)^{i + 1}}$.}}\]
When $s = 0$, this reduces to $\pi_i = bp^i = b(1 - b)^i$, so that the limiting
distribution is $\Geo(b)$.
\end{question}
\end{document}
