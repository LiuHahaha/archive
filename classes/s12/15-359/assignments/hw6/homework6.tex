\documentclass[11pt]{article}
\usepackage{enumerate}
\usepackage{fullpage}
\usepackage{fancyhdr}
\usepackage{amsmath, amsfonts, amsthm, amssymb}
\usepackage{tikz}
\setlength{\parindent}{0pt}
\setlength{\parskip}{5pt plus 1pt}
\pagestyle{empty}

\newcommand*\mycirc[1]{%
  \begin{tikzpicture}
    \node[draw,circle,inner sep=1pt] {#1};
  \end{tikzpicture}}


\def\indented#1{\list{}{}\item[]}
\let\indented=\endlist

\newcounter{questionCounter}
\newcounter{partCounter}[questionCounter]
\newenvironment{question}[2][\arabic{questionCounter}]{%
    \setcounter{partCounter}{0}%
    \vspace{.25in} \hrule \vspace{0.5em}%
        \noindent{\bf #2}%
    \vspace{0.8em} \hrule \vspace{.10in}%
    \addtocounter{questionCounter}{1}%
}{}
\renewenvironment{part}[1][\alph{partCounter}]{%
    \addtocounter{partCounter}{1}%
    \vspace{.10in}%
    \begin{indented}%
       {\bf (#1)} %
}{\end{indented}}

%%%%%%%%%%%%%%%%%%%%%%%%%%%%%%%%%%%%%%%%%%%%%%%%%%%%%%%%%%%
\newcommand{\myname}{Shashank Singh}
\newcommand{\myandrew}{sss1@andrew.cmu.edu}
\newcommand{\myclass}{15-359 Probability and Computing}
\newcommand{\myhwnum}{6}
\newcommand{\mysection}{B}
\newcommand{\duedate}{Friday, March 2, 2012}
%%%%%%%%%%%%%%%%%%%%%%%%%%%%%%%%%%%%%%%%%%%%%%%%%%%%%%%%%%%

\begin{document}
\thispagestyle{plain}

{\Large Assignment \myhwnum} \\
\myclass \\
Name: \myname \\
Email: \myandrew \\
Section: \mysection \\
Due: \duedate
\begin{question}{Problem 1: How we get them}
Since $- \ln X < 0$ if and only if $X > 1$, $P(-ln X < 0) = 0$.
Let $x \in \mathbb{R}$ with $0 \leq x$. Then, since
$0 \leq e^{-x} \leq 1$,
\[P(0 \leq - \ln X \leq x) = P(e^{-x} \leq X \leq 1)
 = \int_{e^{-x}}^1 1 \; dt = 1 - e^{-x}.\]
Differentiating gives the probability density function
$f: \mathbb{R} \rightarrow \mathbb{R}$ such that, $\forall x \in \mathbb{R}$,
\[f_{- \ln X} (x)
 = \left\{
     \begin{array}{c c}
         e^{-x} & x \geq 0 \\
         0      & x < 0
     \end{array}
   \right.
,\]
which is that of an exponential distribution, where $\lambda = 1$.
\qquad $\blacksquare$
\end{question}

\begin{question}{Problem 2: The rain in Spain}
Let $\lambda = 1/25$.
\begin{enumerate}[A.]
\item By definition of the conditional probability of a continuous random
variable and the exponential distribution,
\[E[X | X > 10]
 = \frac{\int_{10}^{\infty} x f_X(x)           \; dx}
        {\int_{10}^{\infty} f_X(x)             \; dx}
 = \frac{\int_{10}^{\infty} x e^{-\lambda x})  \; dx}
        {\int_{10}^{\infty}   e^{-\lambda x}   \; dx}
 = \frac{\int_{10}^{\infty} x e^{-\lambda x})  \; dx}
        {\frac{1}{\lambda}(e^{-10\lambda})}.
\]
Integrating by parts gives
\[\int_{10}^{\infty} x e^{-\lambda x}) \; dx
 = \frac{10e^{-10\lambda}}{\lambda} - \int_{10}^{\infty}\frac{e^{- \lambda x}}{\lambda} \; dx
 = \left(\frac{10}{\lambda} + \frac{1}{\lambda^2}\right) e^{-10\lambda},\]
so that $E[X | X > 10] = 10 + \frac{1}{\lambda} =$ \fbox{$35$}.

\item Since the exponential distribution is memoryless,
$E[X | X > 10] = 10 + E[X] = 10 + \frac{1}{\lambda} =$ \fbox{$35$}.

\item Integration by parts gives
\[\int_{10}^{\infty} x^2 e^{-\lambda x} \; dx
 = \frac{100e^{-10\lambda}}{\lambda}
 - \int_{10}^{\infty}\frac{-2x e^{-\lambda x}}{\lambda} \; dx
 = \frac{100e^{-10\lambda}}{\lambda}
 + \frac{2}{\lambda} \int_{10}^{\infty} x e^{-\lambda x} \; dx,\]
so that, by the integral computed in part A.,
\[\int_{10}^{\infty} x^2 e^{-\lambda x} \; dx
 = \frac{100e^{-10\lambda}}{\lambda}
 + \frac{2}{\lambda}\left(\frac{10}{\lambda} + \frac{1}{\lambda^2}\right) e^{-10\lambda}
 = \left(\frac{100}{\lambda} + \frac{2}{\lambda^2} + \frac{2}{\lambda^3}\right) e^{-10\lambda}.
\]
Thus,
\[E[X^2 | X > 10]
 = \frac{\left((\frac{100}{\lambda} + \frac{2}{\lambda^2} + \frac{2}{\lambda^3}\right) e^{-10\lambda}}
   {\frac{1}{\lambda}(e^{-10\lambda})}
 = 100 + \frac{2}{\lambda} + \frac{2}{\lambda^2} = \mbox{\fbox{$1400$.}}\]

Therefore,
Var$(X | X > 10) = E[X^2 | X > 10] - E[X | X > 10]^2 = 1400 - 35^2 =$
  \fbox{$175$.}
Since $X$ is distributed exponentially with parameter $\lambda$,
Var$(X) = \frac{1}{\lambda^2} = 625 > $ Var$(X | X > 10)$.

\end{enumerate}
\end{question}

\begin{question}{Problem 3: Failure rate}
\begin{enumerate}[A.]
\item Since $X$ is distributed exponentially, for some $\lambda > 0$,
$\forall x \in \mathbb{R}$,
\[f_X (x)
 = \left\{
     \begin{array}{c c}
         \lambda e^{-\lambda x} & x \geq 0 \\
                 0              & x < 0
     \end{array}
   \right.
,\]
and
\[\overline{F}_X (x)
 = \left\{
     \begin{array}{c c}
          e^{-\lambda x} & x \geq 0 \\
          1              & x < 0
     \end{array}
   \right.
.\]
Thus, for $x \geq 0$,
\[r(x) = \frac{f_X(x)}{\overline{F}_X(x)}
 = \frac{\lambda e^{- \lambda x}}{e^{- \lambda x}} = \lambda,\]
so that $r$ is constant on $[0,\infty)$. \qquad $\blacksquare$

\item Suppose that $r = \lambda$, for some constant $\lambda > 0$ (noting that
$r(x)$ is necessarily positive for some $x \in [0,\infty)$). Then,
$\forall x \in [0,\infty)$,
\[\frac{f_X(x)}{1 - F_X(x)} = \lambda,\] so that
$\lambda - \lambda F_X(x) = f_X(x)$. By definition of $F$,
$\lambda - \int_{-\infty}^x f_X(t) \; dt = f_X(x)$. Thus, by the Fundamental
Theorem of Calculus, differentiation gives
\[-\lambda f_X(x) = \frac{d}{dx} f_X(x).\] The unique solution of this well-known
differential equation is the exponential $f_X(x) = e^{- \lambda x},
\forall x \in [0,\infty)$. Thus, the exponential is the unique probability
distribution with a constant failure rate. \qquad $\blacksquare$

\end{enumerate}
\end{question}

\begin{question}{Problem 4: \mycirc{$\lambda$}}
\begin{enumerate}[A.]
\item By definition of half-life, if $t_{1/2}$ is the half-life of the isotope
in question, $\frac{1}{2} = e^{- \lambda t_{1/2}}$. Solving for $t_{1/2}$
gives \fbox{$t_{1/2} = \frac{\ln 2}{\lambda}$.}

\item Since $X_1, X_2, \ldots, X_n$ are independent,
\[F_Y(x) = P(Y < x) = \prod_{i = 1}^n P(X < x) = (1 - e^{-\lambda x})^n.\]
Differentiating gives $f_Y(x) = n(1 - e^{-\lambda x})^{n - 1}$. Thus,
$E[Y] = n\int_0^{\infty} x(1 - e^{-\lambda x})^{n - 1} \; dx$.
\end{enumerate}
\end{question}

\begin{question}{Problem 5: Sparse selection}
\begin{enumerate}[A.]
\item Suppose that, in step $6$., $a^- \leq b^- \leq b^+ \leq a^+$.
By definition, $A_0$ is the set of elements in $A$ in between $b^-$ and $b^+$,
so that $|A_0| \leq \delta(b^+,b^-) \leq \delta(a^+, a^-)$, where,
$\forall x,y \in \mathcal A$, $\delta(x,y)$ denotes the number of elements of
$A$ between $x$ and $y$ (when $A$ is sorted). Thus, since, by definition,
$\lambda^- \geq n/2 - 2n^{3/4}$ and $\lambda^+ \leq n/2 + 2n^{3/4}$,
$\delta(a^+,a^-) \leq n/2 + 2n^{3/4} - (n/2 - 2n^{3/4}) = 4n^{3/4}$, so that
$|A_0| \leq 4n^{3/4}$, the desired result.
\end{enumerate}
\end{question}

\begin{question}{Problem 6: Bayes of our lives}
\begin{enumerate}[A.]
\item \[f_P(p | N > 47)
 = \sum_{i = 48}^{\infty} \frac{P(N = i | P = p) f_P(p)}{P(N = i)}\cdot P(N = i)
 = \sum_{i = 48}^{\infty} P(N = i | P = p) f_P(p)\]
\[ = \sum_{i = 48}^{\infty} (1 - p)^{i - 1}p
 = (1 - p)^{47}p \sum_{i = 0}^{\infty}(1 - p)^i
 = (1 - p)^{47}p \frac{1}{p}
 = \mbox{\fbox{$(1 - p)^{47}$.}}
\]

\item By the result of part A., $E[P | N > 47]
 = \int_0^1 p(1 - p)^{47} \; dp =$ \fbox{$\frac{1}{2352} \approx 0.00043$.}
\end{enumerate}
\end{question}
\end{document}
