\documentclass[11pt]{article}
\usepackage{enumerate}
\usepackage{fullpage}
\usepackage{fancyhdr}
\usepackage{amsmath, amsfonts, amsthm, amssymb}
\setlength{\parindent}{0pt}
\setlength{\parskip}{5pt plus 1pt}
\pagestyle{empty}

\def\indented#1{\list{}{}\item[]}
\let\indented=\endlist

\newcounter{questionCounter}
\newcounter{partCounter}[questionCounter]
\newenvironment{question}[2][\arabic{questionCounter}]{%
    \setcounter{partCounter}{0}%
    \vspace{.25in} \hrule \vspace{0.5em}%
        \noindent{\bf #2}%
    \vspace{0.8em} \hrule \vspace{.10in}%
    \addtocounter{questionCounter}{1}%
}{}
\renewenvironment{part}[1][\alph{partCounter}]{%
    \addtocounter{partCounter}{1}%
    \vspace{.10in}%
    \begin{indented}%
       {\bf (#1)} %
}{\end{indented}}

%%%%%%%%%%%%%%%%%%%%%%%%%%%%%%%%%%%%%%%%%%%%%%%%%%%%%%%%%%%
\newcommand{\myname}{Shashank Singh}
\newcommand{\myandrew}{sss1@andrew.cmu.edu}
\newcommand{\myclass}{15-359 Probability and Computing}
\newcommand{\myhwnum}{5}
\newcommand{\mysection}{B}
\newcommand{\duedate}{Friday, February 24, 2012}
%%%%%%%%%%%%%%%%%%%%%%%%%%%%%%%%%%%%%%%%%%%%%%%%%%%%%%%%%%%

\begin{document}
\thispagestyle{plain}

{\Large Assignment \myhwnum} \\
\myclass \\
Name: \myname \\
Email: \myandrew \\
Section: \mysection \\
Due: \duedate

\begin{question}{Problem 1: Not too concentrated}
Let $f_X : \mathbb{R} \rightarrow \mathbb{R}$ be the probability density
function of $X$, so that $f_X \leq B$. Then, $\forall x, \epsilon \in
\mathbb{R}$ with $\epsilon > 0$,
\begin{eqnarray*}
P(|X - x| \leq \epsilon)
 & =    & P(x - \epsilon \leq X \leq x + \epsilon)        \\
 & =    & \int_{x - \epsilon}^{x + \epsilon} f_X(t) \; dt \\
 & \leq & \int_{x - \epsilon}^{x + \epsilon} B      \; dt \\
 & =    & B(x + \epsilon) - B(x - \epsilon) = \mbox{\fbox{$2\epsilon B$.}}
\end{eqnarray*}
\end{question}

\begin{question}{Problem 2: Mediocristan and Extremistan}
\begin{enumerate}[A.]
\item Let $\lambda = 0.1, \alpha = \frac{10}{9}$. Then,
$M \sim$ Exp$(\lambda)$ ($M$ has an exponential distribution), and
$X \sim$ Pareto$(\alpha)$ ($X$ has a Pareto distribution). As shown
in class, $E[M] = $\fbox{$\frac{1}{\lambda}$.} By definition of expected
value, since $\alpha > 1$,
\begin{eqnarray*}
E[X]
 & = & \int_{1}^{\infty} x \cdot f_X(x)      \; dx \\
 & = & \int_{1}^{\infty} \alpha x^{-\alpha}  \; dx \\
 & = & -\alpha\left(\frac{1^{1 - \alpha}}{1 - \alpha}\right)
   =   \mbox{\fbox{$\displaystyle \frac{\alpha}{\alpha - 1}$.}}
\end{eqnarray*}

\item Since, $\forall m \geq 0$, $\overline{F}_M(m) = P(M > m)$, and,
$\forall x \geq 1$, $\overline{F}_X(x) = P(X > x)$,
\[\frac{1}{100} = e^{-\lambda m_0},\]
and
\[\frac{1}{100} = x_0^{-\alpha}.\]
Thus, \fbox{$m_0 = 10 \ln (100) \approx 46$,} and
\fbox{$x_0 =10^{9/5} \approx 63$.}

\item
\[\rho_M = \lambda^2 \int_{10 \ln 100}^{\infty} mf_M(m) \; dm \approx 5.6 \% .\]
\[\rho_X = \frac{\alpha - 1}{\alpha} \int_{10^{9/5}}^{\infty} xf_X(x) \; dm \approx 63\%.\]
Thus, $\rho_M < \rho_X$. This makes sense because the exponential distribution
decays much more quickly than the Pareto distribution. \qquad $\blacksquare$
\end{enumerate}
\end{question}

\begin{question}{Problem 5: Vanilla search trees}
\begin{enumerate}[A.]
\item Randomly permute the elements of $A$. Insert the elements into a binary
search tree as usual.

\item A clever proof is to notice that the problem is isomorphic to analyzing
the expected depth of an element the recursion tree of randomized quicksort;
the depth of each node in the tree corresponds to the level of recursion at
which it is used as a pivot in the algorithm. The expected depth of this tree
was shown in class to be at most $2H_n$, where $H_n$ denotes the $n^{th}$
harmonic number.

A more rigorous, albeit boring, proof is as follows:
Let $X$ be a random variable denoting the average expected distance of a node
in the tree from the root.
$\forall i \in \{1,2,\ldots,n\}$, let $X_i$ be a random variable denoting the
distance of $a_i$ from the root.
$\forall (i,j) \in \{1,2,\ldots,n\}^2$ with $i \neq j$, let $X_{i,j}$ be an
indicator random variable which is $1$ if $_jj$ is an ancestor of $a_i$ in the
tree. Note that, since the depth of a node is the same as the number of
ancestors it has, $\forall i \in \{1,2,\ldots,n\}$,
$X_i = \sum_{j = 1}^n X_{i,j}$. The $a_j$ is a parent of $a_i$ if and only if
$a_j$ is the first elements of $\{a_j,a_{j + 1},\ldots,a_i\}$ to be inserted
into the tree. This happens with probability $\frac{1}{|j - i| + 1}$,
so that,
\[X = \frac{1}{n} \sum_{i = 1}^n X_i = \frac{1}{n} \sum_{i = 1}^n \sum_{j = 1}^n \frac{1}{|j - i| + 1}
\leq \frac{1}{n} \sum_{i = 1}^n \sum_{j = i + 1}^n \frac{2}{j - i + 1}.\]
This summation can be re-written in terms of $k = j - i$ as
\[E[X] = \frac{1}{n}\sum_{k = 2}^n \sum_{l = 0}^{n - k + 1} \frac{1}{k}
 = \frac{1}{n}\sum_{k - 2}^n (n - k + 1) \frac{2}{k} \leq \sum_{k - 2}^n \frac{2}{k}
 \leq 2H_n \in O(\log n),\]
where $H_n$ denotes the $n^{th}$ harmonic number. \qquad $\blacksquare$
\end{enumerate}
\end{question}

\begin{question}{Problem 6: Have randomness, will predict}
Let $f: \mathbb{N} \rightarrow [0,1] \subseteq \mathbb{R}$ such that,
$\forall n \in \mathbb{N}$, $f(x) = 1 - e^{-x}$. Let $T$ be a real number
randomly sampled from $[0,1]$ (in particular, let $T \sim U(0,1)$). If
$T > f(B)$, then guess that $B < Z$. Otherwise, guess that $B > Z$.

Let $W = \min\{X,Y\}$, let $V = \max\{X,Y\}$, let $C$ be the probability that
you guess correctly. Since $f$ is strictly increasing on its domain,
$f(W) < f(V)$, so that
\begin{eqnarray*}
P(C)
 & = & P(B = W)P(f(B) < T) + P(B = V)P(T < f(B))         \\
 & = & \frac{1}{2}(1 - f(W)) + \frac{1}{2}f(V)           \\
 & = & \frac{1}{2} + \frac{1}{2}\left(f(V) - f(W)\right)
   >  \frac{1}{2}.
\end{eqnarray*}
Thus, guessing in this manner guarantees that you will guess correctly with
probability $\frac{1}{2} + \alpha$, for some $\alpha > 0$.
\qquad $\blacksquare$
\end{question}

\begin{question}{Problem 7: Shooting blanks}
\begin{enumerate}[A.]
\item A deterministic algorithm $A \in \mathcal{A}$ consists of querying the
entries of an input matrix in some order, returning \texttt{true} upon finding
that any column contains only $0$'s, and returning \texttt{false} after
confirming that no column in $A$ contains only $0$'s.

\item No optimal algorithm will query an entry in the matrix more than once or
query any entry in a column in which a $1$ has already been found. Thus, we
consider only those algorithms which do not do this (we can do this because we
are interested only in those algorithms which minimize
$\min_{A \in \mathcal{A}} E[T_A(I_{\tau})]$.

Let $I$ be a boolean matrix with exactly $1$ non-zero entry in each column,
and let $A \in \mathcal{A}$. Let $X$ be a random variable denoting the number
of entries in $I$ queried by $A$ when run on $I$, and,
$\forall i \in \{1,2,\ldots,n\}$, let $X_i$ be a random variable
denoting the number of entries $A$ queries in the $i^{th}$ column. Then,
$\forall i,j \in \{1,2,\ldots,n\}$, $P(X_i = j) = \frac{j}{n}$.

$E[X_i] = \sum_{j = 1}^n j\cdot P(X_i = j) = \frac{1}{n}\sum_{i = 1}^n j
 = \frac{n + 1}{2} $

Thus, by Linearity of Expectation, since $X = \sum_{i = 1}^n X_i$,
\[E[X] = E\left[\sum_{i = 1}^n X_i\right] = \sum_{i = 1}^n E[X_i]
= \frac{n(n + 1}{2}.\]

Therefore, the expected number of queries made by $A$ is at least
$\frac{n(n + 1)}{2}$.

Suppose $L$ is a Las Vegas algorithm. Then, by Yao's Minimax Principle the
expected number of queries made by $L$ in the worst case is bounded below by
$\frac{n(n + 1)}{2}$. \qquad $\blacksquare$

\end{enumerate}
\end{question}
\end{document}
