\documentclass[11pt]{article}
\usepackage{enumerate}
\usepackage{fullpage}
\usepackage{fancyhdr}
\usepackage{amsmath, amsfonts, amsthm, amssymb}
\usepackage{mathrsfs}
\usepackage{graphicx}
\setlength{\parindent}{0pt}
\setlength{\parskip}{5pt plus 1pt}
\pagestyle{empty}

\def\indented#1{\list{}{}\item[]}
\let\indented=\endlist

\newcounter{questionCounter}
\newcounter{partCounter}[questionCounter]
\newenvironment{question}[2][\arabic{questionCounter}]{%
    \setcounter{partCounter}{0}%
    \vspace{.25in} \hrule \vspace{0.5em}%
        \noindent{\bf #2}%
    \vspace{0.8em} \hrule \vspace{.10in}%
    \addtocounter{questionCounter}{1}%
}{}
\renewenvironment{part}[1][\alph{partCounter}]{%
    \addtocounter{partCounter}{1}%
    \vspace{.10in}%
    \begin{indented}%
       {\bf (#1)} %
}{\end{indented}}

%%%%%%%%%%%%%%%%%%%%%%%HEADER%%%%%%%%%%%%%%%%%%%%%%%%%%%%%%
\newcommand{\myname}{Shashank Singh\footnote{sss1@andrew.cmu.edu}}
\newcommand{\myclass}{21-470 Calculus of Variations}
\newcommand{\myhwnum}{4}
\newcommand{\duedate}{Friday, March 28, 2014}
%%%%%%%%%%%%%%%%%%%%%%%%%%%%%%%%%%%%%%%%%%%%%%%%%%%%%%%%%%%

%%%%%%%%%%%%%%%%%%%%CONTENT MACROS%%%%%%%%%%%%%%%%%%%%%%%%%
\renewcommand{\qed}{\quad \ensuremath{\blacksquare}}
\newcommand{\inv}{^{-1}}
\newcommand{\bv}{\mathbf{v}}
\newcommand{\bx}{\mathbf{x}}
\newcommand{\by}{\mathbf{y}}
\newcommand{\bff}{\mathbf{f}}
\newcommand{\bzero}{\mathbf{0}}
\newcommand{\bxi}{\boldsymbol{\xi}}
\newcommand{\boldeta}{\boldsymbol{\eta}}
\newcommand{\dist}{\operatorname{dist}} % distance from or between sets
\newcommand{\area}{\operatorname{area}} % area of a polygon
\newcommand{\Gr}{\operatorname{Gr}}     % graph of a function
\renewcommand{\sp}{\operatorname{span}} % span of a set
\newcommand{\bdry}{\operatorname{bdry}} % boundary of a set
\newcommand{\sminus}{\backslash}        % set difference
\newcommand{\N}{\mathbb{N}}             % natural numbers
\newcommand{\Z}{\mathbb{Z}}             % integers
\newcommand{\Q}{\mathbb{Q}}             % rational numbers
\newcommand{\R}{\mathbb{R}}             % real numbers
\newcommand{\C}{\mathbb{C}}             % complex numbers
\newcommand{\D}{\mathcal{D}}            % domain of an operator
\newcommand{\Cmp}{\mathcal{C}}          % space of compact linear operators\s
\newcommand{\K}{\mathbb{K}}             % underlying field of a linear space
\newcommand{\Ran}{\mathcal{R}}          % range of a linear operator
\newcommand{\Nul}{\mathcal{N}}          % null-space of a linear operator
\renewcommand{\L}{\mathcal{L}}          % space of bounded linear functions
\newcommand{\pow}[1]{\mathcal{P}\left(#1\right)}    % power set of #1
\newcommand{\e}{\varepsilon}            % \varepsilon
\newcommand{\wto}{\rightharpoonup}      % weak convergence
\newcommand{\wsto}{\stackrel{*}{\rightharpoonup}}   % weak-* convergence
\renewcommand{\Re}{\operatorname{Re}}   % real part of a complex number
\newcommand{\tT}{\widetilde{T}}         % for P3
\newcommand{\A}{\mathcal{A}}            % for P3
\renewcommand{\S}{\mathscr{S}}          % constraint set
\newcommand{\X}{\mathscr{X}}            % entire linear space
\newcommand{\Y}{\mathscr{Y}}            % domain of objective functional
\newcommand{\V}{\mathscr{V}}            % set of admissible variations
%%%%%%%%%%%%%%%%%%%%%%%%%%%%%%%%%%%%%%%%%%%%%%%%%%%%%%%%%%%

\begin{document}
\thispagestyle{plain}

{\Large Midterm} \\
\myclass \\
Name: \myname \\
Due: \duedate

\begin{question}{Problem 1}
The $1^{st}$ Euler-Lagrange Equation gives,
\[0 = \frac{d}{dx} e^{x^2}e^{y'(x)}, \quad \forall x \in [0,1]
    \quad \Rightarrow \quad
    c_1 = e^{x^2}e^{y'(x)}, \quad \forall x \in [0,1]
\]
for some $c_1 \in \R$. Solving for $y'(x)$ and integrating gives,
$\forall x \in [0,1]$,
\[y'(x) = \log c_1 - x^2
    \quad \Rightarrow \quad
    y(x) = x\log c_1 - \frac{x^3}{3} + c_2,
\]
for some $c_2 \in \R$. Since $y(0) = 0$, $c_2 = 0$, and so, since $y(1) = 1$,
$\log c_1 = 4/3$. Hence,
\[y(x) = \frac{1}{3} \left( 4x - x^3 \right), \quad \forall x \in [0,1].\]
Since the exponential function is convex, it is clear that $J$ is convex. Hence
this $y$ minimizes $J$ on $\Y$ and $J$ is unbounded above on $\Y$.
\end{question}

\begin{question}{Problem 2}
The $1^{st}$ Euler-Lagrange Equation gives,
\[2 = \frac{d}{dx} \left( -2x^2y'(x) \right), \quad \forall x \in [1,2]
    \quad \Rightarrow \quad
    x + c_1 = -x^2y'(x), \quad \forall x \in [1,2]
\]
for some $c_1 \in \R$. Solving for $y'(x)$ and integrating gives,
$\forall x \in [1,2]$,
\[y'(x) = - \left( x^{-1} + c_1x^{-2}\right)
    \quad \Rightarrow \quad
    y(x) = c_1x^{-1} - \ln x + c_2,
\]
for some $c_2 \in \R$. The constraint $y(1) = 0$ implies $c_1 + c_2 = 0$. The
natural boundary condition is
\[-2x^2y'(x) \bigg|_{x = 2} = 0
    \quad \Rightarrow \quad
    0 = y'(2) = \frac{c_1}{2} - \ln 2 + c_2
    \quad \Rightarrow \quad
    c_1 = - \ln 4, c_2 = \ln 4,
\]
so that
\[y(x) = \frac{- \ln 4}{x} - \ln x + \ln 4, \quad \forall x \in [1,2].\]
Since, $\forall x \in [1,2]$, the function $(y,z) \mapsto 2y - x^2z^2$ is
concave, $J$ is concave. Hence this $y$ maximizes $J$ on $\Y$ and $J$ is
unbounded below on $\Y$.
\end{question}

\begin{question}{Problem 3}
Define $G : C^1[0,1] \to \R$ by $G(y) = \int_0^1 xy(x) \, dx$. For any
$\lambda \in \R$, the $1^{st}$ Euler-Lagrange Equation for $J - \lambda G$
gives
\[-\lambda x = \frac{d}{dx} 2y'(x)
    \quad \Rightarrow \quad
  -\frac{\lambda}{4}x^2 + c_1 = y'(x),
\]
for some $c_1 \in \R$. Integrating gives
\[y(x) = -\frac{\lambda}{12}x^3 + c_1x + c_2, \quad \forall x \in [0,1],\]
for some $c_2 \in \R$.
The boundary conditions $y(0) = 0$ and $y(1) = 1$ imply $c_2 = 0$ and hence
$12 = -\lambda + 12c_1$. The constraint $G(y) = 1$ implies
\[60
    = 60\int_0^1 xy(x) \, dx
    = \int_0^1 -5\lambda x^4 + 60c_1x^2 \, dx
    = -\lambda x^5 + 20c_1x^3 \bigg|_{x = 0}^{x = 1}
    = -\lambda + 20c_1.
\]
Hence, $c_1 = 6$ and $\lambda = 60$, so that
\[y(x) = - 5x^3 + 6x, \quad \forall x \in [0,1].\]
For all $y \in \Y$, $(J - \lambda G)(y) = \int_0^1 y'(x)^2 - 60xy(x) \, dx$.
Since, $\forall x \in [0,1]$, the function $(y,z) \mapsto z^2 - 60xy$ is
convex, $J - \lambda G$ is convex. Hence, this $y$ minimizes $J$ on $\Y$.
\end{question}

\begin{question}{Problem 4}
$J$ is unbounded above on $\Y$, since, for $n \in \N$, if
$y_n \in \Y$ is defined by $y_n(x) = \sin(2nx), \forall x \in [0,\pi/2]$,
\vspace{-5mm}
\begin{align*}
J(y_n)
 &  = \int_0^{\pi/2} 2n \cos^2(2nx) - \sin^2(2nx) + 2e^x\sin(2nx) \, dx \\
 &  \geq \int_0^{\pi/2} 2n \cos^2(2nx) - \sin^2(2nx) - 2e^{\pi/2} \, dx
    =  (2n - 1)\frac{\pi}{4} - \pi e^{\pi/2}
    \to +\infty
\end{align*}
as $n \to \infty$. The $1^{st}$ Euler-Lagrange Equation gives,
\vspace{-2mm}
\[-2y(x) + 2e^x = \frac{d}{dx} 2y'(x),\]
and so $y'$ is continuously differentiable, with $y''(x) = e^x - y(x)$. This
ODE's general solution is
\[y(x) = c_1\cos(x) + c_2\sin(x) + \frac{e^x}{2}.\]
The boundary conditions $y(0) = y\left( \frac{\pi}{2} \right) = 0$ imply
$c_1 = -1/2$ and $c_2 = -e^{\pi/2}/2$, and so
\vspace{-2mm}
\[y(x) = \frac{1}{2} \left( e^x - \cos(x) - e^{\pi/2}\sin(x) \right).\]
I believe, but was not able to show, that $y$ minimizes $J$ on $\Y$.
\end{question}

\begin{question}{Problem 5}
Let $\V := \{v \in C^1[a,b] : \alpha v(a) + \beta v(b) = 0\}$. If $y$ minimizes
$J$ on $\Y$, then
\begin{equation}
0
    = \delta J(y;v)
    = \int_a^b f_{,2}(x,y(x),y'(x))v(x) + f_{,3}(x,y(x),y'(x))v'(x) \, dx
\label{eq:5.1}
\end{equation}
Since $\{v \in C^1[a,b] : v(a) = v(b) = 0\} \subseteq \V$, Lemma 3.4 gives that
the function $x \mapsto f_{,3}(x,y(x),y'(x))$ is continuously differentiable
and we have the $1^{st}$ Euler-Lagrange Equation:
\[\frac{d}{dx} f_{,3}(x,y(x),y'(x)) = f_{,2}(x,y(x),y'(x))
    \quad \forall x \in [a,b].\]
Noting that the case $\alpha = 0$ or $\beta = 0$ is already covered by Lemma
3.5, we assume $\alpha, \beta \neq 0$. Integrating Equation (\ref{eq:5.1}) by
parts, we have, for
\begin{align*}
0
 &  = f_{,3}(x,y(x),y'(x))v(x)\bigg|_{x = a}^{x = b}
    + \int_a^b \left( f_{,2}(x,y(x),y'(x))
    - \frac{d}{dx} f_{,3}(x,y(x),y'(x)) \right) v(x) \, dx  \\
 &  = f_{,3}(a,y(a),y'(a))v(a) - f_{,3}(b,y(b),y'(b))v(b)
\end{align*}
Choosing
\[v(x) = -\left( \frac{\alpha}{\beta} + 1 \right) \frac{x - a}{b - a} + 1,\]
$v \in \V$ and, since $v(a) = 1, v(b) = -\frac{\alpha}{\beta}$, giving the
additional constraint
\[f_{,3}(a,y(a),y'(a)) = \frac{\alpha}{\beta}f_{,3}(b,y(b),y'(b)).\]
\end{question}

\begin{question}{Problem 6}
{\bf Claim:} $g$ must be an affine function. That is, $\exists c_1,c_2 \in \R$
such that $g(x) = c_1x + c_2, \forall x \in [0,1]$.

\emph{Proof:}
First note that every affine function satisfies the given condition, since
\begin{equation}
\int_0^1(c_1x + c_2)v'(x) \, dx
    = -c_1\int_0^1 v(x) \, dx + c_2(v(1) - v(0))
    = 0,
\label{eq:affine}
\end{equation}
integrating the first term by parts and using the conditions on $v$. Put
\[
A   := \int_0^1 g(x) \, dx, \quad
B   := \int_0^1 \int_0^x g(t) \, dt \, dx, \quad
c_1 := 6A - 12B, \quad
c_2 := 6B - 2A,
\]
and define $v \in C^1[0,1]$ by
\[v(x) := \int_0^x g(t) - c_1t - c_2 \, dt, \quad \forall x \in [0,1].\]
Trivially, $v(0) = 0$. Also,
\[
v(1)
    = \int_0^1 g(t) - c_1t - c_2 \, dt
    = A - \frac{c_1}{2} - c_2
    = 0
\]
and
\[
\int_0^1 v(x) \, dx
    = \int_0^1 \int_0^x g(t) - c_1t - c_2 \, dt \, dx
    = B - \int_0^1 \frac{c_1}{2}x^2 - c_2x \, dx
    = B - \frac{c_1}{6} - \frac{c_2}{2}
    = 0.
\]
Hence, $v \in \V$. Then, by Equation (\ref{eq:affine}),
\[0
    = \int_0^1 g(x)v'(x) \, dx
    = \int_0^1 (g(x) - c_1x - c_2)v'(x) \, dx
    = \int_0^1 (g(x) - c_1x - c_2)^2 \, dx,
\]
and so it follows that $g(x) = c_1x + c_2$ for all $x \in [0,1]$.
\end{question}

\begin{question}{Problem 7}
Put
\[\V := \left\{ v \in C^1[a,b] :
                             v(b) = v(a) - \int_a^b v(x) \, dx = 0 \right\}.\]
Then, if $y$ minimizes $J$ on $\Y$, defining $F \in C^1[a,b]$ by
$F(x) = \int_a^x f_{,2}(t,y(t),y'(t)) \, dt, \forall x \in [a,b]$,
\begin{align*}
0
    = \delta J(y;v)
 &  = \int_a^b f_{,2}(x,y(x),y'(x))v(x) + f_{,3}(x,y(x),y'(x))v'(x) \, dx   \\
 &  = \int_a^b \left( f_{,3}(x,y(x),y'(x)) - F(x) \right)v'(x) \, dx,
\end{align*}
using integration by parts and $F(a) = v(b) = 0$. Since
\[\left\{v \in C^1[a,b] : v(a) = v(b) = \int_a^b v(x) \, dx = 0\right\}
    \subseteq \V,\]
by the result of Problem 6, for some $c_1,c_2 \in \R$,
\[f_{,3}(x,y(x),y'(x)) - F(x) = c_1x + c_2, \quad \forall x \in [a,b].\]
Hence, the function $x \mapsto f_{,3}(x,y(x),y'(x))$ is in $C^1[a,b]$ and
differentiating gives
\[\frac{d}{dx} f_{,3}(x,y(x),y'(x)) = f_{,2}(t,y(t),y'(t))
    + c_1, \quad \forall x \in [a,b].\]
\end{question}
%
%%TODO
%\begin{question}{Problem 8}
%Suppose, for sake of contradiction, that $y \in \Y$ minimizes $J$ on $\Y$.
%The $1^{st}$ Euler-Lagrange Equation gives
%\[2y(x) = -\frac{d}{dx} e^{-y'(x)}.\]
%Since the function $z \mapsto e^{z}$ has an analytic inverse, it follows that
%$y' \in C^2[0,1]$, and hence
%\[2y(x) = y''(x) e^{-y'(x)}.\]
%Suppose that, for some $x \in (0,1)$, $x$ maxmimizes $y(x)$ on $[0,1]$. Then,
%\end{question}
\end{document}
