\documentclass[11pt]{article}
\usepackage{enumerate}
\usepackage{fullpage}
\usepackage{fancyhdr}
\usepackage{amsmath, amsfonts, amsthm, amssymb}
\usepackage{color}
\setlength{\parindent}{0pt}
\setlength{\parskip}{5pt plus 1pt}
\pagestyle{empty}

\def\indented#1{\list{}{}\item[]}
\let\indented=\endlist

\newcounter{questionCounter}
\newcounter{partCounter}[questionCounter]
\newenvironment{question}[2][\arabic{questionCounter}]{%
    \setcounter{partCounter}{0}%
    \vspace{.25in} \hrule \vspace{0.5em}%
        \noindent{\bf #2}%
    \vspace{0.8em} \hrule \vspace{.10in}%
    \addtocounter{questionCounter}{1}%
}{}
\renewenvironment{part}[1][\alph{partCounter}]{%
    \addtocounter{partCounter}{1}%
    \vspace{.10in}%
    \begin{indented}%
       {\bf (#1)} %
}{\end{indented}}

%%%%%%%%%%%%%%%%%%%%%%%HEADER%%%%%%%%%%%%%%%%%%%%%%%%%%%%%%
\newcommand{\myname}{Shashank Singh\footnote{sss1@andrew.cmu.edu}}
\newcommand{\myclass}{21-470 Calculus of Variations}
\newcommand{\myhwnum}{1}
\newcommand{\duedate}{Wednesday, January 29, 2014}
%%%%%%%%%%%%%%%%%%%%%%%%%%%%%%%%%%%%%%%%%%%%%%%%%%%%%%%%%%%

%%%%%%%%%%%%%%%%%%%%CONTENT MACROS%%%%%%%%%%%%%%%%%%%%%%%%%
\renewcommand{\qed}{\quad \ensuremath{\blacksquare}}
\newcommand{\inv}{^{-1}}
\newcommand{\bv}{\mathbf{v}}
\newcommand{\bx}{\mathbf{x}}
\newcommand{\by}{\mathbf{y}}
\newcommand{\bff}{\mathbf{f}}
\newcommand{\bzero}{\mathbf{0}}
\newcommand{\bxi}{\boldsymbol{\xi}}
\newcommand{\boldeta}{\boldsymbol{\eta}}
\newcommand{\dist}{\operatorname{dist}} % distance from or between sets
\newcommand{\area}{\operatorname{area}} % area of a polygon
\newcommand{\Gr}{\operatorname{Gr}}     % graph of a function
\renewcommand{\sp}{\operatorname{span}} % span of a set
\newcommand{\bdry}{\operatorname{bdry}} % boundary of a set
\newcommand{\sminus}{\backslash}        % set difference
\newcommand{\N}{\mathbb{N}}             % natural numbers
\newcommand{\Z}{\mathbb{Z}}             % integers
\newcommand{\Q}{\mathbb{Q}}             % rational numbers
\newcommand{\R}{\mathbb{R}}             % real numbers
\newcommand{\C}{\mathbb{C}}             % complex numbers
\newcommand{\D}{\mathcal{D}}            % domain of an operator
\newcommand{\Cmp}{\mathcal{C}}          % space of compact linear operators\s
\newcommand{\K}{\mathbb{K}}             % underlying field of a linear space
\newcommand{\Ran}{\mathcal{R}}          % range of a linear operator
\newcommand{\Nul}{\mathcal{N}}          % null-space of a linear operator
\renewcommand{\L}{\mathcal{L}}          % space of bounded linear functions
\newcommand{\pow}[1]{\mathcal{P}\left(#1\right)}    % power set of #1
\newcommand{\e}{\varepsilon}            % \varepsilon
\newcommand{\wto}{\rightharpoonup}      % weak convergence
\newcommand{\wsto}{\stackrel{*}{\rightharpoonup}}   % weak-* convergence
\renewcommand{\Re}{\operatorname{Re}}   % real part of a complex number
\newcommand{\tT}{\widetilde{T}}         % for P3
\newcommand{\A}{\mathcal{A}}            % for P3
\renewcommand{\S}{\mathcal{S}}          % Schwartz space
\newcommand{\Y}{\mathcal{Y}}            % domain of functional
%%%%%%%%%%%%%%%%%%%%%%%%%%%%%%%%%%%%%%%%%%%%%%%%%%%%%%%%%%%

\begin{document}
\thispagestyle{plain}

{\Large Homework \myhwnum} \\
\myclass \\
Name: \myname \\
Due: \duedate

\begin{question}{Problem 1}
[I wasn't able to get very far with the integrals in parts (b) and (c), and
wasn't able to determine much about the constants in part (d). Based on
numerical computation, I believe (a) offers the slowest solution, followed by
(b), (c), and then (d).]
\begin{enumerate}[(a)]
\item Since $y'(x) = 1$ for all $x \in [0,1]$,
\begin{align*}
J(y)
    = \int_0^1 \frac{\sqrt{1 + y'(x)^2}}{\sqrt{y(x)}} \, dx
    = \sqrt{2} \int_0^1 x^{-1/2} \, dx
    = 2\sqrt{2} x^{1/2} \bigg|_{x = 0}^{x = 1}
    = 2\sqrt{2}.
\end{align*}
 
\item Since
\[y(x) = \sqrt{x - x^2}
    \quad \mbox{ and } \quad
y'(x) = \frac{1 - x}{\sqrt{2x - x^2}}\]
for all $x \in (0,1]$,
\begin{align*}
J(y)
    = \int_0^1 \frac{\sqrt{1 + (1 - x)^2/(2x - x^2)}}{(2x - x^2)^{1/4}} \, dx
 &  = \int_0^1 \frac{\sqrt{2x - x^2 + x^2 - 2x + 1}}{(2x - x^2)^{3/4}} \, dx \\
 &  = \int_0^1 (2x - x^2)^{-3/4} \, dx.
\end{align*}
 
\item Since $y'(x) = x^{-1/2}/2$ for all $x \in (0,1]$, using the inequality
$\sqrt{a + b} \leq \sqrt{a} + \sqrt{b}$ for $a,b \geq 0$,
\begin{align*}
J(y)
    = \int_0^1 \frac{\sqrt{1 + 1/(4x)}}{x^{1/4}} \, dx
 &  = \int_0^1 \sqrt{x^{-1/2} + x^{-3/2}/4} \, dx.
\end{align*}

\item Observing that
\[\frac{dx}{d\theta} = \frac{c^2}{2} (1 - \cos\theta)
\quad \mbox{ and } \quad
\frac{dy}{dx}
    = \left. \frac{dy}{d\theta} \middle/ \frac{dx}{d\theta} \right.
    = \frac{\sin\theta}{1 - \cos\theta},
\]
so that
\[\left( \frac{dy}{dx} \right)^2
    = \left( \frac{\sin\theta}{1 - \cos\theta} \right)^2
    = \frac{1 - \cos^2\theta}{(1 - \cos\theta)^2}
    = \frac{1 + \cos\theta}{1 - \cos\theta},
\]
and changing variables from $x$ to $t$, $J(y)$ simplifies significantly:
\begin{align*}
J(y)
 &  = \int_0^{\theta_1} \frac{\sqrt{1 + \frac{1 + \cos \theta}
                                                {1 - \cos\theta}}}
                        {\sqrt{\frac{c^2}{2}(1 - \cos\theta)}}
      \frac{c^2}{2} (1 - \cos\theta) \, d\theta
    = \frac{c}{\sqrt2} \int_0^{\theta_1} \sqrt{1 + \frac{1 + \cos \theta}
                                                {1 - \cos\theta}}
      \sqrt{1 - \cos\theta} \, d\theta   \\
 &  = \frac{c}{\sqrt2} \int_0^{\theta_1}
                        \sqrt{1 - \cos\theta + 1 + \cos\theta} \, d\theta
    = \frac{c}{\sqrt2} \int_0^{\theta_1}
                        \sqrt{2} \, d\theta = c\theta_1.
\end{align*}
\end{enumerate}
\end{question}

\begin{question}{Problem 2}
Since
\[\frac{d^2}{dx^2} \sqrt{1 + x^2}
    = \frac{d}{dx} \frac{x}{\sqrt{1 + x^2}}
    = \left( 1 + x^2 \right)^{-3/2}
    \geq 0,\]
the function $x \mapsto \sqrt{1 + x^2}$ is convex.
Thus, by Jensen's Inequality and a linear change of variables
\begin{align*}
J(y)
    = \int_a^b \sqrt{1 + y'(x)^2} \, dx
 &  = (b - a) \int_0^1 \sqrt{1 + y'((b - a)u + a)^2} \, du
 &  \mbox{(Change of Variables)}    \\
 &  \geq (b - a) \sqrt{1 + \left( \int_0^1 y'((b - a)u + a) \, du \right)^2}
 &  \mbox{(Jensen's Inequality)}    \\
 &  = \sqrt{(b - a)^2 + \left( \int_a^b y'(x) \, dx \right)^2}
 &  \mbox{(Change of Variables)}    \\
 &  = \sqrt{(b - a)^2 + (B - A)^2},
\end{align*}
where the last step uses that $\int_a^b y'(x) = B - A$ by the Fundamental
Theorem of Calculus. \qed
\end{question}

\begin{question}{Problem 3}
\begin{enumerate}[(a)]
\item The bound represents the `vertical component' of the surface area:
\begin{align*}
J(y)
    = 2 \pi \int_0^b y(x) \sqrt{1 + y'(x)^2} \, dx
 &  \geq 2 \pi \int_0^b y(x) \sqrt{y'(x)^2} \, dx  \\
 &  \geq \pi \int_0^b 2 y(x) y'(x) \, dx
    = \pi y(x)^2 \bigg|_{x = 0}^{x = b}
    = \pi B^2,
\end{align*}
since $y \in \Y$ is non-negative. The bound essentially states that the minimal
surface area is at least the surface area of the disc of radius $B$ centered at
$(b,0)$, normal to the $x$-axis. \qed
 
\item If, for any $x \in [0,b]$, $y(x) > 0$, then
$y(x)\sqrt{1 + y'(x)} > y(x)\sqrt{y'(x)^2}$. Since $y$ and $y'$ are continuous,
it follows that the first inequality in part (a) above can be made strict, so
that no $y \in \Y$ achieves $J(y) = \pi B^2$. \qed
 
\item There is indeed such a sequence. For $n \in \N$, define $y_n \in \Y$ by
$y_n(x) = B(x/b)^n, \forall x \in [0,b]$. Then, using the inequality
$\sqrt{a + b} \leq \sqrt{a} + \sqrt{b}$ for $a,b \geq 0$,
\begin{align*}
J(y_n)
    = 2\pi \int_0^b y_n(x) \sqrt{1 + y_n'(x)^n}
 &  = 2\pi \int_0^b B(x/b)^n \sqrt{1 + (Bnx^{n - 1}/b^n)^2} \, dx  \\
 &  \leq 2\pi B \int_0^b (x/b)^n(1 + Bnx^{n - 1}/b^n) \, dx \\
 &  = 2\pi B \left( \frac{b}{n + 1} + \frac{B}{2} \right)
    \to \pi B^2
\end{align*}
as $n \to \infty$. By the result of part (a), it follows that
$J(y_n) \to \pi B^2$ as $n \to \infty$. \qed
 
\end{enumerate}
\end{question}
\end{document}
