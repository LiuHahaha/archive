\documentclass[11pt]{article}
\usepackage{enumerate}
\usepackage{fullpage}
\usepackage{fancyhdr}
\usepackage{amsmath, amsfonts, amsthm, amssymb}
\usepackage{mathrsfs}
\usepackage{graphicx}
\setlength{\parindent}{0pt}
\setlength{\parskip}{5pt plus 1pt}
\pagestyle{empty}

\def\indented#1{\list{}{}\item[]}
\let\indented=\endlist

\newcounter{questionCounter}
\newcounter{partCounter}[questionCounter]
\newenvironment{question}[2][\arabic{questionCounter}]{%
    \setcounter{partCounter}{0}%
    \vspace{.25in} \hrule \vspace{0.5em}%
        \noindent{\bf #2}%
    \vspace{0.8em} \hrule \vspace{.10in}%
    \addtocounter{questionCounter}{1}%
}{}
\renewenvironment{part}[1][\alph{partCounter}]{%
    \addtocounter{partCounter}{1}%
    \vspace{.10in}%
    \begin{indented}%
       {\bf (#1)} %
}{\end{indented}}

%%%%%%%%%%%%%%%%%%%%%%%HEADER%%%%%%%%%%%%%%%%%%%%%%%%%%%%%%
\newcommand{\myname}{Shashank Singh\footnote{sss1@andrew.cmu.edu}}
\newcommand{\myclass}{21-470 Calculus of Variations}
\newcommand{\myhwnum}{6}
\newcommand{\duedate}{Friday, May 2, 2014}
%%%%%%%%%%%%%%%%%%%%%%%%%%%%%%%%%%%%%%%%%%%%%%%%%%%%%%%%%%%

%%%%%%%%%%%%%%%%%%%%CONTENT MACROS%%%%%%%%%%%%%%%%%%%%%%%%%
\renewcommand{\qed}{\quad \ensuremath{\blacksquare}}
\newcommand{\inv}{^{-1}}
\newcommand{\bv}{\mathbf{v}}
\newcommand{\bx}{\mathbf{x}}
\newcommand{\by}{\mathbf{y}}
\newcommand{\bff}{\mathbf{f}}
\newcommand{\bzero}{\mathbf{0}}
\newcommand{\bxi}{\boldsymbol{\xi}}
\newcommand{\boldeta}{\boldsymbol{\eta}}
\newcommand{\dist}{\operatorname{dist}} % distance from or between sets
\newcommand{\area}{\operatorname{area}} % area of a polygon
\newcommand{\Gr}{\operatorname{Gr}}     % graph of a function
\renewcommand{\sp}{\operatorname{span}} % span of a set
\newcommand{\bdry}{\operatorname{bdry}} % boundary of a set
\newcommand{\sminus}{\backslash}        % set difference
\newcommand{\N}{\mathbb{N}}             % natural numbers
\newcommand{\Z}{\mathbb{Z}}             % integers
\newcommand{\Q}{\mathbb{Q}}             % rational numbers
\newcommand{\R}{\mathbb{R}}             % real numbers
\newcommand{\C}{\mathbb{C}}             % complex numbers
\newcommand{\D}{\mathcal{D}}            % domain of an operator
\newcommand{\Cmp}{\mathcal{C}}          % space of compact linear operators\s
\newcommand{\K}{\mathbb{K}}             % underlying field of a linear space
\newcommand{\Ran}{\mathcal{R}}          % range of a linear operator
\newcommand{\Nul}{\mathcal{N}}          % null-space of a linear operator
\renewcommand{\L}{\mathcal{L}}          % space of bounded linear functions
\newcommand{\pow}[1]{\mathcal{P}\left(#1\right)}    % power set of #1
\newcommand{\e}{\varepsilon}            % \varepsilon
\newcommand{\wto}{\rightharpoonup}      % weak convergence
\newcommand{\wsto}{\stackrel{*}{\rightharpoonup}}   % weak-* convergence
\renewcommand{\Re}{\operatorname{Re}}   % real part of a complex number
\newcommand{\sgn}{\operatorname{sign}}% sign of a real number
\newcommand{\tT}{\widetilde{T}}         % for P3
\newcommand{\A}{\mathcal{A}}            % for P3
\renewcommand{\S}{\mathscr{S}}          % constraint set
\newcommand{\X}{\mathscr{X}}            % entire linear space
\newcommand{\Y}{\mathscr{Y}}            % domain of objective functional
\newcommand{\V}{\mathscr{V}}            % set of admissible variations
%%%%%%%%%%%%%%%%%%%%%%%%%%%%%%%%%%%%%%%%%%%%%%%%%%%%%%%%%%%

\begin{document}
\thispagestyle{plain}

{\Large Homework \myhwnum} \\
\myclass \\
Name: \myname \\
Due: \duedate

\begin{question}{Problem 1}
The $1^{st}$ Euler-Lagrange Equation gives
\[-\frac{1}{2}\sqrt{\frac{1 + y'(x)^2}{(\gamma + y(x))^3}}
    = \frac{d}{dx} \frac{y'(x)}{\sqrt{(\gamma + y(x))(1 + y'(x)^2}}.
\]
Making the substitution $u(x) = \sqrt{\gamma + y(x)}$ (and noting that, since
$y(x) = u(x)^2 - \gamma, y'(x) = 2u(x)u'(x)$),
\[-\frac{1}{2}\sqrt{\frac{1 + 4u(x)^2u'(x)^2}{u(x)^3}}
    = \frac{d}{dx} \frac{2u(x)u'(x)}{u(x)\sqrt{(1 + 4u(x)^2u'(x)^2}}
    = \frac{d}{dx} \frac{2u'(x)}{\sqrt{(1 + 4u(x)^2u'(x)^2}}.
\]
[I guess I wasn't able to see the consequence of the $u$-substitution, as I
wasn't really sure how to proceed from here.]
\end{question}

\newpage
\begin{question}{Problem 2}
We show that, $\forall y \in \Y$, $J(y) \geq \alpha$, for
$\alpha := \left( \frac{4}{7} \right)^{9/2}\left( \frac{1}{2} \right)^{21/8}$.

Let $y \in \Y$. Since $y \in C^1[0,1]$ and $[0,1]$ is compact, $\exists M > 1$
such that $|y(x) - y(z)| < M|x - z|$, $\forall x,z \in [0,1]$. Hence, since
$y(0) = 0$, for $x \in [0,(2M^3)^{-1/2}]$, $y(x)^3 \leq M^3x^3 \leq x/2$ and
thus $|y(x)| \leq (x/2)^{1/3}$. Since $y(1) = 1$, by the Intermediate Value
Theorem, $\exists \beta \in (0, 1)$ such that $y(\beta) = (\beta/2)^{1/3}$
(in particular, we choose the smallest such $\beta$).

For all $x \in [0,\beta]$, since $2|y(x)|^3 \leq x$,
\begin{equation}
|y(x)|^3 \leq x - |y(x)|^3 \leq x - y(x)^3
    \Rightarrow y(x)^6 \leq (y(x)^3 - x)^2.
\label{ineq:2.1}
\end{equation}
Hence, by (a special case of) Jensen's Inequality,
\begin{align*}
J(y)
 &  \geq \int_0^\beta (y(x)^3 - x) |y'(x)|^{9/2} \, dx
    & \mbox{(non-negative integrand)}   \\
 &  \geq \int_0^\beta y(x)^6 |y'(x)|^{9/2} \, dx
    & \mbox{(by (\ref{ineq:2.1}))}   \\
 &  \geq \frac{1}{\beta^{7/2}}\left( \int_0^\beta \left| y(x)^{4/3} y'(x) \right| \, dx \right)^{9/2}
    & \mbox{(Jensen's Inequality)}   \\
 &  = \frac{1}{\beta^{7/2}}\left( \frac{4}{7} y(x)^{7/4} \bigg|_0^\beta \right)^{9/2}
    & \mbox{(Integration)}   \\
 &  = \frac{1}{\beta^{7/2}}\left( \frac{4}{7} (\beta/2)^{7/12} \right)^{9/2}
    & \mbox{($y(0) = 0, y(\beta) = (\beta/2)^{1/3}$)}   \\
 &  = \frac{1}{\beta^{28/8}}\left( \frac{4}{7} \right)^{9/2} (\beta/2)^{21/8}
    \geq \left( \frac{4}{7} \right)^{9/2}\left( \frac{1}{2} \right)^{21/8}.
    & \mbox{($\beta \in (0,1)$)}
\end{align*}
\end{question}

\newpage
\begin{question}{Problem 3}
As usual, define $P, Q : (\alpha,\beta) \times \R \to \R$ by
\begin{align*}
P(x,y) & = f(x,y,\Phi(x,y)) - \Phi(x,y)f_{,3}(x,y,\Phi(x,y))  \\
    \quad \mbox{ and } \quad
    Q(x,y) & = f_{,3}(x,y,\Phi(x,y)),
    \quad\quad \forall x \in (\alpha,\beta), y \in \R.
\end{align*}
Since we are working over $\R$, we trivially have that
$(Q_{,2}(x,y))^T = Q_{,2}(x,y)$.
By Remark 8.4 and the Chain Rule, it suffices to show that
\[P_{,2}(x,y)
    = f_{,3,1}(x,y,\Phi(x,y)) + f_{,3,3}(x,y,\Phi(x,y))\Phi_{,1}(x,y)
    = \frac{d}{dx} f_{,3}(x,y,\Phi(x,y))
    = Q_{,1}(x,y).
\]
Applying the Chain Rule and observing that some terms cancel (again, since we
work in $\R$ and so multiplication commutes),
\begin{align}
\notag
P_{,2}(x,y)
 &  = \frac{d}{dy} f(x,y,\Phi(x,y)) - \Phi(x,y)f_{,3}(x,y,\Phi(x,y))    \\
\notag
 &  = f_{,2}(x,y,\Phi(x,y)) + f_{,3}(x,y,\Phi(x,y))\Phi_{,2}(x,y)
    - \Phi_{,2}(x,y)f_{,3}(x,y,\Phi(x,y))   \\
\notag
 &  - \Phi(x,y) \left( f_{,3,2}(x,y,\Phi(x,y))
    + f_{,3,3}(x,y,\Phi(x,y))\Phi_{,2}(x,y) \right) \\
\label{eq:3a}
 &  = f_{,2}(x,y,\Phi(x,y)) - \Phi(x,y) \left( f_{,3,2}(x,y,\Phi(x,y))
    + f_{,3,3}(x,y,\Phi(x,y))\Phi_{,2}(x,y) \right).
\end{align}
Since $\Phi$ is a stationary field for $f$, if $y'(x) = \Phi(x,y(x))$, we have
\begin{align*}
f_{,2}(x,y(x),\Phi(x,y(x)))
 &  = \frac{d}{dx} f_{,3}(x,y(x),\Phi(x,y(x)))  \\
 &  = f_{,3,1}(x,y(x),\Phi(x,y(x)))
    + f_{,3,2}(x,y(x),\Phi(x,y(x)))\Phi(x,y(x)) \\
 &  + f_{,3,3}(x,y(x),\Phi(x,y(x)))
            \left( \Phi_{,1}(x,y(x)) + \Phi_{,2}(x,y(x))\Phi(x,y(x)) \right).
\end{align*}
Plugging this into Equation (\ref{eq:3a}) gives, after cancelling terms,
\[P_{,2}(x,y)
    = f_{,3,1}(x,y,\Phi(x,y)) + f_{,3,3}(x,y,\Phi(x,y))\Phi_{,1}(x,y). \qed
\]
\end{question}

\begin{question}{Problem 4}

I wasn't able to complete this question.
\end{question}
\end{document}
