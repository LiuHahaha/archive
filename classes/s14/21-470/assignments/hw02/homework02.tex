\documentclass[11pt]{article}
\usepackage{enumerate}
\usepackage{fullpage}
\usepackage{fancyhdr}
\usepackage{amsmath, amsfonts, amsthm, amssymb}
\usepackage{mathrsfs}
\setlength{\parindent}{0pt}
\setlength{\parskip}{5pt plus 1pt}
\pagestyle{empty}

\def\indented#1{\list{}{}\item[]}
\let\indented=\endlist

\newcounter{questionCounter}
\newcounter{partCounter}[questionCounter]
\newenvironment{question}[2][\arabic{questionCounter}]{%
    \setcounter{partCounter}{0}%
    \vspace{.25in} \hrule \vspace{0.5em}%
        \noindent{\bf #2}%
    \vspace{0.8em} \hrule \vspace{.10in}%
    \addtocounter{questionCounter}{1}%
}{}
\renewenvironment{part}[1][\alph{partCounter}]{%
    \addtocounter{partCounter}{1}%
    \vspace{.10in}%
    \begin{indented}%
       {\bf (#1)} %
}{\end{indented}}

%%%%%%%%%%%%%%%%%%%%%%%HEADER%%%%%%%%%%%%%%%%%%%%%%%%%%%%%%
\newcommand{\myname}{Shashank Singh\footnote{sss1@andrew.cmu.edu}}
\newcommand{\myclass}{21-470 Calculus of Variations}
\newcommand{\myhwnum}{1}
\newcommand{\duedate}{Friday, February 7, 2014}
%%%%%%%%%%%%%%%%%%%%%%%%%%%%%%%%%%%%%%%%%%%%%%%%%%%%%%%%%%%

%%%%%%%%%%%%%%%%%%%%CONTENT MACROS%%%%%%%%%%%%%%%%%%%%%%%%%
\renewcommand{\qed}{\quad \ensuremath{\blacksquare}}
\newcommand{\inv}{^{-1}}
\newcommand{\bv}{\mathbf{v}}
\newcommand{\bx}{\mathbf{x}}
\newcommand{\by}{\mathbf{y}}
\newcommand{\bff}{\mathbf{f}}
\newcommand{\bzero}{\mathbf{0}}
\newcommand{\bxi}{\boldsymbol{\xi}}
\newcommand{\boldeta}{\boldsymbol{\eta}}
\newcommand{\dist}{\operatorname{dist}} % distance from or between sets
\newcommand{\area}{\operatorname{area}} % area of a polygon
\newcommand{\Gr}{\operatorname{Gr}}     % graph of a function
\renewcommand{\sp}{\operatorname{span}} % span of a set
\newcommand{\bdry}{\operatorname{bdry}} % boundary of a set
\newcommand{\sminus}{\backslash}        % set difference
\newcommand{\N}{\mathbb{N}}             % natural numbers
\newcommand{\Z}{\mathbb{Z}}             % integers
\newcommand{\Q}{\mathbb{Q}}             % rational numbers
\newcommand{\R}{\mathbb{R}}             % real numbers
\newcommand{\C}{\mathbb{C}}             % complex numbers
\newcommand{\D}{\mathcal{D}}            % domain of an operator
\newcommand{\Cmp}{\mathcal{C}}          % space of compact linear operators\s
\newcommand{\K}{\mathbb{K}}             % underlying field of a linear space
\newcommand{\Ran}{\mathcal{R}}          % range of a linear operator
\newcommand{\Nul}{\mathcal{N}}          % null-space of a linear operator
\renewcommand{\L}{\mathcal{L}}          % space of bounded linear functions
\newcommand{\pow}[1]{\mathcal{P}\left(#1\right)}    % power set of #1
\newcommand{\e}{\varepsilon}            % \varepsilon
\newcommand{\wto}{\rightharpoonup}      % weak convergence
\newcommand{\wsto}{\stackrel{*}{\rightharpoonup}}   % weak-* convergence
\renewcommand{\Re}{\operatorname{Re}}   % real part of a complex number
\newcommand{\tT}{\widetilde{T}}         % for P3
\newcommand{\A}{\mathcal{A}}            % for P3
\renewcommand{\S}{\mathcal{S}}          % Schwartz space
\newcommand{\X}{\mathscr{X}}            % entire linear space
\newcommand{\Y}{\mathscr{Y}}            % domain of objective functional
\newcommand{\V}{\mathscr{V}}            % set of admissible variations
%%%%%%%%%%%%%%%%%%%%%%%%%%%%%%%%%%%%%%%%%%%%%%%%%%%%%%%%%%%

\begin{document}
\thispagestyle{plain}

{\Large Homework \myhwnum} \\
\myclass \\
Name: \myname \\
Due: \duedate

\begin{question}{Problem 1}
For any $y \in \Y$, integrating by parts and using the boundary condition
$y(1) = 1$,
\[\int_0^1 xy(x)^4
    = \frac{x^2}{2} y(x)^4 \bigg|_{x = 0}^{x = 1}
        - \int_0^1 2x^2 y(x)^3 y'(x) \, dx
    = \frac12 - \int_0^1 2x^2 y(x)^3 y'(x) \, dx.
\]
Consequently, $\forall y \in \Y$, $J(y) = 1/2$, and so each $y \in \Y$ is both
a minimizer and a maximizer.
\end{question}

\begin{question}{Problem 2}
First note that $J$ is unbounded above on $\Y$ (it is straightforward to
construct a sequence $\{p_n\}_{n = 1}^\infty$ of second-order polynomials in
$\Y$ with $J(p_n) \to +\infty$ as $n \to \infty$).

For $f : [1,2] \times \R \times \R \to \R$, defined by
\[f(x,y,z) := x^2z^2 + 2y^2 \quad \forall x \in [1,2], y,z \in \R,\]
$J(y) = \int_1^2 f(x,y(x),y'(x)) \, dx$, for all $y \in \Y$.
Since
\[f_{,2}(x,y,z)
    = 4y
\quad \mbox{ and } \quad
f_{,3}(x,y,z)
    = 2x^2z, \quad \forall x \in [0,1], y,z \in \R,
\]
if $y$ minimizes $J$ on $\Y$, the $1^{st}$ Euler-Lagrange Equation gives
\[4y(x)
    = \frac{d}{dx} 2x^2 y'(x)
    = 4xy'(x) + 2x^2y''(x).
\]
Since $x \mapsto x^{-2}$ and $x \mapsto x$ are independent solutions of this
linear second-order differential equation,
\[y(x) = \frac{c_2}{x^2} + c_1 x,\]
for some $c_1,c_2 \in \R$. Plugging in the boundary conditions and solving the
resulting linear system of equations gives $c_1 = 3, c_2 = -4$, so that
$y(x) = -4x^{-2} + 3x, \forall x \in [1,2]$. I wasn't able to show that this
minimizes $J$, but I think a convexity argument should suffice.
\end{question}

\newpage
\begin{question}{Problem 3}
First note that, $J$ is unbounded below on $\Y$. For $n \in \N$, define
$y_n \in \Y$ by $y_n(x) = \sin(nx), \forall x \in [0,\pi]$. Then, since
$\int_0^\pi \sin(x)^2 \, dx = \int_0^\pi \cos(x)^2 \, dx = \pi/2$,
\[J(y_n)
    = \int_0^\pi \sin(nx)^2 - n^2 \cos^2(nx) \, dx
    = (1 - n^2) \frac{\pi}{2} \to -\infty
\]
as $n \to \infty$. It is also apparently the case, although I was unable to
show this, that $J$ is non-positive (i.e., that $\|y\|_2 \leq \|y'\|_2$ for all
$y \in \Y$).

For $f : [0,\pi] \times \R \times \R \to \R$, defined by
\[f(x,y,z) := y^2 - z^2 \quad \forall x \in [0,\pi], y,z \in \R,\]
$J(y) = \int_0^\pi f(x,y(x),y'(x)) \, dx$, for all $y \in \Y$.
Since
\[f_{,2}(x,y,z)
    = 2y
\quad \mbox{ and } \quad
f_{,3}(x,y,z)
    = -2z, \quad \forall x \in [0,1], y,z \in \R,
\]
if $y$ minimizes $J$ on $\Y$, the $1^{st}$ Euler-Lagrange Equation gives
\[2y(x)
    = \frac{d}{dx} - 2y'(x)
    = -2y''(x).
\]
Since $\cos$ and $\sin$ are independent solutions of this linear second-order
differential equation,
\[y(x) = c_1 \cos(x) + c_2 \sin(x), \quad \forall x \in [0,\pi]\]
for some $c_1,c_2 \in \R$. The boundary conditions immediately imply $c_1 = 0$.
On the other hand
\[J(c_2 \sin)
    = \int_0^\pi c_2^2 \sin^2(x) - c_2^2 \cos^2(x) \, dx
    = 0,
\]
so that any multiple of $\sin$ maximizes $J$ on $\Y$.
\end{question}

\begin{question}{Problem 4}
As noted in Problem 3, $J$ is unbounded below on $\Y$. Without the boundary
condition $y(0) = 0$, $J$ is also unbounded above. For $n \in \N$, define
$y_n \in \Y$ by $y_n(x) = n(\pi - x), \forall x \in [0,\pi]$. Then,
\[J(y_n)
    = \int_0^\pi n^2 (\pi - x)^2 - n^2 \, dx
    = n^2 \frac{\pi^3}{3} - n^2\pi \to +\infty
\]
as $n \to \infty$.
\end{question}

\newpage
\begin{question}{Problem 5}
First note that, $J$ is unbounded above on $\Y$. For $n \in \N$, define
$y_n \in \Y$ by $y_n(x) = nx + 1, \forall x \in [0,1]$.
\[J(y_n)
    = \int_0^1 (n - x)^2 + 2x(nx + 1) \, dx \to +\infty
\]
as $n \to \infty$. For $f : [0,1] \times \R \times \R \to \R$, defined by
\[f(x,y,z) := (z - x)^2 + 2xy \quad \forall x \in [0,1], y,z \in \R,\]
$J(y) = \int_0^1 f(x,y(x),y'(x)) \, dx$, for all $y \in \Y$.
Since
\[f_{,2}(x,y,z)
    = 2x
\quad \mbox{ and } \quad
f_{,3}(x,y,z)
    = 2z - 2x, \quad \forall x \in [0,1], y,z \in \R,
\]
if $y$ minimizes $J$ on $\Y$, the $1^{st}$ Euler-Lagrange Equation gives
\[0
    = 2x - \frac{d}{dx} (2y'(x) - 2x)
    = x - y''(x) + 1,
\]
and so $y''(x) = x + 1$. Integrating with respect to $x$ twice gives
\[y(x) = \frac16 x^3 + \frac12 x^2 + c_1 x + c_2,\]
for some $c_1,c_2 \in \R$. Since $y(0) = 1$, $c_2 = 1$. The second boundary
condition derived for the free right endpoint is
$0 = f_{,3}(x,y(x),y'(x)) \big|_{x = 1} = 2y'(1) - 2$, and it follows that
$c_1 = -1/2$.
I wasn't able to show that this minimizes $J$, but I think a convexity argument
should suffice.
\end{question}

\vspace{-3mm}
\begin{question}{Problem 6}
Note that $J$ is unbounded above on $\Y$ (it is straightforward to
construct a sequence $\{p_n\}_{n = 1}^\infty$ of second-order polynomials in
$\Y$ with $J(p_n) \to +\infty$ as $n \to \infty$).
For $f : [1,8] \times \R \times \R \to \R$, defined by
\[f(x,y,z) := xz^4 \quad \forall x \in [1,8], y,z \in \R,\]
$J(y) = \int_1^8 f(x,y(x),y'(x)) \, dx$, for all $y \in \Y$.
Since
\[f_{,2}(x,y,z)
    = 0
\quad \mbox{ and } \quad
f_{,3}(x,y,z)
    = 4xz^3, \quad \forall x \in [1,8], y,z \in \R,
\]
if $y$ minimizes $J$ on $\Y$, the $1^{st}$ Euler-Lagrange Equation gives
\[0
    = - \frac{d}{dx} 4xy'(x)^3
    = - 4 y'(x)^3 - 12xy'(x)^2y''(x)
    = y'(x) + 3x y''(x)
\]
Since $x \mapsto x^{2/3}$ and any non-zero constant function are independent
solutions of this linear second-order differential equation,
$y(x) = c_1 x^{2/3} + c_2$, for some $c_1,c_2 \in \R$. The boundary conditions 
give $c_1 = 1, c_2 = 3$. I wasn't able to show that this minimizes $J$, but I
think a convexity argument should suffice.
\end{question}

\begin{question}{Problem 7}
We conclude that $g$ is constant on $[a,b]$.
Suppose, for sake of contradiction, that $\exists x,y \in (a,b)$ with
$g(x) \neq g(y)$ (without loss of generality, $x < y$ and $g(x) < g(y)$).
Since $g$ is continuous, $\exists \delta > 0$ with
\[a \leq x - \delta < x + \delta \leq y - \delta < y + \delta \leq b,\]
such that, for some $\e > 0$,
\[\inf \{g(z) : z \in (y - \delta, y + \delta)\}
    - \sup \{g(z) : z \in (x - \delta, x + \delta)\}
\geq \e.\]
Define $v : [a,b] \to \R$ for all $x \in [a,b]$ by
\vspace{-3mm}
\[
    v(z) := \left\{
        \begin{array}{cc}
            -\exp \left( -\frac{1}{1 - ((z - x)/\delta)^2} \right)
                                        & : z \in (x - \delta,x + \delta)   \\
            \exp \left( -\frac{1}{1 - ((z - y)/\delta)^2} \right)
                                        & : z \in (y - \delta,y + \delta)   \\
            0                           & \mbox{ else}
        \end{array}
    \right..
\]
Since the bump function $z \mapsto \exp\left(
-\frac{1}{1 - z^2} \right)1_{(-1,1)}$ (where $1_{(-1,1)}$ denotes the indicator
function of $(-1,1)$) is in $C^\infty(\R)$, $v \in C^\infty([a,b])$.
Furthermore, $v(a) = v(b) = 0$, and
\vspace{-3mm}
\[\int_a^b v(z) \, dz
    = \int_{y - \delta}^{y + \delta} 
            \exp \left( -\frac{1}{1 - ((z - y)/\delta)^2} \right) \, dz
    - \int_{x - \delta}^{x + \delta} 
            \exp \left( -\frac{1}{1 - ((z - x)/\delta)^2} \right) \, dz
    = 0,
\]
so that $v \in \overline{\V}$. However, a translating change of variables gives
\vspace{-2mm}
\begin{align*}
\int_a^b g(z)v(z) \, dz
i &  = \int_{y - \delta}^{y + \delta} (g(z) - g(z + x - y))
            \exp \left( -\frac{1}{1 - ((z - y)/\delta)^2} \right) \, dz \\
 &  \geq \e \int_{y - \delta}^{y + \delta} 
            \exp \left( -\frac{1}{1 - ((z - y)/\delta)^2} \right) \, dz
    > 0,
\end{align*}
giving a contradiction. \qed
\end{question}

\vspace{-5mm}
\begin{question}{Problem 8}
At any $y \in \Y$, the set of admissible variations at $y$ is
\vspace{-3mm}
\[\V := \left\{ v \in C^2([a,b])
                            : v(a) = v(b) = \int_a^b v(x) \, dx = 0 \right\}.\]
\vspace{-3mm}
Thus, for any extremum $y \in \Y, v \in \V$, the G\^ateaux variation
satisfies
\begin{align*}
0
    = \delta J(y;v)
    = \int_a^b f_{,2}(x,y(x),y'(x)) v(x) + f_{,3}(x,y(x),y'(x)) v'(x) \, dx \\
    = \int_a^b \left[ f_{,2}(x,y(x),y'(x)) 
                    - \frac{d}{dx} f_{,3}(x,y(x),y'(x)) \right] v(x) \, dx,
\end{align*}
via integration by parts and $v(a) = v(b) = 0$. By the result of
Problem 7, $\exists C \in \R$ such that
\vspace{-3mm}
\[C
   = f_{,2}(x,y(x),y'(x)) - \frac{d}{dx} f_{,3}(x,y(x),y'(x)),
    \quad \forall x \in [a,b]. \qed\]
\end{question}

\begin{question}{Problem 9}
Multiplying the $1^{th}$ Euler-Lagrange Equation by $y'(x)$ on both sides,
$\forall x \in [a,b]$,
\begin{equation}
\label{eq:fEL1}
y'(x)f_{,2}(x,y(x),y'(x)) = y'(x)\frac{d}{dx}f_{,3}(x,y(x),y'(x)).
\end{equation}
The Chain Rule gives
\[\frac{d}{dx} f(x,y(x),y'(x))
    = f_{,1}(x,y(x),y'(x)) + y'(x)f_{,2}(x,y(x),y'(x))
                                                + y''(x)f_{,3}(x,y(x),y'(x))
\]
\[\Rightarrow
y'(x)f_{,2}(x,y(x),y'(x))
    = \frac{d}{dx} f(x,y(x),y'(x))
    - f_{,1}(x,y(x),y'(x)) - y''(x)f_{,3}(x,y(x),y'(x)).
\]
Plugging this into Equation (\ref{eq:fEL1}) and rearranging gives
\begin{align*}
\frac{d}{dx} f(x,y(x),y'(x)) - f_{,1}(x,y(x),y'(x))
 &  = y'(x)\frac{d}{dx}f_{,3}(x,y(x),y'(x)) + y''(x)f_{,3}(x,y(x),y'(x))    \\
 &  = \frac{d}{dx} y'(x) f_{,3}(x,y(x),y'(x)).
\end{align*}
By the product rule. Rearranging again gives
\[
\frac{d}{dx} \left( f(x,y(x),y'(x)) - y'(x) f_{,3}(x,y(x),y'(x)) \right)
    = f_{,1}(x,y(x),y'(x)),
\]
and so integrating with respect to $x$ gives, for some $c \in \R$,
\[
f(x,y(x),y'(x)) - y'(x) f_{,3}(x,y(x),y'(x))
    = c + \int_a^x f_{,1}(t,y(t),y'(t)) \, dt. \qed
\]
\end{question}
\end{document}
