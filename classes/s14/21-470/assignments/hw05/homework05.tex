\documentclass[11pt]{article}
\usepackage{enumerate}
\usepackage{fullpage}
\usepackage{fancyhdr}
\usepackage{amsmath, amsfonts, amsthm, amssymb}
\usepackage{mathrsfs}
\usepackage{graphicx}
\setlength{\parindent}{0pt}
\setlength{\parskip}{5pt plus 1pt}
\pagestyle{empty}

\def\indented#1{\list{}{}\item[]}
\let\indented=\endlist

\newcounter{questionCounter}
\newcounter{partCounter}[questionCounter]
\newenvironment{question}[2][\arabic{questionCounter}]{%
    \setcounter{partCounter}{0}%
    \vspace{.25in} \hrule \vspace{0.5em}%
        \noindent{\bf #2}%
    \vspace{0.8em} \hrule \vspace{.10in}%
    \addtocounter{questionCounter}{1}%
}{}
\renewenvironment{part}[1][\alph{partCounter}]{%
    \addtocounter{partCounter}{1}%
    \vspace{.10in}%
    \begin{indented}%
       {\bf (#1)} %
}{\end{indented}}

%%%%%%%%%%%%%%%%%%%%%%%HEADER%%%%%%%%%%%%%%%%%%%%%%%%%%%%%%
\newcommand{\myname}{Shashank Singh\footnote{sss1@andrew.cmu.edu}}
\newcommand{\myclass}{21-470 Calculus of Variations}
\newcommand{\myhwnum}{5}
\newcommand{\duedate}{Wednesday, April 23, 2014}
%%%%%%%%%%%%%%%%%%%%%%%%%%%%%%%%%%%%%%%%%%%%%%%%%%%%%%%%%%%

%%%%%%%%%%%%%%%%%%%%CONTENT MACROS%%%%%%%%%%%%%%%%%%%%%%%%%
\renewcommand{\qed}{\quad \ensuremath{\blacksquare}}
\newcommand{\inv}{^{-1}}
\newcommand{\bv}{\mathbf{v}}
\newcommand{\bx}{\mathbf{x}}
\newcommand{\by}{\mathbf{y}}
\newcommand{\bff}{\mathbf{f}}
\newcommand{\bzero}{\mathbf{0}}
\newcommand{\bxi}{\boldsymbol{\xi}}
\newcommand{\boldeta}{\boldsymbol{\eta}}
\newcommand{\dist}{\operatorname{dist}} % distance from or between sets
\newcommand{\area}{\operatorname{area}} % area of a polygon
\newcommand{\Gr}{\operatorname{Gr}}     % graph of a function
\renewcommand{\sp}{\operatorname{span}} % span of a set
\newcommand{\bdry}{\operatorname{bdry}} % boundary of a set
\newcommand{\sminus}{\backslash}        % set difference
\newcommand{\N}{\mathbb{N}}             % natural numbers
\newcommand{\Z}{\mathbb{Z}}             % integers
\newcommand{\Q}{\mathbb{Q}}             % rational numbers
\newcommand{\R}{\mathbb{R}}             % real numbers
\newcommand{\C}{\mathbb{C}}             % complex numbers
\newcommand{\D}{\mathcal{D}}            % domain of an operator
\newcommand{\Cmp}{\mathcal{C}}          % space of compact linear operators\s
\newcommand{\K}{\mathbb{K}}             % underlying field of a linear space
\newcommand{\Ran}{\mathcal{R}}          % range of a linear operator
\newcommand{\Nul}{\mathcal{N}}          % null-space of a linear operator
\renewcommand{\L}{\mathcal{L}}          % space of bounded linear functions
\newcommand{\pow}[1]{\mathcal{P}\left(#1\right)}    % power set of #1
\newcommand{\e}{\varepsilon}            % \varepsilon
\newcommand{\wto}{\rightharpoonup}      % weak convergence
\newcommand{\wsto}{\stackrel{*}{\rightharpoonup}}   % weak-* convergence
\renewcommand{\Re}{\operatorname{Re}}   % real part of a complex number
\newcommand{\sgn}{\operatorname{sign}}% sign of a real number
\newcommand{\tT}{\widetilde{T}}         % for P3
\newcommand{\A}{\mathcal{A}}            % for P3
\renewcommand{\S}{\mathscr{S}}          % constraint set
\newcommand{\X}{\mathscr{X}}            % entire linear space
\newcommand{\Y}{\mathscr{Y}}            % domain of objective functional
\newcommand{\V}{\mathscr{V}}            % set of admissible variations
%%%%%%%%%%%%%%%%%%%%%%%%%%%%%%%%%%%%%%%%%%%%%%%%%%%%%%%%%%%

\begin{document}
\thispagestyle{plain}

{\Large Homework \myhwnum} \\
\myclass \\
Name: \myname \\
Due: \duedate

\begin{question}{Problem 1}
For $f : [0,1] \times \R^2 \times \R^2 \to \R$ defined by
$f(x,y,z) = z_1^4 + z_2^2 - z_1y_2 + y_1^3 - 2xy_2$,
\[f_{,2}(x,y,z)
    =   \begin{bmatrix}
            3y_1^2   \\
            -z_1 - 2x
        \end{bmatrix}
    \quad \mbox{ and } \quad
    f_{,3}(x,y,z)
    =   \begin{bmatrix}
            4z_1^3 - y_2    \\
            2z_2
        \end{bmatrix}.
\]
Hence, if $y_*$ minimizes $J$ over $\Y$, then the 1$^{st}$ Euler-Lagrange
Equation gives
\[
    \begin{bmatrix}
        0   \\
        0
    \end{bmatrix}
    = f_{2,}(x,y(x),y'(x)) - \frac{d}{dx} f_{,3}(x,y(x),y'(x))  \\
    =
\begin{bmatrix}
  3y_1(x)^2 - \frac{d}{dx} (4y_1'(x)^3 - y_2(x)\\
  - y_1'(x) - 2x - \frac{d}{dx} 2y_2'(x)
\end{bmatrix}.
\]
We also have the natural boundary conditions
\[0
    =   \begin{bmatrix}
            1   \\
            1
        \end{bmatrix} \cdot
        f_{,3}(x,y(x),y'(x)) \bigg|_{x = 0}
    = 4y_1'(0)^3 - y_2(0) + 2y_2'(0)
\]
and
\[0
    =   \begin{bmatrix}
            1   \\
            -1/2
        \end{bmatrix} \cdot
        f_{,3}(x,y(x),y'(x)) \bigg|_{x = 1}
    = 4y_1'(1)^3 - y_2(1) - y_2'(1).
\]
\end{question}

\newpage
\begin{question}{Problem 2}
For $f : [0,\pi/2] \times \R \times \R \times \R \to \R$ defined by
$f(x,y,z,w) = w^2 - z^2 - 2y$, if $y_*$ minimizes $J$ over $\Y$, then the
1$^{st}$ Euler-Lagrange Equation gives that $y'' \in C^2[0,\pi/2]$ and,
$\forall x \in [0,\pi/2]$,
\begin{align}
\notag
0
 &  = f_{2,}(x,y(x),y'(x),y''(x))
        - \frac{d}{dx} f_{,3}(x,y(x),y'(x),y''(x))
        + \frac{d^2}{dx^2} f_{,4}(x,y(x),y'(x),y''(x))  \\
 &  = -2 + 2y''(x) + 2y^{(4)}(x).
\label{eq:2}
\end{align}
Solutions to $0 = y''(x) + y^{(4)}(x)$ are $\cos$, $\sin$, constant functions,
and linear functions, a particular solution to (\ref{eq:2}) is the function
$x \mapsto x^2/2$, so that the general solution to (\ref{eq:2}) is
\[y(x) = c_1\cos(x) + c_2\sin(x) + c_3x + c_4 + x^2/2,
    \quad \forall x \in [0,\pi/2].
\]
Thus, we have the boundary conditions
\[0
    = f_{,4}(x,y(x),y'(x),y''(x))\big|_{x = 0}
    = 2y''(0)
    = -2c_1 + 2,
\]
\begin{align*}
0     = y(0) = c_1 + c_4,
\quad\quad
0   & = y(\pi/2) = c_2 + c_3\pi/2 + c_4 + \pi^2/8,  \\
\mbox{ and } \quad
0   & = y'(\pi/2) = -c_1 + c_3 + \pi/2,
\end{align*}
giving $c_1 = 1,c_2 = \pi^2/8 - 1 - \pi/2, c_3 = 1 - \pi/2,$ and $c_4 = -1$.
\end{question}

\begin{question}{Problem 3}
For $f : [1,2] \times \R \times \R \times \R \to \R$ defined by
$f(x,y,z,w) = x^3w^2 - 24xy$, if $y_*$ minimizes $J$ over $\Y$, then the
1$^{st}$ Euler-Lagrange Equation gives, $\forall x \in [0,\pi/2]$,
\begin{align*}
0
 &  = f_{2,}(x,y(x),y'(x),y''(x))
        - \frac{d}{dx} f_{,3}(x,y(x),y'(x),y''(x))
        + \frac{d^2}{dx^2} f_{,4}(x,y(x),y'(x),y''(x))  \\
 &  = -24x + \frac{d^2}{dx^2} 2x^3y''(x),
\end{align*}
and hence, integrating twice gives $x^3 y''(x) = 2x^3 + c_1x + c_2$, for some
$c_1,c_2 \in \R$. Dividing both sides by $x^3$ and integrating twice more
gives, for some $c_3,c_4 \in \R$,
\[y(x) = x^2 + c_3x + c_4 - c_1 \ln x + \frac{c_2}{2x},
    \quad \forall x \in [1,2].
\]
The given boundary conditions give
\begin{align*}
6 & = y(1) = 1 + c_3 + c_4 + c_2/2 \\
-1 & = y'(1) = 2 + c_3 - c_1 - c_2/2 \\
8 & = y(2) = 4 + 2c_3 + c_4 - c_1 \ln 2 + c_2/4 \\
4 & = y'(2) = 4 + c_3 - c_1/2 - c_2/8.
\end{align*}
Solving this (nonsingular) linear system gives
$c_1 = c_4 = 0, c_2 = 8, c_3 = 1$, and so
\[y(x) = x^2 + x + 4/x, \quad \forall x \in [1,2].\]
\end{question}

\newpage
\begin{question}{Problem 4}
Let $f : [0,3] \times \R \times \R \to \R$ defined by $f(x,y,z) = z^4 - 8z^2$.
\begin{enumerate}[(a)]
\item The $1^{st}$ Weierstrass-Erdmann Corner Condition gives that
\begin{equation}
4\alpha^3 - 16\alpha
    = f_{,3}(c,y(c),\alpha)
    = f_{,3}(c,y(c),\beta)
    = 4\beta^3 - 16\beta.
\label{eq:WE1}
\end{equation}
The $2^{nd}$ Weierstrass-Erdmann Corner Condition gives that
\begin{equation}
- 3\alpha^4 + 8\alpha^2
    = f(c,y(c),\alpha) - \alpha f_{,3}(c,y(c),\alpha)  
    = f(c,y(c),\beta) - \alpha f_{,3}(c,y(c),\beta)     
    = - 3\beta^4 + 8\beta^2.
\label{eq:WE2}
\end{equation}
Equation (\ref{eq:WE1}) can be written as
\[(\alpha - \beta)(\alpha^2 + \alpha\beta + \beta^2)
    = \alpha^3 - \beta^3
    = 4(\alpha - \beta),
\]
so that, since $c \in S(y)$ and hence $\alpha \neq \beta$,
\begin{equation}
\alpha^2 + \alpha\beta + \beta^2 = 4
\label{eq:WE1simp}
\end{equation}
Equation (\ref{eq:WE2}) can be written as
\[3(\alpha - \beta)(\alpha + \beta)(\alpha^2 + \beta^2)
    = 3(\alpha^4 - \beta^4)
    = 8(\alpha^2 - \beta^2)
    = 8(\alpha - \beta)(\alpha + \beta)
\]
so that (again, since $c \in S(y)$ and hence $\alpha \neq \beta$),
\[3(\alpha + \beta)(\alpha^2 + \beta^2) = 8(\alpha + \beta).\]
Thus, we have two cases: either (C1) $\alpha^2 + \beta^2 = 8/3$ or (C2)
$\alpha = - \beta$.

Case (C1): ($\alpha^2 + \beta^2 = 8/3$) Using Equation (\ref{eq:WE1simp}), we have
\[\alpha\beta = 4 - 8/3 = 4/3,\]
so that $\sgn(\alpha) = \sgn(\beta)$ and $\beta = \frac{4}{3\alpha}$. Plugging
this back into (C1) and rearranging gives
\[9\alpha^4 - 24\alpha^2 + 16 = 0,\]
so that the quadratic formula gives $\alpha^2 = 4/3$. But then (C1) gives
$\beta^2 = 4/3$, and so $\alpha = \beta$ (since $\sgn(\alpha) = \sgn(\beta)$),
contradicting the fact that $c \in S(y)$.

Case 2 (C2): ($\alpha = -\beta$) Using Equation (\ref{eq:WE1simp}) gives
$\alpha^2 = 4$, so that $(\alpha,\beta) \in \{(2,-2),(-2,2)\}$.

This is consistent with the observation that $f(x,y,z) = (z^2 - 4) - 16$, which
is clearly minimized when $z = \pm 2$.

\item Minimizers with exactly $1$ corner must be piecewise linear, with slope
$2$ on $(0,a)$ and slope $-2$ on $(a,3)$ or slope $-2$ on $(0,3-a)$ and slope
$2$ on $(3-a,3)$, for some $a \in [0,3]$. Since $y(0) = 0$ and $y(3) = 2$,
$a = 2$, and the two possible minimizers with exactly one corner are
\[y_1(x) =
    \left\{
        \begin{array}{ll}
            2x & \mbox{ if } x \in [0,2] \\
            8 - 2x & \mbox{ if } x \in [2,3]
        \end{array}
    \right.
    \quad \mbox{ and } \quad
y_2(x) =
    \left\{
        \begin{array}{ll}
            -2x & \mbox{ if } x \in [0,1] \\
            2x - 4 & \mbox{ if } x \in [1,3]
        \end{array}
    \right..    
\]
\end{enumerate}
\end{question}
\end{document}
